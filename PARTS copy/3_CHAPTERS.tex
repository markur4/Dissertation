
% ======================================================================
% == Chapter 1
% ======================================================================
\unnsection{Chapter 1: Modelling Myeloma Dissemination \textit{in vitro}}
% Title: Keep it Together: Modelling Myeloma Dissemination in vitro with
% hMSC- Interacting Subpopulations of INA-6 Cells and their
% Aggregation/Detachment Dynamics 


% == Paper 1 ===========================================================
% ## Import paper here
\addpdf{Research Article: Cancer Research Communications}{PUBLICATIONS/AACR.pdf}


% == Paper 1 Supplemental ==============================================
% ## Import paper here
\addpdf{Methods: Supplementary Figures and Methods}{PUBLICATIONS/? AACR Supplemental.pdf}



% ======================================================================
% == Chapter 2
% ======================================================================
\unnsection{Chapter 2: Semi-Automation of Data Analysis}
% Article title: "plotastic: Bridging Plotting and Statistics in Python"

% == Sub-Introduction ==================================================
% GSLS asked me to nest the second chapter between another introduction
% and discussion


\unnsubsection{Introduction}

Why did I need plotastic?


Why do biologists need plotastic?
- Assays output more data in shorter time, e.g. multiplex qPCR
- example: 20 genes, 3 timepoints, 11 biological replicates, (all 3
technical replicates already averaged)
- 20 * 3 * 11 = 660 data points

this is multidimensional data:  660 data points spread across two dimensions: time
and gene

shortly Describe Main Packages in more detail:
- seaborn: It multidimensional data
- pingouin: It's a statistical package

-in manual analysis e.g. in Excel, the user has to manually select the
data, copy it, paste it into a new sheet, and then perform the
statistical test. In Prism, the user has to select the data, click on
the statistical test, and then select the data again. This is not only
time-consuming, but also prone to



% == Paper 2 ===========================================================
% ## Import paper here
\addpdf[.93]{Software Article: Journal of Open Source Software}{PUBLICATIONS/JOSS_plotastic.pdf}


% == Sub-Discussion ====================================================
\unnsubsection{Discussion}



Is plotastic useful for biologists?
- Yes but use is limited by minimal knowledge of Python
- However, that is subject to change as Python is becoming more popular
in biology and AI assisted coding decreased the barrier to entry
significantly. Tools like github copilot are able to generate code, fix
bugs and suggest improvements. This is a game changer for biologists
that are not familiar with programming.
- Furthermore, installing and using plotastic for biologists is overestimated. These
steps re needed:
- Install anaconda from the internet
- Open the terminal
- Type \texttt{pip install plotastic}
- Check Rea
