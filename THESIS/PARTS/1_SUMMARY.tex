

% ======================================================================
% == Abstract
% ======================================================================

% ## Treat this like a section (\section would add a header)
\addcontentsline{toc}{section}{Summary / Zusammenfassung} % Add to Table of Contents
\markboth{Summary / Zusammenfassung}{} % > Mark this section for headers

% == from paper ========================================================
% Abstract 1st paper (AACR):
% Multiple myeloma involves early dissemination of malignant plasma
% cells across the bone marrow; however, the initial steps of
% dissemination remain unclear. Human bone marrow-derived mesenchymal
% stromal cells (hMSCs) stimulate myeloma cell expansion (e.g., IL-6)
% and simultaneously retain myeloma cells via chemokines (e.g., CXCL12)
% and adhesion factors. Hence, we hypothesized that the imbalance
% between cell division and retention drives dissemination. 

% We present an in vitro model using primary hMSCs co-cultured with
% INA-6 myeloma cells. Time-lapse microscopy revealed proliferation and
% attachment/detachment dynamics. Separation techniques (V-well adhesion
% assay and well plate sandwich centrifugation) were established to
% isolate MSC-interacting myeloma subpopulations that were characterized
% by RNAseq, cell viability and apoptosis. Results were correlated with
% gene expression data (n=837) and survival of myeloma patients (n=536). 

% On dispersed hMSCs, INA-6 saturate hMSC-surface before proliferating
% into large homotypic aggregates, from which single cells detached
% completely. On confluent hMSCs, aggregates were replaced by strong
% heterotypic hMSC-INA-6 interactions, which modulated apoptosis
% time-dependently. Only INA-6 daughter cells (nMA-INA6) detached from
% hMSCs by cell division but sustained adherence to hMSC-adhering mother
% cells (MA-INA6).

% Isolated nMA-INA6 indicated hMSC-autonomy through superior viability
% after IL­6 withdrawal and upregulation of proliferation-related genes.
% MA-INA6 upregulated adhesion and retention factors (CXCL12), that,
% intriguingly, were highly expressed in myeloma samples from patients
% with longer overall and progression-free survival, but their
% expression decreased in relapsed myeloma samples. 

% Altogether, in vitro dissemination of INA-6 is driven by detaching
% daughter cells after a cycle of hMSC-(re)attachment and proliferation,
% involving adhesion factors that represent a bone marrow-retentive
% phenotype with potential clinical relevance.

%%%%%%%%%%%%%%%%%%%%%%%%%%%%%%%%%%%%%%%%%%%%%%%%%%%%%%%%%%%%%%%%%%%%%%%%
% Summary 2nd paper (JOSS):
% plotastic addresses the challenges of transitioning from exploratory
% data analysis to hypothesis testing in Python's data science ecosystem.
% Bridging the gap between seaborn and pingouin, this library offers a
% unified environment for plotting and statistical analysis. It simplifies
% the workflow with a user-friendly syntax and seamless integration with
% familiar seaborn parameters (y, x, hue, row, col). Inspired by seaborn's
% consistency, plotastic utilizes a DataAnalysis object to intelligently
% pass parameters to pingouin statistical functions. The library
% systematically groups the data according to the needs of statistical
% tests and plots, conducts visualisation, analyses and supports extensive
% customization options. In essence, plotastic establishes a protocol for
% configuring statical analyses through plotting parameters. This approach
% streamlines the process, translating seaborn parameters into statistical
% terms, allowing researchers to focus on correct statistical testing and
% less about specific syntax and implementations.

% Statement of need 2nd paper (JOSS):
% Python's data science ecosystem provides powerful tools for both
% visualization and statistical testing. However, the transition from
% exploratory data analysis to hypothesis testing can be cumbersome,
% requiring users to switch between libraries and adapt to different
% syntaxes.

% seaborn has become a popular choice for plotting in Python, offering an
% intuitive interface. Its statistical functionality focuses on
% descriptive plots and bootstrapped confidence intervals (Waskom, 2021).
% The library pingouin offers an extensive set of statistical tests, but
% it lacks integration with common plotting capabilities (Vallat, 2018).
% statannotations integrates statistical testing with plot annotations,
% but uses a complex interface and is limited to pairwise comparisons
% (Charlier et al., 2022).

% plotastic addresses this gap by offering a unified environment for
% plotting and statistical analysis. With an emphasis on user-friendly
% syntax and integration with familiar seaborn parameters, it simplifies
% the process for users already comfortable seaborn. The library ensures a
% smooth workflow, from data import to hypothesis testing and
% visualization


% == English ===========================================================
\renewcommand{\abstractname}{Summary}
\begin{abstract}
    \label{Summary}%
    %

    % This PhD thesis integrates biomedical research and
    % data science, focusing on the development of an \textit{in vitro} model for
    % studying myeloma cell dissemination and the creation of a software tool for
    % semi-automated analysis of multidimensional datasets. The work addresses
    % critical challenges in biomedical research, including (1) understanding the
    % early steps of myeloma dissemination and (2) improving the efficiency of
    % data analysis, which \dashed{shown in this thesis} represents a
    % reproducibility bottleneck in biomedical research.

    This thesis integrates biomedical research and data science, focusing on
    an \textit{in vitro} model for studying myeloma cell dissemination and a
    Python-based tool, \texttt{plotastic}, for semi-automated analysis of
    multidimensional datasets. Two major challenges are adressed: (1)
    understanding the early steps of myeloma dissemination and (2) improving
    data analysis efficiency to address the complexity- and reproducibility
    bottlenecks currently present in biomedical research.

    In the experimental component, primary human mesenchymal stromal cells
    (hMSCs) are co-cultured with INA-6 myeloma cells to study cell
    proliferation, attachment, and detachment via time-lapse microscopy. Key
    findings reveal that detachment often follows cell division, predominantly
    driven by daughter cells. Novel separation techniques were developed to
    isolate myeloma subpopulations for further characterization by RNAseq, cell
    viability, and apoptosis assays. Differential expression of adhesion and
    retention factors upregulated by INA-6 cells correlates with patient
    survival. Overall, this work provides insights into myeloma dissemination
    mechanisms and identifies genes that potentially counteract dissemination
    through adhesion, which could be relevant for the design of new
    therapeutics.

    To manage complex data, a Python-based software named \texttt{plotastic} was
    developed that streamlines analysis and visualization of multidimensional
    datasets. \texttt{plotastic} is built on the idea that statistical analyses
    are performed based on how the data is visualized. This approach not only
    simplifies data analysis, but semi-automates analysis in a standardized
    statistical protocol. The thesis becomes a case study as it reflects on the
    application of \texttt{plotastic} to the \textit{in vitro} model,
    demonstrating how the software facilitates rapid adjustments and refinements
    in data analysis and presentation. Such efficiency is crucial for handling
    semi-big data transparently, which \dashed{despite being managable}
    is complex enough to complicate analysis and reproducibility.

    Together, this thesis illustrates the synergy between experimental
    methodologies and advanced data analysis tools. The \textit{in vitro} model
    provides a robust platform for studying myeloma dissemination, while
    \texttt{plotastic} addresses the need for efficient data analysis. Combined,
    they offer a comprehensive approach to handling complex experiments,
    advancing both cancer biology and research practices, in support of
    exploratory and transparent analysis of challenging phenomena.
\end{abstract}

\newpage
% \cleardoublepage % > Ensure the next part starts on an odd page





% == German ============================================================
\renewcommand{\abstractname}{Zusammenfassung}
\begin{abstract}
    \label{Zusammenfassung}
    Diese Doktorarbeit integriert biomedizinische Forschung und
    Datenwissenschaften und konzentriert sich auf ein \textit{in vitro}-Modell
    zur Untersuchung der Dissemination von \linebreak Myelomzellen sowie ein
    Python-basiertes Werkzeug, \texttt{plotastic}, zur semi-automatisierten
    Analyse multidimensionaler Datensätze. Zwei Hauptprobleme werden bearbeitet:
    (1) das Verständnis der frühen Schritte der Myelomdissemination und (2) die
    Verbesserung der Effizienz der Datenanalyse, um die derzeit in der
    biomedizinischen Forschung vorhandenen Engpässe bezüglich Komplexität und
    Reproduzierbarkeit zu adressieren.

    Im experimentellen Teil werden primäre menschliche mesenchymale Stromazellen
    \linebreak (hMSCs) mit INA-6-Myelomzellen kokultiviert, um Zellproliferation,
    Anhaftung und Ablösung mittels Zeitraffer-Mikroskopie zu untersuchen.
    Zentrale Erkenntnisse zeigen, dass die Ablösung oft auf die Zellteilung
    folgt und vorwiegend von Tochterzellen angetrieben wird. Neue
    Trennungstechniken wurden entwickelt, um Myelom-Subpopulationen für weitere
    Charakterisierungen durch RNAseq, Zellviabilität und Apoptose-Assays zu
    isolieren. Die differentielle Expression von Adhäsions- und
    Retentionsfaktoren, die durch INA-6 Zellen hochreguliert werden, korreliert
    mit dem Überleben der Patienten. Insgesamt liefert diese Arbeit Einblicke in
    die Mechanismen der Myelomdissemination und identifiziert Gene, die
    potenziell die Dissemination durch Adhäsion konterkarieren könnten, was für
    die Entwicklung neuer Therapeutika relevant sein könnte.

    Zur Verwaltung komplexer Daten wurde eine Python-basierte Software namens
    \linebreak \texttt{plotastic} entwickelt, welche die Analyse und Visualisierung
    multidimensionaler Datensätze optimiert. \texttt{plotastic} basiert auf der
    Idee, dass statistische Analysen basierend darauf durchgeführt werden, wie
    die Daten visualisiert werden. Dieser Ansatz vereinfacht nicht nur die
    Datenanalyse, sondern automatisiert sie auch teilweise in einem
    standardisierten statistischen Protokoll. Die Arbeit wird zu einer
    Fallstudie, da sie die Anwendung von \texttt{plotastic} auf das \textit{in
        vitro}-Modell reflektiert und zeigt, wie die Software schnelle Anpassungen
    und Verfeinerungen in der Datenanalyse und -präsentation erleichtert. Eine
    solche Effizienz ist entscheidend für den transparenten Umgang mit
    Semi-Big-Data, die trotz ihrer Handhabbarkeit komplex genug ist, um die
    Analyse und Reproduzierbarkeit zu erschweren.

    Zusammengefasst veranschaulicht diese Dissertation die Synergie zwischen
    experimentellen Methoden und fortgeschrittenen Werkzeugen der Datenanalyse.
    Das \textit{in vitro}-Modell bietet eine robuste Plattform für die
    Untersuchung der Myelomdissemination, während \texttt{plotastic} den Bedarf
    an effizienter Datenanalyse adressiert. Zusammen bieten sie einen
    umfassenden Ansatz für die Bearbeitung komplexer Experimente, fördern sowohl
    die Krebsbiologie als auch die Forschungspraktiken und unterstützen die
    explorative und transparente Analyse herausfordernder Phänomene.


\end{abstract}