



% ======================================================================
% == Chapter 1
% ======================================================================
\unnsection{Chapter 1: Modelling Myeloma Dissemination \textit{in vitro}}
\vspace{-\baselineskip} % > Remove space made by empty lines

% ## Reset reference counters of figs, tabs, so each chapter starts at 1
\setcounter{figure}{0}
\setcounter{table}{0}

% Title: Keep it Together: Modelling Myeloma Dissemination in vitro with
% hMSC- Interacting Subpopulations of INA-6 Cells and their
% Aggregation/Detachment Dynamics 

% ======================================================================
% == Abstract
% ======================================================================
% \addcontentsline{toc}{subsection}{Abstract} % > Add to Table of Contents

% \renewcommand{\abstractname}{Abstract} % > Rename what is displayed
% \begin{abstract}
%     lorem ipsum
% \end{abstract}

% > First argument is the label
% > Second argument is the header name
% > Third argument is the abstract text
\customabstract{c1:abstract}{Abstract}{
    lorem ipsum
}


% ======================================================================
% == Introduction
% ======================================================================
% \unnsubsection{C1:Introduction}{Introduction}
\unnsubsection[c1:introduction]{Introduction}
\ %
lorem ipsum dolor sit amet


% ======================================================================
% == Methods
% ======================================================================
\unnsubsection[c1:methods]{Methods}
\ %
Methods are described in \refapdx{subapdx:materialsmethods}



% ======================================================================
% == Results
% ======================================================================
\unnsubsection[c1:results]{Results}
\ %
Lorem ipsum dolor sit amet, consetetur sadipscing elitr, sed diam nonumy
eirmod tempor invidunt ut labore et dolore magna aliquyam erat, sed diam

% ## Example:
% > \includeimage[scale][left bottom right top]{
% >     pathtopicture
% > }{caption}{label}


Lorem ipsum dolor sit amet, consetetur sadipscing elitr, sed diam nonumy
eirmod tempor invidunt ut labore et dolore magna aliquyam erat, sed diam

% ## Fig. 1
% > use \noindent, for some reason first image is indented otherwise
\noindent\includeimage[1][{\fmleft} 2.8in {\fmright} {\fmtop}]{
    FIGS/figures9_1.pdf
}{
    Lorem ipsum dolor sit amet, consetetur sadipscing elitr, sed diam nonumy
    eirmod tempor invidunt ut labore et dolore magna aliquyam erat, sed diam
    voluptua. At vero eos et accusam et justo duo dolores et ea rebum. Stet
}{fig:1}


Lorem ipsum dolor sit amet, consetetur sadipscing elitr, sed diam nonumy
eirmod tempor invidunt ut labore et dolore magna aliquyam erat, sed diam


% ## Fig. 2
\includeimage[1][{\fmleft} 7.4in {\fmright} {\fmtop}]{
    FIGS/figures9_2.pdf
}{
    Lorem ipsum dolor sit amet, consetetur sadipscing elitr, sed diam nonumy
    eirmod tempor invidunt ut labore et dolore magna aliquyam erat, sed diam
    voluptua. At vero eos et accusam et justo duo dolores et ea rebum. Stet
}{fig:2}


Lorem ipsum dolor sit amet, consetetur sadipscing elitr, sed diam nonumy
eirmod tempor invidunt ut labore et dolore magna aliquyam erat, sed diam


% ## Fig. 3
\includeimage[1][{\fmleft} 5.4in {\fmright} {\fmtop}]{
    FIGS/figures9_3.pdf
}{
    Lorem ipsum dolor sit amet, consetetur sadipscing elitr, sed diam nonumy
    eirmod tempor invidunt ut labore et dolore magna aliquyam erat, sed diam
    voluptua. At vero eos et accusam et justo duo dolores et ea rebum. Stet
}{fig:3}


Lorem ipsum dolor sit amet, consetetur sadipscing elitr, sed diam nonumy
eirmod tempor invidunt ut labore et dolore magna aliquyam erat, sed diam


% ## Fig. 4
\includeimage[1][{\fmleft} 4.3in {\fmright} {\fmtop}]{
    FIGS/figures9_4.pdf
}{
    Lorem ipsum dolor sit amet, consetetur sadipscing elitr, sed diam nonumy
    eirmod tempor invidunt ut labore et dolore magna aliquyam erat, sed diam
    voluptua. At vero eos et accusam et justo duo dolores et ea rebum. Stet
}{fig:4} % > Is referenced in supplemental


Lorem ipsum dolor sit amet, consetetur sadipscing elitr, sed diam nonumy
eirmod tempor invidunt ut labore et dolore magna aliquyam erat, sed diam


% ## Fig. 5
\includeimage[.97][{\fmleft} {\fmbottom} {\fmright} {\fmtop}]{
    FIGS/figures9_5.pdf
}{
    Lorem ipsum dolor sit amet, consetetur sadipscing elitr, sed diam nonumy
    eirmod tempor invidunt ut labore et dolore magna aliquyam erat, sed diam
    voluptua. At vero eos et accusam et justo duo dolores et ea rebum. Stet
}{fig:5}


Lorem ipsum dolor sit amet, consetetur sadipscing elitr, sed diam nonumy
eirmod tempor invidunt ut labore et dolore magna aliquyam erat, sed diam


% ## Fig. 6
\includeimage[1][{\fmleft} 3.3in {\fmright} {\fmtop}]{
    FIGS/figures9_6.pdf
}{
    Lorem ipsum dolor sit amet, consetetur sadipscing elitr, sed diam nonumy
    eirmod tempor invidunt ut labore et dolore magna aliquyam erat, sed diam
    voluptua. At vero eos et accusam et justo duo dolores et ea rebum. Stet
}{fig:6}


Lorem ipsum dolor sit amet, consetetur sadipscing elitr, sed diam nonumy
eirmod tempor invidunt ut labore et dolore magna aliquyam erat, sed diam


    % ## Tab. 1
    {
        \footnotesize
        \newcommand{\myheader}{
            \hline
            \textbf{Regulation during disease progression} & \textbf{Gene} & \textbf{Ensemble ID} & \textbf{Progression Free / Overall Survival} & \textbf{Better Prognosis with high/low expression} & \multicolumn{2}{p{3cm}|}{\textbf{Association of expression with survival}}                    \\
            \hhline{~~~~~--}
            &               &                      &                                              &                                                    & \textbf{[p-unc]}                                                      & \textbf{[p-adj]} \\
            \hline
        }

        \begin{longtable}{|>{\bfseries}p{3cm}|>{\bfseries}p{1.9cm}|p{3cm}|p{2cm}|p{2cm}|p{1.5cm}|p{1.5cm}|}
            \caption{%
            Adhesion and ECM genes (shown in \autoref{fig:6}A) were filtered by
            their association with patient survival (p-adj. < 0.01) and were
            categorized as continuously downregulated during disease
            progression. The complete list is presented in \refapdx{apdx:supplemental}{tab:S2}.
            Bone Marrow Plasma Cells (BMPC), Monoclonal Gammopathy of
            Undetermined Significance (MGUS), smoldering Multiple Myeloma (sMM),
            Multiple Myeloma (MM), and Multiple Myeloma Relapse (MMR). p-unc =
            unadjusted p-values; p-adj: p-values adjusted using the
            Benjamini-Hochberg method with 101 genes. }\label{tab:1}
            \\
            \myheader
            \endfirsthead

            \multicolumn{7}{c}%
            {Appendix \thesection~\tablename\ \thetable\ -- \textit{Continued from previous page}} \\
            \myheader
            \endhead

            \hline
            \multicolumn{7}{r}{\textit{Continued on next page}}                                    \\
            \endfoot

            \endlastfoot

            \hline

            \multirowcell{3}{3cm}{Not Downregulated (or overall low expression)}
             & CCNE2  & ENSG00000175305 & Overall    & low  & 5.34E-04 & 8.64E-03                  \\
            \hhline{~------}
             & MMP2   & ENSG00000087245 & Prog. Free & high & 2.29E-05 & 2.32E-03                  \\
            \hhline{~------}
             & OSMR   & ENSG00000145623 & Prog. Free & high & 5.67E-04 & 7.15E-03                  \\
            \hhline{~------}
            \hline

            \multirowcell{17}{2.7cm}{Continuously Downregulated (BMPC > MGUS > sMM > MM > MMR)}
             & AXL    & ENSG00000167601 & Overall    & high & 3.64E-05 & 1.84E-03                  \\
            \hhline{~------}
             & COL1A1 & ENSG00000108821 & Prog. Free & high & 3.03E-04 & 4.37E-03                  \\
            \hhline{~~~----}
             &        &                 & Overall    & high & 5.93E-04 & 8.64E-03                  \\
            \hhline{~------}
             & CXCL12 & ENSG00000107562 & Prog. Free & high & 1.16E-04 & 2.93E-03                  \\
            \hhline{~~~----}
             &        &                 & Overall    & high & 6.48E-04 & 8.64E-03                  \\
            \hhline{~------}
             & CYP1B1 & ENSG00000138061 & Overall    & high & 6.84E-04 & 8.64E-03                  \\
            \hhline{~------}
             & DCN    & ENSG00000011465 & Overall    & high & 2.47E-04 & 8.33E-03                  \\
            \hhline{~------}
             & LRP1   & ENSG00000123384 & Overall    & high & 4.34E-04 & 8.64E-03                  \\
            \hhline{~------}
             & LTBP2  & ENSG00000119681 & Prog. Free & high & 9.03E-05 & 2.93E-03                  \\
            \hhline{~------}
             & CYP1B1 & ENSG00000138061 & Overall    & high & 6.84E-04 & 8.64E-03                  \\
            \hhline{~------}
             & DCN    & ENSG00000011465 & Overall    & high & 2.47E-04 & 8.33E-03                  \\
            \hhline{~------}
             & LRP1   & ENSG00000123384 & Overall    & high & 4.34E-04 & 8.64E-03                  \\
            \hhline{~------}
             & LTBP2  & ENSG00000119681 & Prog. Free & high & 9.03E-05 & 2.93E-03                  \\
            \hhline{~------}
             & MFAP5  & ENSG00000197614 & Prog. Free & high & 2.43E-04 & 4.09E-03                  \\
            \hhline{~------}
             & MMP14  & ENSG00000157227 & Prog. Free & high & 6.93E-05 & 2.93E-03                  \\
            \hhline{~------}
             & MYL9   & ENSG00000101335 & Prog. Free & high & 1.46E-04 & 2.95E-03                  \\
            \hhline{~~~----}
             &        &                 & Overall    & high & 1.56E-05 & 1.57E-03                  \\


            \hline
        \end{longtable}
    }




% ## Fig. 7
\includeimage[1][{\fmleft} 6.8in {\fmright} {\fmtop}]{ FIGS/figures9_7.pdf }{
    Proposed model of “Detached Daughter Driven Dissemination” (DDDD) in
    aggregating multiple myeloma. \textbf{Heterotypic Interaction:} Malignant
    plasma cells colonize the bone marrow microenvironment by adhering to an MSC
    (or osteoblast, ECM, etc.) to maximize growth and survival through paracrine
    and adhesion mediated signaling, even if contact may trigger initial
    apoptosis. Gene expression will focus on establishing a strong anchor within
    the bone marrow, but also on attracting other myeloma cells (via secretion
    of ECM factors and CXCL12/CXCL8, respectively). \textbf{Cell Division:} Cell
    fission can generate one daughter cell that no longer adheres to the MSC
    (nMA). \textbf{Homotypic Interaction:} If myeloma cells have the capacity to
    grow as aggregates, the daughter cell stays attached to their MSC-adhering
    mother cell (MA). \textbf{Re-Adhesion:} The daughter cell “rolls around” the
    mother cell until it re-adheres to the MSC. Our model estimates the rolling
    duration to be 1-10 h long. \textbf{Proliferation \& Saturation:} We
    estimate that a single myeloma cell covers one MSC completely after roughly
    four population doublings. When heterotypic adhesion is saturated,
    subsequent daughter cells benefit from a homotypic interaction, since they
    stay close to growth-factor secreting MSCs and focus gene expression on
    proliferation (e.g. driven by E2F) and not adhesion (driven by NF-κB).
    \textbf{Critical Size:} Homotypic interaction is weaker than heterotypic
    interaction, and each cell fission destabilizes the aggregate. Hence,
    detachment of myeloma cells may depend mostly on aggregate size.
    \textbf{Dissemination:} After myeloma cells have detached, they gained a
    viability advantage through IL-6-independence (with unknown duration), which
    enhances their survival outside of the bone marrow and allows them to spread
    throughout the body. }{fig:7}





lorem ipsum

V-Well is described in \refapdx{apdx:supplemental}{fig:S1}

V-Well is described in \refapdx{apdx:supplemental}{fig:S1}, \ref{fig:S2}

% Image cytometry is described in \autoref{subapdx:sup_figtabs} \autoref{fig:S2}

Full analysis is shown in \refapdx{subapdx:example_analysis}

% ======================================================================
% == Discussion
% ======================================================================
\unnsubsection[c1:discussion]{Discussion}
\
lorem ipsum

% == Paper 1 ===========================================================
% > You could import .pdf here, but chapter based theses should apply the 
% > manuscripts into the formatting of the thesis
% \addpdf{Research Article: Cancer Research Communications}{PUBLICATIONS/AACR.pdf}





