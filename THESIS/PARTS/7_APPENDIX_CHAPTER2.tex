
% ======================================================================
% == Appendices for Chapter 2
% ======================================================================




% == Class Diagramm of plotastic =======================================
\subsection{Class Diagram}
\label{subapdx:classdiagr}

% ## Save re-used text in a macro
\def\umlconvention{Arrow shapes follow the UML (unified modeling
    language): A hollow triangle indicates inheritance (\textit{``is~a''}) and a
    filled diamond indicates composition (\textit{``has~a''}). }

% ## Upper part
\includeimage[0.422]{
    APPENDIX_CHAPTER2/classdiagr_dataframe.png
}{
    \textbf{Class diagram of \texttt{plotastic} (upper part):} The
    architecture of \texttt{plotastic} begins with classes that are related to
    handling a \texttt{pandas.DataFrame} object which stores the data, and defining
    dimensions to group the data (y, x, hue, col, row). This diagram ends with the
    classes \texttt{SubPlot} and \texttt{StatTest} and is continued on the next
    page. \umlconvention
}{fig:S_classdiagr}
\newpage
\setcounter{figure}{0} % > Reset figure counter
% ## Lower Part
\def\mycap{\textbf{(continued)} The architecture of \texttt{plotastic}
    continues after the class \texttt{DataIntegrity} with classes for plotting
    (\texttt{SubPlot}) and statistical testing (\texttt{StatTest}) and end with
    the class \texttt{DataAnalysis}, which serves as the main user interface.
    \umlconvention }
\begin{figure}[H]
    \centering
    \includegraphics*[scale=.20]{APPENDIX_CHAPTER2/classdiagr_plot+stats.png}
    \caption{\mycap}
\end{figure}



% == Readme from plotastic (PyPi) ======================================
% > Make an empty page with the section title
\def\mytitle{Readme}
\subsection{\mytitle}
\label{subapdx:readme}
\ %
The following pages are the \texttt{README.md} of \texttt{plotastic} found in
the Python Package Index (PyPi) (\url{pypi.org/project/plotastic}), and on
GitHub (\url{github.com/markur4/plotastic}).

\addpdf*[.93]{\mytitle}{APPENDIX_CHAPTER2/README_pypi.pdf}
\newpage




% == Example Analyses plotastic =======================================
% > Make an empty page with the section title
\def\mytitle{Example Analysis ``qpcr''}
% \markboth{Appendix}{: \mytitle}
\subsection{\mytitle}
\label{subapdx:example_analysis}
\ %
The following pages are a jupyter notebook from an example analysis using
\texttt{plotastic}. This notebook and further analyses examples are found on
GitHub (\url{github.com/markur4/plotastic}). The qPCR dataset was derived from
the experiments described in Chapter 1 \autoref{fig:4} and was changed into a public test
dataset by exchanging the original gene names with random ones while preserving
gene classes and quantitative fold changes.

\addpdf*[.97]{\mytitle}{APPENDIX_CHAPTER2/qpcr.pdf}
\newpage
