
% ======================================================================
% == Appendices for Chapter 1: Materials and Methods
% ======================================================================


\subsubsection*{Isolation and Culturing of Primary Human Bone Marrow-Derived Mesenchymal Stromal Cells}
Primary human Mesenchymal Stromal Cells (MSCs) were obtained from the femoral
head of patients (\apdxref{apdx:supplemental}{tab:S1}) undergoing elective hip
arthroplasty. Material was collected with written informed consent of all
patients and the procedure was approved by the local Ethics Committee of the
University of Würzburg (186/18). In brief, bone marrow was washed with
MSC-medium (Dulbecco’s modified Eagle’s medium (DMEM/F12, Thermo Fisher
Scientific, Darmstadt, Germany) supplemented with \SI{10}{\percent} Fetal Calf
Serum (FCS, Bio\&Sell GmbH, Feucht, Germany,
\citet{fernandez-rebolloHumanPlateletLysate2017}), \SI{100}{U/ml} penicillin,
\SI{0.1}{mg/ml} streptomycin (Thermo Fisher Scientific), \SI{50}{\micro\gram/ml}
ascorbate and \SI{100}{nmol/l} sodium selenite (both Sigma-Aldrich GmbH, Munich,
Germany)) and centrifuged at \SI{250}{g} for \SI{5}{\minute}. The pellet was
washed four times with MSC-medium and resulting supernatants containing released
cells were collected. Cells were pelleted and cultured at a density of
\SI{1e9}{cells} per \SI{175}{cm\squared} culture flask. After two days
non-attached cells were washed away and adherent ones were cultivated in
MSC-medium until confluence. Then, they were either frozen in liquid nitrogen or
directly utilized for experiments. hMSC cultures were sustained for a maximum of
two passages. All cells were cultured at \SI{37}{\degreeCelsius} and at
\SI{5}{\percent} CO\textsubscript{2}.


\subsubsection*{Culturing of Myeloma Cell Lines}
The plasmacytoma cell line \INA [\textit{RRID:CVCL\_5209}; DSMZ, Braunschweig,
    Germany, authenticated by DSMZ in 2014
    \cite{burgerGp130RasMediated2001,gramatzkiTwoNewInterleukin61994}] was
cultivated in RPMI1640 medium (Thermo Fisher Scientific) supplemented with
\SI{20}{\percent} (v/v) FCS, \SI{100}{\micro\gram/ml} gentamicin, \SI{2}{mmol/l}
L-glutamine (both Thermo Fisher Scientific), \SI{1}{mmol/l} sodium pyruvate,
\SI{100}{nmol/l} sodium selenite (both Sigma Aldrich GmbH) and \SI{2}{ng/ml}
recombinant human interleukin-6 (IL-6; Miltenyi Biotec, Bergisch Gladbach,
Germany). \INA were passaged three times per week by diluting them to
\SI{1e5}{cells/ml}, \SI{2e5}{cells/ml}, or \SI{4e5}{cells/ml} for 3, 2, and 1
days of culturing, respectively. MM.1S [\textit{RRID:CVCL\_8792}]
\cite{greensteinCharacterizationMMHuman2003}, and U266 cells
    [\textit{CVCL\_0566}] \cite{nilssonEstablishedImmunoglobulinProducing1970} were
propagated and cultivated in RPMI1640 medium comprising \SI{10}{\percent} (v/v)
FCS, \SI{100}{U/ml} penicillin, \SI{100}{\micro\gram/ml} streptomycin,
\SI{2}{mmol/l} L-glutamine, and \SI{1}{mmol/l} sodium pyruvate. All cells were
cultured at \SI{37}{\degreeCelsius} and at \SI{5}{\percent} CO\textsubscript{2}.


\subsubsection*{Co-Culturing of Primary hMSCs and \INA and MSC-Conditioning of Medium}
For each co-culture, hMSCs were seeded out \SI{24}{\hour} prior to \INA
addition to generate MSC-conditioned medium (CM). CM from different donors was
collected separately and used immediately when adding \INA. To ensure that CM
was free of hMSCs, it was strained (\SI{40}{\micro\meter}) and centrifuged for
\SI{15}{\minute} at \SI{250}{g}. \INA cells were washed with PBS
(\SI{5}{\minute}, \SI{1200}{rpm}), resuspended in MSC-medium and added to hMSCs
such that co-culture comprised \SI{33}{\percent} (v/v) of CM gathered directly
from the respective hMSC-donor. Co-cultures did not contain IL-6
\cite{chatterjeePresenceBoneMarrow2002}.


\subsubsection*{Collagen I Coating}
Collagen I solution (isolated from rat tail, Corning, NY, USA) was diluted 1:2
(\SI{75}{ng/mL}) in acetic acid (\SI{0.02}{N}), applied to 96-well plates
(\SI{30}{\micro\liter} in each well) and incubated for \SI{2}{\hour} at room
temperature. Acetic acid was removed and wells were washed once with
\SI{100}{\micro\liter} of PBS. Coated plates were stored dry at
\SI{4}{\degreeCelsius}.


\subsubsection*{Fluorescent Staining of Cells}
For each live staining, cells were strained (\SI{70}{\micro\meter}) to remove
clumps and washed (\SI{5}{\minute}, \SI{250}{g}) once with the respective media
(without FCS) and then resuspended in staining reagents. For
\textit{CellTracker\texttrademark Green CMFDA Dye} and
\textit{CellTracker\texttrademark Deep Red Dye} (Thermo Fisher Scientific)
staining, \SI{1}{mL} staining solution for a maximum of \SI{1e6}{cells} was
prepared. Staining was done at room temperature (RT) for \SI{15}{\minute} using
\SI{5}{\micro M} CMFDA (5-Chlormethyl-fluoresceindiacetat) and \SI{5}{\minute}
of \SIrange{1}{2}{\micro M} DeepRed. To reduce background, stained cells were
pelleted, resuspended in cell medium (containing FCS), incubated for
\SI{30}{\minute} (\SI{37}{\degreeCelsius}, \SI{5}{\percent}
CO\textsubscript{2}), washed in cell medium, resuspended in
\SIrange{100}{1000}{\micro\liter} and counted.

For PKH26 staining (Sigma Aldrich GmbH), a maximum of \SI{1e4}{cells} was
resuspended in \SI{500}{\micro\liter} diluent C before swiftly adding
\SI{500}{\micro\liter} of staining solution (\SI{1}{\micro\liter} diluted in
\SI{500}{\micro\liter} diluent C) and incubating cells for \SI{5}{\minute} at
RT. The staining reaction was stopped by adding \SI{1}{mL} of FCS-containing
medium and adding \SI{3}{mL} of FCS-free medium. Cells were washed with
\SI{10}{mL} of FCS-containing medium, resuspended in
\SIrange{100}{1000}{\micro\liter} cell medium, and counted.

For Calcein-AM (Calcein-O,O′-diacetat-tetrakis-(acetoxymethyl)-ester) (Thermo
Fisher Scientific) staining, end concentrations of \SI{0.5}{\micro M} were used.
\SI{12.5}{\micro\liter} of diluted stock solution (\SI{2.5}{\micro M}) was
carefully added to \SI{50}{\micro\liter} of the co-culture and incubated for
\SI{10}{\minute} at \SI{37}{\degreeCelsius}.

For Hoechst 33342 staining, cells were washed once with PBS, resuspended in a
maximum of \SI{500}{\micro\liter} of PBS, and fixed with \SI{5}{mL} of ice-cold
ethanol (\SI{70}{\percent} v/v) by vigorously pipetting up and down to
dissociate aggregates. Cells were washed once with PBS and stained with
\SI{2.5}{\micro\gram/mL} Hoechst 33342 (Thermo Fisher Scientific) diluted in PBS
for \SI{1}{hour} at \SI{37}{\degreeCelsius}.



% \subsubsection*{Automated Fluorescence Microscopy}
% To remove clumps for microscopic applications, we cultured cells in
% \SI{40}{\micro\meter} strained medium containing FCS. To reduce background
% fluorescence and phototoxicity, we used phenol-red free versions of the
% respective medium, if available. All microscopy equipment was acquired from
% ZEISS. The microscope was an Axio Observer 7 with confocal ApoTome.2 equipped
% with a motorized reflector revolver and motorized scanning table
% (\SIprod{130}{100}{mm}). The microscope was mounted on an Antivibrations-Set (Axio
% Observer (D)) with two antivibration carrier plates, each equipped with two
% vibration dampening feet. The light source was a microLED 2 for transmission
% light and (for fluorescence) Colibri 7 (R[G/Y]B-UV) for five channels of
% incident light (\SIlist{385;475;555;590;630}{nm}). For excitation (EX) and
% emission (EM) light filtering and beam splitting (BS) we used the following
% reflectors: 96 HE BFP shift free (E) (EX: \SI{390/40}{nm}, BS: \SI{420}{nm}, EM:
% \SI{450/40}{nm}), 43 HE Cy 3 shift free (E) (EX: \SI{550/25}{nm}, BS:
% \SI{570}{nm}, EM: \SI{605/70}{nm}), 38 HE eGFP shift free (E) (EX:
% \SI{470/40}{nm}, BS: \SI{495}{nm}, EM: \SI{525/50}{nm}) and 90 HE LED (E) (EX:
% \SIlist{385;475;555;630}{nm}, BS: \SIlist{405;493;575;653}{nm}, EM:
% \SIlist{425/30;514/30;592/30;709/100}{nm}). We used the black and white camera
% Axiocam 506 mono (D) and if not stated otherwise, \SI{2x2}{binning} was used for
% fluorescence imaging. For mosaic acquisitions (“tiles”) we used a tiling overlap
% of \SIrange{8}{10}{\percent} and image tiles were not stitched. Images were
% magnified \SIlist{5x;10x} (Fluar 5x/0.25 M27 and EC Plan-Neofluar 10x/0.3 Ph1
% M27).


\subsubsection*{Automated Fluorescence Microscopy}
To remove clumps for microscopic applications, we cultured cells in
\SI{40}{\micro\meter} strained medium containing FCS. To reduce background
fluorescence and phototoxicity, we used phenol-red free versions of the
respective medium, if available.

All microscopy equipment was acquired from
ZEISS. The microscope was an \textit{Axio Observer 7} with confocal
\textit{ApoTome.2} equipped with a motorized reflector revolver and motorized
scanning table (\SIprod{130}{100}{mm}). The microscope was mounted on an
Antivibrations-Set (Axio Observer (D)) with two antivibration carrier plates,
each equipped with two vibration dampening feet. The light source was a
\textit{microLED~2} for transmission light and (for fluorescence)
\textit{Colibri 7} (R[G/Y]B-UV) for five channels of incident light
(\SIlist{385;475;555;590;630}{nm}).
%
For excitation (EX) and
emission (EM) light filtering and beam splitting (BS) we used the following
reflectors:
\textit{96 HE BFP shift free} (E) (%
EX:~\SIslash{390}{40}{nm},
BS:~\SI{420}{nm},
EM:~\SIslash{450}{40}{nm}),
%
\textit{43 HE Cy 3 shift free} (E) (%
EX:~\SIslash{550}{25}{nm},
BS:~\SI{570}{nm},
EM:~\SIslash{605}{70}{nm}),
%
\textit{38 HE eGFP shift free} (E) (%
EX:~\SIslash{470}{40}{nm},
BS:~\SI{495}{nm},
EM:~\SIslash{525}{50}{nm}) and
%
\textit{90 HE LED} (E) (%
EX:~385 + 475 + 555 + 630 \si{nm},
BS:~405 + 493 + 575 + 653 \si{nm},
EM:~425/30 + 514/30 + 592/30 + 709/100 \si{nm}%
).
We used the black and white camera \textit{Axiocam 506 mono} (D) and if not stated
otherwise, \SIprod{2}{2}{binning} was used for fluorescence imaging. For mosaic
acquisitions (“tiles”) we used a tiling overlap of \SIrange{8}{10}{\percent} and
image tiles were not stitched. Images were magnified 5x and 10x (\textit{Fluar
    5x/0.25 M27} and \textit{EC Plan-Neofluar 10x/0.3 Ph1 M27}).


\subsubsection*{Cell Viability and Apoptosis Assay}
To examine cell viability and apoptosis, cells were seeded in a 96-well plate
(\SI{1e4}{cells} per well) to be measured inside culture wells after respective
incubation time immediately. ATP-amount and Caspase 3/7 activity were used as a
proxy for viability and apoptosis rates, respectively. They were assessed using
the \textit{CellTiter-Glo Luminescent Cell Viability Assay} and the
\textit{Caspase-Glo 3/7 Assay}, respectively (Promega GmbH, Mannheim, Germany),
according to the manufacturer’s instructions. Luminescence was measured with an
Orion II Luminometer (Berthold Detection Systems, Pforzheim, Germany).


\subsubsection*{Microscopic Characterization of hMSC Saturation}
For saturating hMSC with \INA, hMSCs were stained with \textit{CellTracker
    Green}, plated out on 384-well plates (Greiner Bio-One GmbH, Frickenhausen,
Germany) at \SI{5e3}{hMSC/cm^2} and cultured for \SI{24}{\hour}. \INA cells
were stained with \textit{CellTracker DeepRed}, resuspended in MSC-medium, added
to adhering hMSCs in different amounts (\SI{5e3}{INA6/cm^2},
\SI{1e3}{INA6/cm^2}, \SI{2e3}{INA6/cm^2}) and co-cultured for \SI{24}{\hour} and
\SI{48}{\hour}. The complete co-culture was scanned and the number of \INA
cells adhering to one hMSC was counted manually for 100 MSCs for each technical
replicate. Fluorescent images were digitally re-stained (\INA green, hMSC
inverse black).


\subsubsection*{Analysis of \INA Survival and Aggregation Depending on hMSC Confluence}
To describe aggregate growth and survival of \INA depending on hMSC density,
unstained hMSCs were seeded out into 96-well plates (white, clear bottom,
Greiner) at different densities (see Seeding Table). To ensure nutrient supply,
we used lower cell densities for longer co-culturing durations while maintaining
constant ratios of \INA to adhesion surface provided by hMSCs. Those plates
that were to be assessed after \SI{72}{\hour} of co-culturing received an
additional \SI{100}{\micro\liter} of fresh MSC-medium after \SI{24}{\hour} of
co-culturing (total volume of \SI{300}{\micro\liter}), and after \SI{48}{\hour}
of co-culturing, \SI{100}{\micro\liter} was gently removed from the co-culture
and carefully replaced with fresh MSC-medium without disturbing the co-culture
on the bottom.

To describe aggregate growth, complete wells were scanned using
10x magnification, phase contrast, \SIprod{2}{2}{binning}, and autofocus
focusing on each tile both before and after harvesting. Afterwards, \INA cells
were harvested for measuring viability and apoptosis.


\begin{table}[h]
    \footnotesize
    \centering
    \caption*{\textbf{Seeding Table:} Seeding densities for describing growth and survival of \INA depending on hMSC density. Co-cult. dur.: Co culturing duration; MSC-adh. surface: adhesion surface provided by hMSCs; vol.: volume.}
    \label{tab:co_culture_conditions}
    \begin{tabular}{|p{2cm}|p{2cm}|p{2cm}|p{3.5cm}|p{2cm}|p{3cm}|}
        \hline
        \textbf{Co-cult. dur. [h]} & \textbf{hMSC density [1000 hMSC/cm\textsuperscript{2}]} & \textbf{\INA density [1000 INA6/cm\textsuperscript{2}]} & \textbf{Ratios INA : MSC (adh. surface)} & \textbf{Seeding vol. [\textmu L]} & \textbf{End vol. [\textmu L]}                                        \\
        \hline
        24                         & 2, 10, 40                                               & 10                                                      & 1:0.2, 1:1, 1:confluent                  & 200                               & 200                                                                  \\
        \hline
        48                         & 1, 5, 40                                                & 5                                                       & 1:0.2, 1:1, 1:confluent                  & 200                               & 200                                                                  \\
        \hline
        72                         & 1, 5, 40                                                & 5                                                       & 1:0.2, 1:1, 1:confluent                  & 200                               & 300 \newline [after 24 h: +100], \newline [after 48 h: exchange 100] \\
        \hline
    \end{tabular}
\end{table}


For luminescent assessment of cell survival, \INA cells were harvested by
removing co-culture medium, adding \SI{150}{\micro\liter} of MSC-medium, and
then stirred by strongly pipetting up and down twice while aiming the pipette
tip at the upper corner, lower left, and lower right of the well bottom
(‘Mercedes star’). Washing and stirring was repeated once before washing wells
again with \SI{150}{\micro\liter} MSC-medium. Harvested \INA cells were
strained (\SI{40}{\micro\meter} filter), pelleted, and resuspended in
\SI{200}{\micro\liter} MSC-medium. Cells were counted using Neubauer chambers,
re-distributed into 96-well plates (white, clear bottom) with \SI{1e5}{\INA cells} per well, and then subjected to viability and apoptosis assays.

To minimize the loss of sensitive apoptotic cells, another approach was used to
measure viability and apoptosis without harvesting \INA cells. hMSCs and \INA
were seeded out individually in parallel to the co-cultures. Prior to measuring viability and
apoptosis, culture volume was adjusted to \SI{150}{\micro\liter} by removing
\SI{50}{\micro\liter} or \SI{150}{\micro\liter} for the timepoints
\SI{48}{\hour} or \SI{72}{\hour}, respectively (carefully not to stir up culture
on bottom). \SI{100}{\micro\liter} of luminescent reagents were then added
directly to \SI{150}{\micro\liter} of co-culture. The fold change of viability
or apoptosis that is due to MSC interaction (\(FC_{\text{MSC interaction}}\))
was then calculated using the following formula, with \(L\) being the mean of
four technical replicates measured in relative luminescent units per seconds
    [RLU/s], and \(L_{\text{Co Culture}}\), \(L_{\text{MSC}}\), \(L_{\text{INA6}}\) the
luminescence measured in the co-culture, hMSCs alone, and \INA alone,
respectively.

\begin{center}
    $FC_{\text{MSC Interaction}}=\frac{L_{\text{Co Culture}}}{L_{\text{MSC}}  + L_{\text{\INA}}}$
\end{center}



\subsubsection*{Time-Lapse Characterization of \INA Aggregation, Detachment and Division}
In order to record the aggregation and detachment of \INA in contact with
hMSCs, hMSCs (\SI{5e3}{cells/cm^2}) were fluorescently stained with PKH26 and
plated onto 8-well \(\mu\)-Slides (ibidi, Gräfelfing, Germany). hMSCs were
incubated for \SI{24}{\hour} before being placed into an ibidi Stage Top
Incubation System and were equilibrated to the incubation system for a minimum
of \SI{3}{\hour} (\SI{80}{\percent} humidity and \SI{5}{\percent} CO2). \INA
cells (\SI{2e4}{cells/cm^2}) were washed and resuspended in \SI{33}{\percent}
(v/v) MSC-conditioned medium before adding them directly before acquisition
start in a small volume (\SI{10}{\micro\liter}). Brightfield and fluorescence
images of \SI{13}{mm^2} of co-culture were acquired every \SI{15}{minutes} for
\SI{63}{\hour}. Movement speed of the motorized table was adjusted to the lowest
setting that allows acquisition of the complete region within \SI{15}{minutes}.

Respective events of interest were analyzed manually and categorized into
defined event parameters. Events were binned across the time axis using these
boundaries: [0.0, 12.85, 25.7, 38.55, 51.4, 64.25]. We collected a minimum of
events per recording and analysis so that each time bin contained at least 5
values, except when analyzing detachment events, since these did not appear
before \SI{20}{\hour} of incubation for some replicates. For each recording and
event parameter, the event count was normalized by dividing by the total number
of events per time bin.

We determined the frequency and the cause of aggregation by looking for two
interacting \INA cells and went backward in time to see if they were two
daughter cells or if two independent \INA cells had collided. We determined the
frequency of aggregates with detaching cells by tracing their growth across the
complete time-lapse and looking for detachment events. We picked random 100
aggregates by including aggregates from both the border and center of the well.

We characterized detachment events by noting multiple parameters manually: Time
point of detachment, aggregate size (at the time of detachment), the last
interaction partner, and the number of detaching \INA cells.

For characterizing cell division events, we recorded a new set of time-lapse
videos using unstained hMSCs that were grown to confluence for \SI{24}{\hour}
(\SI{4e4}{hMSCs/cm^2}) to provide for unlimited adhesion surface. We categorized
daughter cells in terms of their mobility (mobility being the speed of putative
movements or “rolling”). The mobility criteria were met if one \INA daughter
cell moved farther than half a cell radius within one frame (\SI{15}{\minute})
relative to the MSC-adherent \INA cell which was required to stand still
in-between respective frames. We measured the “rolling” duration by subtracting
the time point of the last perceived movement from the time point of division.
We excluded those division events from the measurement of rolling duration if
\INA cells underwent apoptosis shortly after division.


\subsubsection*{Cell Cycle Synchronization at M-Phase}
\INA cells were arrested at mitosis by double thymidine (\SI{2}{mM}) treatments
followed by \SI{5}{\hour} of nocodazole (\SI{500}{ng/mL}) incubation. In detail:
\SI{3e5}{cells/mL} \INA in \SI{4}{mL} were treated with \SI{2}{mM} thymidine
(Sigma Aldrich GmbH) for \SI{16.5}{\hour}. Cells were released by washing them
in \INA medium once and allowed to cycle for \SI{9}{\hour} before treating them
with \SI{2}{mM} thymidine for \SI{18}{\hour} a second time. Afterwards, cells
were released and allowed to cycle for \SI{2}{\hour} before treating them with
\SI{100}{ng/mL} nocodazole (Sigma Aldrich GmbH) for \SI{5}{\hour}. Arrested
\INA were released by washing them once and resuspending them in MSC-medium
with \SI{33}{\percent} MSC-conditioned medium. Cell cycle profile was checked
using image cytometry (\apdxref{apdx:supplemental}{fig:S2}).

\subsubsection*{V-Well Adhesion Assay}
This assay was modified from \cite{weetallHomogeneousFluorometricAssay2001}.
96 v-well plates were coated with collagen~I (rat tail, Corning). Collagen
coating ensures that confluent hMSCs withstand centrifugation even after hMSCs
in the well tip were removed. hMSCs (\SI{4e4}{cells/cm^2}) were seeded out and
grown to confluence for \SI{24}{\hour} in collagen-coated v-well plates. To
ensure that only \INA are pelleted in the v-well tip, hMSCs were removed from
the well-tip by touching the well-ground with a \SI{10}{\micro\liter} pipette
and roughly pipetting hMSCs away.

Arrested \INA (\SI{1e4}{cells/cm^2}) were released by washing them once in PBS
and resuspending them in \SI{33}{\percent} (v/v) MSC-conditioned medium before
adding them on top of confluent hMSCs. \INA adhered for 1, 2, 3, and
\SI{24}{\hour} before the complete co-culture was stained with \SI{0.5}{\micro
    M} Calcein-AM (\SI{10}{\minute} at \SI{37}{\degreeCelsius}). Non-adherent \INA
were pelleted by centrifugation using a Hettich 1460 rotor (\(r =
\SI{124}{mm}\)) at \SI{2000}{rpm} (\SI{555}{g}) for \SI{10}{\minute}.

The well tip was imaged by fluorescence microscopy with 5x magnification,
\textit{96 HE} emission filter, autofocus configured for maximum signal intensity,
\SIprod{2}{2}{binning} and \SI{14}{bit} grayscale depth. Pellet brightness was
analyzed in ZEN 2.6 (Zeiss) by summing up pixel brightnesses across the complete
pellet image. Background brightness was acquired from a cell culture with only
hMSCs. Reference brightness was acquired from a cell culture with only \INA,
defining \SI{100}{\percent} pellet brightness without adhesion. Background
intensity was subtracted before normalizing by reference. Outliers were removed
from technical replicates (\(n=4\)) if their z-score was larger than
1.5 $\sigma$ technical variation.

After measuring pellet brightnesses, the cell pellet was removed by pipetting
\SI{10}{\micro\liter} from the well tip. Pellets of the same technical
replicates were pooled, washed in PBS, resuspended in \SI{200}{\micro\liter}
PBS, added to \SI{1.8}{mL} ice-cold \SI{70}{\percent} ethanol, and stored at
\SI{-20}{\degreeCelsius}. Remaining non-MSC-adhering \INA cells were removed by
replacing culture medium with \SI{100}{\micro\liter} of medium. MSC-adherent
\INA were manually detached by rapid pipetting and equally pelleted, analyzed,
and isolated.

\subsubsection*{Cell Cycle Profiling}
\INA cells were fixed in \SI{70}{\percent} ice-cold ethanol, washed,
resuspended in PBS, distributed in 96-well plates, and stained with Hoechst
33342 (\SI{2.5}{\micro\gram/mL} in PBS) for \SI{1}{\hour} at
\SI{37}{\degreeCelsius}. For image cytometric cell cycle profiling, plates were
scanned completely using automated fluorescence microscopy with
5x magnification, \textit{96 HE} emission filter, \SIprod{1}{1}{binning}, \SI{14}{bit}
depth, and an illumination time that fills \SI{70}{\percent} of the grayscale
range. The autofocus was configured to re-adjust every second tile. A
pre-trained convolutional neural network (“DeepFeatures 2 reduced”,  \textit{Intellesis},
Zeiss) was fine-tuned to segment scans into background, single nuclei, and
fragmented nuclei. Nuclei were filtered to exclude fragmented nuclei and those
nuclei with extreme size (within the range of
\SIrange{50}{500}{\micro\meter\squared}) and roundness (within the range of
\numrange{0.4}{1.0}). Cell cycle profiles were normalized by the mode of the
nucleus intensities within the G0/G1 peak. To retrieve frequencies of cells
cycling in G0/G1, S, and G2 phase, the brightness distribution of all single
nuclei was fitted to the sum of three Gaussian curves (“Skewed Gaussian Model”
for G0G1 and G2 phase, and “Rectangle Model” for S phase) using the python
package LMFIT \cite{newvilleLMFITNonLinearLeastSquare2014} (\apdxref{apdx:supplemental}{fig:S4}).
The Gaussian curves were used to calculate the cell frequencies for each cell
cycle phase by integration using the composite trapezoidal rule implemented by
numpy.trapz \cite{harrisArrayProgrammingNumPy2020}.

For validation of image cytometry, \SI{5}{mL} of \INA stock culture was removed
and ethanol fixed as described above. Flow cytometry analyses were performed
using an \textit{Attune Nxt Flow} Cytometer (Thermo Fisher Scientific). Data analyses
were performed using FlowJo V10 software (TreeStar, USA).


\subsubsection*{Protocol: Well Plate Sandwich Centrifugation (WPSC)}
96-well plates (flat bottom, clear) were coated with collagen I (rat tail,
Corning) to ensure that confluent hMSCs withstand centrifugation and
repeated washing. hMSCs (\SI{2e4}{cells/cm^2}) were seeded out and grown to
confluence for \SI{72}{hours} in collagen-coated 96-well plates. To remove
aggregates from the medium and prevent clogging of magnetic columns, any
FCS-containing fluid was strained with a \SI{40}{\micro\meter} cell
strainer.

\begin{itemize}
    \item \textbf{Collect MSC-Conditioned Medium and Add \INA:}
          \begin{enumerate}
              \item Collect hMSC-conditioned medium (CM) from the well plates and replace it with \SI{100}{\micro\liter} of fresh hMSC medium. Collect CM from different donors separately.
              \item Strain CM (\SI{40}{\micro\meter}) and centrifuge for \SI{15}{minutes} at \SI{250}{g} to ensure that CM does not contain hMSCs.
              \item Dilute CM by mixing 2 parts of CM with 1 part of MSC-medium (dilute 1.5 fold).
              \item Count \INA cells and retrieve enough cells to fill all 96 wells with \SI{2e4}{INA6/cm^2} (\SI{6.8e4}{cells} per well, covering approximately \SI{65}{\percent} of the well bottom).
              \item Centrifuge \INA (\SI{5}{minutes}, \SI{250}{g}) and resuspend them in a volume of diluted CM to reach a concentration of \SI{6.8e5}{INA6/mL}.
              \item Add \SI{100}{\micro\liter} \INA suspension to hMSCs (end volume: \SI{200}{\micro\liter}; end concentration: \SI{33}{\percent} (v/v) hMSC-conditioned medium).
              \item Incubate for \SI{24}{hours} at \SI{37}{\degreeCelsius} and \SI{5}{\percent} CO$_2$.
          \end{enumerate}

    \item \textbf{Prepare \CMina Reference:}
          \begin{enumerate}
              \setcounter{enumi}{7}
              \item Add \SI{100}{\micro\liter} of fresh MSC-medium into each well of an empty 96-well plate (not coated).
              \item Add \SI{100}{\micro\liter} of \INA suspension (\SI{6.8e5}{\INA/mL} in diluted CM).
              \item Incubate for \SI{24}{hours} at \SI{37}{\degreeCelsius} and \SI{5}{\percent} CO$_2$.
          \end{enumerate}

    \item \textbf{Collect \CMina and \nMAina:}
          \begin{enumerate}
              \setcounter{enumi}{10}
              \item Pre-warm well plate centrifuge to \SI{37}{\degreeCelsius}.
              \item Prepare a counter-weight by filling \SI{200}{\micro\liter} of water into all wells of an empty 96-well plate.
              \item Prepare well-plate sandwiches:
                    \begin{itemize}
                        \item[a.] Turn an empty 96-well plate (“catching plate”) upside down and place one on top of the co-culture-plate, the \CMina reference plate, and the counter-weight so that all well openings align.
                        \item[b.] Fix well plates using tape with reusable adhesive (e.g., \textit{Leukofix}).
                    \end{itemize}
              \item Turn both plates around. Medium will spill from the co-culture plate into the catching plate.
              \item Centrifuge plate for \SI{40}{seconds} at \SI{1000}{rpm} with the catching plate facing the ground.
              \item Remove the adhesive tape and the co-culture plate.
              \item Turn the co-culture plate around and add \SI{30}{\micro\liter} of washing medium (MSC-medium 0\% FCS, \SI{3}{mM} EDTA) gently by touching the wall of each well and pressing the pipette slowly.
                    \begin{itemize}
                        \item[a.] Work quickly to ensure that co-culture does not dry. We recommend using a multipette (Eppendorf).
                        \item[b.] Many \nMAina are removed by physical force applied by adding \SI{30}{\micro\liter} of medium and not just by centrifugation. Hence, it is critical to apply the same dispensing technique across all replicates. We recommend using a multipette (Eppendorf) that can apply \SI{30}{\micro\liter} with controllable pressure, since its push-button retains a long pushing path even for dispensing small volumes, unlike push-buttons from the usual \SI{100}{\micro\liter} pipettes that reduce the pushing-path for smaller volumes.
                        \item[c.] Centrifugation minimizes technical variability by replacing one step of manual pipetting. Also, it ensures that confluent MSCs remain unharmed. Manual pipetting on the other hand would require touching the well-bottom to remove all fluids which damages the adhesive hMSC layer.
                    \end{itemize}
              \item Turn the co-culture plate upside down, place it onto the catching plate and re-apply adhesive tape to fix the well plate sandwich.
              \item Repeat steps 14-18 two more times until the catching plate contains \SI{290}{\micro\liter} of medium in each well.
              \item Pool \CMina from the catching plate that was fixed to the reference plate.
              \item Pool \nMAina from the catching plate that was fixed to the co-culture plate.
              \item Collect remaining \INA by adding \SI{100}{\micro\liter} of PBS into each well of the catching plates, collect and pool with \CMina or \nMAina.
              \item Strain \CMina and \nMAina using \SI{40}{\micro\meter} cell strainer.
              \item Isolate \MAina by continuing with either accutase dissociation or rough pipetting.
          \end{enumerate}

    \item \textbf{Collect \MAina by Accutase Dissociation Followed by MAC Sorting:}
          \begin{enumerate}
              \setcounter{enumi}{24}
              \item Block \SI{2}{mL} tubes with sorting buffer (PBS, \SI{2}{mM} EDTA, \SI{1}{\percent} BSA) for \SI{1}{hour} at \SI{4}{\degreeCelsius}.
              \item Dilute accutase (Sigma Aldrich GmbH) (\SI{400}{to}{600} units/mL) 4-fold in cold PBS. Always keep accutase on ice, since accutase loses activity at room temperature.
              \item Add \SI{50}{\micro\liter} of cold accutase (directly after the last centrifugation step) and incubate co-culture plate for \SI{5}{minutes} at \SI{37}{\degreeCelsius}.
              \item Place a co-culture plate onto a shaker and shake for \SI{1}{minute} at \SI{300}{rpm}.
              \item Collect cell suspension from wells and stop the reaction by adding \SI{500}{\micro\liter} of FCS to pooled cell suspension.
              \item Evaluate presence of adherent \INA cells and the integrity of confluent hMSCs under the microscope.
              \item Repeat steps 27-30 until all \INA cells have dissociated or until confluent hMSCs start to tear.
              \item Strain cell suspension (\SI{30}{\micro\meter}). This yields \MAina.
              \item Pellet \MAina, \nMAina, and \CMina (\SI{1200}{rpm}, \SI{10}{minutes}).
              \item Resuspend \MAina in \SI{86}{\micro\liter} sorting buffer (PBS, \SI{2}{mM} EDTA, \SI{1}{percent} BSA).
              \item Resuspend \CMina and \nMAina in \SI{300}{\micro\liter} cold diluted accutase and incubate for \SI{3}{minutes} at \SI{37}{\degreeCelsius} to ensure equal treatment for all samples.
              \item Stop accutase by adding \SI{200}{\micro\liter} of FCS (\SI{100}{percent}).
              \item Pellet \CMina and \nMAina (\SI{1200}{rpm}, \SI{10}{minutes}) and resuspend in \SI{86}{\micro\liter} sorting buffer (PBS, \SI{2}{mM} EDTA, \SI{1}{percent} BSA).
              \item Transfer samples into \SI{2}{mL} tubes that were blocked with sorting buffer.
              \item Add \SI{10}{\micro\liter} of CD45 coated magnetic beads (Miltenyi Biotec B.V. \& Co. KG, Bergisch Gladbach).
              \item Place tubes into rotator and incubate for \SI{15}{minutes} at \SI{4}{\degreeCelsius}.
              \item Continue with MAC sorting according to the manual. Use an MS column and wash 3 times.
              \item Improve purity of eluted \MAina by straining eluate (\SI{30}{\micro\meter}) (wash strainer using \SI{1}{mL} of sorting buffer) and applying it onto an MS column a second time. Wash three times.
              \item Collect \SI{20}{\micro\liter} per eluate and apply it onto a \SI{96}{well} plate to evaluate purity.
                    \begin{itemize}
                        \item[a.] Incubate plate for \SI{24}{hours}.
                        \item[b.] Count the number of adherent cells (hMSCs) per \INA using phase contrast microscopy.
                        \item[c.] We reached a mean purity of \SI{3.2e-4} (±\SI{2.2e-4}) hMSCs per \MAina.
                        \item[d.] hMSC contamination did not have an impact on RNAseq, since those genes that are highly expressed in hMSCs (VCAM1, ALPL, FGF5, FGFR2), did not appear as differentially expressed in \MAina (data not shown). RNAseq detected 0.44~±0.16~{CPM-normalized} counts of VCAM1 transcripts in \MAina, however, it was excluded like all genes with less than 1 count in at least 2 of 5 replicates.
                    \end{itemize}
              \item Count cells using a Neubauer chamber.
              \item Pellet samples (\SI{250}{g} for \SI{5}{minutes}).
              \item Resuspend in respective medium or lysis buffer (e.g., RA1 for RNA extraction).
          \end{enumerate}

    \item \textbf{Collect \MAina by Rough Pipetting (No MAC Sorting):}
          \begin{enumerate}
              \setcounter{enumi}{46}
              \item After the last centrifugation step, add hMSC-medium to each well of the co-culture plate to reach a volume of \SI{150}{\micro\liter}.
                    \begin{itemize}
                        \item[a.] Since the yield of \MAina was large, we dissociated \MAina cells from hMSCs by vigorous pipetting (for further samples after RNAseq, see Supplementary Table 1). Since no enzymatic digestion is used, we reckoned that there would be no need for MAC sorting. Confluent hMSCs withstand this procedure and don’t dissociate as single cells, which can be removed by straining cells (\SI{30}{\micro\meter}). We reached similar purities as for MAC-sorting (data not shown).
                    \end{itemize}
              \item Using a multi-channel pipette (\SI{100}{\micro\liter}), gently raise \SI{90}{\micro\liter} into the tips.
              \item Lean pipette tip on the upper well-border and roughly pipette up and down once.
              \item Repeat step 48 at the lower right and lower left well border (Total of 3 pipetting steps “Mercedes Star”).
              \item Attach a catching plate onto the co-culture and centrifuge for \SI{40}{seconds} at \SI{500}{rpm} (\SI{28}{g}).
              \item Repeat steps 46-50 until a sufficient amount of \MAina is removed.
              \item Control purity of \MAina by placing out aliquot onto an empty \SI{96}{well} plate.
              \item Collect \MAina from catching plate.
              \item Remove hMSCs by straining cell suspension (\SI{30}{\micro\meter}).
              \item Count cells using a Neubauer chamber.
              \item Pellet \MAina (\SI{250}{g} for \SI{5}{minutes}).
              \item Resuspend in respective medium or lysis buffer.
          \end{enumerate}
\end{itemize}


\textbf{Centrifugal Force:} We used a Hettich 1460 rotor (\(r = \SI{124}{mm}\))
(Hettich GmbH \& Co. KG, Tuttlingen, Germany). For calculating the centrifugal
force that acts onto the co-culture within well plate sandwiches, we subtracted
the height of the catching plate (\SI{14.4}{mm}, Greiner 96-well plate) and the
depth of each well (\SI{10.9}{mm}). This yields a radius of \SI{98.7}{mm}, which
translates to the following centrifugal forces: \SI{500}{rpm}: \SI{28}{g};
\SI{1000}{rpm}: \SI{110}{g}; \SI{2000}{rpm}: \SI{441}{g}.

\textbf{Washing medium containing EDTA:} Washing medium containing EDTA: EDTA
removes calcium from integrins, which are required for adhesion. It is not
strong enough to dissociate \INA from hMSCs, but could help with removing \INA
from other \INA. For generating samples for RNAseq, we added \SI{3}{mM} of EDTA
to the washing medium. For further samples, we did not add EDTA to the washing
medium, since we found that it does not increase yield for all biological
replicates consistently (data not shown). We suspect that integrin-mediated
adhesion depends on hMSC donor or internal variance of \INA. We recommend using
\SI{3}{mM} of EDTA, however, this requires further optimizations like including
an incubation time at \SI{37}{\degreeCelsius} after the addition of washing
medium to account for biological variance. However, this could take long
incubation times of up to \SI{60}{minutes}  \cite{laiDifferentMethodsDetaching2022}.

\subsubsection*{Track Cell Number During WPSC}
To track the cell count during WPSC, \INA cells were stained with \textit{CellTracker
    green}, and both co-culturing and catching plates were scanned after each
centrifugation step. For each round of centrifugation, an empty catching plate
was used. A pre-trained convolutional neural network (\textit{Intellesis}, Zeiss) was
fine-tuned to segment the scans into background, cells, and cell borders. Single
cells were counted, and the cumulative sum for each catching plate was
calculated.

\subsubsection*{Sub-Culturing After WPSC of MSC-Interacting \INA Subpopulations}
After \CMina, \nMAina, and \MAina were isolated, they were counted with a
Neubauer chamber using all nine quadrants and diluted to \SI{1e5}{cells/mL} in
MSC-medium (10\% FCS, no IL-6 except for control). \SI{100}{\micro\liter} of
cell suspension was applied to 96-well plates, incubated for \SI{48}{hours} at
\SI{37}{\degreeCelsius} and \SI{5}{\percent} CO$_2$, and then subjected to
viability and apoptosis assays.

\subsubsection*{RNA Isolation}
Total RNA was isolated from \INA cells using the \textit{NucleoSpin RNA II
    Purification} Kit (Macherey-Nagel, Düren, Germany) according to the
manufacturer’s instructions.




\subsubsection*{RNAseq, Differential Expression and Functional Enrichment Analysis of \INA cells}
FASTQ files were merged to the respective sample. The quality of FASTQ files was
assessed with \texttt{FastQC} \cite{Andrews:2010tn} and a joint report was
created with \texttt{MultiQC} \cite{ewelsMultiQCSummarizeAnalysis2016}. FASTQ
files were aligned with \texttt{STAR} \cite{dobinSTARUltrafastUniversal2013} to
the GRCh38 reference genome build \cite{zerbinoEnsembl20182018}. Quality and
alignment statistics of final BAM files were assessed with \texttt{samtools
    stats} \cite{liSequenceAlignmentMap2009}, and a joint report with
\texttt{FastQC} reports by \texttt{MultiQC} was generated.

Raw read counts were generated with \texttt{HTSeq}
\cite{andersHTSeqPythonFramework2015} using the union method. \texttt{HTSeq}
runs internally in \texttt{STAR}. Differential gene expression analysis was
performed with \texttt{edgeR} \cite{Robinson:2010:Bioinformatics:19910308} in R
3.6.3 \cite{rcoreteamLanguageEnvironmentStatistical2018}, according to the \texttt{edgeR} manual.

Counts were merged and genes with zero counts in all samples were removed
(number of genes: 36380). The whole count table was annotated with R
Bioconductor \cite{gentlemanBioconductorBiocViews} human annotation data package
\texttt{org.Hs.eg.db} \cite{carlsonOrgHsEg2016}. A DGEList Element was created
with the raw counts, gene information, i.e., Ensembl GeneIDs, HUGO Symbol,
Genename, and ENTREZ GeneIDs and a sample grouping metadata table.

\begin{lstlisting}[language=R,style=defaultstyle]
y <- DGEList(counts=ct2[,-1:4], group=meta.data$group, genes=ct2[,1:4])
\end{lstlisting}

Counts were filtered to keep only those genes which have at least 1 read per
million in at least 2 samples (number of genes: 14136). Afterwards,
normalization factors were recalculated.

\begin{lstlisting}[language=R,style=defaultstyle]
keep <- rowSums(cpm(y)>1) >=2
y <- y[keep, , keep.lib.size=FALSE]
y1 <- calcNormFactors(y)
\end{lstlisting}

A design matrix was created with the grouping factor by treatment condition
(group=F1, F2, F3, which are abbreviations for \CMina, \nMAina, \MAina,
respectively)

\begin{lstlisting}[language=R,style=defaultstyle]
design = model.matrix(~0+group)
\end{lstlisting}

Dispersion was estimated, the resulting coefficient of biological variation
(BCV) is 0.135, i.e., BCV expression values vary up and down by 13.5\% between
samples.

\begin{lstlisting}[language=R,style=defaultstyle]
y1.1 <- estimateDisp(y1, design)
BCV <- sqrt(model.F$y1.1$common.dispersion)
\end{lstlisting}

A generalized linear model (\texttt{glmQLFit} function) was fitted.

\begin{lstlisting}[language=R,style=defaultstyle]
fit <- glmQLFit(y1.1, design)
\end{lstlisting}

and pairwise comparisons were made, e.g.

\begin{lstlisting}[language=R,style=defaultstyle]
F1vsF2 <- glmQLFTest(fit, contrast=makeContrasts(groupF1 - groupF2, levels=design))
DE.F1vsF2 <- topTags(F1vsF2, n=nrow(F1vsF2), p.value=0.05)
\end{lstlisting}

Afterwards, gene list of differentially expressed genes were used for functional
enrichment analysis with \texttt{Metascape}
\cite{zhouMetascapeProvidesBiologistoriented2019}.


\subsubsection*{RT-qPCR}
For cDNA synthesis, \SI{1}{\micro\gram} of total RNA was reverse transcribed
with Oligo(dT)15 primers and Random Primers (both Promega GmbH) and
\textit{Superscript IV} reverse transcriptase (Thermo Fisher
Scientific) according to the manufacturer’s instructions. For quantitative PCR,
the cDNA was diluted 1:10, and qPCR was performed in \SI{20}{\micro\liter} using
\SI{2}{\micro\liter} of cDNA, \SI{10}{\micro\liter} of \textit{GoTaq qPCR Master
    Mix} (Promega GmbH), and \SI{5}{pmol} of sequence-specific primers
obtained from biomers.net GmbH (Ulm, Germany) or Qiagen GmbH
(Hilden, Germany) (see Supplementary Table 3 for primer sequences and PCR
conditions). qPCR conditions were as follows: \SI{95}{\degreeCelsius} for
\SI{3}{minutes}; 40 cycles: \SI{95}{\degreeCelsius} for \SI{10}{seconds};
respective annealing temperature for \SI{10}{seconds}; \SI{72}{\degreeCelsius}
for \SI{10}{seconds}; followed by melting curve analysis for the specificity of
qPCR products using an \textit{TOptical Gradient 96 PCR Thermal Cycler}
(Analytik Jena AG, Jena, Germany). Samples that showed unspecific
byproducts were discarded. Ct values were measured in three technical replicates
(triplicates). Non-detects were discarded. One of three technical replicates was
treated as an outlier and excluded if its z-score crossed 1.5 $\sigma$
technical variation. We normalized expression by the housekeeping gene \textit{36B4}.
Efficiencies were determined in each reaction by linear regression of
log-transformed amplification curves
\cite{ramakersAssumptionfreeAnalysisQuantitative2003}. Differential expression
was calculated based on a modified $\Delta\Delta Ct$ formula that separated
exponents to apply individual efficiencies to each $Ct$ value:



\begin{minipage}{0.5\textwidth}
    \begin{align*}
        \text{Fold Change} & = \frac{E_{\text{tar}}^{\Delta Ct_{\text{tar} (co\ -\ treated)}}}{E_{\text{ref}}^{\Delta Ct_{\text{ref} (co\ -\ treated)}}}                                                                         \\
                           & = \frac{E_{\text{tar,co}}^{Ct_{\text{tar,co}}} \div E_{\text{tar,treated}}^{Ct_{\text{tar,treated}}}}{E_{\text{ref,co}}^{Ct_{\text{ref,co}}} \div E_{\text{ref,treated}}^{Ct_{\text{ref,treated}}}}
    \end{align*}
\end{minipage}
\begin{minipage}{0.5\textwidth}
    \footnotesize
    \begin{itemize}
        \item $E_{\text{tar,co}}$ = Efficiency of the target gene \\measured in the control sample
        \item $Ct_{\text{tar,co}}$ = Ct value of the target gene \\measured in the control sample
        \item $tar$ = Target Gene
        \item $ref$ = Reference Gene
        \item $treated$ = Treated sample
        \item $co$ = Control Sample
    \end{itemize}
\end{minipage}

Fold change expression was normalized by the median of \CMina (and not
sample-wise, as commonly used in $\Delta\Delta C_t$) since some genes were
not expressed without direct hMSC contact (e.g., \textit{MMP2}), and also in
order to display variation of \CMina next to \nMAina and \MAina.



\subsubsection*{Statistics}
For molecular analyses, each data point represents one biological replicate,
defined as the mean of all technical replicates of co-cultures seeded from the
same batch of hMSCs and/or \INA cells on the same day. For analyses of
time-lapse recordings, each data point represents the normalized event count
from one co-culture recording. Unique hMSCs were prioritized for each biological
replicate or recording (see \apdxref{apdx:supplemental}{tab:S1}). Bars and lines represent the
mean, and error bars represent the standard deviation of all hMSC donors or
recordings (= all biological replicates).

Metric, normally distributed, dependent data were analyzed using factorial
RM-ANOVA and paired Student’s t-tests. Results of RM-ANOVA are reported as
follows
\begin{center}
    \([F(df_1, df_2) = F; p = p\text{-value}]\),
\end{center}
where \(df_1\) is the degrees of freedom of the observed effect, \(df_2\) is the
degrees of freedom of the error, and \(F\) is the F-statistic
\cite{vallatPingouinStatisticsPython2018}. If sphericity was met, \(p\)-values
were not corrected using the Greenhouse-Geisser method (\(p\)-unc).
\begin{align*}
    df_1 & = k - 1                                                      &  & \text{(k = number of groups)}                     \\
    df_2 & = (k - 1)(n - 1)                                             &  & \text{(n = number of samples in each group)}      \\
    F    & = \frac{SS_{\text{Effect}} / df_1}{SS_{\text{Error}} / df_2} &  & \text{(SS = Sums of squares for effect or error)}
\end{align*}
If data points within dependent sample pairs were missing, such pairs were
excluded from paired t-tests while others from the same subject remained.

Metric non-normal distributed, independent data was analyzed using the
Kruskal-Wallis H-test and Mann–Whitney U tests. Results of the Kruskal-Wallis
H-test are reported as
\[H(df) = H\], with \(df\) being the degrees of freedom
and \(H\) being the Kruskal-Wallis H statistic, corrected for ties
\cite{vallatPingouinStatisticsPython2018}.

Metric bivariate non-normal distributed data was correlated using Spearman’s
rank correlation and reported as: \[\rho (df) = \rho, p = p\text{-value}\],
where \(\rho\) is Spearman’s rank correlation coefficient. \(df\) is calculated
as
\[
    df = n - 2 \quad \text{(n = number of observations)}
\]
These tests were applied using Python (3.10) packages \texttt{pingouin} (0.5.1)
and \texttt{statsmodels} (0.14.0)
\cite{seaboldStatsmodelsEconometricStatistical2010,vallatPingouinStatisticsPython2018}. Data was plotted
using \texttt{seaborn} \cite{waskomSeabornStatisticalData2021} and \texttt{plotastic}
\cite{kuricPlotasticBridgingPlotting2024}. Sphericity was ensured by Mauchly’s
test. Normality was checked with the Shapiro-Wilk test for \(n > 3\).

Data points were log10 transformed to convert the scale from multiplicative
("fold change") to additive, or to fulfill sphericity requirements. \(P\)-values
derived from patient survival data were corrected using the Benjamini-Hochberg
procedure. For other post-hoc analyses, \(p\)-values were not adjusted for
family-wise error rate in order to minimize type I errors. To prevent type II
errors, the same conclusions were validated by different experimental setups and
through varying hMSCs donors across experiments (see \apdxref{apdx:supplemental}{tab:S1}).

Significant \(p\)-values from pairwise tests were annotated as stars between
data groups
\[ p-value = 0.05 > \text{*} > 0.01 > \text{**} > 10^{-3} >
    \text{***}> 10^{-4} > \text{****}. \]
If too many significant pairs were
detected, only those pairs of interest were annotated.

No power calculation was performed to determine sample size since samples were
limited by availability of primary hMSC donors. Experiments were repeated until
a minimum of three biological replicates were gathered.


\subsubsection*{Patient Cohort, Analysis of Survival and Expression}
Patient samples (\(n=873\)) were collected at the University Hospital Heidelberg
and processed as described
\cite{seckingerTargetExpressionGeneration2017b,seckingerCD38ImmunotherapeuticTarget2018},
and are available at the European Nucleotide Archive (ENA) via accession numbers
PRJEB36223 and PRJEB37100. Consecutive patients with monoclonal gammopathy of
unknown significance (MGUS) (\(n = 62\)), asymptomatic (\(n = 259\)),
symptomatic, therapy-requiring (\(n = 764\)), and relapsed/refractory myeloma
(\(n = 90\)), as well as healthy donors (\(n = 19\)) as comparators were
included in the study approved by the ethics committee (\#229/2003,
\#S-152/2010) after written informed consent.

Gene expression was measured by RNA sequencing as previously described
\cite{seckingerCD38ImmunotherapeuticTarget2018}. Gene expression is defined as
the log\(_2\) transformed value of normalized counts~+~1 (as pseudocount).
Progression-free (PFS) and overall survival (OS) were analyzed for the subset of
previously untreated symptomatic myeloma patients. For delineating “high” and
“low” expression of target adhesion (\(n=101\)) and cell cycle (\(n=173\))
genes, thresholds per gene were calculated with maximally selected rank
statistics by the \texttt{maxstat} package in R
\cite{hothornMaximallySelectedRank}. PFS and OS were analyzed for high
vs. low expression with the Kaplan-Meier method \cite{kaplanNonparametricEstimationIncomplete1958}.
Significant differences between the curves were analyzed with log-rank tests
\cite{harringtonClassRankTest1982}. \(P\)-values were corrected for multiple testing
by the Benjamini-Hochberg method. Analyses were performed with R version 3.6.3
\cite{rcoreteamLanguageEnvironmentStatistical2018}.
