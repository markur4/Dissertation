

\unnsubsection{\textit{\textbf{Aims}}}% > Aims are part of Introduction
\label{sec:aims}%
This PhD thesis is designed to bridge significant gaps in the understanding and
analysis of myeloma cell behavior and the handling of complex biomedical
datasets. The specific aims are as follows:

\begin{itemize}
    \item Develop an \textit{in vitro} model to elucidate the mechanisms of
          myeloma cell dissemination in interaction with mesenchymal stromal cells
          (hMSCs), focusing particularly on:
          \begin{itemize}
              \item Observing and quantifying cell proliferation, attachment, and
                    detachment dynamics using time-lapse microscopy.
              \item Isolating and characterizing distinct myeloma subpopulations
                    interacting with hMSCs to understand differential gene expression
                    related to cell adhesion and patient survival.
          \end{itemize}

    \item Design and implement a Python-based software tool, \texttt{plotastic},
          to facilitate the analysis of multidimensional datasets
          generated in biomedical research. This tool will aim to:
          \begin{itemize}
              \item Streamline the data analysis process, making it more efficient and
                    reproducible.
              \item Integrate visualization and statistical analysis capabilities to
                    ensure that data analysis protocols are aligned with the ways in which
                    data is visualized.
              \item Provide a case study demonstrating the application of
                    \texttt{plotastic} in the analysis of \textit{in vitro} dissemination
                    experiments, emphasizing the tool's ability to handle semi-big data and
                    enhance reproducibility.
          \end{itemize}

    \item Synthesize the findings from the experimental and software development
          components to advance the understanding of myeloma dissemination and improve
          research practices in biomedical data analysis.
\end{itemize}

These aims are crafted to address both the biological and technical challenges
in current cancer research methodologies and data science applications in
biomedicine, fostering advancements that could lead to novel therapeutic
strategies and more robust scientific inquiries.