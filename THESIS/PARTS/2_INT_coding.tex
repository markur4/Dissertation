

\unnsubsection{Multidimensional Data in Biomedical Research}%
\label{sec:intro_multidimensional_data}%
As modern biosciences advance, researchers increasingly encounter datasets that
are influenced by a variety of independent variables, such as time, dosage, and
environmental conditions. These variables introduce multidimensional complexity
into datasets, challenging traditional analysis methods. For instance, cell
adhesion studies, which are crucial for understanding cellular interactions and
cancer metastasis, often require analyses across multiple time points and
varying adhesion molecule concentrations, demonstrating a time-dependent
variability that significantly impacts biological interpretations
\cite{reblTimedependentMetabolicActivity2010, mckayCellcellAdhesionMolecules1997,bolado-carrancioPeriodicPropagatingWaves2020a}.


Multidimensional data encompass datasets where multiple \emph{independent
    variables} (here reffered to as \emph{factors}) can influence one
\emph{dependent variables} (\emph{outcomes}) \cite{krzywinskiMultidimensionalData2013}.
% Statistically speaking, an experiment assesses if an \emph{independent variable}
% (here reffered to as \emph{factors}) has an influence on a \emph{dependent
% variable} (\emph{outcome}), e.g. ``Does \emph{Time} influence \emph{Gene
% Expression}?'' \cite{motulskyIntuitiveBiostatisticsNonmathematical2018}. 
In biomedicine, dependent variables are often continuous (intervals or ratios),
whereas independent variables are often categorical (ordinal or nominal),
respectively. Categorical variables comprise discrete values called categories
or \emph{levels}, which are assigned to experimental conditions or measurement
modalities, for example the factor \textit{`time'} could comprise three
\linebreak levels: \textit{`\SI{0}{\hour}'}, \textit{`\SI{24}{\hour}'}, and
\textit{`\SI{48}{\hour}'}. Such setups are attractive, because they are
compatible with common hypothesis tests, such as ANOVA etc.
\cite{motulskyIntuitiveBiostatisticsNonmathematical2018}: If the levels of one
factor are associated with a different outcome, that factor is considered to
have an influence on the dependent variable. Multiple factors address multiple
hypotheses, including the influence from each individual factor, but also
potential interactions between factors. This makes it crucial to design analysis
strategies that can reveal the true structure and value of the data
\cite{krzywinskiMultidimensionalData2013}.


% TODO: Add examples of multidimensional data in biosciences
% (put the following examples into discussion? Use one example, rest into discussion?)

A primary example of multidimensional data is multiplex RT-qPCR, where the
expression levels of various genes are measured across different samples under
varying conditions \cite{bustinReproducibilityBiomedicalResearch2014}. Here, the
dependent variable is typically the fold change expression values derived from
$\Delta\Delta$Ct calculations
\cite{brankatschkSimpleAbsoluteQuantification2012}. The independent variables
include the genes being measured and the experimental conditions under which the
samples are processed.

Microscopy data further illustrate the complexity of multidimensional datasets
\cite{ruedenImageJ2ImageJNext2017}. In this context, the dependent variable
might be a quantifiable feature, such as cell count or morphological metrics
extracted from image analyses. The independent variables can expand immensely to
include factors such as well-plate coordinates in a 96-well plate, Z-positions
in confocal microscopy, and time points in time-lapse studies.

Lastly, big-data aggregation tools like \texttt{Metascape} provide a rich source
of multidimensional data by integrating various dependent variables, such as
gene expression fold changes and associated $p$-values, with independent variables
spanning gene identifiers, gene ontology terms, and ontology classes derived from
multiple databases \cite{zhouMetascapeProvidesBiologistoriented2019}. Despite
the provision of summarized graphical outputs, the raw data often remain in
complex, nested formats within Excel sheets, posing significant challenges for
hypothesis-driven research.

This extensive integration of multiple dimensions requires sophisticated
visualization and analysis techniques. While basic statistical visualizations
suffice for one- or two-dimensional data, more complex data sets necessitate
advanced techniques, which allow researchers to visualize and interact with data
in ways that elucidate the underlying patterns and relationships
\cite{dunnExploringVisualizingMultidimensional2017}. However, the gap between
available visualization tools and the needs of clinicians or biologists without
extensive bioinformatics training remains wide, emphasizing the need for
intuitive, user-friendly tools that bridge this knowledge gap and enhance the
accessibility of complex data analyses
\cite{dunnExploringVisualizingMultidimensional2017}.




% \unnsubsection{Multidimensional Data in Biomedical Research}%
% \label{sec:multidimensional_data}%
% - This is the first paragraph after the sections describing cancer dissemination

% - Introduce why multidimensional data
% Modern biosciences describe processes (e.g. cell-adhesion) that are highly
% dependent on multiple experimental conditions, like time and concentration
% \cite{reblTimedependentMetabolicActivity2010, mckayCellcellAdhesionMolecules1997}
% - Go into more detail how and why cell adhesion is highly time-dependent, providing real citable sources for these claims  
% - The awareness for complex datasets is rising in biosciences, as tutorials for
% multidimensional data are available
% \cite{krzywinskiMultidimensionalData2013, dunnExploringVisualizingMultidimensional2017}.

% - What is multidimensional data
% - Explain in detail, introduce terms like dependent variable, independent variable, factors

% - Describe in three examples on methods how multidimensional data is generated in biosciences
% - Provide real citable sources that show examples for every example!

% An example of multidimensional data is (multiplex) RT-qPCR, a method that
% measures the expression of multiple genes in multiple samples. The data can be
% structured with the dependent variable being the foldchange expression values
% (resulting from delta delta Ct according to
% \citet{brankatschkSimpleAbsoluteQuantification2012}), and the independent
% variables being the genes and the samples. The data is grouped into foldchange
% expression values as dependent variables, and

% Another hallmark example for multidimensional data are microscopy data. Stacks
% of images themselves are three-dimensional (single black-white channel),
% however, multidimensional data refers mostly to the resulting data after image
% analysis. For example, the dependent variable could a feature quantified from
% each image (e.g. number of cells). The independent factors can increase
% drastically, including experimental conditions grouped into well-plate
% coordinates (e.g. 96 wellplates), or Z-position (confocal microscopy), or time
% (time-lapse microscopy).


% Another source of multidimensional data are aggregated big-data, such as the
% result from GO-term enrichment analysis tools like \texttt{Metascape}
% \cite{zhouMetascapeProvidesBiologistoriented2019}. This Data integrates multiple
% dependent variables like foldchange expression and p-values, and the data is
% grouped into factors like gene, gene ontology terms and ontology classes from
% multiple online databases. Since these factors apply to every experimental
% conditions, GO analysis can result in highly complex and nested datasets.
% Although \texttt{Metascape} returns a graphical user interface with numerous
% summarizing graphs, the supplied Excel sheets can be non-readable for
% humans, making it difficult to extract the information that is relevant for
% specific research hypotheses.


% ======================================================================
\unnsubsection{Nontransparencies in Biomedical Data Analyses}%
\label{sec:intro_nontransparencies}%
%%%%%%%%%%%%%
The advent of advanced technologies in biosciences has ushered in an era of \emph{big
    data}, characterized by unprecedented volumes and complexities of
data~\cite{bubendorfHighthroughputMicroarrayTechnologies2001, yangScalabilityValidationBig2017,
    ekmekciIntroductionProgrammingBioscientists2016}. This rise has been paralleled
by significant challenges in data analysis, particularly impacting the
reproducibility of scientific research. Studies such as the
\citet{baker500ScientistsLift2016} survey revealed that more than
\SI{70}{\percent} of researchers have tried and failed to reproduce another
scientist's experiments, highlighting a reproducibility crisis that questions
the reliability of scientific findings
\cite{begleyReproducibilityScienceImproving2015, ioannidisWhyMostPublished2005}.

\emph{Reproducibility} is considered foundational to scientific research,
ensuring that findings are reliable and verifiable. Still, its meaning requires
precise definition \cite{goodmanWhatDoesResearch2016}. The common understanding
of scientific reproduction implies not only that detailed information is
provided to enable independent repetition (\emph{transparency}), but also that
time and effort is invested into repeating the experiments
(\emph{corroboration}). However, since modern biomedical journals are demanding
novelty research, and since experiments have become highly specialized and
time-intensive, repeating someone else's work is considered neither interesting%
\footnotequote{There are no scientists with the interest, resources, or
    incentives to “repeat” or confirm this vast sea of published work, so whether
    the findings they report are reproducible will simply never be assessed.}{flierProblemIrreproducibleBioscience2022}
nor possible for most publications \cite{flierProblemIrreproducibleBioscience2022,
    pengReproducibleResearchComputational2011}. Hence, the meaning of
reproducibility is confined to \emph{transparency}, a concept that has
been applied to many fields, including clinical trials
\cite{goodmanWhatDoesResearch2016,
    committeeonstrategiesforresponsiblesharingofclinicaltrialdataSharingClinicalTrial2015}.

Nevertheless, there is a surprising amount of evidence for nontransparencies in
biomedical data analyses: For Microarray-based miRNA profiling, raw data was not
reported in more than \SI{40}{\percent} of 127 articles, making independent
verification impossible \cite{witwerDataSubmissionQuality2013}. The same study
also found that re-analysis of data often times did not support the original
conclusions. Furthermore, \SI{44}{\percent} of 233 preclinical articles describe
statistical tests insufficiently, while few don't describe them at all
\cite{gosselinInsufficientTransparencyStatistical2021}. Another study reviewed
147 papers in the field of optometrics and found that \SI{91}{\percent} did not
discuss their rationale of correcting $p$-values for multiple comparisons (e.g.
Bonferroni correction) \cite{armstrongWhenUseBonferroni2014}. However, given
that the exact use of multiple comparisons corrections has been under debate for
decades, it is reasonable to assume that researchers lack the confidence to
report their technique in detail \cite{pernegerWhatWrongBonferroni1998,
    moranArgumentsRejectingSequential2003, sullivanFactsFictionsHandling2021}. In
general, $p$-values are target of extreme scrutiny and also the cause of many
arguments, which themselves are of questionable statistical reasoning%
\footnotequote{Arguing about the P value is like focusing on a single
    misspelling, rather than on the faulty logic of a
    sentence}{leekStatisticsValuesAre2015}%
\cite{leekStatisticsValuesAre2015}. Additionally, statistical illiteracy is a
well-known problem among clinicians
\cite{lakhlifiIllusionKnowledgeStatistics2023}. Among biomedical researchers,
\SI{77}{\percent} state that they have not received formal training in data
literacy, including visualization and public deposition of data, although they
understand its high relevance \cite{federerDataLiteracyTraining2016}.
Correspondingly, it has been communicated that there is a lack of intuitive
tools to embed computational work into publications, but also a lack of
bioinformaticians to translate computation into clinics
\cite{mesirovAccessibleReproducibleResearch2010, smithJournalOpenSource2018,gomez-lopezPrecisionMedicineNeeds2019}.
Therefore, nontransparencies in biomedical analyses are not only caused by a
habit%
\footnotequote{[...] old habits die hard, and many will be unwilling to discard the
    hours spent learning existing systems.}{pengReproducibleResearchComputational2011}%
of insufficient reporting, but could be exacerbated by the confusions caused
by currently available methodologies and the lack of proper training.



\unnsubsection{Semi-Big Data: Big Enough to Cause Problems}%
\label{sec:intro_semi_big_data}%
Recent advances in big data analysis have significantly improved the
standardization of both raw data availability and processing pipelines \cite{gomez-cabreroDataIntegrationEra2014}.
Particularly in RNAseq analysis, automation and the use of sophisticated
software have established standards that enhance reproducibility across
studies. For example, tools such as \texttt{STAR} and \texttt{HISAT} for
sequence alignment, and \texttt{Cufflinks} and \texttt{DESeq} for differential
expression analysis, rely on scripts that standardize processing steps to
produce repeatable and verifiable results \cite{dobinSTARUltrafastUniversal2013,
    kimHISATFastSpliced2015,trapnellDifferentialGeneTranscript2012,loveModeratedEstimationFold2014}.
These frameworks not only automate data handling but also ensure that data
analysis protocols are followed consistently, reducing human error and
variability between different users or laboratories.

However, this level of standardization and automation has not been mirrored in
the analysis of \emph{semi-big data}. Semi-big data, as introduced in this thesis,
describes datasets that are on the cusp of manageability: substantial enough to
overwhelm manual analysis methods yet not sufficiently large or uniform to
justify the heavy computational frameworks developed for big
data. Such data are frequently generated in experiments like automated
microscopy or multiplex qPCR, where the scale and complexity of the data can
vary significantly depending on the experimental design and objectives
\cite{krzywinskiMultidimensionalData2013}.

Researchers often revert to basic tools such as \textit{Microsoft Excel} for
analyzing these semi-big datasets \cite{incertiYouStillUsing2019a}. While Excel
provides familiarity and immediate accessibility, it lacks the sophisticated
data handling capabilities necessary for efficient and error-free processing of
complex (multidimensional) datasets. This reliance on manual methods not only
makes the analysis laborious and prone to mistakes but also significantly
impedes the reproducibility of research findings. The time and effort required
to replicate analyses done manually mean that validating findings from semi-big
data can be prohibitively challenging for peer reviewers and other researchers
in the field.

Given these challenges, there is a critical need for developing new tools and
frameworks specifically tailored for semi-big data. These tools should bridge
the gap between the simplicity of user-friendly software like Excel and the
robust, script-based automation seen in big data frameworks. By providing
standardized, repeatable, and easy-to-use methods for handling complex datasets,
such tools could significantly enhance the reliability and efficiency of
research involving semi-big data, ultimately supporting broader scientific
inquiry and verification.





\unnsubsection{The Shortcomings of Common Biomedical Analysis Tools}%
\label{sec:intro_nontransparent_tools}%
%%%%%%%%%%%%%%
Interactive software systems commonly used for exploratory data analysis in
biomedical research often lack mechanisms to track and reproduce the
researcher's actions systematically. Even when analysis is performed using
scripting languages, the integration of results from multiple packages without a
coherent record of the commands and code used undermines reproducibility. This
practice can obscure analysis, making it difficult, if not impossible, for other
researchers to replicate the results \cite{leekStatisticsValuesAre2015,
    pengReproducibleResearchComputational2011,
    mesirovAccessibleReproducibleResearch2010, localioStatisticalCodeSupport2018}.

A particularly illustrative example is \textit{GraphPad Prism}, a tool
ubiquitously employed across biomedical disciplines for statistical analysis.
Despite its widespread use, it does contribute to data analysis
nontransparencies due to \textit{Prism}'s closed-source nature and the common
journal practice of not requiring detailed methodological transparency in its
usage, a practice that is common in biostatistics literature
\cite{gosselinInsufficientTransparencyStatistical2021,
    localioStatisticalCodeSupport2018}. Furthermore, \textit{GraphPad Prism} still
requires manual data entry and lacks the robustness and automation necessary for
handling multidimensional or semi-big data. Although, \textit{GraphPad Prism} is
compatible \textit{``multiple variable tables''} \dashedsentence{similar to
    long-form tables known from \citet{wickhamTidyData2014}}, but does not
automatically graph these kinds of tables, but only user specified subsets
\cite{GraphPadPrism102024}.

Moreover, \textit{Microsoft Excel}, another staple in data processing in
biomedicine, is notoriously inadequate for handling multidimensional data and
complex statistical analyses. Its limitations include poor error tracking,
absence of change documentation (audit trails), and a propensity for introducing
errors that often go unnoticed, such as converting gene names to dates
\cite{ziemannGeneNameErrors2016}. To compensate for these shortcomings,
\textit{Microsoft} has recently integrated a Python interpreter into
\textit{Excel}, allowing researchers to automate tasks and analyze data
efficiently and correctly \cite{microsoftexcelAnnouncingPythonExcel2023}.

Indeed, many common tools in biomedicine allow for scripting or automation to
handle semi-big data more effectively. For example, \texttt{Fiji/ImageJ}, a
popular image processing platform, supports extensive macro and scripting
capabilities \cite{ruedenImageJ2ImageJNext2017}. These features enable researchers to automate batch processing of
image data, streamlining tasks that would otherwise require laborious manual
input. Similarly, \texttt{PyMOL}, a leading tool in protein structural biology,
utilizes Python scripting to automate complex tasks, allowing for detailed
molecular modeling and visualization that are reproducible and scalable across
datasets \cite{PyMOL2024, rigsbyUsingPyMOLApplication2016}.

Although automation scripts used in tools like \texttt{Fiji/ImageJ} and
\texttt{PyMOL} improve transparency for publishing singular data analysis
pipelines, they still face challenges that can impede their reproducibility
\cite{pengReproducibleResearchComputational2011, sandveTenSimpleRules2013}:
These scripts sometimes require specialized software environments, where setting
up dependencies and configurations can be complex enough to discourage
replication efforts. Additionally, these scripts do not always provide
comprehensive outputs of intermediate steps, which is crucial for verifying and
understanding the progression of data analysis \cite{sandveTenSimpleRules2013}.

On the other hand, when scripts are designed to be more generalized and
distributed \linebreak\dashedsentence{for instance, as a \texttt{Fiji/ImageJ}
    plugin or a standalone application} they can make substantial contributions to
scientific research by enabling other researchers to apply these tools to their
own data sets
\cite{narztReusabilityConceptProcess1998,wilkinsonFAIRGuidingPrinciples2016}.
However, this approach also comes with its own set of challenges
\cite{sandveTenSimpleRules2013}. These generalized tools often lack
comprehensive user-manuals (\emph{documentation}) and are not thoroughly tested
across different platforms or data sets, which can lead to unexpected errors
that can not be fixed by the user. Moreover, even when these tools are
available, they frequently suffer from low adoption rates, meaning that few
people are familiar with the details of such tools, further decreasing the
confidence and reproducibility in the final results.

Given these complexities, there is a pressing need for new analytical tools
specifically designed for semi-big data. These tools must strike a balance
between the ease of use found in basic software and the robust, analytical
capabilities of more sophisticated systems. By providing standardized workflows,
comprehensive documentation, and ensuring cross-platform compatibility, these
tools can significantly enhance reproducibility. They not only allow researchers
to perform analyses more efficiently but also ensure that these analyses are
robust, transparent, and easily verifiable by the broader scientific community.

This thesis presents a software environment developed in Python, designed to
bridge this gap. It demonstrates that even minimal coding skills can be
leveraged to create powerful tools that standardize and accelerate the analysis
of semi-big data, ultimately fostering more reproducible and trustworthy
scientific research.







% ======================================================================
\unnsubsection{Modern Standards of Software Development}
\label{sec:intro_code_quality}
%%%%%%%
A main reason to write software is to define reusable instructions for task
automation~\cite{narztReusabilityConceptProcess1998}.
The complexity of software code makes it prone to errors, which can prevent
its usage by persons other than the author himself. This is a problem for the
general scientific community, as the software is often essential for
reproduction~\cite{sandveTenSimpleRules2013}. Hence, modern journals aim to
enforce standards to software development, including software written and used
by biological researchers~\cite{smithJournalOpenSource2018}. Here, we provide a
brief overview of the standards utilized by \texttt{plotastic} that ensure
its reliability and reproducibility by the scientific
community~\cite{pengReproducibleResearchComputational2011}.


Modern software development is a long-term commitment of maintaining and
improving code after initial release~\cite{boswellArtReadableCode2011}. Hence,
it is good practice to write the software such that it is \emph{scalable},
\emph{maintainable} and \emph{usable}. \emph{Scalability} or, to be precise,
\emph{structural scalability} means that the software can easily be expanded
with new features without major modifications to its architecture
\cite{bondiCharacteristicsScalabilityTheir2000}. This is achieved by writing the
software in a modular fashion, where each module is responsible for a single
function. \emph{Maintainability} means that the software can easily be fixed
from bugs and adapted to new requirements \cite{kazmanMaintainability2020}. This
is achieved by writing the code in a clear and readable manner, and by writing
tests that ensure that the code works as
expected~\cite{boswellArtReadableCode2011}. \emph{Usability} is hard to
define~\cite{brookeSUSQuickDirty1996}, yet one can consider a software as usable
if the commands have intuitive names and if the software's manual, termed
\emph{documentation}, is up-to-date and easy to understand for new users with
minimal coding experience. A software package that has not received an update
for a long time (approx. one year) could be considered abandoned. Abandoned
software is unlikely to be fully functional, since it relies on other software
(dependencies) that has changed in functionality or introduce bugs that were not
expected by the developers of all dependencies. Together, software that's
scalable, maintainable and usable requires continuous changes to its codebase.
There are best practices that standardize the continuous change of the codebase,
including version control, continuous integration (often referred to as CI), and
software testing.

Version control is a system that records changes to the codebase line by line,
documenting of the detailed history of the codebase, including the person and
timepoint of every change. This is required to isolate new and experimental
features into newer versions and away from the stable version that's known to
work. The most popular version control system is \texttt{Git}, which is
considered the industry standard for software
development~\cite{chaconGitBook2024}. \texttt{Git} can use GitHub.com as a
platform to store and host codebases in the form of software repositories.
GitHub's most famous feature is called ``pull request''. A pull request is a
request from anyone registered on GitHub to include changes to the codebase (as
in \textit{``please pull this into your main code''}). One could see pull
requests as the identifying feature of the open source community, since it
exposes the codebase to potentially thousands of independent developers,
reaching a workforce that is impossible to achieve with closed source models
used by paid software companies.

\emph{Continuous integration} (CI) is a software development practice in which
developers integrate code changes into a shared repository several times a
day~\cite{duvall2007continuous}. Each integration triggers the test suite,
aiming to detect errors as soon as possible. The test suite includes building
the software, setting up an environment for the software to run, and then
executing the programmed tests, ensuring that the software runs as a whole.
Continuous integration is often used together with software branches. Branches
are independent copies of the codebase that are meant to be merged back into the
original code once the changes are finished. Since branches accumulate multiple
changes over time, this can lead to minor incompatibilities between the branches
of all developers (integration conflicts), which is something that CI helps to
prevent.

Continuous integration especially relies on a thorough software testing suite.
Software testing is the practice of writing code that checks if the codebase
works as expected~\cite{10.5555/2161638}. The main type of software testing is
unit testing, which tests the smallest units of the codebase (functions and
classes) in isolation (\autoref{lst:unit_test}). Software testing is automated
by specialized frameworks that execute the tests and report the results, a
popular example being \texttt{pytest}, which is utilized by
\texttt{plotastic}~\cite{pytestx.y}.

\def\mycaption{ Example of an arbitrary Python function and its respective unit
    test function. The first function simply returns the number 5. The second
    function tests if the first function indeed returns the number 5. The test
    function is named with the prefix ``\texttt{test\_}'' and is placed in a
    file that ends with the suffix ``\texttt{\_test.py}''. Testing frameworks
    such as \texttt{pytest} scan the repository for files that end with
    ``\texttt{\_test.py}'' and execute the functions that start with
    ``\texttt{test\_}''. Note that code after ``\texttt{\#}'' is considered a
    comment and won't be executed.}
\begin{lstlisting}[
    language=Python, 
    style=pythonstyle,
    label=lst:unit_test, 
    caption=\mycaption,
    ]
# Define a function called "give_me_five" that returns the number 5
def give_me_five():
    return 5
# Define a test function asserting that "give_me_five" returns 5
def test_give_me_five():
    assert give_me_five() == 5 
\end{lstlisting}

\pagebreak

The quality of the software testing suite is measured by the code coverage, the
precision of the tests, and the number of test-cases that are checked. The code
coverage is the percentage of the codebase that is called by the testing
functions, which should be as close to 100\% as possible, although it does not
measure how well the code is tested. The precision of the test is not a
measurable quantity, but it represents if the tests truly checks if the code
works as expected. The number of test-cases is the number of different scenarios
that are checked by the testing functions, for example testing every possible
option or combinations of options for functions that offer multiple options.

Together, the standards of software development, including version control,
continuous integration, and software testing, ensure that the software is
scalable, maintainable, and usable. This is especially important for software
that is used by the scientific community, as it ensures that the software is
working as expected at defined versions years after publishing scientific
results.

% ======================================================================
% ======================================================================
\unnsubsection{What makes Python an ``Easy'' Programming Language?}%
\label{sec:intro_python}%
%%%%%%%%%%
Here, we provide a general overview of the Python programming language,
explaining terms like \textit{``type''}, \textit{``method''}, etc., in order to
prepare readers without prior programming experience for the following chapters.
We also describe the design principles of Python to lay out the key concepts
that differentiate Python compared to other programming languages. A more
detailed tutorial on Python that's specialized for bioscientists is found
in~\citealt{ekmekciIntroductionProgrammingBioscientists2016}


Languages such as Python are considered \textit{``high-level''}, which means
that it is designed to be easy to read and write, but also independent of
hardware by hiding (\textit{``abstracting''}) underlying
details~\cite{PythonLanguageReference2024}. A key principle of Python is the
emphasis on implementing a syntax that is concise and close to human language
(\autoref{lst:readable}, \autoref{lst:not_readable}).

\def\mycaption{ Example of
    readable Python code. This one-line code returns the words (string)
    \texttt{"Hello, World!"} when executed. The command is straightforward and easy
    to understand.}
\begin{lstlisting}[
    language=Python, 
    style=pythonstyle,
    label=lst:readable,
    caption=\mycaption
    ]
print("Hello, World!")
# Expected output: Hello, World!
\end{lstlisting}

\def\mycaption{ Example of less readable code written in the low-level
    programming language C. This code is doing exactly the same as the Python
    code in \autoref{lst:readable}, but is harder to understand because more
    steps are needed, including the import of a library \texttt{stdio.h} and the
    definition of a function called \texttt{main}. Note that C uses \texttt{//}
    to begin comment sections.}
\begin{lstlisting}[
    language=C, 
    style=defaultstyle,
    label=lst:not_readable, 
    caption=\mycaption
    ]
#include <stdio.h>          // Import functions for standard input & output
int main() {                // Define a function called 'main'
    printf("Hello, World!");
    return 0;
}
// Expected output: Hello, World!
\end{lstlisting}

Furthermore, Python is an \textit{interpreted} language, which means that the
code is executed line by line. This makes coding easier because the programmer
can see the results of the code immediately after writing it, and error messages
point to the exact line where the error occurred (\autoref{lst:errormessage}). This is in contrast to
\textit{compiled} languages, where the code has to be compiled into machine code
before it can be executed. The advantage of compiled languages is that the code
runs faster, because the machine code is optimized for the hardware.

\def\mycaption{Example of an error message. Since Python is case-sensitive, the
    error is caused by misspelling the variable name \texttt{hi} as \texttt{Hi}.
    The error message begins with the type of error (\texttt{NameError}),
    followed by a traceback that shows the sequence of function calls that led
    to the error. The traceback located the source of the error in the file
    \texttt{errormessage.py} and points to the line that caused the error.
    Finally, the error message explains the error, stating that \texttt{'Hi'} is
    undefined. }
\begin{lstlisting}[
    language=Python, 
    style=pythonstyle,
    label=lst:errormessage,
    caption=\mycaption
    ]
hi = "Hello World"  # Define a variable 'hi' with the value "Hello World"
print(Hi)           # Print the value of the undefined variable 'Hi'

# Expected Output:
# ---------------------------------------------------------------------------
# NameError                                 Traceback (most recent call last)
# File /Users/martinkuric/Documents/errormessage.py:2
#       1 hi = "Hello World"
# ----> 2 print(Hi)

# NameError: name 'Hi' is not defined
\end{lstlisting}

Python automates tasks that would otherwise require an advanced understanding of
computer hardware, like the need for manual allocation of memory space. This is
achieved by using a technique called \textit{``garbage collection''}, which
automatically frees memory space that is no longer needed by the program. This
is a feature that is not present in low-level programming languages like C or
C++, that were designed to maximize control over hardware.

Another hallmark of Python is its \textit{dynamic typing system}. In Python the
type is inferred automatically during code execution
(\autoref{lst:dynamic_typing}). This is in contrast to \textit{statically} typed
languages like C, where the type of a variable has to be declared explicitly and
cannot be changed during code execution
(\autoref{lst:static_typing})~\cite{PythonLanguageReference2024}.

\def\mycaption{ Example of dynamic typing in Python. The variable ``\texttt{a}''
    is assigned the value 5, which is of type integer. The variable
    ``\texttt{a}'' is then overwritten with the value ``\texttt{Hello,
        World!}'', which is of type string. Python allows dynamic re-assignment of
    variables with different types. Note that code after ``\texttt{\#}'' is
    considered a comment and won't be executed.}
\begin{lstlisting}[
    language=Python,
    style=pythonstyleNonbreaking,
    label=lst:dynamic_typing,
    caption=\mycaption,
    belowskip=-\vhalf % > Remove space because two listings are close
    ]
a = 5  # Type integer
a = 5.0  # Type float
a = 'Hello, World!'  # Type string
a = True  # Type boolean
a = False  # Type boolean
a = [1, 2, 3]  # Type list of integers
a = {'name': 'Regina'}  # Type dictionary
\end{lstlisting}

\def\mycaption{ Example of static typing in C. The variable ``\texttt{a}'' is
    declared as an integer (\texttt{int}), and can only store integers. The
    variable ``\texttt{a}'' is then assigned the value 5, which is an integer.
    The variable ``\texttt{a}'' is then assigned the value \texttt{'Hello,
        World!'}, which is a string. This results in a compilation error, because
    the variable ``\texttt{a}'' can only store integers. Note that code after
    ``\texttt{//}'' is considered a comment and won't be executed. }
\begin{lstlisting}[
    language=C,
    style=defaultstyleNonbreaking,
    label=lst:static_typing,
    caption=\mycaption,
    ]
int a;  // Declare type as integer
a = 5;
a = 'Hello, World!';  // Compilation error!
\end{lstlisting}

Dynamic typing makes Python a very beginner-friendly language, since one does
not have to keep track of the type of each variable. However, this also makes
Python a slower language, because the interpreter has to check the type of each
variable during code execution. Also, developing code with dynamic typing
systems is prone to introducing bugs (\texttt{TypeError}), because it allows
unexperienced developers to convert variables from one type to another without
noticing, leading to unexpected behavior. Hence, larger Python projects require
disciplined adherence to programming conventions. One such convention is
\textit{type hinting}, which is a way to explicitly note the type of a
variable. Type hinting does not have an effect on the code, but it makes the
code more readable and understandable for other developers, and allows for
development environments to detect type errors before execution
(\autoref{lst:type_hint})~\cite{vanrossumPEP484Type2014}.

\def\mycaption{ Example of type hints used in Python. Explicitly stating the
    type of the variable is optional and does not change the behavior of the
    code, but behaves exactly as shown in \autoref{lst:dynamic_typing}.}
\begin{lstlisting}[
    language=Python,
    style=pythonstyleNonbreaking,
    label=lst:type_hint,
    caption=\mycaption,
    ]
a: int = 5
a: str = 'Hello, World!'
\end{lstlisting}

%%%%%%%%%%%%%%%%%%%%%%%%


To make Python as easy as possible, python packages aim to reduce the amount of
code that has to be written by the user. For example, the package
\texttt{matplotlib} is a plotting library where every command is written such
that the user immediately understands its purpose, like plotting a line or
labeling an axis (\autoref{lst:matplotlib}). Hence \texttt{matplotlib} code is a
sequence of simple function calls, where the state of the plot is modified and
saved in the background line by line.

\def\mycaption{ Example of using pre-written functions of a Python package. The
    functions of the package \texttt{matplotlib.pyplot} become accessable by
    importing the package as \texttt{plt}, where plt serves as an alias (or rather
    shortcut) to access the functions of the package. Then, two arbitrary lists are
    defined, \texttt{x} and \texttt{y}. These datapoints are plotted (scatterplot)
    using the function \texttt{plot} The plots x- and y-axes are then labeled and
    saved as an image. The code is written in a sequence of function calls, where
    the state of the plot is saved in the background. The plot is then displayed
    using the function \texttt{show}. }
\begin{lstlisting}[
    language=Python,
    style=pythonstyleNonbreaking,
    label=lst:matplotlib,
    caption=\mycaption,
    ]
import matplotlib.pyplot as plt  # Make functions accessible via plt
x = [1, 2, 3, 4, 5]              # Define arbitrary x values
y = [1, 4, 9, 16, 25]            # Define arbitrary y values
plt.plot(x, y)                   # Plot x against y
plt.xlabel('Timepoint')          # Add a label to the x axis
plt.ylabel('Foldchange')         # Add a label to the y axis
plt.title('Gene Expression')     # Add a title above the plot
plt.savefig('plot.png')          # Save plot as image onto harddrive
plt.show()                       # Show the plot preview
    
\end{lstlisting}

However, when no pre-written functions or packages are available, Python offers
the tools of a general purpose programming language to write and deploy custom
code easily. Programming styles can be classified into two main paradigms:
\emph{functional} and \emph{object-oriented} programming, which can be
understood as different ways to structure code. Python supports both paradigms.
In \emph{functional} programming, the code is written in a way that the program
is a sequence of function calls, where each function call returns a value that
is used in the next function call (\autoref{lst:functional}). This approach is
useful when multiple actions have to be performed on the same data and the
structure of the data is relatively simple, for example a string of a DNA
sequence.

\def\mycaption{ Example of functional programming in Python. The code
    defines a function called ``\texttt{find\_restriction\_site}'' that
    finds the position of a restriction site in a DNA sequence. The function
    ``\texttt{cut}'' uses the function ``\texttt{find\_restriction\_site}''
    to cut the sequence at the restriction site. To execute both functions,
    we first define an arbitrary DNA sequence and then call the function
    \texttt{cut} passing the sequence as an argument.
}
\begin{lstlisting}[
    language=Python,
    style=pythonstyleNonbreaking,
    label=lst:functional,
    caption=\mycaption,
    ]
def find_restriction_site(sequence: str):  # Define a function
    return sequence.find('GCGC')           # Find the position of 'GCGC'
    
def cut(sequence: str):                        # Define another function
    position = find_restriction_site(sequence) # Use the function above
    return sequence[position:]                 # Cut sequence at the position
    
seq1 = 'TGAGCTGAGCTGATGCGCTATATTTAGGCG'  # Define an arbitrary sequence
seq1_cut = cut(seq1)                     # Cut the sequence
print(seq1_cut)                          # Show the result
# Expected output: GCGCTATATTTAGGCG
    
    
\end{lstlisting}


When the data itself gains in complexity, for example when storing not just the
gene sequence, but also the promoter sequence, an \emph{object-oriented}
approach is more suitable (\autoref{lst:oop}). Object-oriented programming is a
programming paradigm that uses objects and classes. An object is a collection of
both data and functions, and a class is a blueprint for creating objects. The
data of an object is stored as \emph{attributes}. Functions that are associated
with an object are called \emph{methods}. A major benefit of using an object
oriented versus a functional approach is that the data itself programmable,
enabling the programmer to define the behavior of the data through methods. This
is achieved by using `\texttt{self}' to reference the objects themselves inside
the class. \texttt{self} can be understood as \textit{``this object''}, and is a
placeholder for objects that are to be created from the blueprint. \texttt{self}
is required to access attributes and methods before specific objects are created
in order to program how the objects are to be changed when calling methods.


% and acts
% as a placeholder within the blueprint (class) for the object , allowing the
% programmer to access the attributes and methods of the object without knowing
% the object's name that is assigned by the user when creating the object.

% allowing
% the programmer to access the attributes and methods of the object without knowing

% to program how the object is to be changed during creation or calling
% methods. ``\texttt{self}'' can be understood as \textit{``this object''},
% allowing the programmer to access the attributes and methods of the object
% without knowing the object's name that is assigned by the user when creating the
% object.


\def\mycaption{ Example of object oriented programming in Python. The class is
    called ``\texttt{Gene}'' and acts as a blueprint to create \texttt{Gene}
    objects. \texttt{Gene} has four methods, ``\texttt{\_\_init\_\_}'',
    ``\texttt{find\_promoter}'', ``\texttt{find\_restriction\_site}'' and
    ``\texttt{cut}''. The method ``\texttt{\_\_init\_\_}'' is called when
    creating (``initializing'') an object, which fills the object with
    user-defined data. The parameter ``\texttt{self}'' is a placeholder for the
    objects that are to be created. ``\texttt{find\_promoter}'' is a method that
    finds the position of the promoter in the gene and is called during object
    initialization. }
\begin{lstlisting}[
    language=Python,
    style=pythonstyleNonbreaking,
    label=lst:oop,
    caption=\mycaption,
    ]
class Gene:                               # Define a Gene class
    def __init__(self, sequence: str):    # Define how a Gene object is created
        self.sequence: str = sequence     # Save sequence as attribute
        self.promoter: str = self.find_promoter()  # Automatically find promoter
        # Add further Gene attributes here
    def find_promoter(self):              # Define how to find the promoter
        return self.sequence.find('TATA')
    def find_restriction_site(self):      # Define how to find restriction site
        return self.sequence.find('GCGC') # Find the position of 'GCGC'
    def cut(self):                        # Define how to cut the DNA sequence
        position = self.find_restriction_site()   # Call the method above
        return self.sequence[position:]   # Cut the gene at the position

gene1 = Gene(sequence='TGAGCTGAGCTGATGCGCTATATTTAGGCG')   # Create Gene object
print(gene1.cut())                        # Cut gene and show result 
# Expected output: GCGCTATATTTAGGCG
\end{lstlisting}



When designing software, both functional and object oriented programming can be
used together, where object oriented programming is often used to design the
program's overall architecture, and functional programming is used to implement
the algorithms of the program's features. This allows for scalability of the
software, as every single class is extended through the addition of new methods.


Furthermore, classes can be expanded in their functionalities through
\emph{inheritance} (\autoref{lst:inheritance}). Inheritance is a feature of
object-oriented programming that allows a class to access every attribute and
method of a parent class. For example, one could extend the class
``\texttt{Gene}'' with a class ``\texttt{mRNA}'', by writing a class
``\texttt{mRNA}'' that inherits from the class ``\texttt{Gene}''.

\vspace{\vhalf}

\def\mycaption{ Example of inheritance in Python. The class ``\texttt{mRNA}''
    inherits from the class ``\texttt{Gene}'', and has
    two methods, ``\texttt{\_\_init\_\_}'' and ``\texttt{find\_stopcodon}''. ``\texttt{find\_stopcodon}'' loops through a list of stop codons (\texttt{for x in list}):
    For each stop codon, the position is found. If a codon
    wasn't found, the \texttt{.find()} method returns -1, which leads to a
    different print message. Note that \texttt{f""} formats strings
    to include variables.}
\begin{lstlisting}[
    language=Python,
    style=pythonstyle,
    label=lst:inheritance,
    caption=\mycaption,
    ]
class mRNA(Gene):          # Define the mRNA class, inheriting from Gene class
    def __init__(self, sequence: str):   # Define how an mRNA object is created
        super().__init__(sequence)       # Get attributes from parent class
        self.sequence = self.sequence.replace('T', 'U')  # Replace T with Uracil
    def find_stopcodons(self):           # Define how to find stop codons
        for stopcodon in ['UGA','UAA','UAG']:   # Loop over stop codons
            position: int = self.sequence.find(stopcodon)   # Find the position
            if position == -1:           # If position is -1, codon wasn't found
                print(f"{stopcodon} not found") # Message if not found
            else:                        # If position isn't -1, codon was found
                print(f"{stopcodon} found at {position}")         
            
mrna1 = mRNA(sequence='TGAGCTGAGCTGATGCGCTATATTTAGGCG')  # Create an mRNA object
mrna1.find_stopcodons()                  # Show the position of stop codons
# Expected outputs: 
# UGA found at 0
# UAA not found
# UAG found at 24
\end{lstlisting}


Together, Python is not just beginner-friendly, but also well respected for its
ease in development, which is why it is widely used in professional settings for
web development, data analysis, machine learning, biosciences and more
\cite{ekmekciIntroductionProgrammingBioscientists2016,rayhanRisePythonSurvey2023}.


% ======================================================================
\unnsubsection{The Potential of Python Data Science Packages for Biomedicine}%
\label{sec:intro_python_packages}%
%%%
Python includes a vast number of built-in packages used for basic data-types,
software development, simple math operations, etc.,
\cite{PythonLanguageReference2024}. Still, Python relies on packages developed
by its users to provide specialized tools for data analysis. A Python package
consists of multiple Python \emph{modules}, where each module is a text-file
with~a~\texttt{.py} ending containing Python code. Famous examples of such
packages are \texttt{pytorch} and \texttt{tensorflow}, that are used to build
models of artificial intelligence, including \textit{ChatGPT}
\cite{paszkePyTorchImperativeStyle2019, abadiTensorFlowLargeScaleMachine2016,
    radfordLanguageModelsAre2019}. Here, we outlay the most important packages used
for \texttt{plotastic} in Chapter 2 and present examples how these packages are
utilized in modern biomedical research.

\textbf{\texttt{\textbf Interactive Python}:} The standard Python interface is
insufficient for data science, because it lacks the tools to quickly and
conveniently visualize and explore data. \texttt{IPython} can be understood as
an enhanced version of the standard Python interpreter, designed to improve the
interactivity of Python code execution \cite{perezIPythonSystemInteractive2007}.
\texttt{IPython} introduces features like rich media support to display
graphics, but also helps users to use correct python data types through dynamic
type introspection, detecting errors in the code. This functionality is akin to
what \textit{MATLAB} and \textit{RStudio} provide through their advanced
graphical user interfaces and extensive debugging tools. \texttt{IPython} is most
often utilized in the form of \emph{Jupyter Notebooks}.



\textbf{\texttt{Jupyter}:} \texttt{Jupyter} is an evolution of \texttt{IPython},
introducing the \emph{Jupyter notebook} format, which has the
file-ending~\texttt{.ipynb} \cite{kluyverJupyterNotebooksPublishing2016}.
Jupyter Notebooks are documents that combine both code and text structured as
\emph{code cells} and \emph{markdown cells}, respectively. Markdown cells allow
the author to provide additional information with text formatting, for example
structuring the document with headings and subheadings, adding hyperlinks,
images and mathematical formulas. Code cells can be executed individually,
displaying the output directly below the cell. This allows for an interactive
exploration of data, but also makes Jupyter Notebooks a very human-readable
format that outlays data analysis in a clear manner with precise and
reproducible documentation of all data processing steps. Another major benefit
of Jupyter Notebooks are interchangable \textit{Kernels}, allowing the execution
of code in different programming languages, such as R, Julia, and C++
\cite{giorgiLanguageEngineBioinformatics2022}. Today, Jupyter Notebooks have
become a standard format compatible with collaborative platforms like
\textit{Google Colab} and \textit{JupyterLab}, but also professional software
development tools like \textit{VS Code}, and \textit{PyCharm}. For biomedical
research, Jupyter Notebooks are a powerful solution for improving
reproducibility: They elegantly combine both documentation and code execution
into a concise presentation of the data analysis process, hence being an
intuitive tool to both capture and embed computational work directly into
papers, a requirement postulated by
\citet{mesirovAccessibleReproducibleResearch2010}. Jupyter notebooks are
increasingly found in the supplemental of modern publications of both
bioinformatics and wet-lab research
\cite{taskiranCelltypedirectedDesignSynthetic2024,
    bosch-queraltFluorescenceMicroscopybasedProtocol2022,
    howeChallengesOpportunitiesUnderstanding2015}.



\textbf{\texttt{NumPy}:} Central processing units (CPU) usually execute one
instruction on one data point at a time. For manipulating tabular data, this is
inefficient as the same instruction must be repeatedly loaded for every data
point. \texttt{NumPy} accelerates the mathematical capabilities of Python by
enabling large-scale operations on multi-dimensional arrays and matrices with
high efficiency \cite{harrisArrayProgrammingNumPy2020}. One key feature of
\texttt{NumPy} is the implementation of ``vectorization'' or SIMD (Single
Instruction, Multiple Data) instructions. SIMD allows multiple data points to be
processed simultaneously, significantly speeding up operations that are
inherently parallelizable, such as matrix addition or multiplication.
\texttt{NumPy}’s syntax and functional approach to array manipulation have set a
standard for matrix computation, influencing the design of advanced AI
frameworks such as \texttt{PyTorch} and \texttt{mlx}, which mirrors several of
\texttt{NumPy’s} functionalities to facilitate ease of use for those familiar
with \texttt{NumPy} \cite{paszkePyTorchImperativeStyle2019,mlx2023}.
This standardization has made \texttt{NumPy} an attractive tool not only in
genomics \cite{dingPyComplexHeatmapPythonPackage2023}, but also for modern
clinical applications like imaging technologies and augmented-reality in
surgery \cite{thompsonSciKitSurgeryCompactLibraries2020}.


\textbf{\texttt{Pandas}:} Tables are the most common way to store experimental
results. \texttt{Pandas} extends Python with a tabular datatype, called
\texttt{DataFrame}, which allows for easy data manipulation with integrated
indexing \cite{mckinneyPandasFoundationalPython2011}. The intuitive interface of
Pandas can be likened to \textit{Microsoft Excel}; however, it is vastly more
powerful due to its speed, functionality, and ability to handle larger datasets,
e.g. by running efficient \texttt{numpy} vectorization in the background. Unlike
\textit{Excel}, Pandas enables automation by summarizing processing commands
into scripts, documenting each step, and ensuring reproducibility.
\texttt{Pandas} is used in biomedicine for data wrangling, data cleaning, and
data analysis, as it allows for the integration of multiple data sources into a
single table \cite{santosCOVID19ScholarlyProduction2020}.

\textbf{\texttt{matplotlib}:} \texttt{matplotlib} is a plotting library that
provides a wide range of static, animated, and interactive plots and graphs
(\autoref{lst:matplotlib}) \cite{hunterMatplotlib2DGraphics2007}. It serves as the
foundation for many visualization tools and is particularly valued for its
flexibility and customization options. For example, \texttt{Pandas} uses
\texttt{matplotlib} to plot column datapoints directly from a \texttt{DataFrame}
object, creating histograms or scatter plots, which is useful for preliminary
data analysis and checking data distributions. However, \texttt{matplotlib} uses
a low-level syntax, hence plots generated by \texttt{matplotlib} can be
cumbersome to format and customize.


\textbf{\texttt{seaborn}:} While the low-level syntax of \texttt{matplotlib} is
valued for its flexibility, formatting publication grade plots can be laborious,
and its inconsistent syntax can make it difficult to remember the correct
commands for different plot types. \texttt{seaborn} is a high-level interface on
top of \texttt{matplotlib} that offers a more intuitive and highly standardized
syntax across a wide array of plot types
\cite{waskomSeabornStatisticalData2021}. \texttt{seaborn} also integrates
closely with \texttt{Pandas} data structures: It automatically groups
datapoints, calculates measures of both central tendency (e.g. mean, median) and
variance (e.g. standard deviation), and displays them into the plot (e.g. error
bars). This completely replaces manual calculation of descriptive statistics.
\texttt{seaborn} also offers intuitive grouping (\textit{facetting}) of data
points, which simplifies the creation of complex visualizations involving
multidimensional data, making it easier to reveal patterns and relationships via
color encoding, faceting, and automated statistical fits. This is particularly
useful in biomedical research for visualizing and understanding complex
datasets, such as large quantities of protein data
\cite{krzywinskiMultidimensionalData2013,weissVisualizingProteinBig2022}.
\texttt{seaborn} could indirectly contribute to improving reproducibility in
biomedical research by making visualizations of complex data very accessible
through an easy and standardized syntax.



\textbf{\texttt{Pingouin}:} Integrating both data visualization and statistical
analysis is beneficial for researchers who wish to conduct advanced statistical
analysis without switching between different software environments.
\texttt{Pingouin} is designed to be a user-friendly statistical tool that offers
a straightforward syntax for performing statistical tests, which are commonly
implemented in R \cite{vallatPingouinStatisticsPython2018}. Unlike R,
\texttt{Pingouin} integrates seamlessly within the Python ecosystem, which
allows combining data manipulation, analysis, and visualization all in one
platform. This improves reproducibility by reducing the number of software tools
required to analyze data. Despite its potential to streamline the data analysis
process, \texttt{Pingouin} has not been widely adopted by biomedical research,
yet. One example of a study that utilized \texttt{Pingouin} is the work of
\citet{kellyEthicalMatrixMethod2023} in the field of Patient Public Involvement
(PPI), producing an ethical matrix that allows for the inclusion of stakeholder
opinion in medical research design. This lack of \texttt{Pingouin}'s adoption in
biomedicine could be due to recent development and the dominance of R in the
field. However, since Python offers multiple benefits over R in syntax, software
development, runtime performance and integration with other tools (like
including performant C++ code), \texttt{Pingouin} is an attractive standard for
future statistical analyses in biomedicine
\cite{gorelickHighPerformancePython2020}.

\textbf{\texttt{statannotations}:} While \texttt{seaborn} and
\texttt{matplotlib} offer a wide range of statistical plots, they lack the
ability to automatically annotate plots with $p$-values often abreviated by
stars. \texttt{statannotations} is a package that extends \texttt{seaborn} by
providing an interface to add statistical annotations to plots, but also perform
simple statistical tests \cite{charlierTrevismdStatannotationsV02022}. It is
limited to only pairwise comparisons such as t-tests and can not perform omnibus
tests like ANOVA.


Together, these python packages form the backbone of modern data analysis in Python,
often times combining software from different languages to accelerate certain
features, while retaining the ease of use and readability that Python is known
for. This is particularly advantageous in the field of biomedicine, where the
requirements of modern data analysis are often complex and require a high degree
of flexibility and customization.
