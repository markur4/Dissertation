

% From paper 1 (AACR): Multiple myeloma arises from clonal expansion of
% malignant plasma cells in the bone marrow (BM). At diagnosis, myeloma cells
% have disseminated to multiple sites in the skeleton and, in some cases, to
% “virtually any tissue” (1,2). However, the mechanism through which myeloma
% cells initially disseminate remains unclear.

% Dissemination is a multistep process involving invasion, intravasation,
% intravascular arrest, extravasation, and colonization (3). To initiate
% dissemination, myeloma cells overcome adhesion, retention, and dependency on
% the BM microenvironment, which could involve the loss of adhesion factors such
% as CD138 (4,5). BM retention is mediated by multiple factors: First,
% chemokines (CXCL12 and CXCL8) produced by mesenchymal stromal cells (MSCs),
% which attract plasma cells and prime their cytoskeleton and integrins for
% adhesion (6,7). Second, myeloma cells must overcome the anchorage and physical
% boundaries of the extracellular matrix (ECM), consisting of e.g. fibronectin,
% collagens, and proteoglycans such as decorin (8–11). Simultaneously, ECM
% provides signals inducing myeloma cell cycle arrest or progression the cell
% cycle (8,11). ECM is also prone to degradation, which is common in several
% osteotropic cancers, and is the cause of osteolytic bone disease. This is
% driven by a ‘vicious cycle' that maximizes bone destruction by extracting
% growth factors (EGF and TGF-β) that are stored in calcified tissues (12).
% Third, direct contact with MSCs physically anchors myeloma cells to the BM
% (3,13). Fourth, to disseminate to distant sites, myeloma cells require, at
% least partially, independence from essential growth and survival signals
% provided by MSCs in the form of soluble factors or cell adhesion signaling
% (5,14,15). For example, the VLA4 (Myeloma)–VCAM1 (MSC)-interface activates
% NF-$\kappa$B in both myeloma and MSCs, inducing IL-6 expression in MSCs. The
% independence from MSCs is then acquired through autocrine survival signaling
% (16,17). In short, anchorage of myeloma cells to MSCs or ECM is a
% ‘double-edged sword': adhesion counteracts dissemination, but also presents
% signaling cues for growth, survival, and drug resistance (18).

% To address this ambiguity, we developed an in vitro co-culture system modeling
% diverse adhesion modalities to study dissemination, growth, and survival of
% myeloma cells and hMSCs. Co-cultures of hMSCs and the myeloma cell line INA-6
% replicated tight interactions and aggregate growth, akin to "microtumors" in
% Ghobrial's metastasis concept (19). We characterized the growth conformations
% of hMSCs and INA­6 as homotypic aggregation vs. heterotypic hMSC adherence and
% their effects on myeloma cell survival. We tracked INA­6 detachments from
% aggregates and hMSCs, thereby identifying a potential "disseminated"
% subpopulation lacking strong adhesion. We developed innovative techniques
% (V-well adhesion assay and well plate sandwich centrifugation) to separate
% weakly and strongly adherent subpopulations for the subsequent analysis of
% differential gene expression and cell survival. Notably, our strategy resolves
% the differences in gene expression and growth behavior between cells of one
% cell population in "direct" contact with MSCs. In contrast, previous methods
% differentiated between "direct" and "indirect" cell-cell contact using
% transwell inserts (20). To evaluate whether genes mediating adhesion and
% growth characteristics of INA-6 were associated with patient survival, we
% analyzed publicly available datasets (21,22).






% ======================================================================
% ======================================================================
\unnsubsection{Multiple Myeloma and Other Monoclonal Gammopathies}%
\label{sec:intro_myeloma}%
\ac{MM} is a hematological malignancy characterized by the
clonal expansion of malignant plasma cells primarily within multiple sites of
the \ac{BM}. This cancer arises from plasma cells, the
antibody-producing cells of the immune system, which undergo malignant
transformation resulting in uncontrolled growth and disruption of normal bone
marrow function \cite{yangPathogenesisTreatmentMultiple2022}. The prevalence of
multiple myeloma has tripled across both europe and USA from 1980 to 2014 due to
an ageing population \cite{ociasTrendsHematologicalCancer2016,
    turessonRapidlyChangingMyeloma2018}. For 2024, 35780 new \ac{MM} cases and 12540
deaths are estimated for the USA alone \cite{siegelCancerStatistics20242024}.

To understand the progression of a healthy plasma cell to \ac{MM}, one can
review other \emph{monoclonal gammopathies}. These are defined by the presence
of monoclonal immunoglobulin in the blood serum which is indicative of abnormal
plasma cell clones overexpressing the same type of dysfunctional antibody.
\cite{kyleMonoclonalGammopathyUndetermined1997,
    fermandMonoclonalGammopathyClinical2018}. When no further disease manifestations
are present, the condition is termed \acfi{MGUS}, which is the most commonly
diagnosed monoclonal gammopathy \cite{kyleMonoclonalGammopathyUndetermined1997}.
\ac{MGUS} has a 1-\SI{5}{\percent} annual risk of progression to \ac{MM}
\cite{rajkumarInternationalMyelomaWorking2014}.

To distinguish \ac{MM} from other monoclonal gammopathies, diagnosis of \ac{MM}
requires not only identification of a minimum of clonal plasma cells, but also a
\emph{myeloma defining event} which is evidence of malignancy or end-organ
damage, such as hypercalcemia, renal insufficiency, anemia, or bone lesions
\cite{rajkumarInternationalMyelomaWorking2014}. A localized
smaller\footquote{Solitary plasmacytoma with \SI{10}{\percent} or more
    clonal plasma cells is regarded as multiple myeloma. [...] If bone marrow has
    less than \SI{10}{\percent} clonal plasma cells, more than one bone lesion is
    required to distinguish [\ac{MM}] from solitary plasmacytoma with minimal marrow
    involvement.}{rajkumarInternationalMyelomaWorking2014} mass of clonal plasma
cells together with a singular primary bone lesion is diagnosed as \ac{SP}.
\ac{SP} can progress to \ac{MM} in \SI{32}{\percent} of cases with a median
follow-up of 9.7 years
\cite{thumallapallySolitaryPlasmacytomaPopulationbased2017,
    gaoSolitaryBonePlasmacytoma2024}. Studies from
\citet{kyleMonoclonalGammopathyUndetermined1997} show that \ac{SP} cases are
rare, constituting only \SI{2.5}{\percent} of monoclonal gammopathy diagnoses,
whereas \ac{MM} represent \SI{18}{\percent}. Another rare precursor of \ac{MM}
is \acfi{aMM} \dashed{also termed \acfi{sMM}}, representing \SI{3}{\percent} of
monoclonal gammopathies \cite{kyleMonoclonalGammopathyUndetermined1997}.
\ac{aMM} is diagnosed when no myeloma defining event is detected, although the
quantities of clonal plasma cells or monoclonal protein align with respective
criteria for \ac{MM} diagnosis \cite{rajkumarInternationalMyelomaWorking2014}.
Recent reports show that if left untreated, \SI{72}{\percent} of \ac{aMM}
patients progress to \ac{MM}, whereas early treatment can lower the progression
rate to \SI{11}{\percent} within up to 7.6 years until last
follow-up\footdagger{For non-high risk \ac{aMM} patients, treatment lowered
    \ac{MM} progression rate to \SI{9}{\percent}, compared to \SI{31}{\percent} for
    untreated patients (within up to 6.7 and 7.6 years of follow-up, respectively).
    For high-risk \ac{aMM} patiens, treatment lowered \ac{aMM} progression rate to
    \SI{11}{\percent}, compared to \SI{72}{\percent} for untreated patients (within
    up to 5.2 years of follow-up and median time to progression of 2.2 years,
    respectively) \cite{abdallahModeProgressionSmoldering2024}.}
\cite{abdallahModeProgressionSmoldering2024,
    mateosmaria-victoriaLenalidomideDexamethasoneHighRisk2013}. \ac{MM} itself can
progress to advanced stages, such as \emph{extramedullary involvement/disease}
which describes colonization of soft tissues outside the bone marrow
\cite{bladeExtramedullaryDiseaseMultiple2022}, but also \ac{PCL} which is
characterized by high levels of circulating plasma cells
\cite{jungUpdatePrimaryPlasma2022}. However, the most common cause of death is
renal failure during the \ac{MM} stage, caused by excess immunoglobulins or
hypercalcemia due to bone degradation \cite{kunduMultipleMyelomaRenal2022}.

With a 5-year surival rate of \SI{50}{\percent}
\cite{turessonRapidlyChangingMyeloma2018}, \ac{MM} can be considered incurable
and deadly. \ac{MM} relapses within the first year in \SI{16}{\percent} of
patients [\ac{MMR}], others face relapse at a later time or only continued
response to treatment \cite{majithiaEarlyRelapseFollowing2016}. Although
treatments have improved, the age-adjusted mortality rate of \ac{MM} has
decreased from 1999 to 2020 by only \SI{-1.6}{\percent}
\cite{doddiDisparitiesMultipleMyeloma2024}.
\citet{engelhardtFunctionalCureLongterm2024} describes the current standard care
for transplant-eligible \ac{NDMM} patients as follows: Induction
with a CD38 antibody, proteasome inhibitor, immunomodulatory drug, and
dexamethasone, potentially followed by bone marrow transplantation and
lenalidomide maintenance \cite{rajkumarMultipleMyelomaCurrent2020}. A major
challenge to these treatments is the continued cycle of remission and relapse,
with each relapse generally being harder to treat
\cite{podarRelapsedRefractoryMultiple2021}. Development of such resistance is
well described in the literature, often arising from the intraclonal genetic
heterogeneity within the myeloma cell population and the protective niche
provided by the \ac{BMME}
\cite{solimandoDrugResistanceMultiple2022}.



% ======================================================================
% ======================================================================
\unnsubsection{Dissemination of Myeloma Cells}%
\label{sec:intro_myeloma_dissemination}%
As the name suggests, \emph{multiple} myeloma involves spreading of clonal
plasma cells in multiple sites within the body, a process that's described with
the term \emph{dissemination}. Although a single large plasmacytoma is still
classified as \ac{MM} \cite{rajkumarInternationalMyelomaWorking2014}, the
presence of multiple tumor lesions within the \ac{BM} is very common. More than
one or 25 such lesions predict poor prognosis for asymptomatic and symptomatic
\ac{MM} patients, respectively \cite{kastritisPrognosticImportancePresence2014,
    maiMagneticResonanceImagingbased2015a}. Additionally, \ac{MM} cells can
disseminate to extramedullary sites of virtually any tissue, highlighting
\ac{MM} as a systemic disease with potential multi-organ impact
\cite{rajkumarMultipleMyelomaCurrent2020,
    bladeExtramedullaryDiseaseMultiple2022}. Hence, dissemination is a major
contributor to \ac{MM} progression and poor prognosis, enabling \ac{MM} cells to
colonize new niches that favor survival, quiescent states or are less accessible
for therapy, especially with high subclonal heterogeneity
\cite{forsterMolecularImpactTumor2022, keatsClonalCompetitionAlternating2012}.

Dissemination of \ac{MM} is reminiscent of \emph{metastasis}, a
term typically associated with solid tumors describing the spread of cancer
cells to distant sites. However, it substantially differs from metastasis due to
the hematological or ``liquid'' nature of \ac{MM}. Long-lived plasma cells originate
from migratory B-cells, negating the need for extensive transformative processes
such as \ac{EMT}, which is required for
escaping tightly connected solid tissues to enter the bloodstream
\cite{ribattiEpithelialMesenchymalTransitionCancer2020}. Although referred to as
``liquid tumor'', \ac{MM} cells still accumulate as distinct foci within the bone
marrow, somewhat mirroring the localized growth of solid tumors. This
characteristic has led to \ac{MM} being proposed as a model for studying solid
``micrometastases'' \cite{ghobrialMyelomaModelProcess2012}, highlighting its
unique blend of liquid and solid tumor properties and providing insights into
the mechanisms of cancer dissemination and colonization of new niches.

The exact mechanism of \ac{MM} dissemination is not entirely understood.
Nevertheless, attempts to structure this process have been made by
\citet{zeissigTumourDisseminationMultiple2020}, describing \ac{MM} dissemination in
five steps:
\begin{enumerate}
    \item Retention in the \ac{BM}
    \item Release from the \ac{BM}
    \item Intravasation
    \item Extravasation
    \item Colonization.
\end{enumerate}

% \emph{Retention in the \ac{BM}}, \emph{release from the \ac{BM}},
% \emph{intravasation}, \emph{extravasation}, and \emph{colonization}. 

According to \citet{zeissigTumourDisseminationMultiple2020}, \ac{MM}
dissemination begins with \ac{MM} cells overcoming retention and adhesion within
the \ac{BMME}. Following release, \ac{MM} cells undergo \emph{intravasation}
into the bloodstream, where they can circulate before extravasating into new
\ac{BM} sites. This migration is directed by chemokines and growth factors
produced by \ac{BM} cells. For instance, CXCL12 and IGF-1 are critical in
guiding \ac{MM} cells back to the \ac{BM}, a process called \emph{homing}
\cite{vandebroekExtravasationHomingMechanisms2008}. In the \ac{BM} they can
\emph{colonize} and form new tumor foci.

The review by \citet{zeissigTumourDisseminationMultiple2020} implies a
sequential order of such steps, yet direct proof of this is lacking. Still, the
review provides a framework that integrates multiple complex research topics
into one coherent context. For instance,
\citet{zeissigTumourDisseminationMultiple2020} states that two adhesive
processes are critical for succesful dissemination: Lowered adhesion to the \ac{BM},
but increased adhesion to the endothelium to initiate extravasation
\cite{asosinghUniquePathwayHoming2001a,
    mrozikTherapeuticTargetingNcadherin2015}. This alone implies stringent
separation of different adhesive processes during the dissemination process.
Given that \acp{CAM} have become attractive targets for
treating \ac{MM} \cite{bouzerdanAdhesionMoleculesMultiple2022,
    katzAdhesionMoleculesLifelines2010}, such detailed understanding of cell
adhesion is crucial for developing successful therapies.


% ======================================================================
% ======================================================================
\unnsubsection{Retention of Myeloma Cells in the Bone Marrow}%
\label{sec:intro_myeloma_retention}%
According to \citet{zeissigTumourDisseminationMultiple2020}, overcoming
retention and adhesion to the BME is critical to \ac{MM} dissemination. Retention of
plasma cells to the BME is mediated by multiple mechanisms, which are
categorized here into \emph{direct adhesion}, \emph{soluble survival factors}
and \emph{chemotaxis}. A fourth notable mechanism is the physical boundary that
is bone tissue and \ac{ECM}, which could become important for
\ac{MM} dissemination once degradation of bone tissue has progressed.

\emph{Direct adhesion} of \ac{MM} cells to the \ac{BM} is mediated through
\ac{ECM} components and cell adhesion to other \ac{BM} resident cells like
osteoblasts, osteoclasts and \acp{BMSC} \cite{teohINTERACTIONTUMORHOST1997,
    bouzerdanAdhesionMoleculesMultiple2022}. \ac{ECM} components include
fibronectin, collagens, and proteoglycans such as decorin
\cite{huDecorinmediatedSuppressionTumorigenesis2021,
    huangHigherDecorinLevels2015, katzAdhesionMoleculesLifelines2010,
    kiblerAdhesiveInteractionsHuman1998}. \acp{BMSC} are vital in this niche,
supporting cell adhesion through \acp{CAM} but also by secretion of (\ac{ECM})
components \cite{katzAdhesionMoleculesLifelines2010}. Such adhesion acts both as
physical anchorage but also provides signaling cues for growth, survival, and
drug resistance \cite{chenContributionBoneMarrow2020}. A classic example is the
binding of \ac{MM} cell integrins to VCAM-1 on \acp{BMSC}, such as $\alpha4\beta1$ (VLA-4)
\cite{bouzerdanAdhesionMoleculesMultiple2022}. Since direct adhesion promotes
both retention and tumour growth, it could play an ambiguous role during \ac{MM}
dissemination.

\emph{Soluble survival factors} contribute to \ac{BM} retention, since plasma
cells can not survive outside the bone marrow without them. For example,
deleting BCMA \dashed{a receptor for survival factors} leads to loss of
\acp{BMPC}  due to unsustained maintenance of cell survival
\cite{oconnorBCMAEssentialSurvival2004}. Soluble survival factors include IL-6,\,
IGF-1,\,BAFF,\,APRIL,\,and\,VEGF, although IGF-1 has proven to be the primary
survival factor \cite{sprynskiRoleIGF1Major2009}. These signals are secreted by
\acp{BMSC} and adipocytes \cite{kiblerAdhesiveInteractionsHuman1998,
    garcia-ortizRoleTumorMicroenvironment2021}. \emph{Chemotaxis} is also crucial
for \ac{BM} retention \cite{ullahRoleCXCR4Multiple2019}. CXCL12 and CXCL8 are
soluble chemotactic signals produced by \acp{BMSC} and attract \ac{MM} cells,
but also primes their cytoskeleton and integrins for adhesion
\cite{aggarwalChemokinesMultipleMyeloma2006,
    alsayedMechanismsRegulationCXCR42007}.
\citet{roccaroCXCR4RegulatesExtraMedullary2015} demonstrated that inhibiting
CXC4R \dashed{the receptor of CXCL12/SDF-1$\alpha$} with the monoclonal antibody
Ulocuplumab decreased \ac{MM} cell homing to the \ac{BM} and reduced
dissemination via factors associated with \ac{EMT}. Consequently, CXCL12:CXCR4
signaling has emerged as a promising target of several treatments aiming at
preventing myeloma cell migration, as reviewed by
\citet{itoRoleTherapeuticTargeting2021} who conclude a need for further studies
to develop combined therapies explicitly against tumor cell dissemination.

Together, \acp{BMSC} play a critical role in \ac{MM} retention, providing direct
adhesion, soluble survival factors, and chemotactic signals.


% ======================================================================
% ======================================================================
\unnsubsection{Release of Myeloma Cells from the Bone Marrow}%
\label{sec:intro_myeloma_release}%
\citet{zeissigTumourDisseminationMultiple2020} describes the release of myeloma
cells from the \ac{BMME} as all steps required for overcoming bone marrow
retention, but also putative triggers leading to migration out of the BME. To
the author's knowledge, release of \ac{MM} cells is the least understood among
disseminative processes. Still, in order to gain understanding of how \ac{MM}
dissemination is initiated, one can summarize reports that could be involved.


Studies suggest that \acp{CAM} expression indeed plays a role in \ac{MM}
dissemination. For instance, studies demonstrate that circulating \ac{MM} cells
exhibit reduced levels of integrin $\alpha4\beta1$, in contrast to those located
in the \ac{BM} \cite{paivaDetailedCharacterizationMultiple2013,
    paivaCompetitionClonalPlasma2011}. Given that dissemination can be induced in
mice by overexpressing the \ac{CAM} shedding protease heparanase
\cite{yangHeparanasePromotesSpontaneous2005}, it seems reasonable that dynamic
loss of adhesive strength is causing release of \ac{MM} cells from the \ac{BM}.
Another useful comparison is that of \acp{CAM} expression at different disease
stages, indicative of their role in disease progression, which in turn could
serve as a proxy for dissemative potential. For example,
\citet{terposIncreasedCirculatingVCAM12016}
reported an increase in adhesion molecule expression of ICAM-1 and VCAM-1 in
patients with \ac{MM} compared to those with \ac{MGUS} and \ac{aMM}. However,
\citet{perez-andresClonalPlasmaCells2005} reported that CD40 is downregulated in
\ac{PCL} patients, hence, different \acp{CAM} could serve ambiguous roles in
\ac{MM} progression. Together, the regulation of \acp{CAM} can depend on both
momentary microenvironmental factors, but also on the stage of the disease,
while the specific stimuli regulating their expression are not fully defined. A
study from \citet{akhmetzyanovaDynamicCD138Surface2020} presents CD138
(\textit{aka} Syndecan-1) as a potential \textit{switch} between adhesion and
release, as mice treated with CD138 blocking antibodies exhibited
rapid mobilization of \ac{MM} cells from the \ac{BM} to peripheral blood,
confirming that alterations in adhesion molecule expression are sufficient to
cause \ac{MM} cell release. \citet{brandlJunctionalAdhesionMolecule2022}
builds on that finding, showing that JAM-C inversely correlates with CD138
expression while promoting \ac{MM} progression in a mouse model.


Another often overlooked requirement for \ac{MM} cell release is the need for
independence from essential growth and survival signals provided by \ac{BMSC}.
Autocrine signaling has been proposed as a key mechanism through which myeloma
cells gain independence from essential survival factors such as IL-6
\cite{frassanitoAutocrineInterleukin6Production2001,
    urashimaCD40LigandTriggered1995}. Autocrine signaling could also disrupt
responsiveness to \ac{MSC}-derived chemotactic signals, since \ac{MM} cells from
\ac{BM} biopsies were shown to express CXCL12 under hypoxic conditions induced
by HIF-2$\alpha$ \cite{martinHypoxiainducibleFactor2Novel2010}. Since \ac{MM}
niches turn increasingly hypoxic and circulating myeloma cells upregulate
hypoxia associated genes, hypoxia is a promising factor for understanding the
release of \ac{MM} \cite{garcesTranscriptionalProfilingCirculating2020}.



The degradation of bone tissue could also play a critical role in myeloma cell
release by eliminating adhesive anchorage within the ECM, considering that
\ac{ECM} is remodeled even at the \ac{MGUS} stage
\citet{glaveyProteomicCharacterizationHuman2017}. This degradation is part of a
‘vicious cycle' that is well described in osteotropic cancer types and is the
key pathway of bone destruction, dissolving bone-resident growth factors like
TGF-$\beta$ that further drive tumor growth
\cite{haradaMyelomaBoneInteraction2021, siclariMolecularInteractionsBreast2007,
    wangProstateCancerPromotes2019}. Notably, it is reasonable to assume that bone
destruction drives dissemination by removing physical barriers, yet such concept
was not proven yet.


In summary, the release of \ac{MM} cells from the \ac{BM} is a complex process
that involves dynamic regulation of \acp{CAM}, autocrine signaling, and hypoxia,
but also the degradation of bone tissue. These processes are not fully
understood and require further investigation to formulate strategies that
prevent uncontrolled spread of \ac{MM} cells and support modern therapies.







% ======================================================================
% ======================================================================
\unnsubsection{MSCs: Mesenchymal Stromal (Stem) Cells}%
\label{sec:intro_hMSCs}%
The previous sections mentioned \acp{MSC} several times as \acp{BMSC}, being a
crucial component of the \ac{BMME} in the context of multiple myeloma
\cite{mangoliniBoneMarrowStromal2020}. Before discussing their role in \ac{MM}
specifically, it is important to understand what an \ac{MSC} is and what impact
this cell type has on biomedical research.

Explaining what an \ac{MSC} is, can be challenging. MSCs
are derived from multiple different sources, serve a wide array of functions and
are always isolated as a heterogenous group of cells. This makes it particularly
challenging to find a consensus on their exact definition, nomenclature, exact
function and \textit{in vivo} differentiation potential. Therefore, the
following paragraphs provide a brief overview of the biology of MSCs set within
a historical context.

\acp{MSC} first gained popularity as a stem cell. Stem cells lay the foundation
of multicellular organisms. Embryonic stem cells orchestrate the growth and
patterning during embryonic development, while adult stem cells are responsible
for regeneration during adulthood. The classical definition of a stem cell is
that of a relatively undifferentiated cell that divides asymmetrically,
generating one daughter cell with maintained stemness, and one differentiated
daughter cell \cite{cooperCellProliferationDevelopment2000,
    shenghuiMechanismsStemCell2009}. Because of their significance in biology and
regenerative medicine, stem cells have become a prominent subject in modern
research. \acp{hMSC} have been presented as promising candidate in the context
of regeneration, given that they feature also intriguing immunomodulatory
capabilities, easy isolation and \textit{in vitro} expansion, and safety for
both autologous and allogeneic transplantation
\cite{ullahHumanMesenchymalStem2015}.


\emph{Mesenchyme} is a type of tissue first appearing during embryonic
development \dashleft{spefically during gastrulation}. It emerges from
epithelial cells after prior committment to a mesodermal fate, followed by a
loss of cell junctions and free migration away from the epithelial layer. This
process is called \acf{EMT}, a process often exploited by cancer cells to
metastasize \cite{tamFormationMesodermalTissues1987,
nowotschinCellularDynamicsEarly2010}. Hence, the term mesenchyme describes
non-epithelial embryonic tissue differentiating into mesodermal lineages such as
bone, muscles and blood. Interestingly, it was shown nearly twenty years earlier
that cells within adult bone marrow seemed to have mesenchymal properties as
they were able to differentiate into bone tissue
\cite{friedensteinOsteogenesisTransplantsBone1966,
friedensteinOsteogenicPrecursorCells1971, biancoMesenchymalStemCells2014}. This
was the origin of the \emph{``mesengenic process''}-hypothesis: This concept
states that mesenchymal stem cells serve as progenitors for multiple mesodermal
tissues (bone, cartilage, muscle, marrow stroma, tendon, fat, dermis and
connective tissue) during both adulthood and embryonic
development~\cite{caplanMesenchymalStemCells1991,caplanMesengenicProcess1994}.
The mesenchymal nature of these cells (termed bone marrow stromal cells:
\acp{BMSC}) was confirmed later when they were shown to differentiate into
adipocytic (fat) and chondrocytic (cartilage)
lineages~\cite{pittengerMultilineagePotentialAdult1999}. Since then, the term
\emph{``mesenchymal stem cell''} (MSC) has grown popular as an adult multipotent
precursor to a couple of mesodermal tissues. \acp{MSC} derived from bone marrow
(\acp{BMSC}) were shown to differentiate into osteocytes, chondrocytes,
adipocytes and cardiomyocytes \cite{gronthosSTRO1FractionAdult1994,
muruganandanAdipocyteDifferentiationBone2009, xuMesenchymalStemCells2004}. Most
impressively, these cells also exhibited ectodermal and endodermal
differentiation potential, as they produced neuronal cells, pancreatic cells and
hepatocytes \cite{barzilayLentiviralDeliveryLMX1a2009,
wilkinsHumanBoneMarrowderived2009, gabrInsulinproducingCellsAdult2013,
stockHumanBoneMarrow2014}.

It was later established that cultures with MSC-like properties can be isolated
from \textit{`virtually every post-natal organs and tissues'}, and not just bone
marrow \cite{dasilvameirellesMesenchymalStemCells2006}. However, depending on
which tissue they originated from, \acp{MSC} can differ greatly in their
transcription profile and \textit{in vivo} differentiation potential
\cite{jansenFunctionalDifferencesMesenchymal2010,
    sacchettiNoIdenticalMesenchymal2016}.

Since \acp{MSC} are a heterogenous group of cells, they were defined by their
\textit{in vitro} characteristics. A minimal set of criteria are the following
\cite{dominiciMinimalCriteriaDefining2006}: First, \acp{MSC} must be plastic
adherent. Second, they must express or lack a set of specific surface antigens
(positive for CD73, CD90, CD105; negative for CD45, CD34, CD11b, CD19). Third,
\acp{MSC} must differentiate to osteoblasts, adipocytes and chondroblasts
\textit{in vitro}. Together, \acp{MSC} exhibit diverse differentiation
potentials and can be isolated from multiple sources of the body.

Today, the potential in \acp{MSC} lies not their stemness, but rather in their
immunomodulatory capabilities, which could be the reason why conventionally the
\emph{`S'} in \ac{MSC} stands for \emph{Stromal} instead of \emph{Stem}.
Althogh, \acp{MSC} are not yet established in routine clinical practice
\dashed{despite thousands of clinical trials covering most of the human
    body's organs}, they still are among the most studied cell types and are topic
of vast research \cite{abdelrazikMesenchymalStemCells2023}. \acp{MSC} are valued
for their very high treatment tolerance, but also as an adaptive platform for
modifications to improve their therapeutic effects
\cite{dsouzaMesenchymalStemStromal2015}. For example,
\citet{chenTreatmentIschemicStroke2022} boldly announced the translation of
modified \acp{MSC} into the clinical practice of treating ischemic strokes,
whereas \citet{monivasgallegoMesenchymalStemCell2024} conclude that
\emph{further studies are needed}, a statement that's ubiquitous in publications
on \acp{MSC} based therapies\footquote{Altogether, the articles published in
    this Special Issue raise more questions than they answer, given that most of the
    conclusions carry the statement \textbf{‘further studies are
        needed'}.}{abdelrazikMesenchymalStemCells2023}. Still, many fields of research
benefit from this vast general understanding of \acp{MSC} biology, including
this work and the study of the \ac{BMME} in the context of cancer pathologies
such as \ac{MM}.






% ======================================================================
% ======================================================================
\unnsubsection{Molecular Interactions between MSC­s Myeloma Cells}
\label{sec:intro_myeloma_hMSC_interactions}
As mentioned in previous sections, \acp{MSC} are key drivers of \ac{MM}
progression through mediating retention and survival of \ac{MM} cells in the
bone marrow through e.g. cell adhesion and chemoattractaction
\cite{zeissigTumourDisseminationMultiple2020}. However, the impact of \ac{MM}
cells on \acp{BMSC} is a lot more far-reaching, influencing cell cycle
progression, immune response and bone metabolism
\cite{dotterweichContactMyelomaCells2016,
    fernandoTranscriptomeAnalysisMesenchymal2019}. Hence, \acp{BMSC} play a complex
role in the overall pathology of \ac{MM} \cite{mangoliniBoneMarrowStromal2020}.
Since bone tissue represents a sturdy physical barrier that \ac{MM} cells might
have to overcome to disseminate, it is crucial to revisit the impact that
myeloma cells leave on the \ac{BMME}.

In healthy bone tissue, there's an equilibrium between bone formation and
degradation to maintain turnover, repair and remodelling.
\cite{vaananenMechanismBoneTurnover1993}: Mesenchymal stromal cells (BMSCs)
differentiate into bone forming osteoblasts, while bone degrading osteoclasts
are derived from hematopoietic stem cells. Myeloma cells shift the equilibrium
of bone turnover towards degradation, leading to \ac{MBD}
\cite{hideshimaUnderstandingMultipleMyeloma2007}. MBD is present in 80\% of
patients at diagnosis and is characterized by osteolytic lesions, osteopenia and
pathological fractures \cite{terposPathogenesisBoneDisease2018}.


\ac{MM} cells establish this shift in bone turnover on multiple levels: They
directly stimulate osteoclast activity by secreting MIP1$\alpha$
\cite{obaMIP1alphaUtilizesBoth2005}, but also indirectly through reprogramming
\acp{BMSC} by having them produce osteoclast stimulating factor RANKL
\cite{tsubakiHGFMetNFkB2020}. This is mediated by NF-$\kappa$B signaling, a
pathway that's crucial for MM pathology which is activated through direct
cell-cell contact between myeloma cells and \acp{BMSC} via e.g. VCAM1
\cite{cippitelliRoleNFkBSignaling2023, royNFkBActivatingPathways2018}.
NF-$\kappa$B is activated in both myeloma cells and \acp{BMSC}, but with
different outcomes: In myeloma cells, NF-$\kappa$B transduces survival signaling
and is also a key driver of cell adhesion mediated drug resistance
\cite{royNoncanonicalNFkBMutations2017, landowskiCellAdhesionmediatedDrug2003}.
\acp{BMSC} however react with stress-induced senescence and secretion of
multiple factors that drive \ac{MM} pathology, such as RANKL and components of
the \ac{SASP} \cite{chauhanMultipleMyelomaCell1996,
    fairfieldMultipleMyelomaCells2020}.


Another fundamental factor contributing to bone destruction is the suppresion of
osteogenic differentiation of \acp{BMSC} by myeloma cells: \ac{MM} cells secrete
DKK1, which inhibits Wnt signaling that otherwise induces the key osteogenic
transcription factor RUNX2  \cite{gaurCanonicalWNTSignaling2005,
    qiangDkk1inducedInhibitionWnt2008, zhouDickkopf1KeyRegulator2013}. Instead,
\acp{BMSC} are driven towards the adipogenic lineage \ac{MM} by VCAM1 signaling,
with MM cells stabilizing the adipogenic transcription factor PPAR-$\gamma$
\cite{dotterweichContactMyelomaCells2016, liuMyelomaCellsShift2020}.
Furthermore, \ac{MM} inhibit osteogenic differentiation of \acp{BMSC} on an
epigenetic level \cite{allegraEpigeneticCrosstalkMalignant2022}. Key mediator
GFI1, a zinc finger protein activated by TNF-$\alpha$ which recruits the
chromatin modifier HDAC1 to the promoter of RUNX2
\cite{dsouzaGfi1ExpressedBone2011, adamikEZH2HDAC1Inhibition2017}. Intriguingly,
the same group was able to prevent activation of GFI1 and rescue osteogenic
differentiation by inhibition of p62, an adapter protein involved in autophagy
that bridges several signaling pathways, including NF-$\kappa$B
\cite{adamikXRK3F2InhibitionP62ZZ2018}. \citet{teramachiBlockingZZDomain2016}
previously treated mice with the same inhibitor XRK3F2, which resulted in bone
formation restricted to MM containing bones. This approach is intriguing, since
NF-$\kappa$B has remained an untreatable target given its ubiquitous role in
cell survival. However, a review by \citet{verzellaNFkBPharmacopeiaNovel2022}
clearly disagrees with this notion\footquote{Collectively, this bulk of
    evidence demonstrates that the safe and cancer-selective inhibition of the
    NF-$\kappa$B pathway is clinically achievable and promises profound benefit to
    patients with NF-$\kappa$B-driven cancers}{verzellaNFkBPharmacopeiaNovel2022},
shifting the focus onto upstream activators and downstream effectors of
NF-$\kappa$B signaling, a principle that was exemplified by the work of
\citet{adamikXRK3F2InhibitionP62ZZ2018} on the GFI1/p62 interface. This
highlights the need for a thorough characterization of the transcriptional
programs elicited by NF-$\kappa$B signaling in \ac{MM} cells and \acp{BMSC},
especially during direct cell adhesion.


Overall, \acp{MSC} are pivotal within the \ac{BMME}, serving both structural and
responsive roles in \ac{MM}. NF-$\kappa$B signaling, essential in MSC-myeloma
interactions, plays a crucial role in myeloma survival and MSC function
modification \cite{cippitelliRoleNFkBSignaling2023,
    royNFkBActivatingPathways2018}. Direct contact is necessary to activate these
pathways, exacerbating \ac{MM} pathology. Targeting components of NF-$\kappa$B
signaling offers therapeutic potential, aiming to disrupt key interactions and
inhibit myeloma progression. Deeper molecular understanding of the MSC--Myeloma
intreactions could lead to therapies that effectively target a cancer
microenvironment that's driving \ac{MM} progression.


% ======================================================================
% ======================================================================






