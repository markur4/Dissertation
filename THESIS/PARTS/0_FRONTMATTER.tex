


% ======================================================================
% == Institue and Project Period
% ======================================================================
\vspace*{\fill} % > Shift it to bottom

\thispagestyle{empty} % > Don't apply pagestyles to this page

\begin{center}
    % \noindent
    A digital version of this thesis will be made available at:\\
    \url{https://github.com/markur4/Dissertation}

    The \LaTeX~source code is found in this subfolder:\\
    \texttt{./THESIS}

    Videos are found in this subfolder:\\
    \texttt{./publications/Chapter1 Supplementary/}

    \vspace{\vdouble}

    % \noindent
    This work was conducted at the Department of Musculoskeletal Tissue
    Regeneration (Bernhard-Heine-Centre for Locomotive Research), University
    of Würzburg from 08.10.2018 to
    \DTMtoday~ under the supervision of Prof Dr.\,rer.\,nat.~Regina Ebert.
    % 31.03.2024 
\end{center}




% ======================================================================
% == Acknowledgements
% ======================================================================

\newpage
\thispagestyle{empty} % > Don't apply pagestyles to this page

\section*{Acknowledgements} % > The * means it's not listed in TOC
 % \vspace{-\vhalf}
 % \begin{center}%
 {
  \small%
  { \centering    I am deeply grateful to the following supporters who have
      contributed to the completion of this PhD thesis in myriad ways:}

  \vspace{\vhalf}

  \acknowledge{Doris Schneider}{%
      for her exceptional assistance in laboratory work. Her patience, focus, and
      dedication were invaluable to the experimental success of this
      research.%
  }

  \acknowledge{Regina Ebert}{%
      for her unwavering support and insightful guidance. Her patience and
      encouragement were instrumental in making \texttt{plotastic} a reality,
      and her feedback significantly improved the quality of this thesis.%
  }

  \acknowledge{Franziska Jundt \& Torsten Blunk}{%
      for their invaluable supervision of this project. Their professional
      feedback and honest critiques pushed me to strive for the highest quality in
      my work.%
  }

  \acknowledge{Marie-Nicole Kobsar}{%
      my beloved significant other, for her endless patience, kindness, and
      support throughout the entire doctoral journey. Her encouragement and
      understanding were a source of strength and motivation.%
  }

  \acknowledge{My Lab Colleagues \& Collaborators}{%
      for their countless contributions, insights and constructive feedback
      throughout my doctorate.
      % Special thanks to Melanie Krug, Jutta Meißner-Weigl, Sabine Zeck, Marietta
      % Herrman, Sigrid Müller-Deubert, Wyonna Rindt, Franz Jakob, Denitsa
      % Docheva, Drenka Trivanović, Ellen Leich, Franziska Jundt, Dirk Hose,
      % Tanja Nicole Hartmann, Torsten Blunk, and Maximilian Rudert.%
  }

  \acknowledge{My Family}{%
      for their unconditional support, encouragement, and feedback throughout
      my academic journey. Their belief in me provided the foundation upon
      which this achievement was built.%
  }

  
  \noindent Additionally, I wish to extend my gratitude to all my friends who
  listened, discussed, and helped in various ways.
  
  \vspace{\vhalf}

  \noindent I would also like to acknowledge the funding bodies and
  institutions that supported this research, including the German Research
  Foundation (DFG), the Graduate School of Life Sciences (GSLS), and the University of
  Würzburg, whose contributions made this work possible.

  \noindent Finally, a special thanks goes to the Elite Network Bavaria (ENB)
  and the University of Bayreuth for my education in molecular cell biology
  and biological physics. I felt well-prepared for my doctorate at the
  University of Würzburg, and am very grateful for the opportunities for
  interdisciplinary exchange with amazing physicists and other scientists.

  \begin{center}
      {\textbf{Thank you all for your incredible support and encouragement.}}
  \end{center}
 }
% \end{center}