

% ======================================================================
\unnsubsection{\textit{\textbf{Overall Conclusion}}}%
\label{sec:discussion_overall_conclusion}%

% \textbf{Time as a Key Parameter:}
% The area Thermodynamics of started with scientists measuring how long it takes
% for gases to cool down. The author claims, by measuring the time it takes for
% cancer cells to detach could lead to breakthroughs in research of myeloma
% dissemination.

% Time also follows this thesis like a red thread, in the form of handling it
% as another dimension during data analysis, the microscopic time-dependent
% changes in cell adhesion and detachment events, or the highly dynamic
% interactions proposed in the steps of multiple myeloma dissemination. 

% - ECM secretion



This work demonstrates the transformative potential of integrating
semi-automated tools and advanced imaging techniques in the study of cancer cell
biology, specifically multiple myeloma. The employment of tools such as
\texttt{seaborn} and \texttt{plotastic} for data analysis, alongside innovative
microscopy methods like live-cell imaging and image cytometry, has enabled a
deeper understanding of the complex behaviors and interactions of myeloma cells
within the \acf{BMME}. These methodologies have revealed the adaptive nature of
myeloma cells, characterized by dynamic adhesion capabilities that could
facilitate their survival and dissemination across different niches. This
plasticity is crucial for the progression of multiple myeloma and represents a
potential target for therapeutic interventions.

By automating complex and time-intensive processes, this thesis not only
enhances the efficiency and accuracy of experimental research but also allows
for the exploration of intricate cellular dynamics that were previously
unattainable with traditional methods. The integration of semi-automation and
sophisticated imaging provides a comprehensive view of cellular interactions and
behaviors, 

Future research should continue
to expand the use of these advanced technologies to further unravel the
complexities of myeloma cell biology and its interaction with the tumor
microenvironment, ultimately contributing to the development of more effective,
personalized treatment strategies.

This convergence of bioinformatics, advanced imaging, and cell biology not only
facilitates a more nuanced understanding of myeloma but also sets a precedent
for the application of these technologies in broader biomedical research,
promising significant advancements in the diagnosis, study, and treatment of
complex diseases.




####



This thesis has engaged deeply with the challenge of advancing our understanding of cellular dynamics in cancer, particularly multiple myeloma, through innovative methodologies spanning coding, cancer research, and microscopy. These sections collectively underscore the integration of advanced technologies and the potential they hold for refining both experimental and analytical practices in biomedical research.

Technological Proposals and Observations

The discussion within the cancer research section proposes the potential utility of multimodal large language models (LLMs) like ChatGPT-4o to refine the terminology for describing complex biological phenomena. This proposition is theoretical and suggests a future direction rather than demonstrating current usage, aiming to ensure that complex observations are more accessible and accurately captured within the scientific community.

In coding, the emphasis has been on developing automated tools to handle and analyze extensive data sets effectively. This focus addresses the need for precision and efficiency in data handling, which is critical in managing the voluminous and complex data generated by modern biomedical research.

Microscopy’s contributions have been vital in illustrating the limitations of static data presentation and advocating for a shift towards dynamic, real-time data analysis through live-cell imaging. This approach offers a clearer insight into the cellular mechanisms at play, particularly those involved in disease progression and response to treatment.

Shared Challenges and Forward Directions

Common across all discussions is the challenge of bridging the gap between advanced technological capabilities and their practical application in research. Each section identifies the need for a balance between manual insight and automated efficiency, suggesting a hybrid approach as essential for future research advancements.

The potential for machine learning and automation to streamline research processes and enhance the accuracy of experimental outcomes is a recurring theme. This technology is poised to revolutionize how data is analyzed and presented, suggesting significant implications for both current and future research methodologies.

Final Remarks

Overall, this thesis advocates for a continued evolution towards integrating more sophisticated computational tools and real-time data analysis techniques in biomedical resea