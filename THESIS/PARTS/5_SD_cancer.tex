



% ======================================================================
\unnsubsection{Isolating \& Quantifying Subpopulations within Cells in Direct
      Contact with MSCs}%
\label{sec:discussion_novel_methods}%
This project aimed to develop methodologies for isolating cells after direct
contact with \acp{hMSC}. The primary challenge was the scarcity of \textit{in
      vitro} methods that could effectively separate and isolate adhering cell
subpopulations for subsequent molecular analysis. Most available techniques
predominantly focus on the quantification of cell adhesion
\cite{khaliliReviewCellAdhesion2015, kashefQuantitativeMethodsAnalyzing2015},
and often employ indirect contact setups, complex micromanipulation \dashed{such
      as Atomic Force Microscopy (AFM)}, or are unsuitable for using live \acp{hMSC}
as the immobilizing surface. To address the limitations of current adhesion
assays, we developed and enhanced innovative methodologies, specifically the
\acf{WPSC} and V-Well adhesion assays.

\textbf{Variability of Washing Steps}: Given the complexity of the requirements,
this project first attempts relied on simple and traditional adhesion assays
that rely on manual washing steps \cite{humphriesCellAdhesionAssays2009}.
Washing involves aspirating the medium, dispensing washing buffer, and
potentially repeating these steps multiple times. This introduces variability
due to differences in pipetting techniques, which affect the accuracy of volume
transfer \cite{guanAssessingVariationsManual2023,
      pushparajRevisitingMicropipettingTechniques2020}. However, adhesion assays don't
rely on precise volume transfer, but accurate detachment of cells adhering at
the well bottom. This introduces a new set of considerations for the pipetting
technique, especially since cells are highly sensitive to shear forces applied
by fluid flow. From the author's experience with washing experiments and
subsequent microscopic evaluations (data not shown), several factors could
contribute to the variability of washing steps:
\begin{enumerate}
      \item The distance of the pipette tip from the well bottom, which
            decreases during aspiration.
      \item The position of the pipette tip relative to the well bottom (center or edge).
      \item The angle of the pipette tip.
      \item The speed of aspiration.
      \item Accidental or intended contact between the pipette tip and the cell layer.
      \item The residual volume left after aspiration.
      \item \textit{The same considerations apply when dispensing the washing buffer.}
\end{enumerate}


\noindent In addition to user-dependent factors, other variables such as the cells'
position on the well bottom can significantly impact the outcome. To the
author's experience, cells located at the edge of the well don't detach as
easily as those in the center, while cells touching the edge are almost
impossible to remove. This phenomenon is likely related to the \textit{boundary
      layer effect}, where fluids slow down near the edges of the well
\cite{weyburneNewThicknessShape2014}.

Together, since both user-dependent and independent factors can affect the
outcome of washing steps, adhesive assays that replace washing are highly
desirable. Still, since washing is straightforward and some variability is
alleviated by the disciplined execution of washing protocols, it remains a
common method for adhesion assays.



\textbf{Directly Interacting Cells Contain Unexplored Interaction Scenarios}: It
is evident that direct and indirect contact to \acp{MSC} have varying effects on
myeloma cells. That difference is crucial for understanding changes in the
\ac{BMME} during MM progression \cite{fairfieldMultipleMyelomaCells2020,
      dziadowiczBoneMarrowStromaInduced2022}. These studies utilize well-inserts to
co-culture myeloma cells in close \dashed{indirect} contact with MSCs. However,
such comparison of indirect \textit{vs.} direct co-culturing methods might not
fully represent the complexity of intercellular interactions
scenarios%
\footterm{\footinteractionscenario}{\label{foot:interactionscenario}}
found in the \ac{BMME}. This is exemplified by
this project, as it relied on the complex growth behavior: \INA cells aggregated
homotypically in direct proximity to those adhering heterotypically to
\acp{hMSC}, and detached through cell division. Furthermore, such methods fail
to capture the subtle variations in paracrine signaling concentrations, where
even a few micrometers of distance could significantly alter cellular responses.

Such knowledge shifted this project's point of view as well: Initially, our
hypothesis focused on direct heterotypic interactions, not expecting a \nMAina
population, but rather subpopulations within \MAina cells that are separable by
varying adhesion strengths. Hence, our assay employed strict conditions favoring
one growth scenario \dashed{heterotypic interactions}, with co-cultures
providing unlimited hMSC-surface availability causing predominantly heterotypic
adhesion, while the short incubation time prevented the formation of aggregates.
Despite these measures, our assay still captured cells emerging from recent cell
divisions rather than from weak heterotypic adherence as initially hypothesized.
This demonstrates the robustness of our method in separating subpopulations that
arising from unexpected interaction scenarios\footref{foot:interactionscenario}.
This can be a major advantage over methods that summarize direct interactions
as one population. Analysing the non-adhering subpopulation within directly
interacting cells could provide valuable insights not just in multiple myeloma,
but also metastasis of other cancer types.


\textbf{Minimizing Variability}:
There are innovative adhesion assays that both
support the isolation of nonadherent subpopulations from directly interacting
cells, and avoid variability introduced by washing steps.

One simple method involves flipping over a 96-well plate, with surface tension
preventing medium spills as non-adhering cells fall to the surface for
collection \cite{zepeda-morenoInnovativeMethodQuantification2011}. However, we
found that the medium in fact did spill occasionally (not shown). Other
approaches involve sealing the plate, such as with PCR plate seals, and using
centrifugation to separate cells \cite{reyesCentrifugationCellAdhesion2003,
      chenHighThroughputScreeningTest2021}. Despite our efforts, we could not
consistently avoid air bubbles, which, after flipping, would contact the cell
layer and create dry regions during centrifugation.

The V-Well adhesion assay does not flip, but collects non-adhering cells into
the nadir of V-shaped wells during centrifugation
\cite{weetallHomogeneousFluorometricAssay2001}. This work profited greatly from
this method, while \dashed{to our knowledge} being the first to use cell
monolayers as the immobilizing surface. We value this method for its precision,
as centrifugation applies a uniform and configurable force, while the readout
remains straightforward, relying on the total fluorescent brightness rather than
individual cell counting.




\textbf{Specializing in Quantifying Adhesion or Isolating Subpopulations}: Most
adhesion assays primarily focus on quantification rather than isolation. The
author attempted to combine both quantification and isolation, but found that
the two goals can be mutually exclusive. The author summarizes the key
differences between quantification and isolation approaches as such:

\begin{itemize}
      \item Cell Manipulation for Harvest \textit{vs.} Readout:
            \begin{itemize}
                  \item Isolation methods are designed to manipulate cells for easy
                        harvest. For instance, the \ac{WPSC} method uses a catching
                        plate to collect non-adherent cells for subsequent analysis.
                  \item Quantification methods, on the other hand, manipulate cells
                        to simplify the readout process. For example, the V-Well
                        assay, which pellets cells into a single location, allowing
                        for a pooled fluorescence measurement without the need for
                        extensive cell handling.
            \end{itemize}

      \item Optimization for Subsequent Analysis \textit{vs.} Sample Throughput:
            \begin{itemize}
                  \item Isolation methods are optimized for detailed subsequent
                        analyses, such as RNA or protein analysis. For example,
                        \ac{WPSC} minimizes the introduction of biases such as those
                        from fluorescent staining, making it suitable for downstream
                        molecular assays.
                  \item Quantification methods are optimized for high sample
                        throughput. The V-Well assay, as an end-point assay, is
                        designed to efficiently handle multiple treatments
                        simultaneously, providing quick and comparative results with
                        lower cell numbers.
            \end{itemize}
      \item Handling of Cell Numbers:
            \begin{itemize}
                  \item Isolation methods, such as \ac{WPSC}, require multiple wells
                        (e.g., 96 wells) to gather a sufficient amount of cells per
                        subpopulation, which is crucial for robust downstream
                        analyses.
                  \item Quantification methods, exemplified by the V-Well assay, are
                        highly efficient even with low cell numbers.
            \end{itemize}
\end{itemize}



\noindent Thus, this adopted two distinct techniques for isolating and
quantifying directly interacting subpopulations, each optimizing for different
outcomes, but also supporting the separation of subpopulations within direct
intercellular interactions. Still, it is theoretically possible to insert
microscopy steps into the \ac{WPSC} method to scan the well bottom for later
cell counting. Also, this work effectively isolated cell pellets from the V-well
plate for subsequent fixation and cell cycle profiling. The process was tedious
and required multiple technical replicates to achieve sufficient cell numbers
for analysis. It also required removing \ac{hMSC} from the V-well nadir to
prevent contamination during pellet aspiration.

Together, while both methods can combine quantification and isolation, they
are optimized towards either of them. Knowing these strengths and weaknesses
could help to advance these methods in future studies.





\textbf{Rationales of the Well Plate Sandwich Centrifugation}: Inspired by the
principles of both flipping and V-Well adhesion assays, we developed the Well
Plate Sandwich Centrifugation (\acf{WPSC}) method to address the challenges of
isolating cell populations. This method innovatively combines elements from both
techniques to provide a more reliable approach to cell isolation. One of the key
advantages of WPSC is its ability to reduce the variability commonly introduced
by manual pipetting. Instead of relying on aspiration, which introduce
variability in cell collection and requires touching the well bottom for
complete removal of medium, WPSC employs centrifugation to remove non-adhering
cells. Medium is then returned by pipetting to repeat the process and maximize
non-adhering cell collection, as the number of detachable cells plateau after
few rounds of centrifugation. Hence, this approach compromises
between minimizing washing variability and isolating larger quantities of cells.

The 96 well plate format has advantages, reducing spilling when flipping the
sandwich, as surface tension kept fluids in place. The 96 well plate format also
reduces per-well variability by performing the same washing procedure up to 96
times.

The slow centrifugation speeds used during \ac{WPSC} are also decided after
thorough consideration. For this, one has to discuss how exactly
non-adhering cells detach during centrifugation. While centrifugal force is an
obvious factor, the properties of cell adhesion are unclear under dry conditions
during centrifugation. The author assumed that the cells are being pulled along
by the medium as it is centrifuged into the catching plate. Hence, the centrifugation
speed was chosen as fast enough to transfer the medium, without completely
drying the co-culture plate and minimizing overall cell stress.

A significant challenge in \ac{WPSC} is the dissociation of \MAina from the hMSC
monolayer. WPSC employs two distinct techniques to achieve this dissociation.
The first technique involves repeated treatment with the gentle digestive enzyme
Accutase followed by \acp{MACS}. \ac{MACS}, despite being effective, is costly,
time-consuming, reduces overall cell yield, and potentially introduces biases
due to CD45 antibody selection and the requirement for cold-treatment. The
second technique utilizes strong pipetting to physically detach non-adhering
cells (termed \emph{`Wash'}). It is important to note that these techniques did
not affect the protocol on detaching \nMAina from the co-culture, hence
providing for a consistent ratio of isolated \MAina to \nMAina across all
experiments. Ultimately, we preferred \emph{Wash}, as \ac{MACS} had to be
performed on all samples to ensure comparability, reducing overall cell yield
which became limiting for downstream applications, especially for \nMAina cells.
Both methods achieved comparable purity of \MAina cells, with few hMSCs per
10\textsuperscript{4} \MAina cells (purity assessment not shown). \emph{Wash}
probably pofited from the highly durable nature of primary hMSC monolayers,
whereas \emph{MACS} required dissociation of the co-culture.

Together, \ac{WPSC} offers a versatile solution for isolating hMSC-interacting
myeloma cells. It successfully balances the need for precision with the ability
to handle larger cell quantities. \ac{WPSC} could be adapted to other cell types
that combines monolayer forming and suspension cells.



\textbf{Key Points on Adhesion Assays:} Ultimately, this work established two
methodologies that could represent a significant advancement in the field of
adhesion assays, providing cost-effective, precise, reliable, and reproducible
techniques for both isolating and quantifying subpopulations within co-cultures
of directly interacting cell types. They offered valuable insights into the
mechanisms of MM detachment and are potentially applicable to other research
questions that focus on growth and interaction scenarios involving multiple cell
types.




%%%%%%%%%%%%
% ======================================================================


\newpage %! corrective

\unnsubsection{Integrating Evidence and Hypotheses for a Mechanistic
      Understanding of Dissemination}%
\label{sec:discussion_framework}%
The results outlined in Chapter\,1 encompass various aspects of multiple myeloma
research, including colonization of the \ac{BMME}, myeloma-\ac{MSC}
interactions, and the association of adhesion factor%
\footterm{\footadhesionfactor}{\label{foot:adhesionfactor}}%
expression with patient survival and disease stages. Such a broad scope invites
the formulation of generalized conclusions, potentially compromising scientific
rigor. The following sections aim to clearly separate hypotheses from evidence
to guide further research on myeloma dissemination.

\textbf{Integrating Observations of \INA in the Multistep Dissemination Model:}
The results gained in this work fit well into the multistep model proposed by
\citet{zeissigTumourDisseminationMultiple2020}. For most steps, observations
were made that could inspire further hypotheses and research:


\begin{enumerate}
      \item \textbf{Retention:}
            \begin{itemize}
                  \item \textit{This work's Observation:} \INA cells attach quickly and
                        strongly to \acp{hMSC}, forming stable aggregates.
                  \item \textit{Hypothesis:} Myeloma cells are retained in the
                        bone marrow microenvironment (BMME) through strong adhesion to
                        \acp{hMSC} and stable homotypic aggregation.
                  \item \textit{Suggested Experiment:} Inject \INA cells into mice and
                        examine bone lesions. Compare the growth patterns in mice
                        co-injected with an ICAM-1 or LFA-1$\alpha$ antibody, which
                        dissolve homotypic aggregates \textit{in vitro} and prevent
                        \INA growth \textit{in vivo}
                        \cite{kawanoHomotypicCellAggregations1991a,
                              klauszNovelFcengineeredHuman2017}. If disrupting aggregation
                        leads to diffuse bone colonization rather than focal lesions,
                        it supports the hypothesis that strong adhesion and
                        aggregation are crucial for retention in the \ac{BMME}.
            \end{itemize}
      \item \textbf{Release:}
            \begin{itemize}
                  \item \textit{This work's Observation:} \INA cells detach from \acp{hMSC}
                        through cell division, and external forces can detach single
                        cells from \INA aggregates.
                  \item \textit{Hypothesis:} Myeloma cells detach from the BMME
                        through cell division and external forces after reaching a
                        minimal aggregate size.
                  \item \textit{Suggested Experiment:} Inject Bromodeoxyuridine (BrdU)
                        stained \INA cells into mice and compare the cell cycle
                        profiles and BrdU signals of circulating cells versus
                        those in the bone marrow. Enrichment of G1/G0 cells
                        among circulating cells would support the hypothesis
                        that detachment is more likely shortly after cell
                        division.
            \end{itemize}
      \item \textbf{Intra-/Extravasation:}
            \begin{itemize}
                  \item This study did not make experiments to study for
                        intra-/extravasation, but these phenomena could be
                        explored with similar methods, if MSCs were replaced by
                        endothelial cells, similar to
                        \citet{solimandoJAMAPrognosticFactor2018}.
            \end{itemize}
      \item \textbf{Colonization:}
            \begin{itemize}
                  \item \textit{This work's Observation:} \INA cells exhibit quick
                        attachment to \acp{hMSC} within one hour and rapidly
                        upregulate numerous adhesion factors, including \ac{ECM}
                        factors.
                  \item \textit{Hypothesis:} Quick attachment and fast
                        expression of adhesion factors enhance the potential to
                        colonize new niches. This is particularly relevant as \INA
                        cells were isolated from the pleura, indicating an ability to
                        colonize extramedullary sites
                        \cite{burgerGp130RasMediated2001c}.
                  \item \textit{Suggested Experiment:} Inject \INA cells into mice and
                        observe if they colonize extramedullary sites. Compare
                        this to \INA cells with reduced adaptability to test the
                        hypothesis. Research is required to find techniques to
                        reduce such putative adaptability: One potential option
                        is using XRK3F2 to inhibit p62, an upstream activator of
                        NF-$\kappa$B \cite{adamikXRK3F2InhibitionP62ZZ2018}. In
                        fact, NF-$\kappa$B signaling seems a robust target,
                        given that it plays a role both in MM patients
                        \cite{sarinEvaluatingEfficacyMultiple2020}, and inducing
                        adhesion factor expression in \INA (this work). Other
                        targetable genes are those proposed by
                        \citet{shenProgressionSignatureUnderlies2021} to be
                        master regulators of myeloma progression.
            \end{itemize}
\end{enumerate}


These hypotheses \dashed{based on observations from \INA cells} provide a
starting point for understanding myeloma dissemination. While these insights are
specialized for the \INA cell line, they inspire the development of a more
generalized framework applicable to a broader range of myeloma cases.



\textbf{Constructing a Generalizable Hypothetical Framework of Dissemination:}
A mechanistic understanding of myeloma dissemination remains elusive. Although
\citet{zeissigTumourDisseminationMultiple2020} described dissemination as a
multistep process, evidence is largely collected for individual steps, leaving
the connections between these steps unproven. As a result, the process of
dissemination is a patchwork of evidence fragments. The following sections aim
to integrate such fragments, especially those derived from the \INA cell line in
this work, to construct a more coherent understanding of myeloma dissemination.
To do so, this work speficies new terminology, including
% \emph{\ac{CAD}}\footterm{\footcad\label{foot:cad}}
\emph{\acf{CAD}}\footterm{\footcad}{\label{foot:cad}}
and \emph{\ac{CAD} dramatype}\footterm{\footcaddt}{\label{foot:caddt}}
or short \emph{adhesion dramatype}\footterm{\footadhesiondt}{\label{foot:adhesiondt}}.




\textbf{Distinguishing Phenotype and Dramatype:} \INA cells exhibited great
reactivity to \acp{hMSC}. Describing this new state as a \emph{phenotype} would
correctly imply the influence of both genetic and environmental factors.
However, this usage overloads the term \emph{environmental factors}, as it
includes the donor’s history, \textit{in vitro} culturing conditions, the
experimental model simulating the \ac{BMME}, and experimental
conditions\,---such as the ratio of \acp{MSC} to \INA cells. Animal studies
faced a similar issue and thus introduced the term \emph{dramatype}
\cite{zutphenPrinciplesLaboratoryAnimal2001}. A dramatype describes the state
resulting from proximate environmental factors, while a phenotype summarizes the
overall environmental background prior to encountering that environment. In
cancer research, the term dramatype is rarely used
\cite{hinoStudiesFamilialTumors2004}, with some researchers preferring terms
like \emph{phenotype switching} \cite{woutersRobustGeneExpression2020}. However,
this blurs the distinction between clonal heterogeneity and transient cell
signaling. The author proposes using \emph{dramatypes} in cell biology to focus
on transient states within the bounds of transcriptional plasticity, while
\emph{phenotypes} describe relatively persistent genetic and epigenetic
backgrounds. These \emph{dramatypes} could then define distinct adhesive
behaviors of myeloma cells observed at each step of dissemination, considering
the microenvironmental and adhesional changes encountered.


\textbf{Introducing Adhesion Dramatypes:} The concept of
\ac{CAD}\footref{foot:cad} describes the time-dependent changes in cell adhesion
and detachment, linking these phases to molecular signatures such as \ac{CAM}
expression or microenvironment-mediated cell signaling. Emphasizing the time
component is particularly useful for predicting the behavior of suspension cells
with complex attachment and detachment dynamics, such as \INA. In this context,
\MAina and \nMAina represent two distinct \textit{in vitro} adhesion dramatypes.
The \MAina dramatype is characterized by the expression of adhesion factors and
stable heterotypic adhesion to \acp{hMSC}, addressing the retention and
colonization steps in the multistep model of dissemination. \MAina cells then
transition to the \nMAina dramatype through cell division and the loss of MSC
adhesion, characterized by unstable homotypic aggregation from which single
cells detach. This may represent the release step in the dissemination process.


\textbf{Key Hypotheses:}
The author introduces the \emph{Dynamic Adhesion Hypothetical Framework for
      Myeloma Dissemination}, which leverages direct observations of
\ac{CAD}\footref{foot:cad}, and is structured around four key hypotheses. Each
address fundamental aspects of myeloma cell dissemination based on both
literature and the results of this work:



\newcommand{\caddramatype}{ %
      \textbf{Myeloma cells change their adhesion dramatype during dissemination.}
      In response to different environmental cues faced during dissemination,
      myeloma cells switch, change or adapt their \ac{CAD}. These states are
      characterized by adhesion dramatpyes\footref{foot:adhesiondt}. Different
      steps in dissemination involve distinct adhesion dramatypes, or instance,
      one for specialized colonizing new sites and one specialized for vascular
      interactions. %
}%
\newcommand{\caddramatypetitle}{ %
      \textit{Hypothesis 1}: Cells Change their Adhesion Dramatype during Dissemination%
}%


\newcommand{\cadplasticity}{ %
      \textbf{Rapid changes of adhesional dramatypes drives aggressive dissemination in myeloma.}
      Adhesional plasticity describes the overall repertoire of adhesion
      dramatypes\footref{foot:adhesiondt} that individual myeloma cells can
      deploy. However, such plasticity is limited by the rapidness of deploying
      a specialized adhesion dramatype during steps of dissemination.
      %
}%
\newcommand{\cadplasticitytitle}{ %
      \textit{Hypothesis 2}: Rapid Adhesional Plasticity Drives Aggression in Myeloma %
}%


\newcommand{\cadddiversity}{%
      \textbf{\ac{CAD} is highly diverse between myeloma patients.}
      % Transcriptional plasticity and clonal heterogeneity introduce variability
      % into myeloma cell populations. These variations are determind by
      % patient-specific factors of even greater variability, such as disease
      % stage and genomic background. Consequently, the combinations of these
      % factors suggest a myriad of manifestations of \ac{CAD} and
      % differing dissemination mechanisms among patients.
      Transcriptional plasticity and clonal heterogeneity introduce variability
      into myeloma cell populations. These variations are primarily determined
      by other patient-specific factors such as disease stage and genomic
      background. Consequently, the interactions between these factors, along
      with other influences like the tumor microenvironment and therapeutic
      interventions, suggest a myriad of manifestations of \ac{CAD} and
      differing dissemination mechanisms among patients. }%
\newcommand{\cadddiversitytitle}{ %
      \textit{Hypothesis 3}: CAD is Highly Diverse Between Myeloma Patients%
}%


\newcommand{\caddtrigger}{%
      \textbf{Detachment is caused by multiple cues of varying nature.} Given
      the diversity of myeloma \ac{CAD}, detachment could be both a consequence
      of ongoing processes, but also triggered by timely defined events. Both
      could combine external mechanical forces, cell division, loss
      of \ac{CAM} expression, or even pure chance. }%    
\newcommand{\caddtriggertitle}{ %
      \textit{Hypothesis 4}: Detachment is Caused by Multiple Cues of Varying
      Nature %
}%





\begin{enumerate}[parsep=4pt]
      \item \caddramatype
      \item \cadplasticity
      \item \cadddiversity
      \item \caddtrigger
\end{enumerate}


This framework sets the stage for a detailed exploration of each hypothesis,
linking empirical data with hypothetical constructs to provide a comprehensive
framework that can help to identify commonalities in myeloma dissemination, but
also inform the development of targeted therapies.




% ======================================================================
\unnsubsection{\caddramatypetitle}%
\label{sec:discussion_caddadaptation}%
As presented in Chapter\,1, \MAina cells exhibited upregulation of both adhesion
factors and chemoattractants (\mypageref{fig:5}), switching their \ac{CAD}
adhesion dramatype
% \footterm{\footadhesiondt\label{foot:adhesiondt}} 
from homotypic aggregation to \acp{MSC} adhesion. Given that \INA cells were
isolated from an extramedullary site \dashed{the pleura}
\cite{burgerGp130RasMediated2001}, such changes likely facilitate colonization
of new microenvironments. This section explores the hypothesis that MM cells
adapt or change their adhesion dramatype not only during colonization, but at
each step of dissemination.


\textbf{Adhesion Dramatypes Assume Distinguishable Niches:} The multistep model
proposed by \citet{zeissigTumourDisseminationMultiple2020} posits that myeloma
cells acquire regulatory mechanisms specialized for each step of dissemination.
The author hypothesizes that the different niches involved in these steps are
unique enough to induce distinct adhesion dramatypes. This requires thorough
knowledge of separate niches. \citet{granataBoneMarrowNiches2022} categorizes
the \ac{BM} into sinusoidal, arteriolar, and endosteal niches, each spatially
and molecularly distinguishable. The endosteal niche is home to \ac{MSC} and a
majority of plasma cells%
\footquote{We suggest that it is reasonable to approach
      the notion of physical plasma cell survival niches with some skepticism. It is
      clear that most BM plasma cells rely heavily on access to APRIL or BLyS (66,
      70), and it appears that mature plasma cells are relatively stationary (59).
      However to us, that plasma cells must remain indefinitely in physical survival
      niches to survive is less obvious.}{wilmoreHereThereAnywhere2017}%
, and the vascular niches \dashed{sinusoidal and arteriolar} include endothelial
cells \cite{zehentmeierStaticDynamicComponents2014,
      wilmoreHereThereAnywhere2017}. Other niches encountered during dissemination
include peripheral blood, lymph nodes, and extramedullary sites. Comprehensive
mapping and characterization of these niches, including their adhesion molecules
and soluble factors, is necessary to understand the adhesion requirements for
each niche. This is a highly complex task, yet summarizing available information
per niche could provide a powerful basis.


\textbf{Distinct Adhesion Dramatypes Transitioning between Niches:}
Adhesion processes are well-documented in MM progression, particularly within
the \ac{BMME} \cite{bouzerdanAdhesionMoleculesMultiple2022}. However, the
dynamism of these processes remains unclear. Overall,
\citet{fredeDynamicTranscriptionalReprogramming2021} have shown that individual
myeloma cells can switch between alternate transcriptional states through
differential epigenetic regulation. Such states were associated with distinct
transcriptioal signatures like those of endothelial progenitors or enhancers
linked to CXCR4. This is indicative of myeloma cells having different \ac{CAD}
dramatypes. In other cancers, different adhesive phenotypes and transitions,
such as those seen in \ac{EMT}, are common
\cite{gengDynamicSwitchTwo2014}. For myeloma, EMT-like phenotypes have been
described, but a clear association with distinct adhesion behaviors is hindered
by the cells maintaining their suspension state
\cite{roccaroCXCR4RegulatesExtraMedullary2015,
      qianSETDB1InducesLenalidomide2023}. This work might be the first to identify
adhesion dramatypes through functional separation of detachable myeloma cells.
As presented earlier, exploring these findings further could reveal transitions
between adhesion dramatypes during MM dissemination, such as attaching for
colonization, or initiating release.


\textbf{Extramedullary Niche:}
Changing adhesion dramatypes predicts a specialized set of adhesion factors for
extramedullary niches. A distinct phenotype has been proposed for extramedullary
myeloma\footquote{Our analysis concluded that the gain of CD44, loss of
      CD56, loss of very late antigen-4 (VLA-4), imbalance of the chemokine
      receptor-4-chemokine ligand-12 (CXCR4-CXCL12) axis, [...] show an increased
      propensity [...] to leave the bone marrow and hone in extramedullary sites
      giving rise to more aggressive extramedullary diseases.
}{guptaExtramedullaryMultipleMyeloma2022}, characterized by changes in
expression of CD44, CD56, VLA-4, and CXCR4
\cite{guptaExtramedullaryMultipleMyeloma2022}. In support of this,
\cite{hathiAblationVLA4Multiple2022} demonstrated that VLA-4 seems to direct
myeloma cells to the BM, since ablating VLA-4 reduced medullary disease, but
increased extramedullary involvement. Furthermore, the role of CXCR4 in
mediating adhesion factor expression is well established, particularly in
extramedullary MM cells \cite{roccaroCXCR4RegulatesExtraMedullary2015,
      guptaExtramedullaryMultipleMyeloma2022}: Extramedullary myeloma cells
overexpress CXCR4, making them more responsive to cues that induce adhesion
factor expression, such as CD44/H-CAM.



\textbf{Vascular Niche:}
Changing adhesion dramatypes predicts a specialized set of adhesion factors for
endothelial interaction, supporting intravasation and extravasation. Although
not assessed in this thesis, the vascular niche is a popular therapeutic target
for preventing dissemination \cite{neriTargetingAdhesionMolecules2012}. Key
adhesion factors like JAM-A, JAM-C and N-Cadherin have been highlighted as
potential targets \cite{solimandoHaltingViciousCycle2020,
      mrozikTherapeuticTargetingNcadherin2015, brandlJunctionalAdhesionMolecule2022}.
Such factors were not differentially expressed between subpopulations isolated
in Chapter\,1. \citet{brandlJunctionalAdhesionMolecule2022} has described JAM-C
co-localizing with blood vessels in the \ac{BM} and dynamic expression by MM
cells in both patients and mice. These results suggest distinct adhesive
mechanisms for vascular \textit{versus} \ac{MSC} interactions.


\textbf{Circulating MM:}
Changing adhesion dramatypes would predict that circulating MM cells lose adhesion
factors. Studies confirm that \dashed{compared to BM-resident cells} circulating
\ac{MM} cells exhibit reduced expression%
\footquote{Our results show that CTCs typically represent a unique
      subpopulation of all BM clonal PCs, characterized by downregulation ($P$ <
      .05) of integrins (CD11a,\,CD11c,\,CD29,\,CD49d,\,CD49e), adhesion
      (CD33,\,CD56, CD117,\,CD138), and activation molecules
      (CD28/CD38/CD81).}{paivaDetailedCharacterizationMultiple2013}%
of multiple adhesion factors, including $\alpha4\beta1$ and CD138/Syndecan-1
\cite{paivaDetailedCharacterizationMultiple2013,
      paivaCompetitionClonalPlasma2011, akhmetzyanovaDynamicCD138Surface2020}.
Evidence suggests that a dynamic loss of CD138/Syndecan-1 and gain of JAM-C
causes intravasation, circulation, and dissemination of MM cells
\cite{akhmetzyanovaDynamicCD138Surface2020,
      brandlJunctionalAdhesionMolecule2022}. This thesis also shows that \nMAina
cells, after emerging as daughter cells from \MAina, not only lose adhesion
factor expression but also exhibit increased survival during IL-6 deprivation,
potentially aiding survival in circulation.


\textbf{\textit{Intermediary Conclusion:} Available Evidence of Adhesion
      Phenotypes Lacks Functional Characterization and Proof of Dramatypic
      Transitions:} The concept of changing adhesion dramatypes is supported by the
existence of distinct BM niches and the identification of separable adhesion
phenotypes. However, the author identifies two major gaps in the current
literature: First, most phenotypic characterizations of adhesion phenotypes are
limited to surface \ac{CAM} expression, ignoring potential secretion of \ac{ECM}
proteins. Second, the transitions between these phenotypes during dissemination
are unexplored.

Functional characterization of adhesive phenotypes \dashed{including \ac{ECM}
      factors} and their transitions could provide a robust framework for
understanding dissemination as a multistep process, reinforcing the dynamic
adhesion hypothetical framework. Mapping adhesive properties for each involved
niche could aid this endeavor.




\textbf{Considerations for Research on Myeloma Cell Adhesion:}
The evidence presented here sets the stage for a more detailed exploration of
adhesion factors in MM. Characterizations of bulk myeloma will not capture
the dynamic changes in adhesion factor expression that occur during dissemination.
Studying adhesion factors in MM \textit{in vitro} requires considering the
specific microenvironmental context. Some adhesion factors are not present in MM
cells but can be rapidly expressed with appropriate signals.

Also, further studies should differentiate between initial adhesion and
upregulated adhesion factors. For example, performing a \ac{WPSC} assay after 30
minutes of adhesion could separate \INA cells based on initial adhesion
capability, with RNAseq of \nMAina \textit{vs} \MAina identifying initial
adhesion factors. Distinguishing between initial and upregulated adhesion
factors could be crucial for predicting colonization potential across niches.
Such initial adhesion is likely to be essential for subsequent growth in \ac{BM}
or extramedullary environments.

Most importantly, it is imperative that characterizations of adhesional
properties should not only include surface \acp{CAM}, but also \ac{ECM} proteins
secreted by myeloma cells. The role of \ac{ECM} proteins in myeloma
dissemination is well-established \cite{ibraheemBMMSCsderivedECMModifies2019},
and this thesis identified a potential role for \ac{ECM} proteins in myeloma
adhesion and colonization. Further research on \ac{ECM}-expression of pheno- and
dramatypes could prove pivotal in understanding myeloma dissemination.


\textbf{Implications for Therapy:}
Adhesion molecules have been targeted for therapy for over a decade
\cite{nairChapterSixEmerging2012, neriTargetingAdhesionMolecules2012}.
Especially inhibiting adhesion molecules involved in interaction with the
endothelium effectively reduces tumor burden in mouse models
\cite{asosinghUniquePathwayHoming2001a,
      mrozikTherapeuticTargetingNcadherin2015}. A deeper understanding of how myeloma
cells regulate \ac{CAD} could be key to predicting and preventing dissemination.

Changing adhesion dramatypes suggests that different adhesion factors should be
either antagonized or agonized depending on their role. For instance, adhesion
factors involved in intravasation and extravasation should be antagonized, while
those facilitating BM retention%
% \footterm{\footretentiveadhesionfactors\label{foot:retentiveadhesionfactors}}%
\footterm{\footretentiveadhesionfactors}{\label{foot:retentiveadhesionfactors}}%
% \label{foot:retentiveadhesionfactors}
should be agonized. However, care should be taken to not agonize adhesion
factors that also provide survival signals. For instance, the short polypeptide
SP16 can activate the receptor LRP1 \dashed{its high expression being associated
      with improved survival of MM patients in this work} passed phase I clinical
trial for treating inflammatory diseases \cite{wohlfordPhaseClinicalTrial2021},
but could potentially increase survival of MM through PI3K/Akt signaling
\cite{potereDevelopingLRP1Agonists2019, heinemannInhibitingPI3KAKT2022}.
However, the risk of promoting cell survival when agonizing adhesion factors can
be mitigated by using targets identified in Chapter\,1, since \autoref{tab:S2}
(\mypageref{tab:S2}) provides a list of retention
targets\footref{foot:retentiveadhesionfactors} that were associated with
improved patient survival when highly expressed. This could also mitigate the
risk of promoting colonization of new sites through increased adhesion, but since
colonization is limited by initial release, this risk is probably low.

Most intriguingly, CAR-T cell therapy could benefit from the concept of adhesion
dramatypes: Arming CAR-T cells against targets listed in \autoref{tab:S2}
(\mypageref{tab:S2}) could specialize in targeting colonizing cells, while those
factors upregulated in \nMAina cells could specialize in targeting detached
cells. Specifically targeting circulating MM cells could effectively reduce
dissemination, as demonstrated in a proof-of-concept study aimed at preventing
metastasis using Granzyme B-based CAR-T cells \cite{sunGranzymeBbasedCAR2024}.


\textbf{\textit{Concluding Remarks and Future Directions:}} Evidence of changing
adhesion phenotypes across various niches reveals a complex interplay between
myeloma cells and their environments, characterized by dynamic regulation of
adhesion factors. Introducing the concept of dramatypes to distinguish between
phenotypic and dynamic adhesion behaviors provides a more detailed framework for
understanding the intricacies myeloma dissemination. Available evidence supports
the hypothesis that myeloma cells adapt their adhesion dramatype in response to
different microenvironments encountered during dissemination. This suggests
potential therapeutic strategies targeting these specific adhesion mechanisms.
Since the majority of currently available phenotypic characterizations have
ignored \ac{ECM} factor secretion, an important axis of potential adhesive
interactions has been overlooked.

Distinguishing adhesion dramatypes among vascular, bone marrow, and
extramedullary niches highlights the need for targeted therapy to either promote
retention or prevent dissemination. Identifying bone marrow retentive factors
that do not induce survival signaling is crucial \dashed{such as CXCR4 or
      CXCL12}, with the gene-list from this work providing a strong starting point
(\apdxref{subapdx:tabs}{tab:S1}).

Future research should include characterization of \ac{ECM} factor expression to
fully clarify the functional roles and transitions of these adhesion dramatypes.
This would validate the changing adhesion dramatype hypothesis and identify
therapeutic targets to disrupt dissemination at various stages. Controlled
\textit{in vitro} studies simulating specific microenvironments, integrating RNA
sequencing and live-cell imaging, will enhance understanding of adhesion factor
regulation and inform the development of precise interventions for multiple
myeloma management.



% ======================================================================
\unnsubsection{\cadplasticitytitle}%
\label{sec:discussion_caddadaptability}%
Chapter\,1 presented diverse observations of rapid transitions between adhesion
dramatypes: Within three days, \INA cells transitioned from homotypic
aggregation to \ac{MSC} adhesion, then back to aggregation followed by
detachment of single cells. This not only shows diverse adhesional
plasticity\footterm{\footadhesionplasticity}{\label{foot:adhesionplasticity}}
but also an intriguing capacity for speed. Since \INA were isolated from highly
advanced \ac{PCL} \cite{burgerGp130RasMediated2001}, such rapid adhesional
plasticity could be driving an aggressive phenotype of myeloma.



\textbf{Associating Adhesion Factors with Disease Progression\,\&\,Aggressiveness:}
The hypothesis of rapid adhesional plasticity is predicated on the association
between adhesion factors and cancer aggressiveness. The transformative processes
in MM pathogenesis have been recognized for decades, typically observed over
months or years \cite{hallekMultipleMyelomaIncreasing1998}. Much of this
research has focused on transformations in resistance mechanisms acquired during
chemotherapy, with cell adhesion factors being well-established drivers of
survival signaling via NF-$\kappa$B, contributing to the selection of
drug-resistant myeloma clones \cite{landowskiCellAdhesionmediatedDrug2003,
      solimandoDrugResistanceMultiple2022}.


Recent research has provided detailed characterizations of adhesion factors
driving myeloma aggressiveness. For instance, specific adhesion and migration
factors have been proposed as master regulators of myeloma progression %
% \footquote{A total of 28 genes were then computationally predicted to be
%     master regulators (MRs) of MM progression. HMGA1 and PA2G4 were validated
%     \textit{in vivo} [...], indicating their role in MM progression and
%     dissemination. Loss of HMGA1 and PA2G4 also compromised the proliferation,
%     migration, and adhesion abilities of MM cells \textit{in
%         vitro}.}{shenProgressionSignatureUnderlies2021}%
\cite{shenProgressionSignatureUnderlies2021}. %
Additionally, a recent study identified 18 adhesion factors as the basis for a
prognostic model to identify high-risk variants in newly diagnosed MM patients
\cite{huDevelopmentCellAdhesionbased2024}. Another recent study demonstrated the
prognostic value of mutated \ac{ECM} proteins expressed by myeloma
\cite{eversPrognosticValueExtracellular2023}.

This thesis contributes to this field by showing that bone-retentive adhesion
factors and \ac{ECM} proteins are continuously downregulated during \ac{MGUS},
\ac{aMM}, \ac{MM}, and \ac{MMR}. Other studies of bulk myeloma biopsies confirm
changes in adhesion factor expression at some point between \ac{MGUS} and
\ac{PCL}%
\footquote{Patients with NDMM had increased VCAM-1 and ICAM-1 compared with MGUS
      and sMM patients. [...] MM patients at first relapse had increased levels
      of ICAM-1 and L-selectin, even compared with NDMM patients and had
      increased levels of VCAM-1 compared with MGUS and sMM.%
}{terposIncreasedCirculatingVCAM12016}\,%
\footquote{Clonal PC from all MG [Monoclonal Gammopathies] displayed
      significantly increased levels of CD56, CD86 and CD126, and decreased
      amounts of CD38 ($P$ < 0.001). Additionally, HLA-I and
      $\beta$2-microglobulin were abnormally highly expressed in MGUS, while
      CD40 expression was decreased in MM and PCL ($P$ < 0.05). Interestingly, a
      progressive increase in the soluble levels of $\beta$2-microglobulin was
      found from MGUS to MM and PCL patients ($P$ < 0.03). In contrast, all
      groups showed similar surface and soluble amounts of CD126, CD130 and
      CD95, except for increased soluble levels of CD95 observed in PCL.%
}{perez-andresClonalPlasmaCells2005},~%
reporting %
increased levels of VCAM-1,\,ICAM-1,\,L-selectin,\,CD56,\,CD86,
CD126,\,\&\,CD95, %
decreased levels of CD38,\,HLA-I,\,$\beta$2-microglobulin,\,\&\,CD40, and %
no changes in CD130 %
\cite{terposIncreasedCirculatingVCAM12016, perez-andresClonalPlasmaCells2005}.

Intriguingly, not only the surface expression of adhesion factors plays a role
during progression, but also the surrounding \ac{ECM}. \ac{ECM} from myeloma
patients exhibits tumor-promoting properties, in stark contrast to the
tumor-abrogating \ac{ECM} from healthy donors
\cite{ibraheemBMMSCsderivedECMModifies2019}.

Together, recent advances have effectively associated changes in cell adhesion
expression phenotypes with aggressive myeloma. Further insights can be drawn
from the databases used in Chapter\,1 (\mypageref{fig:6})
\cite{seckingerCD38ImmunotherapeuticTarget2018}, identifying adhesion genes
differentially expressed between cohorts of different disease stages, followed by
functional categorization into GO-terms associated with dissemination steps.
However, these studies do not focus on a mechanistic understanding of how cell
adhesion drives aggressive dissemination, nor do they discuss the speed of
adaptations in adhesion phenotypes.


\textbf{Adhesional Plasticity and Speed:}
Clonal dynamics have established that rapid mutations can drive aggressive
progression on a genomic scale \cite{keatsClonalCompetitionAlternating2012,
      eversPrognosticValueExtracellular2023}. The hypothesis of rapid adhesional
plasticity extends this idea to the activity of adhesion factors, encompassing
dynamic (post-)transcriptional regulation and adhesion kinetics regulated at the
protein level. In solid cancers, sufficient information on regulatory dynamics
of \ac{EMT} exists to train a deep learning model
\cite{tongLearningTranscriptionalRegulatory2023}. This model infers
transcriptional changes over time from static single-cell RNAseq data,
demonstrating that dynamic phenotypic changes can be projected along selected
transcriptional trajectories \dashed{in this case \ac{EMT}}, and aiding in the
prediction of future detachment dynamics.


However, for hematological cancers such as MM, high time resolutions of up to
minutes or even seconds might be required. As previously discussed, the speed of
dynamic changes is inherent to adhesional processes
(\mypageref{sec:discussion_semi_automated_analysis}), but time remains an often
overlooked dimension in molecular cancer research%
\footquote{These predominant snapshot approaches are fundamental limiting
factors in the advancement of precision oncology since they are causal
agnostic, i.e., they remove the notion of time (dynamics) from cancer
datasets. [...] The lack of time-series measurements in single-cell
multi-omics [...] and cell population fluctuations (i.e., ecological
dynamics), in patient-derived tumor and liquid biopsies, remains a central
roadblock in reconstructing cancer networks as complex dynamical
systems.%
}{uthamacumaranReviewMathematicalComputational2022}
\cite{uthamacumaranReviewMathematicalComputational2022}. The author hypothesizes
that this rapidness is crucial for colonizing new sites. Adhesional plasticity
alone might not be sufficient for successful attachment or survival: \INA cells
failed to adhere to \ac{MSC} during live-cell imaging if the motorized stage top
was moving too fast, necessitating decelerated microscopy configuration (data
not shown). This indicates that colonization attempts are thwarted by
fast-moving environments, despite strong \ac{MSC} adhesion potential.

Similar to Hypothesis\,1, the evidence for this hypothesis is limited by the
current understanding of transitions between adhesion dramatypes. Hypothesizing
the rapidity of such transitions adds further complexity, requiring a
time-dimension for every experiment. Nonetheless, there is evidence of rapid
transitions towards detaching and invasive dramatypes, implying swiftness in
these processes: A sudden loss of the adhesion factor CD138 can
occur either through antibody treatment or shedding by heparanase
\cite{yangHeparanasePromotesSpontaneous2005,
      akhmetzyanovaDynamicCD138Surface2020}. Exploring such rapid dynamisms is a major
challenge for future research but also presents a significant opportunity to
establish a new field of research in myeloma dissemination focused on \ac{CAD}.


\textbf{Potential Mechanisms Facilitating the Fast Switch of Adhesion Dramatypes:}
The hypothesis of rapid adhesional plasticity suggests that aggressive myeloma
cells can swiftly alter their adhesion dramatype, as observed in \INA-\ac{MSC}
co-cultures. To facilitate such rapid changes, several molecular mechanisms
might be utilized: Integrins can undergo rapid conformational changes from
active to inactive forms, a process used to detach B-cell leukemia cells through
small molecule treatment \cite{ruanVitroVivoEffects2022}. Additionally, myeloma
cells can express proteases like heparanase to shed adhesion factors from their
cell surface \cite{yangHeparanasePromotesSpontaneous2005}. For \INA cells, the
author proposes three mechanisms that could explain this swiftness:

\textbf{Rapid NF-$\kappa$B Signaling:}
First, NF-$\kappa$B signaling is enriched in \MAina cells.
NF-$\kappa$B is known as one of the fastest signaling pathways, capable of
regulating the transcription of target genes within seconds
\cite{gallego-sellesFastRegulationNFkB2022,
      zarnegarNoncanonicalNFkBActivation2008}. This signaling pathway is relevant for
both \textit{in vitro} experiments and MM patients
\cite{sarinEvaluatingEfficacyMultiple2020}, making its downstream effectors
robust targets for treatment.

\textbf{Asymmetric Cell Divisions?}
Second, asymmetric cell division could explain the rapid loss of adhesion gene
mRNA transcripts observed in \nMAina cells that emerged from \MAina cells
through cell division. This process \dashed{popularized by stem cell research for
      facilitating self-renewal \cite{shenghuiMechanismsStemCell2009}}, has
underlying molecular mechanisms conserved in asymmetrically dividing
cells and cellular polarization processes as well \cite{inabaAsymmetricStemCell2012,
      stjohnstonCellPolarityEggs2010}. Asymmetry can be established
\emph{intrinsically}, where factors are segregated between daughter cells, or
\emph{extrinsically}, by placing two daughters into distinct microenvironments
\cite{inabaAsymmetricStemCell2012}. It is debatable whether the definition of
extrinsic asymmetric cell division is fulfilled by this work's observation:
\nMAina daughter cells emerging out of range of \acp{MSC}, thereby delaying
direct adhesion until the \nMAina re-attaches to an \ac{MSC}.

Intrinsic mechanisms could be explored by live-cell imaging of cell division
events in \INA-\ac{MSC} co-cultures, followed by \textit{in situ} hybridization
using fluorescently labeled antisense RNA probes. If successful, this could
represent the first evidence for asymmetric cell division in MM, which could be
useful for the popular \emph{cancer stem cell hypothesis}. However, \MAina cells
do not currently fulfill the multipotency criterion required by stem cell
terminology \cite{johnsenMyelomaStemCell2016, liAsymmetricCellDivision2022}.

\textbf{Speed and Flexibility of ECM Interactions:}
Third, the role of \ac{ECM} in facilitating rapid adhesional plasticity must be
considered. The \ac{ECM} regulates cell adhesion and migration rapidly, and its
composition is altered by myeloma cells, which include various mutated \ac{ECM}
proteins \cite{ibraheemBMMSCsderivedECMModifies2019,
      eversPrognosticValueExtracellular2023}. For the case of myofibroblasts,
contractile cell-matrix interactions can involve calcium signalling, where
transduction takes less than \SI{1}{\second} \cite{yamadaCell3DMatrix2022}.
Additionally, myeloma cells can remodel the \ac{ECM} on site, reducing the need
for adaptations in cell surface factor expression, providing an additional axis
for potential adhesive interactions and improved flexibility in changing
adherent sites. Although \MAina cells never detached from \ac{MSC} themselves,
\ac{ECM} is a viable candidate for facilitating rapid adhesional plasticity in
myeloma cells.



\textbf{Implications for Research on Myeloma Cell Adhesion:}
Rapid adhesional plasticity could explain the high variance in adhesion factor
expression that's independent of donor-to-donor variability: Even within the
same disease stage and niche, subsets of myeloma cells could exhibit rapidly
interchanging adhesion dramatypes. This underscores the relevance of \textit{in
      vitro} studies involving direct contact with stromal or endothelial cells, as
they can capture subpopulations with different adhesion dramatypes, akin to
\MAina and \nMAina.

Rapid adhesional plasticity also explains the arguably paradoxical trend of
decreasing expression of bone retentive adhesion
factors\footref{foot:retentiveadhesionfactors} during disease progression, as
described in Chapter\,1: Hypothetically speaking, partial loss of adhesion
factors improves overcoming \ac{BM} retention, while quick re-upregulation of
these factors allows for rapid reattachment \dashleft{if required}. Quick
re-upregulation could then facilitate exploration of new niches or an acute need
of \ac{CAM} mediated survival signaling. This dynamic switching could give
myeloma cells a competitive advantage in various microenvironments. Adhesional
plasticity further gains flexibility, if cells not only regulate surface
\ac{CAM} espression, but also utilize secretion of \ac{ECM} as an additional
axis for adhesive interactions.



\textbf{Implications for Therapy:}
Rapid adhesional plasticity could significantly impact the development of
targeted therapies: Different myeloma dramatypes might lack traditional
prognostic markers but still possess the ability to rapidly express these
markers, potentially leading to the misidentification of high-risk patients.
Therefore, targeted therapies should incorporate multiple markers obtained from
various tissue sources to enhance the accuracy of prognostic predictions. This
could ensure that high-risk patients receive the most appropriate therapeutic
interventions.



\textbf{\textit{Concluding Remarks:}} The hypothesis of rapid \ac{CAD}
plasticity is supported by the association between adhesion factors and myeloma
aggressiveness, although direct evidence for the speed of these adaptations
remains limited. Advanced stages of myeloma and aggressive phenotypes are linked
to distinct adhesion dramatypes. The scarce evidence of dramatype transitions
only imply rapidness, lacking precise dynamics require. If proven true, this
hypothesis underscores the need for future research to focus on the mechanisms
and speed of these adhesion changes to develop robust personalised therapies.




% ======================================================================
\unnsubsection{\cadddiversitytitle}%
\label{sec:discussion_cadddiversity}%
Adhesion factor expression in myeloma cells exhibits large variability: The
interquartile range of CXCL12 fold-change expression spans more than one order
of magnitude (Chapter\,1, \autoref{fig:6}, \mypageref{fig:6}). Such
between-patient variance further adds to the previously discussed adhesional
plasticity\footref{foot:adhesionplasticity}. High
variance poses both a challenge and an opportunity for cancer research, as
dissecting the sources of this variability can reveal how specific forms
of \ac{CAD} contribute to myeloma progression in various ways.


\textbf{Prognostic Power of Genomic Variants:}
Genetic diversity is a major source of between-patient variability. Ongoing
genomic research continues to identify recurrent patterns of chromosomal
aberrations and mutational signatures, defining both structural and single
nucleotide variants \cite{kumarMultipleMyelomasCurrent2018a,
      hoangMutationalProcessesContributing2019}. The prognostic value of these genetic
variants in MM is well established \cite{sharmaPrognosticRoleMYC2021}, and their
identification is becoming increasingly cost-effective, paving the way for
targeted therapies \cite{zouComprehensiveApproachEvaluate2024,
      budurleanIntegratingOpticalGenome2024}. Recent advances associating high-risk
myeloma with \ac{ECM} mutations or adhesion factor expression, as discussed in
Hypothesis\,2 (\mypageref{sec:discussion_caddadaptability}), could potentially
explain the diversity of adhesion dramatypes between patients
\cite{eversPrognosticValueExtracellular2023,
      huDevelopmentCellAdhesionbased2024}.

However, while these prognostic associations are valuable, they do not fully
explain the mechanisms by which these genetic variants drive myeloma
progression.



\textbf{Integrating \textit{in vitro} \ac{CAD} Characteristics into a
      Mechanistic Understanding:} %
% \INA cells form aggregates, and such growth behavior was shwon to be fundamental
% in proposing the mechanism of how these cells would disseminate \textit{in vivo}
% (\autoref{fig:7}, \mypageref{fig:7}). Primary myeloma cell cultures are known to
% show aggregation behavior \cite{kawanoHomotypicCellAggregations1991a,
%       okunoVitroGrowthPattern1991}. The \ac{CAD} of other cell lines are also very
% diverse: MM1.S being plastic adhering, moderately MSC-adhering non-aggregating,
% \INA being non adhering aggregate forming and MSC-adhering, U266 being plastic
% adhering, non MSC-adhering and non-aggregating. Given these diverse behaviors,
% it is likely that the \ac{CAD} of myeloma cells \textit{in vitro} shares similar
% complexity.
\INA cells form aggregates, a behavior that was fundamental in Chapter\,1 for
understanding how these cells might disseminate \textit{in vivo}
(\autoref{fig:7}, \mypageref{fig:7}). Not just \INA cells, but also primary
myeloma cell cultures are known to exhibit aggregation behavior
\cite{kawanoHomotypicCellAggregations1991a, okunoVitroGrowthPattern1991}. The
\textit{in vitro} adhesion phenotype of various myeloma cell lines also varies
widely, differing in plastic/MSC adherence and aggregation behavior
(\autoref{tab:cad_characteristics}). This diversity suggests that the \ac{CAD}
of myeloma cells is complex and variable \textit{in vivo}.

\newcolumntype{B}{>{\bfseries}l}
\def\myheader{\textbf{Cell\,Line} & \textbf{Plastic\,Adhering} & \textbf{MSC\,Adhering} & \textbf{H.\,Aggregating} }
\begin{table}[h]
      \centering
      \begin{tabular}{Bccc}
            \hline
            \myheader                         \\
            \hline
            MM1.S & Yes      & Moderate & No  \\
            \INA  & No       & Strong   & Yes \\
            U266  & Moderate & Weak     & No  \\
            AMO1  & No       & Moderate & No  \\
            OPM2  & No       & Moderate & No  \\
            \hline
      \end{tabular}
      \caption{\textit{In vitro} adhesion phenotypes of myeloma cell lines.
            MSC adhesion for MM1.S, INA-6 and U266 was measured in
            \apdxref{subapdx:figs}{fig:S1} (\mypageref{fig:S1}); other data is
            based on laboratory experience. H.\,Aggregating: Homotypically
            Aggregating.}
      \label{tab:cad_characteristics}
\end{table}


Given these insights, it would be informative to examine if other myeloma cell
lines exhibit behavior similar to \INA cells, especially with aggregating cell
lines. By characterizing their \ac{CAD} in terms of plastic/MSC adherence,
aggregation behavior, detachments under live-cell imaging, and gene expression
profiles, followed by comparative \textit{in vivo} studies on dissemination
behavior, researchers could associate these \textit{in vitro} \ac{CAD}
parameters with dissemination patterns observed after injecting these cells into
mice. This approach could provide a deeper understanding of how different
\textit{in vitro} \ac{CAD} patterns contribute to myeloma dissemination.



% ======================================================================
\unnsubsection{\caddtriggertitle}%
\label{sec:discussion_caddtrigger}%
Detachment mechanisms observed in Chapter\,1 primarily involved mechanical
forces. Myeloma cells, \dashed{growing as homotypic aggregates} remained stable,
yet it seemed that they progressively lost adhesion force with each cell
division. Eventually, convective streams were sufficient to detach single \INA
cells from homotypic aggregates. While this process was visibly mechanical, it
was predisposed by cellular interactions that destabilized adhesive strength
through the saturation of hMSC surfaces and changes in aggregate shape due to
cell division. This complexity suggests a multifaceted mechanism behind cell
detachment, warranting exploration of various triggers. The following paragraphs
discuss potential mechanisms that could trigger myeloma cell detachment.

\noindent\textbf{Other Potential Detachment Mechanisms:}%

\begin{itemize}

      \item\textbf{Intercellular interaction scenarios:} \INA cells demonstrated
            that saturation of MSC adhesion and unstable aggregates ultimately
            contribute to detachment \textit{in vitro}. It is reasonable to
            question if similar scenarios apply \textit{in vivo}, where MSCs are
            less abundant and ECM provides more substrates for adhesion.
            However, \INA cells adhere to MSC-derived osteoblasts as well,
            potentially providing ample surface for intercellular interactions
            \cite{dotterweichContactMyelomaCells2016}. The principle that
            adhesion surfaces are limited and can become saturated has not been
            thoroughly explored in the literature, yet it could be a critical
            piece of understanding detachment events.

      \item\textbf{Rapid loss of surface \acp{CAM}:} The loss of CD138, either
            through antibody treatment or intrinsic expression of heparanase,
            highlights rapid changes in adhesion molecules
            \cite{yangHeparanasePromotesSpontaneous2005,
                  akhmetzyanovaDynamicCD138Surface2020}. This suggests that detachment
            might not always be a gradual process but can occur swiftly due to
            biochemical changes.

      \item\textbf{Slow loss of surface \acp{CAM}:} Since bone marrow-retentive
            adhesion molecules gradually decrease (\autoref{fig:6},
            \mypageref{fig:6}), it is plausible that the final detachment of MM
            cells is a slow culmination of diminishing adhesion, with the actual
            separation triggered by other events, such as external forces.

      \item\textbf{Sudden gain of endothelial \acp{CAM}:} Conversely,
            \citet{brandlJunctionalAdhesionMolecule2022} has shown that JAM-C is
            dynamically expressed in MM cells co-localizing with blood vessels.
            One can assume that release could be triggered when endothelial
            cells are in close vicinity and competes with \ac{BM} stroma for MM
            cell adhesion.


      \item\textbf{Loss of substrate adhesion:} Myeloma cells actively
            contribute to the degradation of the bone matrix
            \cite{terposPathogenesisBoneDisease2018}, which could directly
            facilitate detachment. This mechanism is straightforward but might
            be insufficient to explain early-stage dissemination where extensive
            bone degradation hasn't occurred yet. However, in cases of myeloma with
            severe bone disease, this aspect could be critical, as weakened
            or destroyed physical barriers may be an overlooked contributor to
            dissemination.

      \item\textbf{Soluble signals:} The role of cytokines and chemokines
            \dashed{such as MIP-1$\alpha$, MCP-1, IL-8, and CXCL12/SDF-1}
            in influencing MM adhesion within the BM is well established
            \cite{aggarwalChemokinesMultipleMyeloma2006,
                  alsayedMechanismsRegulationCXCR42007}, For instance, myeloma cells
            overexpress MIP-1$\alpha$ constitutively, reducing adhesion and
            triggering migration in an autocrine fashion
            \cite{lentzschMacrophageInflammatoryProtein2003,
                  abeRoleMacrophageInflammatory2002}. When expressed constitutively,
            these signals could prime MM for detachment. Also, if such signals
            accumulate and pass a certan threshold, one could assume that they
            cause detachment as a timely isolated trigger.

      \item\textbf{Soluble signals:} Cytokines and chemokines \dashed{such as
                  MIP-1$\alpha$, MCP-1, IL-8, and CXCL12/SDF-1} play a
            well-established role in influencing MM adhesion within the bone
            marrow (BM) \cite{aggarwalChemokinesMultipleMyeloma2006,
                  alsayedMechanismsRegulationCXCR42007}. For instance, myeloma cells
            overexpress MIP-1$\alpha$ constitutively, which reduces adhesion and
            triggers migration in an autocrine manner
            \cite{lentzschMacrophageInflammatoryProtein2003,
                  abeRoleMacrophageInflammatory2002}. Constitutive expression of these
            signals may prime MM cells for detachment. Additionally, if the
            accumulation of such signals surpasses a certain threshold, it is
            reasonable to assume they could act as an isolated trigger for
            detachment.

      \item\textbf{Purely mechanical forces:} It is conceivable that physical
            changes in the bone matrix, such as bending or breaking, could
            mechanically dislodge myeloma cells from their niche. This process
            could become more pronounced with advancing bone destruction, but
            its direct impact on cell detachment remains speculative at this
            point. It is of particular note, that mechanical loading of bone has
            been shown to enhance bone health in a myeloma mouse model, as the
            beneficial mechanoresponse positively modulates bone turnover
            \cite{rummlerMechanicalLoadingPrevents2021}.

      \item\textbf{Pure chance:} Detachment might occasionally occur randomly,
            without a specific trigger, although this notion is purely
            speculative and included for completeness.

\end{itemize}

\textbf{Implications for Future Research:} %
Detachment events are critical not only as isolated key events in dissemination
but also for their implications on subsequent steps in the process. Cells that
detach due to soluble signals are likely to assume different adhesion dramatypes
influenced by downstream signaling compared to cells detached by mechanical
forces. Understanding these nuances can inform targeted interventions.

A rational categorization of disease stages could be instrumental in
understanding detachment mechanisms. However, there is currently no solid
mechanistic basis for such categorizations. Possible approaches could involve
weighing mechanical versus molecular contributions to detachment, or
distinguishing between direct detachment signals, or indirect detachment due to
substrate destruction. This would be particularly useful if the severity of bone
disease indeed influences the detachment mechanism, as advanced bone disease
implies indirect detachment after substrate destruction.

While \textit{in vivo} studies offer valuable snapshots, a mechanistic
understanding of detachment probably requires a high time-resolution, such as
that provided by \textit{in vitro} live-cell imaging. In this work, \textit{in
      vitro} studies were limited by the absence of surrounding 3D substrate. However,
this setup provided sufficient insights into detachment mechanisms that seem at
least reasonably inferable to an \textit{in vivo} context. Most importantly, the
identified targets and their association with clinical outcomes remained
consistent regardless of the experimental setup. Therefore, this approach could
bridge the gap between \textit{in vivo} and \textit{in vitro} studies, offering
a more controlled environment to study specific detachment mechanisms with
specialized setups only possible in \textit{in vitro} studies.


\textbf{Implications for Therapy:}%
Understanding the specific reasons behind myeloma cell detachment could be
crucial for predicting subsequent steps of dissemination. For instance, as
myeloma progresses, the degradation of bone and loss of physical barriers could
alter detachment mechanisms. Therefore, advanced disease states may require
specialized treatment strategies that address these unique detachment processes.


\textbf{\textit{Concluding Remarks:}}%
These paragraphs elucidated the complex interplay of mechanical and molecular
factors in myeloma cell detachment, highlighting the multifaceted nature of this
process. Mechanical forces, such as those mediated by cell division and
convective streams, alongside molecular dynamics like the modulation of cell
adhesion molecules and bone matrix integrity, could play crucial roles in cell
detachment. The process is probably not governed by a singular molecular
mechanism but results from the dynamic expression of adhesion factors, changes
within the bone marrow microenvironment, and external mechanical forces. These
insights underscore the need to categorize detachment mechanisms based on their
triggers, for instance distinguishing between either directed cell signals or
substrate-dependent mechanical contributions.

The variability introduced by patient-specific factors, such as the onset or
severity of bone destruction, suggests that categorizing detachment mechanisms
could significantly impact therapeutic strategies. Our findings advocate for an
integrated approach that combines \textit{in vitro} temporal precision with
\textit{in vivo} relevance, aiming to precisely counteract the early stages of
myeloma dissemination. Future research should continue to explore these
mechanisms, potentially using advanced imaging and 3D culture systems, to
further refine our understanding of detachment processes and their implications
for treatment.


% ======================================================================
\newpage
\unnsubsection{\texorpdfstring{% > Required when using tex commands in titles
            \textit{\textbf{Conclusion\,3:} The Dynamic Adhesion Hypothetical
                  Framework for Myeloma Dissemination} %
      }{%
            Conclusion 1: The Dynamic Adhesion Hypothetical Framework for
            Myeloma Dissemination %
      }%
}%
\label{sec:discussion:conclusioncancer}%

This thesis demonstrates the plasticity of myeloma
\acf{CAD}\footref{foot:cad}, with findings indicating rapid adaptability of
myeloma cells to diverse microenvironments. This adaptability is encapsulated in
the concept of \emph{“adhesion dramatype”}\footref{foot:caddt}, introduced to
describe dynamic states of adhesion due to proximate environmental factors,
distinguishing it from more persistent \emph{phenotypic} characteristics.
Observations from \INA cells support the idea that cell detachment can result
from mechanical forces, cell division, and the instability of homotypic
aggregates.
This work has also advanced methodologies for adhesion
assays, providing new tools to isolate and quantify subpopulations within
co-cultures, crucial for understanding the nuances of MM detachment.

While the evidence from this work provides a comprehensive foundation for
understanding \ac{CAD} in multiple myeloma, many aspects remain speculative,
particularly concerning the speed and precise mechanisms of \ac{CAD} changes.
Literature supports these findings, but evidence remains fragmented across many
fields, including genetic diversity in adhesion and \ac{ECM} factors, signaling
pathways modulating adhesion, and differential expression of \acp{CAM} between
microenvironmental niches or disease stages. The integration of such fragments
highlight a complex interplay yet to be fully deciphered. For instance, although
the prognostic value of \ac{CAM} and \ac{ECM} proteins is well-established, the
detailed pathways through which these variants contribute to myeloma
dissemination remain less clear, necessitating more functional validation.
Recurring concepts, such as the plasticity of \ac{CAD}, unexplored mechanical
contributions, and the influence of microenvironmental cues emphasize
the need for a mechanistic understanding of dissemination.

Future research should prioritize the development of precise \textit{in vitro}
models that mimic specific microenvironments like the \ac{BMME}, integrating
advanced live-cell imaging and adhesion assays. The novel assays developed in
this work, particularly for myeloma-MSC interactions, could be adapted to study
other niches such as vascular environments. This approach will enhance our
understanding of how different adhesional patterns or dramatypes influence
myeloma progression and dissemination, providing deeper insights that could lead
to targeted therapeutic interventions.

The dynamic nature of \ac{CAD} underscores the need for personalised therapeutic
strategies that consider specific adhesion dramatypes and niche-specific
interactions. Targeting \ac{CAD} could prevent dissemination, especially in
advanced disease stages where bone degradation modifies detachment mechanisms.
Therapies could also benefit from a multifactorial approach that includes
strengthening \ac{ECM} or cell adhesion to enhance bone marrow retention.
However, it is critical to ensure that these strategies do not inadvertently
promote survival signaling, colonization of extramedullary sites or aggravate
myeloma bone disease. Understanding the triggers and mechanisms of cell
detachment informs the design of effective interventions that could adapt to the
progression of the disease and the specific needs of the patient.
