



% ======================================================================
\unnsubsection{Isolating \& Quantifying Subpopulations within Cells in Direct Contact with MSCs}%
\label{sec:discussion_novel_methods}%
This project aimed to develop methodologies for isolating cells after direct
contact with \acp{hMSC}. The primary challenge was the scarcity of \textit{in
      vitro} methods that could effectively separate and isolate adhering cell
subpopulations for subsequent molecular analysis. Most available techniques
predominantly focus on the quantification of cell adhesion
\cite{khaliliReviewCellAdhesion2015, kashefQuantitativeMethodsAnalyzing2015},
and often employ indirect contact setups, complex micromanipulation, or are
unsuitable for using live \acp{hMSC} as the immobilizing surface. To address the
limitations of current adhesion assays, we developed and enhanced innovative
methodologies, specifically the \acf{WPSC} and V-Well adhesion assays.

\textbf{Variability of Washing Steps}: Given the complexity of the requirements,
this project first attempts relied on simple and traditional adhesion assays
that rely on manual washing steps \cite{humphriesCellAdhesionAssays2009}.
Washing involves aspirating the medium, dispensing washing buffer, and
potentially repeating these steps multiple times. This introduces variability
due to differences in pipetting techniques, which affect the accuracy of volume
transfer \cite{guanAssessingVariationsManual2023,
      pushparajRevisitingMicropipettingTechniques2020}. However, adhesion assays don't
rely on precise volume transfer, but accurate detachment of cells adhering at
the well bottom. This introduces a new set of considerations for the pipetting
technique, especially since cells are highly sensitive to shear forces applied
by fluid flow. From the author's experience with washing experiments and
subsequent microscopic evaluations (data not shown), several factors could
contribute to the variability of washing steps:
\begin{enumerate}
      \item The distance of the pipette tip from the well bottom, which decreases during aspiration.
      \item The position of the pipette tip relative to the well bottom (center or edge).
      \item The angle of the pipette tip.
      \item The speed of aspiration.
      \item Accidental or intended contact between the pipette tip and the cell layer.
      \item The residual volume left after aspiration.
      \item \textit{The same considerations apply when dispensing the washing buffer.}
\end{enumerate}


In addition to user-dependent factors, other variables such as the cells'
position on the well bottom can significantly impact the outcome. To the
author's experience, cells located at the edge of the well don't detach as
easily as those in the center, while cells touching the edge are almost
impossible to remove. This phenomenon is likely related to the \textit{boundary
      layer effect}, where fluids slow down near the edges of the well
\cite{weyburneNewThicknessShape2014}.

Together, since both user-dependent and independent factors can affect the
outcome of washing steps, adhesive assays that replace washing are highly
desirable. Still, since washing is straightforward and some variability is
alleviated by the disciplined execution of washing protocols, it remains a
common method for adhesion assays.




\textbf{Directly Interacting Cells Contain Unexplored Interaction Scenarios}: It
is evident that direct and indirect contact to \acp{MSC} have varying effects on
myeloma cells. That difference is crucial for understanding changes in the
\ac{BMME} during MM progression \cite{fairfieldMultipleMyelomaCells2020,
      dziadowiczBoneMarrowStromaInduced2022}. These studies utilize well-inserts to
co-culture myeloma cells in close \dashed{indirect} contact with MSCs. However,
such comparison of indirect \textit{vs.} direct co-culturing methods might not
fully represent the complexity of intercellular interactions scenarios found in
the \ac{BMME}. This is exemplified by this project, as it relied on the complex
growth behavior: \INA cells aggregated homotypically in direct
proximity to those adhering heterotypically to \acp{hMSC}, and detached
through cell division. Furthermore, such methods fail to capture the subtle
variations in paracrine signaling concentrations, where even a few micrometers
of distance could significantly alter cellular responses.

Such knowledge shifted this project's point of view as well: Initially, our
hypothesis focused on direct heterotypic interactions, not expecting a \nMAina
population, but rather subpopulations within \MAina cells that are separable by
varying adhesion strengths. Hence, our assay employed strict conditions favoring
one growth scenario \dashed{heterotypic interactions}, with co-cultures providing unlimited hMSC-surface
availability causing predominantly heterotypic adhesion, while the short
incubation time prevented the formation of aggregates. Despite these measures,
our assay still captured cells emerging from recent cell divisions rather than
from weak heterotypic adherence as initially hypothesized. This demonstrates the
robustness of our method in separating subpopulations that arising from unexpected
intercellular interaction scenarios. This can be a major an advantage over
methods that summarize direct interactions as one population. Analysing the
non-adhering subpopulation within directly interacting cells could provide
valuable insights not just in multiple myeloma, but also metastasis of other
cancer types.


\textbf{Minimizing Variability}:
There are innovative adhesion assays that both
support the isolation of nonadherent subpopulations from directly interacting
cells, and avoid variability introduced by washing steps.

One simple method involves flipping over a 96-well plate, with surface tension
preventing medium spills as non-adhering cells fall to the surface for
collection \cite{zepeda-morenoInnovativeMethodQuantification2011}. However, we
found that the medium in fact did spill occasionally (not shown). Other
approaches involve sealing the plate, such as with PCR plate seals, and using
centrifugation to separate cells \cite{reyesCentrifugationCellAdhesion2003,
      chenHighThroughputScreeningTest2021}. Despite our efforts, we could not
consistently avoid air bubbles, which, after flipping, would contact the cell
layer and create dry regions during centrifugation.

The V-Well adhesion assay does not flip, but collects non-adhering cells into
the nadir of V-shaped wells during centrifugation
\cite{weetallHomogeneousFluorometricAssay2001}. This work profited greatly from
this method, while \dashed{to our knowledge} being the first to use cell
monolayers as the immobilizing surface. We value this method for its precision,
as centrifugation applies a uniform and configurable force, while the readout
remains straightforward, relying on the total fluorescent brightness rather than
individual cell counting.




\textbf{Specializing in Quantifying Adhesion or Isolating Subpopulations}: Most
adhesion assays primarily focus on quantification rather than isolation. The
author attempted to combine both quantification and isolation, but found that
the two goals can be mutually exclusive. The author summarizes the key
differences between quantification and isolation approaches as such:

\begin{itemize}
      \item Cell Manipulation for Harvest \textit{vs.} Readout:
            \begin{itemize}
                  \item Isolation methods are designed to manipulate cells for easy
                        harvest. For instance, the \ac{WPSC} method uses a catching
                        plate to collect non-adherent cells for subsequent analysis.
                  \item Quantification methods, on the other hand, manipulate cells
                        to simplify the readout process. For example, the V-Well
                        assay, which pellets cells into a single location, allowing
                        for a pooled fluorescence measurement without the need for
                        extensive cell handling.
            \end{itemize}

      \item Optimization for Subsequent Analysis \textit{vs.} Sample Throughput:
            \begin{itemize}
                  \item Isolation methods are optimized for detailed subsequent
                        analyses, such as RNA or protein analysis. For example,
                        \ac{WPSC} minimizes the introduction of biases such as those
                        from fluorescent staining, making it suitable for downstream
                        molecular assays.
                  \item Quantification methods are optimized for high sample
                        throughput. The V-Well assay, as an end-point assay, is
                        designed to efficiently handle multiple treatments
                        simultaneously, providing quick and comparative results with
                        lower cell numbers.
            \end{itemize}
      \item Handling of Cell Numbers:
            \begin{itemize}
                  \item Isolation methods, such as \ac{WPSC}, require multiple wells
                        (e.g., 96 wells) to gather a sufficient amount of cells per
                        subpopulation, which is crucial for robust downstream
                        analyses.
                  \item Quantification methods, exemplified by the V-Well assay, are
                        highly efficient even with low cell numbers.
            \end{itemize}
\end{itemize}



Thus, this adopted two distinct techniques for isolating and quantifying
directly interacting subpopulations, each optimizing for different outcomes, but
also supporting the separation of subpopulations within direct intercellular
interactions.

Still, it is theoretically possible to insert microscopy steps into the
\ac{WPSC} method to scan the well bottom for later cell counting. Also, this
work effectively isolated cell pellets from the V-well plate for subsequent
fixation and cell cycle profiling. The process was
tedious and required multiple technical replicates to achieve sufficient cell
numbers for analysis. It also required removing \ac{hMSC} from the V-well nadir
to prevent contamination during pellet aspiration.

Together, while both methods can combine quantification and isolation, they
are optimized towards either of them. Knowing these strengths and weaknesses
could help to advance these methods in future studies.






\textbf{Rationales of the Well Plate Sandwich Centrifugation}: Inspired by the
principles of both flipping and V-Well adhesion assays, we developed the Well
Plate Sandwich Centrifugation (\acf{WPSC}) method to address the challenges of
isolating cell populations. This method innovatively combines elements from both
techniques to provide a more reliable approach to cell isolation. One of the key
advantages of WPSC is its ability to reduce the variability commonly introduced
by manual pipetting. Instead of relying on aspiration, which introduce
variability in cell collection and requires touching the well bottom for
complete removal of medium, WPSC employs centrifugation to remove non-adhering
cells. Medium is then returned by pipetting to repeat the process and maximize
non-adhering cell collection, as the number of detachable cells plateau after
few rounds of centrifugation. Hence, this approach compromises
between minimizing washing variability and isolating larger quantities of cells.

The 96 well plate format has advantages, reducing spilling when flipping the
sandwich, as surface tension kept fluids in place. The 96 well plate format also
reduces per-well variability by performing the same washing procedure up to 96
times.

The slow centrifugation speeds used during \ac{WPSC} are also decided after
thorough consideration. For this, one has to discuss how exactly
non-adhering cells detach during centrifugation. While centrifugal force is an
obvious factor, the properties of cell adhesion are unclear under dry conditions
during centrifugation. The author assumed that the cells are being pulled along
by the medium as it is centrifuged into the catching plate. Hence, the centrifugation
speed was chosen as fast enough to transfer the medium, without completely
drying the co-culture plate and minimizing overall cell stress.

A significant challenge in \ac{WPSC} is the dissociation of \MAina from the hMSC
monolayer. WPSC employs two distinct techniques to achieve this dissociation.
The first technique involves repeated treatment with the gentle digestive enzyme
Accutase followed by \acp{MACS}. \ac{MACS}, despite being effective, is costly,
time-consuming, reduces overall cell yield, and potentially introduces biases
due to CD45 antibody selection and the requirement for cold-treatment. The
second technique utilizes strong pipetting to physically detach non-adhering
cells (termed \emph{`Wash'}). It is important to note that these techniques did
not affect the protocol on detaching \nMAina from the co-culture, hence
providing for a consistent ratio of isolated \MAina to \nMAina across all
experiments. Ultimately, we preferred \emph{Wash}, as \ac{MACS} had to be performed
on all samples to ensure comparability, reducing overall cell yield which became
limiting for downstream applications, especially for \nMAina cells. Both methods
achieved comparable purity of \MAina cells, with few hMSCs per $10e4$
\MAina cells (purity assessment not shown). \emph{Wash} probably pofited from
the highly durable nature of primary hMSC monolayers, whereas \emph{MACS}
required dissociation of the co-culture.

Together, \ac{WPSC} offers a versatile solution for isolating hMSC-interacting
myeloma cells. It successfully balances the need for precision with the ability
to handle larger cell quantities. \ac{WPSC} could be adapted to other cell types
that combines monolayer forming and suspension cells.



\textbf{Key Points:} Ultimately, this work established two methodologies
that could represent a significant advancement in the field of adhesion assays,
providing cost-effective, precise, reliable, and reproducible techniques for
both isolating and quantifying subpopulations within co-cultures of directly
interacting cell types. They offered valuable insights into the mechanisms of MM
detachment and are potentially applicable to other research questions that focus
on multicellular interactions and complex growth scenarios.


\unnsubsection{A Bottom Up Approach to Understanding Myeloma Dissemination}%
\label{sec:discussion_order_adhesion}%

%%%%%%%%%%%%%%%%%%%%%%%%%%%%%%%

Predicting when a myeloma cell starts regulating adhesion factors is a key
question in understanding dissemination.



Myeloma cells are isolated at certain sources at certain stages. This work
boldly defines disease stage and location as two dimensions with different implications
for adhesive behaviors.

The following paragraphs construct a narrative and checks for every step if there is
evidence for it in this work or the literature.

First let's construct a framework that's at least reasonable, but not necessarily
backed up by evidence:

Three dimensions where changes in adhesion factors are expected. These dimensions
make up a space, where every point describes an adhesive behavior of myeloma cells.
1 Location of Myeloma Cells (BM, vascular)
2 Disease Stage (asymptomatic MM, MM, MM relapse)
3 Cues that might trigger changes, or processes associated with changes or detachment


Why are these dimensions important and how could they be studied?

1 Location: Knowing how an MM cell can change their adhesive properties during its course of
dissemination is crucial for understanding the process itself. These changes
could be studied by tracking the expression of adhesion factors in MM cells at
different locations in mouse models. For humans, designing studies that gather
biopsies at different locations from the same patient, e.g. bone marrow and cirulating
myeloma cells could be a starting point.

2 Studying the adhesive changes during MM progression is
interesting, as it could unravel a specialized treatment strategy
that could maybe prevent dissemination.

3 The cues that trigger the detachment of MM cells are not well understood. It
could be that MM cells detach due to a combination of factors, such as loss of
adhesion factors, changes in the BM microenvironment, or cell division or
even completely random. Knowing specific dissemination signals helps preventing
dissemination.


How could these dimensions they be studied?

1 Location: These changes could be studied by tracking the expression of adhesion factors in MM cells at
different locations in mouse models. For humans, designing studies that gather
biopsies at different locations from the same patient, e.g. bone marrow and cirulating
myeloma cells could be a starting point.

2 Progression: Databases of expression from Myeloma cells gathered from bone
marrow \ac{MGUS}, \ac{aMM}, \ac{MM}, \ac{MMR} already exist
\citet{akhmetzyanovaDynamicCD138Surface2020, seckingerCD38ImmunotherapeuticTarget2018}

3 Cues: Identifying such signals might be challenging without
having understood the other two dimensions first.



What biological implications do these dimensions have?

1 Location of Myeloma Cells:
- Different locations could require different adhesion factors:
- Circulating MM cells do not need adhesion, probably losing adhesion factors
- BM cells express adhesion factors to adhere to the Bone marrow microenvironment (MSCs, adipocytes, and osteoblasts)
- Extravasating/intravasating cells need adhesion factors for endothelium
- Extramedullary cells need adhesion factors for respective tissues

2 Disease Stage:
- Higher disease stages imply changes in adhesion factors that favor aggressiveness.
- Aggressiveness includes:
- Better Colonization of new niches, including extramedullary ones
- This implies a more diverse set of available adhesion factors
- Faster regulation to adapt to new niches
- Better survival in circulation

3 Cues or associated processes:
- Different cues could trigger different adhesional changes
- Soluble signals?
- Loss of CD138 \cite{akhmetzyanovaDynamicCD138Surface2020}
- Detachment through cell division
-


What evidence is there that supports this framework?





\textbf{1 Location of Myeloma Cells}
\begin{itemize}
      \item \textbf{Other Findings}
            \begin{itemize}
                  \item The review by
                        \citet{zeissigTumourDisseminationMultiple2020} could be a
                        starting point. She does not discuss adhesion factors, but
                        seeing dissemination as a multistep process does imply
                        different adhesion factors for different steps.
                  \item Malignant Plasma Cells express different adhesion factors
                        than normal plasma cells \cite{cookRoleAdhesionMolecules1997, bouzerdanAdhesionMoleculesMultiple2022}.
                  \item For B-Cell Chronic Lymphocytic Leukemia, adhesion
                        molecule expression patterns define distinct phenotypes in
                        disease subsets \cite{derossiAdhesionMoleculeExpression1993}
            \end{itemize}

      \item \textbf{Extramedullary Involvement}
            \begin{itemize}
                  \item Extramedullary involvement: HCAM dramatic upregulation of HCAM
                  \item CXCR4, the homing receptor, mediates production of
                        adhesion factors in extramedullary MM cells \cite{roccaroCXCR4RegulatesExtraMedullary2015}
            \end{itemize}

      \item \textbf{Extravasation of Myeloma Cells}
            \begin{itemize}
                  \item Blocking Endothelial Adhesion through JAM-A decreases progression: \cite{solimandoHaltingViciousCycle2020}
                        % \item During the extravasation process, cells require specific
                        %       adhesion factors like JAM-C and CD138 to navigate and
                        %       adhere to the endothelium
                        %       \cite{asosinghUniquePathwayHoming2001a,
                        %             brandlJunctionalAdhesionMolecule2022}.

            \end{itemize}

      \item \textbf{Circulating Myeloma Cells}
            \begin{itemize}
                  \item Circulating plasma cells are rare, but detectable in peripheral blood
                        \cite{witzigDetectionMyelomaCells1996}
                        % \item Circulating MM cells typically lose adhesion factors like
                        %       integrin $\alpha4\beta1$ and CD138, which are crucial for their
                        %       transition from the BM to peripheral blood.

            \end{itemize}

      \item \textbf{BM-Resident Myeloma Cells}
            \begin{itemize}
                  \item INA-6 cells are highly adhesive to hMSCs, dynamically
                        upregulating adhesion factors when in direct contact with
                        hMSCs, and subsequently losing adhesion factor expression after
                        cell division [this work,
                                    \citet{kuricModelingMyelomaDissemination2024}].

                        % \cite{zeissigTumourDisseminationMultiple2020, akhmetzyanovaDynamicCD138Surface2020}.
                        % \item BM-resident MM cells maintain high levels of adhesion
                        %       molecules to interact with MSCs, adipocytes, and osteoblasts
                        %       within the BM niche \cite{bouzerdanAdhesionMoleculesMultiple2022, burgerGp130RasMediated2001, chatterjeePresenceBoneMarrow2002}.


                        % Extramedullary MM cells adapt by expressing adhesion
                        % molecules that facilitate attachment to the respective tissues
                        % they colonize \cite{blonskaJunRegulatedGenesPromote2015, blonskaMultipleMyelomaBone2022}.
            \end{itemize}

\end{itemize}

\begin{enumerate}
      \item \textbf{Disease Stage}
            \begin{itemize}
                  \item The idea that MM pathogenesis involves transformative
                        processes has been around for decades
                        \cite{hallekMultipleMyelomaIncreasing1998}, but a
                        detailed understanding of changing adhesive properties
                        is still lacking, especially during the progression of
                        MM.

                  \item At early stages, such as asymptomatic MM, cells primarily
                        retain their BM niche with minimal alteration in adhesion factor
                        expression \cite{abdallahModeProgressionSmoldering2024, akhmetzyanovaDynamicCD138Surface2020}.
                        % \item As MM progresses to symptomatic stages, there is a
                        %       diversification in adhesion factor expression, enabling the cells
                        %       to adhere to various new niches, including extramedullary sites
                        %       \cite{asosinghUniquePathwayHoming2001a, akhmetzyanovaDynamicCD138Surface2020, dotterweichContactMyelomaCells2016}.
                        % \item In relapse, MM cells show enhanced regulatory flexibility in
                        %       adhesion factor expression, facilitating rapid dissemination and
                        %       colonization of different tissues
                        %       \cite{keatsClonalCompetitionAlternating2012, bobinRecentAdvancesTreatment2022}.
            \end{itemize}

      \item \textbf{Cues or Processes}
            \begin{itemize}
                  \item Soluble signals within the BM microenvironment, such as
                        cytokines and chemokines, play significant roles in modulating
                        adhesion factor expression in MM cells
                        \cite{aggarwalChemokinesMultipleMyeloma2006, alsayedMechanismsRegulationCXCR42007}.
                        % \item The dynamic expression of CD138, in particular, has been
                        %       implicated in the regulation of MM cell detachment and circulation
                        %       \cite{akhmetzyanovaDynamicCD138Surface2020}.
                        % \item Additionally, cell division-related detachment can
                        %       contribute to MM cell dissemination, as cells divide and detach,
                        %       facilitating movement to new niches
                        %       \cite{dotterweichContactMyelomaCells2016, chatterjeePresenceBoneMarrow2002}.
            \end{itemize}
\end{enumerate}

% Circulating MM cells typically lose adhesion factors like integrin α4β1 and
% CD138, which are crucial for their transition from the BM to peripheral
% blood \citet{zeissigTumourDisseminationMultiple2020},
% \citet{akhmetzyanovaDynamicCD138Surface2020}. Conversely, BM-resident MM cells
% maintain high levels of adhesion molecules to interact with MSCs, adipocytes,
% and osteoblasts within the BM
% niche \citet{bouzerdanAdhesionMoleculesMultiple2022},
% \citet{burgerGp130RasMediated2001}, \citet{chatterjeePresenceBoneMarrow2002}.
% During the extravasation process, cells require specific adhesion factors like
% JAM-C and CD138 to navigate and adhere to the
% endothelium\citet{asosinghUniquePathwayHoming2001a},
% \citet{brandlJunctionalAdhesionMolecule2022}. Extramedullary MM cells adapt by
% expressing adhesion molecules that facilitate attachment to the respective
% tissues they colonize\citet{blonskaJunRegulatedGenesPromote2015},
% \citet{blonskaMultipleMyelomaBone2022}.

% 2. Disease Stage

% At early stages, such as asymptomatic MM, cells primarily retain their BM niche
% with minimal alteration in adhesion factor
% expression\citet{abdallahModeProgressionSmoldering2024},
% \citet{akhmetzyanovaDynamicCD138Surface2020}. As MM progresses to symptomatic
% stages, there is a diversification in adhesion factor expression, enabling the
% cells to adhere to various new niches, including extramedullary
% sites\citet{asosinghUniquePathwayHoming2001a},
% \citet{akhmetzyanovaDynamicCD138Surface2020},
% \citet{dotterweichContactMyelomaCells2016}. In relapse, MM cells show enhanced
% regulatory flexibility in adhesion factor expression, facilitating rapid
% dissemination and colonization of different
% tissues\citet{keatsClonalCompetitionAlternating2012},
% \citet{bobinRecentAdvancesTreatment2022}.

% 3. Cues or Processes

% Soluble signals within the BM microenvironment, such as cytokines and
% chemokines, play significant roles in modulating adhesion factor expression in
% MM cells\citet{aggarwalChemokinesMultipleMyeloma2006},
% \citet{alsayedMechanismsRegulationCXCR42007}. The dynamic expression of CD138,
% in particular, has been implicated in the regulation of MM cell detachment and
% circulation\citet{akhmetzyanovaDynamicCD138Surface2020}. Additionally, cell
% division-related detachment can contribute to MM cell dissemination, as cells
% divide and detach, facilitating movement to new
% niches\citet{dotterweichContactMyelomaCells2016},
% \citet{chatterjeePresenceBoneMarrow2002}.



%%%%%%%%%%%%%%%%%%%%%%%%


% ======================================================================
\unnsubsection{Dynamic and Niche-Dependent Regulation of Adhesion Factors}%
\label{sec:discussion_dynamic_regulation}%


% \textbf{Colonizing New Niches:}
This work showed that \INA cells dynamically upregulate adhesion factors when in
direct contact with \acp{hMSC}. Such adhesion factors are not expressed by \INA
cells without contact to \acp{hMSC}, or by \INA cells emerging as daughter cells
from \MAina cells. This implies that myeloma cells are capable of rapid changes in
adhesion factor expression that are substantially dynamic.
Predicting when a myeloma cell starts regulating adhesion factors is a key
question in understanding dissemination.

The following paragraphs
discuss how the idea of dynamic adhesion factor expression holds up
against current knowledge.



This is in line
substantial dynamics of
myeloma cells to regulate adhesion factors according to their environment.



This implies that myeloma cells
dynamically regulate adhesion factors during colonization of new niches.




INA-6 was initially isolated from plasma cell leukemia as an extramedullary
plasmacytoma located in the pleura from a donor of age.




% \textbf{Extrapolating Dynamic Adhesion Factor Expression onto other Disseminative Steps?:}

For example,
circulating MM cells show lower levels of integrin $\alpha4\beta1$
compared to those residing in the BM. Furthermore, treatment with a syndecan-1 blocking antibody
has been shown to rapidly induce the mobilization of MM cells from the BM to
peripheral blood in mouse models, suggesting that alterations in adhesion
molecule expression facilitate MM cell release
\cite{zeissigTumourDisseminationMultiple2020}.



% \textbf{Losing Adhesion Factors During Progression:}
There is not much more information available on the background of that patient \cite{TwoNewInterleukin6,burgerGp130RasMediated2001}.
But assuming that
This is a highly advanced
stage of myeloma.
However,  Chapter 1 shows that adhesion factors are
lost during MM progression. INA-6 are highly adhesive to hMSCs.
This is a contradiction that needs to be resolved.





% However, INA-6 do not express adhesion factors. They do that only in hMSC presence
% Hence MAINA-6 could be a smaller fraction of MM cells, specialized on preparing a new niche
% for the rest of the MM cells. This could be a reason why they are so adhesive.

This assumption dictates that aggressive myeloma cells gain the ability
to dynamically express adhesion factors.
It could be that INA-6 has gained the capability to express adhesion factors
fast in order to colonize new niches, such as pleura from which they were
isolated.

This shows that not just the stage of the disease, but also the location of the
myeloma cells plays a role when considering adhesion factors.

According to this, this thesis
predicts a low expression of adhesion factors in circulating myeloma cells,
but a high expression in adhesive cells, e.g. non-circulating, or rather those

indeed CD138 paper isolated cells from circulating MM cells \cite{akhmetzyanovaDynamicCD138Surface2020}

indeed, temporal subclones have been identified \cite{keatsClonalCompetitionAlternating2012}.

% ======================================================================
\unnsubsection{Subsets of Adhesion Factors Contribute To Different Steps of Dissemination}%
\label{sec:discussion_subsets_adhesion_factors}% 

- adhesion molecules during vascular involvement have these adhesion molecules: JAM-C
and CD138.
- NONE of Them were shown in Chapter 2 of this study, (except for JAM-B)


- One has to consider that intravasation and/or extravasation would require a different
set of adhesion factors than adhesion to BM or extramedullary environments.

This has great implications for targeting adhesion factors for therapy, as it
suggests that different adhesion factors should either be antagonized or
agonized depending on the function of the adhesion factor. According to this
assumption, adhesion factors involved in intra- and extravasation adhesion should be
antagonized, while adhesion factors involved in BM adhesion \dashed{as
      identified in Chapter 2} should be agonized. Indeed, Adhesion factors for endothelium
were shown to decrease tumour burden in mouse models \cite{asosinghUniquePathwayHoming2001a,mrozikTherapeuticTargetingNcadherin2015}

\citet{bouzerdanAdhesionMoleculesMultiple2022}: "Classically, the BMM has been
divided into endosteal and vascular niches"

Together, a detailed mapping of the niches available in the bone marrow is required
to understand the adhesion factors required for each niche. This is a highly
complex task, as the bone marrow is a highly complex organ.



% ======================================================================
\unnsubsection{What Triggers Release: One Master Switch, Many Small Switches, or is it just Random?}%
\label{sec:discussion_many_small_switches}%

Papers like \citet{akhmetzyanovaDynamicCD138Surface2020} make it seem as if
there is one molecule that decides if a myeloma cell is circulating or not.

It's less about one clear (molecular) mechanism that decides that a myeloma cell
decides to become a disseminating cell, but rather a indirect consequence of a combination of many
processes.
These processes are:
- Loss of adhesion factors or dynamic expression of adhesion factors
- Loss of dependency from bone marrow microenvironment
- asdf

Our thesis postulates that there is no big switch that decides if a myeloma cell
detaches from the bone marrow, \emph{it simply happens} once these processes are
present.


% ======================================================================
\unnsubsection{Outlook: High-Value Research Topics for Myeloma Research Arising from this Work}
\label{sec:discussion_potential_breakthroughs}
As an Outlook, the Author lists research topics arising from this work that have
great potential for breakthroughs in myeloma research.

\textbf{Anti tumor effects of MSCs:}
This thesis has discussed the pro-tumor effects of MSCs. However, MSCs have also
been shown to have anti-tumor effects \cite{galderisiMyelomaCellsCan2015}. This
work has also shown that primary \acp{hMSC} can induce apoptosis in \INA6 cells
initially \dashed{probably through the action of death domain receptors},
but inhibit apoptosis during prolonged culturing.

This shows that hMSCs could be leveraged
as a therapeutic target that could prevent myloma progression.




\textbf{Cell Division as a Mechanism for Dissemination Initiation:}
The author describes how the detachment of daughter cells from the mother cell
after a cycle of hMSC-(re)attachment and proliferation could be a key mechanism
in myeloma dissemination. This mechanism was shown in other studies of
extravasation. The author sees great potential in this mechanism as a target for
future research. It is probably under-researched due to requirement of
sophisticated time-lapse equipment, yet the simplicity of detachment through
cell division is intriguing through its simplicity. It implies asymmetric cell
division. Cancer cells are known to divide asymmetrically, e.g. moving miRNAs to
one daughter cell.

% \textbf{Time as a Key Parameter:}
% The area Thermodynamics of started with scientists measuring how long it takes
% for gases to cool down. The author claims, by measuring the time it takes for
% cancer cells to detach could lead to breakthroughs in research of myeloma
% dissemination.

% - Cell adhesion is highly time-dependent.
% - Cell detachment is required for metastasis and dissemination
% -

% key mechanistic insights

% measuring the minimum time
% for detachments to begin, or the time required for daughter cells to re-attach
% to the hMSC monolayer. Such mechanistic insights



% Time-resolution was mostly
% limited by available disk space. Investing into more hard drives is worth it,
% since

\textbf{Lists of Adhesion Gene Associated With Prolonged Patient Survival:}
The author lists adhesion genes that are associated with prolonged patient
survival. These genes are highly expressed in myeloma samples from patients with
longer overall

At this time we could be on the verge of a new era of myeloma therapy,
including bi-specific antibodies and cell based approaches
\cite{moreNovelImmunotherapiesCombinations2023,
      engelhardtFunctionalCureLongterm2024}. Currently, available CAR-T Cell therapies
(ide-cel, cilta-cel) are extremely expensive, but show complete remission rates
of up to \SI{80}{\percent} and a 18-month progression free survival rate of
\SI{66}{\percent} \cite{bobinRecentAdvancesTreatment2022}. An affordable
``off-the-shelf'' CAR-T Cell product could become reality since the problem of
deadly graft-versus-host disease during allogeneic transplantation seems to be
solvable \cite{qasimMolecularRemissionInfant2017}, hence, research groups and
biotech companies are racing towards developing a safe allogeneic CAR-T Cell
technology \cite{depilOfftheshelfAllogeneicCAR2020}.


the list of genes could be good targets because the BM niche is highly hypoxic, car t cells
are not well, but directing them to the BM niche could increase efficacy.



% ======================================================================
\unnsubsection{\textit{\textbf{Conclusion\,2:} Cancer \& Myeloma \& Dissemination is bad}}%
\label{sec:discussion_conclusion_cancer}%

lorem ipsum yes yes very bad








