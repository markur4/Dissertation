



% ======================================================================
\unnsubsection{Isolating \& Quantifying Subpopulations within Cells in Direct Contact with MSCs}%
\label{sec:discussion_novel_methods}%
This project aimed to develop methodologies for isolating cells after direct
contact with \acp{hMSC}. The primary challenge was the scarcity of \textit{in
      vitro} methods that could effectively separate and isolate adhering cell
subpopulations for subsequent molecular analysis. Most available techniques
predominantly focus on the quantification of cell adhesion
\cite{khaliliReviewCellAdhesion2015, kashefQuantitativeMethodsAnalyzing2015},
and often employ indirect contact setups, complex micromanipulation, or are
unsuitable for using live \acp{hMSC} as the immobilizing surface. To address the
limitations of current adhesion assays, we developed and enhanced innovative
methodologies, specifically the \acf{WPSC} and V-Well adhesion assays.

\textbf{Variability of Washing Steps}: Given the complexity of the requirements,
this project first attempts relied on simple and traditional adhesion assays
that rely on manual washing steps \cite{humphriesCellAdhesionAssays2009}.
Washing involves aspirating the medium, dispensing washing buffer, and
potentially repeating these steps multiple times. This introduces variability
due to differences in pipetting techniques, which affect the accuracy of volume
transfer \cite{guanAssessingVariationsManual2023,
      pushparajRevisitingMicropipettingTechniques2020}. However, adhesion assays don't
rely on precise volume transfer, but accurate detachment of cells adhering at
the well bottom. This introduces a new set of considerations for the pipetting
technique, especially since cells are highly sensitive to shear forces applied
by fluid flow. From the author's experience with washing experiments and
subsequent microscopic evaluations (data not shown), several factors could
contribute to the variability of washing steps:
\begin{enumerate}
      \item The distance of the pipette tip from the well bottom, which decreases during aspiration.
      \item The position of the pipette tip relative to the well bottom (center or edge).
      \item The angle of the pipette tip.
      \item The speed of aspiration.
      \item Accidental or intended contact between the pipette tip and the cell layer.
      \item The residual volume left after aspiration.
      \item \textit{The same considerations apply when dispensing the washing buffer.}
\end{enumerate}


In addition to user-dependent factors, other variables such as the cells'
position on the well bottom can significantly impact the outcome. To the
author's experience, cells located at the edge of the well don't detach as
easily as those in the center, while cells touching the edge are almost
impossible to remove. This phenomenon is likely related to the \textit{boundary
      layer effect}, where fluids slow down near the edges of the well
\cite{weyburneNewThicknessShape2014}.

Together, since both user-dependent and independent factors can affect the
outcome of washing steps, adhesive assays that replace washing are highly
desirable. Still, since washing is straightforward and some variability is
alleviated by the disciplined execution of washing protocols, it remains a
common method for adhesion assays.




\textbf{Directly Interacting Cells Contain Unexplored Interaction Scenarios}: It
is evident that direct and indirect contact to \acp{MSC} have varying effects on
myeloma cells. That difference is crucial for understanding changes in the
\ac{BMME} during MM progression \cite{fairfieldMultipleMyelomaCells2020,
      dziadowiczBoneMarrowStromaInduced2022}. These studies utilize well-inserts to
co-culture myeloma cells in close \dashed{indirect} contact with MSCs. However,
such comparison of indirect \textit{vs.} direct co-culturing methods might not
fully represent the complexity of intercellular interactions scenarios found in
the \ac{BMME}. This is exemplified by this project, as it relied on the complex
growth behavior: \INA cells aggregated homotypically in direct
proximity to those adhering heterotypically to \acp{hMSC}, and detached
through cell division. Furthermore, such methods fail to capture the subtle
variations in paracrine signaling concentrations, where even a few micrometers
of distance could significantly alter cellular responses.

Such knowledge shifted this project's point of view as well: Initially, our
hypothesis focused on direct heterotypic interactions, not expecting a \nMAina
population, but rather subpopulations within \MAina cells that are separable by
varying adhesion strengths. Hence, our assay employed strict conditions favoring
one growth scenario \dashed{heterotypic interactions}, with co-cultures providing unlimited hMSC-surface
availability causing predominantly heterotypic adhesion, while the short
incubation time prevented the formation of aggregates. Despite these measures,
our assay still captured cells emerging from recent cell divisions rather than
from weak heterotypic adherence as initially hypothesized. This demonstrates the
robustness of our method in separating subpopulations that arising from unexpected
intercellular interaction scenarios. This can be a major an advantage over
methods that summarize direct interactions as one population. Analysing the
non-adhering subpopulation within directly interacting cells could provide
valuable insights not just in multiple myeloma, but also metastasis of other
cancer types.


\textbf{Minimizing Variability}:
There are innovative adhesion assays that both
support the isolation of nonadherent subpopulations from directly interacting
cells, and avoid variability introduced by washing steps.

One simple method involves flipping over a 96-well plate, with surface tension
preventing medium spills as non-adhering cells fall to the surface for
collection \cite{zepeda-morenoInnovativeMethodQuantification2011}. However, we
found that the medium in fact did spill occasionally (not shown). Other
approaches involve sealing the plate, such as with PCR plate seals, and using
centrifugation to separate cells \cite{reyesCentrifugationCellAdhesion2003,
      chenHighThroughputScreeningTest2021}. Despite our efforts, we could not
consistently avoid air bubbles, which, after flipping, would contact the cell
layer and create dry regions during centrifugation.

The V-Well adhesion assay does not flip, but collects non-adhering cells into
the nadir of V-shaped wells during centrifugation
\cite{weetallHomogeneousFluorometricAssay2001}. This work profited greatly from
this method, while \dashed{to our knowledge} being the first to use cell
monolayers as the immobilizing surface. We value this method for its precision,
as centrifugation applies a uniform and configurable force, while the readout
remains straightforward, relying on the total fluorescent brightness rather than
individual cell counting.




\textbf{Specializing in Quantifying Adhesion or Isolating Subpopulations}: Most
adhesion assays primarily focus on quantification rather than isolation. The
author attempted to combine both quantification and isolation, but found that
the two goals can be mutually exclusive. The author summarizes the key
differences between quantification and isolation approaches as such:

\begin{itemize}
      \item Cell Manipulation for Harvest \textit{vs.} Readout:
            \begin{itemize}
                  \item Isolation methods are designed to manipulate cells for easy
                        harvest. For instance, the \ac{WPSC} method uses a catching
                        plate to collect non-adherent cells for subsequent analysis.
                  \item Quantification methods, on the other hand, manipulate cells
                        to simplify the readout process. For example, the V-Well
                        assay, which pellets cells into a single location, allowing
                        for a pooled fluorescence measurement without the need for
                        extensive cell handling.
            \end{itemize}

      \item Optimization for Subsequent Analysis \textit{vs.} Sample Throughput:
            \begin{itemize}
                  \item Isolation methods are optimized for detailed subsequent
                        analyses, such as RNA or protein analysis. For example,
                        \ac{WPSC} minimizes the introduction of biases such as those
                        from fluorescent staining, making it suitable for downstream
                        molecular assays.
                  \item Quantification methods are optimized for high sample
                        throughput. The V-Well assay, as an end-point assay, is
                        designed to efficiently handle multiple treatments
                        simultaneously, providing quick and comparative results with
                        lower cell numbers.
            \end{itemize}
      \item Handling of Cell Numbers:
            \begin{itemize}
                  \item Isolation methods, such as \ac{WPSC}, require multiple wells
                        (e.g., 96 wells) to gather a sufficient amount of cells per
                        subpopulation, which is crucial for robust downstream
                        analyses.
                  \item Quantification methods, exemplified by the V-Well assay, are
                        highly efficient even with low cell numbers.
            \end{itemize}
\end{itemize}



Thus, this adopted two distinct techniques for isolating and quantifying
directly interacting subpopulations, each optimizing for different outcomes, but
also supporting the separation of subpopulations within direct intercellular
interactions.

Still, it is theoretically possible to insert microscopy steps into the
\ac{WPSC} method to scan the well bottom for later cell counting. Also, this
work effectively isolated cell pellets from the V-well plate for subsequent
fixation and cell cycle profiling. The process was
tedious and required multiple technical replicates to achieve sufficient cell
numbers for analysis. It also required removing \ac{hMSC} from the V-well nadir
to prevent contamination during pellet aspiration.

Together, while both methods can combine quantification and isolation, they
are optimized towards either of them. Knowing these strengths and weaknesses
could help to advance these methods in future studies.





\textbf{Rationales of the Well Plate Sandwich Centrifugation}: Inspired by the
principles of both flipping and V-Well adhesion assays, we developed the Well
Plate Sandwich Centrifugation (\acf{WPSC}) method to address the challenges of
isolating cell populations. This method innovatively combines elements from both
techniques to provide a more reliable approach to cell isolation. One of the key
advantages of WPSC is its ability to reduce the variability commonly introduced
by manual pipetting. Instead of relying on aspiration, which introduce
variability in cell collection and requires touching the well bottom for
complete removal of medium, WPSC employs centrifugation to remove non-adhering
cells. Medium is then returned by pipetting to repeat the process and maximize
non-adhering cell collection, as the number of detachable cells plateau after
few rounds of centrifugation. Hence, this approach compromises
between minimizing washing variability and isolating larger quantities of cells.

The 96 well plate format has advantages, reducing spilling when flipping the
sandwich, as surface tension kept fluids in place. The 96 well plate format also
reduces per-well variability by performing the same washing procedure up to 96
times.

The slow centrifugation speeds used during \ac{WPSC} are also decided after
thorough consideration. For this, one has to discuss how exactly
non-adhering cells detach during centrifugation. While centrifugal force is an
obvious factor, the properties of cell adhesion are unclear under dry conditions
during centrifugation. The author assumed that the cells are being pulled along
by the medium as it is centrifuged into the catching plate. Hence, the centrifugation
speed was chosen as fast enough to transfer the medium, without completely
drying the co-culture plate and minimizing overall cell stress.

A significant challenge in \ac{WPSC} is the dissociation of \MAina from the hMSC
monolayer. WPSC employs two distinct techniques to achieve this dissociation.
The first technique involves repeated treatment with the gentle digestive enzyme
Accutase followed by \acp{MACS}. \ac{MACS}, despite being effective, is costly,
time-consuming, reduces overall cell yield, and potentially introduces biases
due to CD45 antibody selection and the requirement for cold-treatment. The
second technique utilizes strong pipetting to physically detach non-adhering
cells (termed \emph{`Wash'}). It is important to note that these techniques did
not affect the protocol on detaching \nMAina from the co-culture, hence
providing for a consistent ratio of isolated \MAina to \nMAina across all
experiments. Ultimately, we preferred \emph{Wash}, as \ac{MACS} had to be performed
on all samples to ensure comparability, reducing overall cell yield which became
limiting for downstream applications, especially for \nMAina cells. Both methods
achieved comparable purity of \MAina cells, with few hMSCs per $10e4$
\MAina cells (purity assessment not shown). \emph{Wash} probably pofited from
the highly durable nature of primary hMSC monolayers, whereas \emph{MACS}
required dissociation of the co-culture.

Together, \ac{WPSC} offers a versatile solution for isolating hMSC-interacting
myeloma cells. It successfully balances the need for precision with the ability
to handle larger cell quantities. \ac{WPSC} could be adapted to other cell types
that combines monolayer forming and suspension cells.



\textbf{Key Points:} Ultimately, this work established two methodologies
that could represent a significant advancement in the field of adhesion assays,
providing cost-effective, precise, reliable, and reproducible techniques for
both isolating and quantifying subpopulations within co-cultures of directly
interacting cell types. They offered valuable insights into the mechanisms of MM
detachment and are potentially applicable to other research questions that focus
on multicellular interactions and complex growth scenarios.




%%%%%%%%%%%%
% ======================================================================

\newcommand{\footcadd}{%
      \emph{\acf{CADD}} (defined in this work): The observation and measurement
      of time-dependent changes in cell adhesion and detachment events.
      \ac{CADD} characterizes the time cells spend attached, migrating or
      detached and associates these phases with molecular signatures, such as
      \ac{CAM} expression or cell signaling mediated by \acp{CAM} or the
      microenvironment. \ac{CADD} expands traditional \emph{cell adhesion} by a
      time component and implies an intention to predict attachment and
      detachment events.
      %
}


% \unnsubsection{Constructing a Hypothetical Framework of Dissemination}%
\unnsubsection{Integrating Evidence and Hypotheses for a Mechanistic Understanding of Dissemination}%
\label{sec:discussion_framework}%
The results outlined in Chapter 1 encompass various aspects of multiple myeloma
research, including colonization of the \ac{BMME}, myeloma-\ac{MSC}
interactions, and the association of adhesion factor expression with patient
survival and disease stages. Such a broad scope invites the formulation of
generalized conclusions, potentially compromising scientific rigor. The
following sections aim to clearly separate hypotheses from evidence to guide
further research on dissemination.

\textbf{Integrating Observations of \INA in the Multistep Dissemination Model:}
The results gained in this work fit well into the multistep model proposed by
\citet{zeissigTumourDisseminationMultiple2020}. For most steps, observations
were made that could inspire further hypotheses and research:

% 1.	Retention:
% •	Observation: \INA cells attach quickly and strongly to \acp{hMSC}, forming stable aggregates.
% •	Hypothesis: Myeloma cells are retained in the bone marrow microenvironment (BMME) through strong adhesion to \acp{hMSC} and stable homotypic aggregation.
% •	Experiment: Inject \INA cells into mice and examine bone lesions. Compare the growth patterns in mice co-injected with an ICAM-1 or LFA-1$\alpha$/ITGB2 antibody, which dissolve homotypic aggregates \textit{in vitro} and prevent \INA growth \textit{in vivo} \cite{kawanoHomotypicCellAggregations1991a, klauszNovelFcengineeredHuman2017}. If disrupting aggregation leads to diffuse bone colonization rather than focal lesions, it supports the hypothesis that strong adhesion and aggregation are crucial for retention in the BMME.
% 2.	Release:
% •	Observation: \INA cells detach from \acp{hMSC} through cell division, and external forces can detach single cells from aggregates.
% •	Hypothesis: Myeloma cells detach from the BMME through cell division and external forces after reaching a minimal aggregate size.
% •	Experiment: Inject \INA cells into mice and compare the cell cycle profiles of circulating cells versus those in the bone marrow. Enrichment of G1/G0 cells among circulating cells would support the hypothesis that detachment is more likely shortly after cell division.
% 3.	Intra-/Extravasation:
% •	Observation: This study did not make specific predictions for intra-/extravasation, but these could be explored if MSCs were replaced by endothelial cells.
% 4.	Colonization:
% •	Observation: \INA cells exhibit quick attachment to \acp{hMSC} within one hour and rapidly upregulate numerous adhesion factors, including \ac{ECM} factors.
% •	Hypothesis: Quick attachment and fast expression of adhesion factors enhance the potential to colonize new niches. This is particularly relevant as \INA cells were isolated from the pleura, indicating an ability to colonize extramedullary sites \cite{burgerGp130RasMediated2001c}.
% •	Experiment: Inject \INA-6 cells into mice and observe if they colonize extramedullary sites. Compare this to \INA-6 cells with reduced adaptability to test the hypothesis.

\begin{enumerate}
      \item \textbf{Retention:}
            \begin{itemize}
                  \item \textit{Observation:} \INA cells attach quickly and
                        strongly to \acp{hMSC}, forming stable aggregates.
                  \item \textit{Hypothesis:} Myeloma cells are retained in the
                        bone marrow microenvironment (BMME) through strong adhesion to
                        \acp{hMSC} and stable homotypic aggregation.
                  \item \textit{Experiment:} Inject \INA cells into mice and
                        examine bone lesions. Compare the growth patterns in mice
                        co-injected with an ICAM-1 or LFA-1$\alpha$ antibody, which
                        dissolve homotypic aggregates \textit{in vitro} and prevent
                        \INA growth \textit{in vivo}
                        \cite{kawanoHomotypicCellAggregations1991a,
                              klauszNovelFcengineeredHuman2017}. If disrupting aggregation
                        leads to diffuse bone colonization rather than focal lesions,
                        it supports the hypothesis that strong adhesion and
                        aggregation are crucial for retention in the \ac{BMME}.
            \end{itemize}
      \item \textbf{Release:}
            \begin{itemize}
                  \item \textit{Observation:} \INA cells detach from \acp{hMSC}
                        through cell division, and external forces can detach single
                        cells from \INA aggregates.
                  \item \textit{Hypothesis:} Myeloma cells detach from the BMME
                        through cell division and external forces after reaching a
                        minimal aggregate size.
                  \item \textit{Experiment:} Inject \INA cells into mice and
                        compare the cell cycle profiles of circulating cells versus
                        those in the bone marrow. Enrichment of G1/G0 cells among
                        circulating cells would support the hypothesis that detachment
                        is more likely shortly after cell division.
            \end{itemize}
      \item \textbf{Intra-/Extravasation:}
            \begin{itemize}
                  \item This study did not make experiments to study for
                        intra-/extravasation, but these phenomena could be
                        explored with similar methods, if MSCs were replaced by
                        endothelial cells.
            \end{itemize}
      \item \textbf{Colonization:}
            \begin{itemize}
                  \item \textit{Observation:} \INA cells exhibit quick
                        attachment to \acp{hMSC} within one hour and rapidly
                        upregulate numerous adhesion factors, including \ac{ECM}
                        factors.
                  \item \textit{Hypothesis:} Quick attachment and fast
                        expression of adhesion factors enhance the potential to
                        colonize new niches. This is particularly relevant as \INA
                        cells were isolated from the pleura, indicating an ability to
                        colonize extramedullary sites
                        \cite{burgerGp130RasMediated2001c}.
                  \item \textit{Experiment:} Inject \INA cells into mice and
                        observe if they colonize extramedullary sites. Compare this to
                        \INA cells with reduced adaptability to test the hypothesis.q
            \end{itemize}
\end{enumerate}


These hypotheses  \dashed{based on observations from \INA cells} provide a starting
point for understanding myeloma dissemination. While these insights are
specialized for the \INA cell line, they inspire the development of a more
generalized framework applicable to a broader range of myeloma cases.

\textbf{Constructing a Generalizable Hypothetical Framework of Dissemination:}
A mechanistic understanding of myeloma dissemination remains elusive. Although
\citet{zeissigTumourDisseminationMultiple2020} described dissemination as a
multistep process, evidence is largely collected for individual steps, leaving
the connections between these steps unproven. As a result, the process of
dissemination is a patchwork of evidence fragments. The following sections aim
to integrate such fragments, especially those derived from the \INA cell line in
this work, to construct a more coherent understanding of myeloma dissemination.

In this context, the author introduces the \emph{Dynamic Adhesion Hypothetical Framework for
      Myeloma Dissemination}, which leverages direct observations of
\acf{CADD}\footnote{\footcadd\label{foot:cadd}}. \ac{CADD} characterizes
the time-dependent changes in cell adhesion and detachment, associating these
phases with molecular signatures like \ac{CAM} expression or cell signaling
mediated by \acp{CAM} and the microenvironment. By adding a temporal component,
\ac{CADD} aims to predict attachment and detachment events.



\newcommand{\caddadaptation}{ %
      \ac{CADD} is adapted in response to different microenvironments faced
      during dissemination %
}
\newcommand{\caddadaptationtitle}{ %
      \textit{Hypothesis 1}: \ac{CADD} is Adapted during Dissemination%
}%


\newcommand{\caddadaptibility}{ %
      High adaptability of \ac{CADD} is a hallmark of aggressive myeloma %
}%
\newcommand{\caddadaptabilitytitle}{ %
      \textit{Hypothesis 2}: High Adaptability of \ac{CADD} is a Hallmark of
      Aggressive Myeloma %
}%


\newcommand{\cadddiversity}{%
      \ac{CADD} is highly diverse within both patients and cell lines %
}%
\newcommand{\cadddiversitytitle}{ %
      \textit{Hypothesis 3}: \ac{CADD} is Highly Diverse Within both Patients
      and Cell Lines%
}%


\newcommand{\caddtrigger}{%
      Detachment is caused by multiple cues of varying nature, including
      external mechanical forces, cell division, loss of \ac{CAM} expression, or
      even pure chance. }%    
\newcommand{\caddtriggertitle}{ %
      \textit{Hypothesis 4}: Detachment is Caused by Multiple Cues of Varying
      Nature %
}%


\textbf{Key Hypotheses:}
The Dynamic Adhesion Hypothetical Framework is structured around four key
hypotheses, each addressing fundamental aspects of myeloma cell dissemination
based on both literature and the results of this work. These hypotheses are as
follows:

\begin{enumerate}[parsep=4pt]
      \item \caddadaptation
      \item \caddadaptibility
      \item \cadddiversity
      \item \caddtrigger
\end{enumerate}


This framework sets the stage for a detailed exploration of each hypothesis,
linking empirical data with hypothetical constructs to provide a comprehensive
framework that can help to identify commonalities in myeloma dissemination, but
also inform the development of targeted therapies.




% ======================================================================
\unnsubsection{\caddadaptationtitle}%
\label{sec:discussion_caddadaptation}%


Chapter 1 demonstrates a paradox in multiple myeloma (MM) progression: adhesion
factors are typically lost as the disease advances, yet \INA cells exhibit high
adhesion to \acp{hMSC}. This contradiction suggests that adhesion factor
expression in myeloma cells may be dynamically regulated. Specifically, \INA
cells do not inherently express adhesion factors; rather, they do so only in the
presence of \acp{hMSC}. This indicates that \INA cells may represent a
specialized subpopulation of MM cells, primed to prepare new niches for
colonization.

The concept of dynamic adhesion factor expression is supported by the behavior
of \INA cells, which upregulate these factors only when in direct contact with
\acp{hMSC}. This dynamic regulation implies that myeloma cells can rapidly alter
their adhesion properties in response to their microenvironment. This
adaptability is crucial for various stages of dissemination, including
intravasation and extravasation, which require different sets of adhesion
factors compared to those needed for bone marrow (BM) adhesion or extramedullary
colonization.

Research indicates that plasma cells dynamically upregulate adhesion factors
upon reaching target tissues, a process that appears to be mirrored in myeloma
cells. Predicting the regulation of adhesion factors is vital for understanding
myeloma dissemination. Different stages of dissemination necessitate distinct
adhesion molecules: circulating MM cells likely downregulate adhesion factors,
BM-resident cells upregulate them to interact with the microenvironment, and
extravasating/intravasating cells require specific adhesion factors for
endothelial interaction.

The significance of adhesion in myeloma dissemination is exemplified in this
study, where \INA cells dynamically upregulate adhesion factors in the presence
of \acp{hMSC}. This dynamic regulation aligns with the idea that dissemination
is a multistep process, as described by
\citet{zeissigTumourDisseminationMultiple2020}. During their lifetime, myeloma
cells may alter their adhesion profiles to adapt to different microenvironments,
facilitating tasks such as exiting the BM or intravasation.

Understanding how myeloma cells change their adhesion properties during
dissemination is crucial. Tracking the expression of adhesion factors in MM
cells at various locations in mouse models and designing studies to gather
biopsies from different patient sites, such as bone marrow and circulating
myeloma cells, can provide valuable insights.
\citet{bouzerdanAdhesionMoleculesMultiple2022} highlighted the complexity of the
BM microenvironment, dividing it into endosteal and vascular niches, each
requiring specific adhesion molecules.

For instance, malignant plasma cells express different adhesion factors than
normal plasma cells, and these molecules have been targeted for therapy for over
a decade \cite{cookRoleAdhesionMolecules1997,
bouzerdanAdhesionMoleculesMultiple2022, nairChapterSixEmerging2012}. In other
cancers, distinct adhesive subtypes, separated through processes like
epithelial-mesenchymal transition (EMT), are common
\cite{gengDynamicSwitchTwo2014}. This variability suggests that myeloma cells
may similarly possess diverse adhesion profiles, adapting dynamically to their
environment.

Extramedullary involvement of myeloma cells further underscores the importance
of dynamic adhesion regulation. CXCR4, a homing receptor, mediates the
production of adhesion factors in extramedullary MM cells, supporting the idea
that different environments prompt distinct adhesion responses
\cite{roccaroCXCR4RegulatesExtraMedullary2015}. Blocking endothelial adhesion
via molecules like JAM-A has been shown to decrease disease progression,
emphasizing the therapeutic potential of targeting specific adhesion pathways
\cite{solimandoHaltingViciousCycle2020}.

Additionally, the role of CXCL12, highly expressed by \acp{hMSC}, in inducing
adhesion factors in MM is well established. This work demonstrates that \INA-6
cells are highly adhesive to \acp{hMSC}, dynamically upregulating adhesion
factors upon direct contact and subsequently losing these factors post-cell
division \cite{ullahRoleCXCR4Multiple2019, burgerGp130RasMediated2001,
chatterjeePresenceBoneMarrow2002}. Such findings suggest that myeloma cells
maintain high levels of adhesion molecules to interact with the BM niche,
adapting their adhesive properties as needed.

The implications for therapy are significant. Understanding the dynamic nature
of adhesion factor expression suggests that targeting these factors could be
beneficial. However, it requires a nuanced approach: adhesion factors involved
in intravasation and extravasation should be antagonized, while those
facilitating BM adhesion should be agonized. Studies have shown that targeting
endothelial adhesion molecules can decrease tumor burden in mouse models,
underscoring the therapeutic potential of this strategy
\cite{asosinghUniquePathwayHoming2001a,
mrozikTherapeuticTargetingNcadherin2015}.

In conclusion, the dynamic adaptation of \ac{CADD} during myeloma dissemination
highlights the complexity and variability of adhesion factor expression. A
detailed mapping of the BM niches and a comprehensive understanding of the
required adhesion factors for each niche are essential. This work lays the
foundation for future studies to explore these dynamics, ultimately informing
targeted therapies aimed at disrupting myeloma dissemination.



% However,  Chapter 1 shows that adhesion factors are
% lost during MM progression. INA-6 are highly adhesive to hMSCs.
% This is a contradiction that needs to be resolved.

% One explanation is the dynamic change of adhesion factor expression.

% However, INA-6 do not express adhesion factors. They do that only in hMSC presence
% Hence MAINA-6 could be a smaller fraction of MM cells, specialized on preparing a new niche
% for the rest of the MM cells. This could be a reason why they are so adhesive.

% - One has to consider that intravasation and/or intra-/extravasation would require a different
% set of adhesion factors than adhesion to BM or extramedullary environments.

% Extravasation: Plasma cells are known to upregulate adhesion factors dynamically
% once they reach a target tissue (???).

% This work showed that \INA cells dynamically upregulate adhesion factors when in
% direct contact with \acp{hMSC}. Such adhesion factors are not expressed by \INA
% cells without contact to \acp{hMSC}, or by \INA cells emerging as daughter cells
% from \MAina cells. This implies that myeloma cells are capable of rapid changes in
% adhesion factor expression that are substantially dynamic.
% Predicting when a myeloma cell starts regulating adhesion factors is a key
% question in understanding dissemination.

% The following paragraphs
% discuss how the idea of dynamic adhesion factor expression holds up
% against current knowledge.

% What biological implications does CADD adaption have?
% - Different locations could require different adhesion factors:
% - Circulating MM cells do not need adhesion, probably losing adhesion factors
% - BM cells express adhesion factors to adhere to the Bone marrow microenvironment (MSCs, adipocytes, and osteoblasts)
% - Extravasating/intravasating cells need adhesion factors for endothelium
% - Extramedullary cells need adhesion factors for respective tissues


% \citet{bouzerdanAdhesionMoleculesMultiple2022}: "Classically, the BMM has been
% divided into endosteal and vascular niches"

% Overall, cell adhesion play a pivotal role in the attachment/detachment dynamics of
% myeloma, hence influencing the dissemination of myeloma cells. This is
% exemplified in this work, where \INA cells dynamically upregulate adhesion
% factors in direct contact with \acp{hMSC}. Predicting how and when myeloma cells
% regulate adhesion activity is a key question in understanding dissemination,
% since that

% Myeloma cells are isolated from patients at a certain stage from a certain
% location. As summarized by \citet{zeissigTumourDisseminationMultiple2020},
% dissemination could be a dynamic process during the lifetime of a myleoma cell
% that managed to exit the \ac{BMME} into blood circulation. This implies that
% myeloma cells could change their adhesion factors during their course of
% dissemination to adapt to their current location for specialized tasks like
% exiting the \ac{BMME} or intra-/extravasation.

% why important?

% Knowing how an MM cell can change their adhesive properties during its course of
% dissemination is crucial for understanding the process itself. These changes
% could be studied by tracking the expression of adhesion factors in MM cells at
% different locations in mouse models. For humans, designing studies that gather
% biopsies at different locations from the same patient, e.g. bone marrow and cirulating
% myeloma cells could be a starting point.

% How studied?
% These changes could be studied by tracking the
% expression of adhesion factors in MM cells at different locations in mouse
% models. For humans, designing studies that gather biopsies at different
% locations from the same patient, e.g. bone marrow and cirulating myeloma cells
% could be a starting point.


Literature:

\textbf{1 Location of Myeloma Cells}
\begin{itemize}
      \item \textbf{Other Findings}
            \begin{itemize}
                  \item The review by
                        \citet{zeissigTumourDisseminationMultiple2020} could be
                        a starting point. She does not discuss adhesion factors,
                        but seeing dissemination as a multistep process does
                        imply different adhesion factors for different steps.
                  \item Malignant Plasma Cells express different adhesion factors
                        than normal plasma cells \cite{cookRoleAdhesionMolecules1997, bouzerdanAdhesionMoleculesMultiple2022}.

                  \item Adhesion molecules have been a popular target for therapy for a decade \cite{nairChapterSixEmerging2012}
                  \item In other cancers different adhesive subtypes are common and are molecularly clearly separated through \ac{EMT} \cite{gengDynamicSwitchTwo2014}
            \end{itemize}

      \item \textbf{Extramedullary Involvement}
            \begin{itemize}
                  \item Extramedullary involvement:  dramatic upregulation of H-CAM/CD44 \cite{}
                  \item CXCR4, the homing receptor, mediates production of
                        adhesion factors in extramedullary MM cells \cite{roccaroCXCR4RegulatesExtraMedullary2015}
            \end{itemize}

      \item \textbf{Intra-/Extravasation of Myeloma Cells}
            \begin{itemize}
                  \item Blocking Endothelial Adhesion through JAM-A decreases progression: \cite{solimandoHaltingViciousCycle2020}
                  \item N-Cadherin is upregulated in MM compared to healthy plasma cells, and has been shown to be a potential target for therapy \cite{mrozikTherapeuticTargetingNcadherin2015}
                  \item - NONE of Them were shown in Chapter 1 of this study, (except for JAM-B)
            \end{itemize}

      \item \textbf{Circulating Myeloma Cells}
            \begin{itemize}
                  \item This work shows that \nMAina have increased survival
                        during IL-6 deprivation, which could be a mechanism for
                        surviving in circulation.
                  \item Circulating plasma cells are rare, but detectable in
                        peripheral blood
                        \cite{witzigDetectionMyelomaCells1996}
                  \item studies demonstrate that circulating \ac{MM} cells
                        exhibit reduced levels of integrin $\alpha4\beta1$, in
                        contrast to those located in the \ac{BM}
                        \cite{paivaDetailedCharacterizationMultiple2013,
                              paivaCompetitionClonalPlasma2011}
                  \item circulating MM cells were CD138/Syndecan-1 negative
                        \cite{akhmetzyanovaDynamicCD138Surface2020}

            \end{itemize}

      \item \textbf{BM-Resident Myeloma Cells}
            \begin{itemize}
                  \item The role of CXCL12 \dashed{which is highly expressed by
                              MSCs} in inducing adhesion factors in MM is well established
                        \cite{ullahRoleCXCR4Multiple2019}
                  \item THIS WORK: INA-6 cells are highly adhesive to hMSCs, dynamically
                        upregulating adhesion factors when in direct contact with
                        hMSCs, and subsequently losing adhesion factor expression after
                        cell division
                  \item BM-resident MM cells maintain high levels of adhesion
                        molecules to interact with MSCs, adipocytes, and osteoblasts
                        within the BM niche \cite{bouzerdanAdhesionMoleculesMultiple2022, burgerGp130RasMediated2001, chatterjeePresenceBoneMarrow2002}.
            \end{itemize}

\end{itemize}


% According to this, this thesis
% predicts a low expression of adhesion factors in circulating myeloma cells,
% but a high expression in adhesive cells, e.g. non-circulating, or rather those


% This has huge implications for studying adhesion factors in MM \textit{in
%       vitro}. Given that some factors are not present in MM cells, but are potentially
% rapidly expressed with the right signal. Hence, further studies focusing on
% adhesion factor expression \textit{in vitro} should provide one specific
% microenvironmental context, and not generalize to all available niches.


% This has great implications for targeting adhesion factors for therapy, as it
% suggests that different adhesion factors should either be antagonized or
% agonized depending on the function of the adhesion factor. According to this
% assumption, adhesion factors involved in intra- and extravasation adhesion should be
% antagonized, while adhesion factors involved in BM adhesion \dashed{as
%       identified in Chapter 2} should be agonized. Indeed, Adhesion factors for endothelium
% were shown to decrease tumour burden in mouse models \cite{asosinghUniquePathwayHoming2001a,mrozikTherapeuticTargetingNcadherin2015}

% Together, a detailed mapping of the niches available in the bone marrow is required
% to understand the adhesion factors required for each niche. This is a highly
% complex task, as the bone marrow is a highly complex organ.


% ======================================================================
\unnsubsection{\caddadaptabilitytitle}%
\label{sec:discussion_caddadaptability}%

biological implications:
Disease Stage:
- Higher disease stages imply changes in adhesion factors that favor aggressiveness.
- Aggressiveness includes:
- Better Colonization of new niches, including extramedullary ones
- This implies a more diverse set of available adhesion factors
- Faster regulation to adapt to new niches
- Better survival in circulation



This assumption dictates that aggressive myeloma cells gain the ability
to dynamically express adhesion factors.
It could be that INA-6 has gained the capability to express adhesion factors
fast in order to colonize new niches, such as pleura from which they were
isolated.


indeed, 3 temporal subtypes have been identified, associating higher risk with
faster changes over time \cite{keatsClonalCompetitionAlternating2012}.

Is Disease stage a proxy for tumor aggressiveness?


yes, adhesion has prognostic value: A recent study by
\citet{huDevelopmentCellAdhesionbased2024} developed a cell adhesion-based
prognostic model for MM, calculating an adhesion-related risk score (ARRS) based
on expression of only twelve adhesion related genes.


Supporting Literature:

\begin{enumerate}
      \item \textbf{Disease Stage}
            \begin{itemize}
                  \item THIS WORK: Expression decreases during progression from
                        \ac{MGUS} to \ac{MMR} of adhesion factors involved in hMSC
                        adhesion.
                  \item The idea that MM pathogenesis involves transformative
                        processes has been around for decades
                        \cite{hallekMultipleMyelomaIncreasing1998}, but a
                        detailed understanding of changing adhesive properties
                        is still lacking, especially during the progression of
                        MM.
                  \item It is discussed that myeloma cell lines derived from
                        advanced stages show different expression than newly
                        diagnosed patients, discussing that they come from
                        multiply relapsed patients
                        \cite{sarinEvaluatingEfficacyMultiple2020}. This work
                        also shows that Myeloma cell lines have the lowest
                        expression of adhesion factors compared to all stages of
                        \ac{MM} and \ac{MGUS}.
                  \item For B-Cell Chronic Lymphocytic Leukemia, adhesion
                        molecule expression patterns define distinct phenotypes in
                        disease subsets \cite{derossiAdhesionMoleculeExpression1993}.
                  \item \citet{terposIncreasedCirculatingVCAM12016} reported an
                        increase in adhesion molecule expression of ICAM-1 and
                        VCAM-1 in patients with \ac{MM} compared to those with
                        \ac{MGUS} and \ac{aMM}.
                  \item However, \citet{perez-andresClonalPlasmaCells2005}
                        reported that CD40 is downregulated in \ac{PCL}
                        patients. Hence, different \acp{CAM} could serve
                        ambiguous roles in \ac{MM} progression.
            \end{itemize}

\end{enumerate}



How could this be studied?

Databases of expression from Myeloma cells gathered from bone
marrow \ac{MGUS}, \ac{aMM}, \ac{MM}, \ac{MMR} already exist
\citet{akhmetzyanovaDynamicCD138Surface2020,
      seckingerCD38ImmunotherapeuticTarget2018}. Going through such databases gives a
good overview. One could categorize genelists using curated databases, get lists
associated with extravasation, intravasation, Bone marrow adhesion. For every
gene of these genelists, they could be filtered for significant differences
between the stages. Further categorizations of pairwise comparisons of stages
are required. but overall, these genelists could be a starting point for This
approach is similar to the genelists published in chapter 1, with the difference
that the genelist was furthere filtered by the RNAseq results of \textit{in
      vitro} experiments.



What new implications do these dimensions have on targeting adhesion factors for
therapy?

- Specialized treatment for each stage?
- Aggressive MM cells have potantial improved control over adhesion factor expression,
regulating a more diverse set of adhesion factors faster. This poses further challenges to targeting.
It could be smarter to not target effector-molecules, but rather upstream regulators of adhesion.
THis work shows that NF-kappaB signaling, which by itself is not treatable, but regulators
downstream of NF-kappaB were shown to be effective \cite{adamikEZH2HDAC1Inhibition2017,adamikXRK3F2InhibitionP62ZZ2018}




% ======================================================================
\unnsubsection{\cadddiversitytitle}%
\label{sec:discussion_cadddiversity}%

- Describe different cell lines: MM1.S being plastic adhering non-aggregating and moderately MSC-adhering, INA-6 being non adhering aggregate forming and MSC-adhering, U266 being plastic adhering, non-aggregating and non MSC-adhering.
- Results from this work: CXCL12 expresion varies from QM between QM

One important dimension that is missing here is the genetic background of the
myeloma cells. These are based on recurrent patterns of chromosomal aberrations
or mutational signatures, defining structural and single nucleotide variants
\cite{kumarMultipleMyelomasCurrent2018a,
      hoangMutationalProcessesContributing2019}. The prognostic value of genetic
variants in MM is well established \cite{sharmaPrognosticRoleMYC2021}, and their
identification is becoming precise and cost-effective using \emph{optical
      genome mapping}, making progress towards personalized therapies
\cite{zouComprehensiveApproachEvaluate2024,
      budurleanIntegratingOpticalGenome2024}. The prognostic value of adhesion factor
expression is nowhere nearly as advanced, with establishing cell adhesion as a
reliable prognostic factor only recently
\cite{huDevelopmentCellAdhesionbased2024}.



% ======================================================================
\unnsubsection{\caddtriggertitle}%
\label{sec:discussion_caddtrigger}%

biological implications:
- Different cues could trigger different adhesional changes
- Soluble signals?
- Loss of CD138 \cite{akhmetzyanovaDynamicCD138Surface2020}
- Detachment through intercellular effects: cell division, Saturation of hMSC adhesion surface
- Detachment with mechanical influence: External forces and instability after aggregate size
-


why is this important?
The cues that trigger the detachment of MM cells are not well understood. It
could be that MM cells detach due to a combination of factors, such as loss of
adhesion factors, changes in the BM microenvironment, or cell division or
even completely random. Knowing specific dissemination signals helps preventing
dissemination.


Papers like \citet{akhmetzyanovaDynamicCD138Surface2020} make it seem as if
there is one molecule that decides if a myeloma cell is circulating or not.

It's less about one clear (molecular) mechanism that decides that a myeloma cell
decides to become a disseminating cell, but rather a indirect consequence of a combination of many
processes.
These processes are:
- Loss of adhesion factors or dynamic expression of adhesion factors
- Loss of dependency from bone marrow microenvironment
- asdf

Our thesis postulates that there is no big switch that decides if a myeloma cell
detaches from the bone marrow, but rather a prolonged process of continuisly
downregulating adhesion factors, a dynamic upregulation of adhesion factors when
they're needed, but the ultimate event that triggers release is better
explained by external mechanical forces intercellular effects (cell division,
saturation of adhesive surface and rising instability of aggregates after
reaching a minimum size).

Detachment is triggered by external mechanical forces on cell
conglomerates previously sensitized by changes in cell adhesion behaviour

Supporting Literature:

\begin{enumerate}
      \item \textbf{Cues or Processes}
            \begin{itemize}
                  \item This work showed that detachment happened mostly
                        mechanically and cell biologically through cell
                        division. - Detachment through intercellular effects:
                        cell division, Saturation of hMSC adhesion surface -
                        Detachment with mechanical influence: External forces
                        and instability after aggregate size.
                  \item Soluble signals within the BM microenvironment, such as
                        cytokines and chemokines, play significant roles in modulating
                        adhesion factor expression in MM cells
                        \cite{aggarwalChemokinesMultipleMyeloma2006, alsayedMechanismsRegulationCXCR42007}.
                  \item CD138 was proposed as a switch between adhesion and
                        migration in MM cells, its blockage triggering migration
                        and intravasation
                        \cite{akhmetzyanovaDynamicCD138Surface2020}.
            \end{itemize}
\end{enumerate}



How can this be studied?

Identifying such signals might be challenging without
having understood the other two hypotheses about adaptability first.




What new implications do these dimensions have on targeting adhesion factors for
therapy?


- It could represent a valid strategy to
strengthen myeloma adhesion, provided that targeted adhesion molecule is proven
to not be involved in other steps of dissemination, such as extravasation.
Stimulating adhesion factor expression or activity is harder than inhibition,
yet not impossible. For instance, the short polypeptide SP16 can activate the
receptor LRP1 \dashed{its high expression being associated with improved
      survival of MM patients in this work}, showing promising results during phase I
clinical trial \cite{wohlfordPhaseClinicalTrial2021}, but could potentially
increase survival of MM through PI3K/Akt signaling
\cite{potereDevelopingLRP1Agonists2019, heinemannInhibitingPI3KAKT2022} -

- One could also accept that many cues are simply not controllable, and hope for
systemic therapies like CAR- T Cells






%%%%%%%%%%%%%%%%%%%%%%%%%%%%%



% ======================================================================
\unnsubsection{Outlook: High-Value Research Topics for Myeloma Research Arising from this Work}
\label{sec:discussion_potential_breakthroughs}
As an Outlook, the Author lists research topics arising from this work that have
great potential for breakthroughs in myeloma research.

\textbf{Anti tumor effects of MSCs:}
This thesis has discussed the pro-tumor effects of MSCs. However, MSCs have also
been shown to have anti-tumor effects \cite{galderisiMyelomaCellsCan2015}. This
work has also shown that primary \acp{hMSC} can induce apoptosis in \INA6 cells
initially \dashed{probably through the action of death domain receptors},
but inhibit apoptosis during prolonged culturing.

This shows that hMSCs could be leveraged
as a therapeutic target that could prevent myloma progression.




\textbf{Cell Division as a Mechanism for Dissemination Initiation:}
The author describes how the detachment of daughter cells from the mother cell
after a cycle of hMSC-(re)attachment and proliferation could be a key mechanism
in myeloma dissemination. This mechanism was shown in other studies of
intra-/extravasation. The author sees great potential in this mechanism as a target for
future research. It is probably under-researched due to requirement of
sophisticated time-lapse equipment, yet the simplicity of detachment through
cell division is intriguing through its simplicity. It implies asymmetric cell
division. Cancer cells are known to divide asymmetrically, e.g. moving miRNAs to
one daughter cell.

% \textbf{Time as a Key Parameter:}
% The area Thermodynamics of started with scientists measuring how long it takes
% for gases to cool down. The author claims, by measuring the time it takes for
% cancer cells to detach could lead to breakthroughs in research of myeloma
% dissemination.

% - Cell adhesion is highly time-dependent.
% - Cell detachment is required for metastasis and dissemination
% -

% key mechanistic insights

% measuring the minimum time
% for detachments to begin, or the time required for daughter cells to re-attach
% to the hMSC monolayer. Such mechanistic insights



% Time-resolution was mostly
% limited by available disk space. Investing into more hard drives is worth it,
% since

\textbf{Lists of Adhesion Gene Associated With Prolonged Patient Survival:}
The author lists adhesion genes that are associated with prolonged patient
survival. These genes are highly expressed in myeloma samples from patients with
longer overall

At this time we could be on the verge of a new era of myeloma therapy,
including bi-specific antibodies and cell based approaches
\cite{moreNovelImmunotherapiesCombinations2023,
      engelhardtFunctionalCureLongterm2024}. Currently, available CAR-T Cell therapies
(ide-cel, cilta-cel) are extremely expensive, but show complete remission rates
of up to \SI{80}{\percent} and a 18-month progression free survival rate of
\SI{66}{\percent} \cite{bobinRecentAdvancesTreatment2022}. An affordable
``off-the-shelf'' CAR-T Cell product could become reality since the problem of
deadly graft-versus-host disease during allogeneic transplantation seems to be
solvable \cite{qasimMolecularRemissionInfant2017}, hence, research groups and
biotech companies are racing towards developing a safe allogeneic CAR-T Cell
technology \cite{depilOfftheshelfAllogeneicCAR2020}.


the list of genes could be good targets because the BM niche is highly hypoxic,
car t cells are not well, but directing them to the BM niche could increase
efficacy.


\textbf{Find MSC and Myeloma crosstalk:}
Do another GSEA analysis using the list from factors upregulated in
\citet{dotterweichContactMyelomaCells2016}, since there, \INA and primary
\ac{hMSC} were used as well. Redoing an analysis with the background of the
associated procceses gained here could reveal insights on the communication
between \ac{hMSC} and \INA cells.



% ======================================================================
\unnsubsection{\textit{\textbf{Conclusion\,3:} The Dynamic Adhesion Hypothetical Framework for Myeloma Dissemination}}%
\label{sec:discussion_conclusion_cancer}%



How does limited understanding of one dimension prevent the understanding of the
other dimensions?

Location \& Progression: If we don't know the expression profile of an MM cell depending on their
source, results become incomparable.

Location \& Cues: If we don't know the cues that trigger detachment, we can't
predict where the MM cells will detach.



