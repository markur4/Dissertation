



% ======================================================================
\unnsubsection{Isolating \& Quantifying Subpopulations within Cells in Direct Contact with MSCs}%
\label{sec:discussion_novel_methods}%
This project aimed to develop methodologies for isolating cells after direct
contact with \acp{hMSC}. The primary challenge was the scarcity of \textit{in
      vitro} methods that could effectively separate and isolate adhering cell
subpopulations for subsequent molecular analysis. Most available techniques
predominantly focus on the quantification of cell adhesion
\cite{khaliliReviewCellAdhesion2015, kashefQuantitativeMethodsAnalyzing2015},
and often employ indirect contact setups, complex micromanipulation, or are
unsuitable for using live \acp{hMSC} as the immobilizing surface. To address the
limitations of current adhesion assays, we developed and enhanced innovative
methodologies, specifically the \acf{WPSC} and V-Well adhesion assays.

\textbf{Variability of Washing Steps}: Given the complexity of the requirements,
this project first attempts relied on simple and traditional adhesion assays
that rely on manual washing steps \cite{humphriesCellAdhesionAssays2009}.
Washing involves aspirating the medium, dispensing washing buffer, and
potentially repeating these steps multiple times. This introduces variability
due to differences in pipetting techniques, which affect the accuracy of volume
transfer \cite{guanAssessingVariationsManual2023,
      pushparajRevisitingMicropipettingTechniques2020}. However, adhesion assays don't
rely on precise volume transfer, but accurate detachment of cells adhering at
the well bottom. This introduces a new set of considerations for the pipetting
technique, especially since cells are highly sensitive to shear forces applied
by fluid flow. From the author's experience with washing experiments and
subsequent microscopic evaluations (data not shown), several factors could
contribute to the variability of washing steps:
\begin{enumerate}
      \item The distance of the pipette tip from the well bottom, which decreases during aspiration.
      \item The position of the pipette tip relative to the well bottom (center or edge).
      \item The angle of the pipette tip.
      \item The speed of aspiration.
      \item Accidental or intended contact between the pipette tip and the cell layer.
      \item The residual volume left after aspiration.
      \item \textit{The same considerations apply when dispensing the washing buffer.}
\end{enumerate}


In addition to user-dependent factors, other variables such as the cells'
position on the well bottom can significantly impact the outcome. To the
author's experience, cells located at the edge of the well don't detach as
easily as those in the center, while cells touching the edge are almost
impossible to remove. This phenomenon is likely related to the \textit{boundary
      layer effect}, where fluids slow down near the edges of the well
\cite{weyburneNewThicknessShape2014}.

Together, since both user-dependent and independent factors can affect the
outcome of washing steps, adhesive assays that replace washing are highly
desirable. Still, since washing is straightforward and some variability is
alleviated by the disciplined execution of washing protocols, it remains a
common method for adhesion assays.




\textbf{Directly Interacting Cells Contain Unexplored Interaction Scenarios}: It
is evident that direct and indirect contact to \acp{MSC} have varying effects on
myeloma cells. That difference is crucial for understanding changes in the
\ac{BMME} during MM progression \cite{fairfieldMultipleMyelomaCells2020,
      dziadowiczBoneMarrowStromaInduced2022}. These studies utilize well-inserts to
co-culture myeloma cells in close \dashed{indirect} contact with MSCs. However,
such comparison of indirect \textit{vs.} direct co-culturing methods might not
fully represent the complexity of intercellular interactions scenarios found in
the \ac{BMME}. This is exemplified by this project, as it relied on the complex
growth behavior: \INA cells aggregated homotypically in direct
proximity to those adhering heterotypically to \acp{hMSC}, and detached
through cell division. Furthermore, such methods fail to capture the subtle
variations in paracrine signaling concentrations, where even a few micrometers
of distance could significantly alter cellular responses.

Such knowledge shifted this project's point of view as well: Initially, our
hypothesis focused on direct heterotypic interactions, not expecting a \nMAina
population, but rather subpopulations within \MAina cells that are separable by
varying adhesion strengths. Hence, our assay employed strict conditions favoring
one growth scenario \dashed{heterotypic interactions}, with co-cultures providing unlimited hMSC-surface
availability causing predominantly heterotypic adhesion, while the short
incubation time prevented the formation of aggregates. Despite these measures,
our assay still captured cells emerging from recent cell divisions rather than
from weak heterotypic adherence as initially hypothesized. This demonstrates the
robustness of our method in separating subpopulations that arising from unexpected
intercellular interaction scenarios. This can be a major an advantage over
methods that summarize direct interactions as one population. Analysing the
non-adhering subpopulation within directly interacting cells could provide
valuable insights not just in multiple myeloma, but also metastasis of other
cancer types.


\textbf{Minimizing Variability}:
There are innovative adhesion assays that both
support the isolation of nonadherent subpopulations from directly interacting
cells, and avoid variability introduced by washing steps.

One simple method involves flipping over a 96-well plate, with surface tension
preventing medium spills as non-adhering cells fall to the surface for
collection \cite{zepeda-morenoInnovativeMethodQuantification2011}. However, we
found that the medium in fact did spill occasionally (not shown). Other
approaches involve sealing the plate, such as with PCR plate seals, and using
centrifugation to separate cells \cite{reyesCentrifugationCellAdhesion2003,
      chenHighThroughputScreeningTest2021}. Despite our efforts, we could not
consistently avoid air bubbles, which, after flipping, would contact the cell
layer and create dry regions during centrifugation.

The V-Well adhesion assay does not flip, but collects non-adhering cells into
the nadir of V-shaped wells during centrifugation
\cite{weetallHomogeneousFluorometricAssay2001}. This work profited greatly from
this method, while \dashed{to our knowledge} being the first to use cell
monolayers as the immobilizing surface. We value this method for its precision,
as centrifugation applies a uniform and configurable force, while the readout
remains straightforward, relying on the total fluorescent brightness rather than
individual cell counting.




\textbf{Specializing in Quantifying Adhesion or Isolating Subpopulations}: Most
adhesion assays primarily focus on quantification rather than isolation. The
author attempted to combine both quantification and isolation, but found that
the two goals can be mutually exclusive. The author summarizes the key
differences between quantification and isolation approaches as such:

\begin{itemize}
      \item Cell Manipulation for Harvest \textit{vs.} Readout:
            \begin{itemize}
                  \item Isolation methods are designed to manipulate cells for easy
                        harvest. For instance, the \ac{WPSC} method uses a catching
                        plate to collect non-adherent cells for subsequent analysis.
                  \item Quantification methods, on the other hand, manipulate cells
                        to simplify the readout process. For example, the V-Well
                        assay, which pellets cells into a single location, allowing
                        for a pooled fluorescence measurement without the need for
                        extensive cell handling.
            \end{itemize}

      \item Optimization for Subsequent Analysis \textit{vs.} Sample Throughput:
            \begin{itemize}
                  \item Isolation methods are optimized for detailed subsequent
                        analyses, such as RNA or protein analysis. For example,
                        \ac{WPSC} minimizes the introduction of biases such as those
                        from fluorescent staining, making it suitable for downstream
                        molecular assays.
                  \item Quantification methods are optimized for high sample
                        throughput. The V-Well assay, as an end-point assay, is
                        designed to efficiently handle multiple treatments
                        simultaneously, providing quick and comparative results with
                        lower cell numbers.
            \end{itemize}
      \item Handling of Cell Numbers:
            \begin{itemize}
                  \item Isolation methods, such as \ac{WPSC}, require multiple wells
                        (e.g., 96 wells) to gather a sufficient amount of cells per
                        subpopulation, which is crucial for robust downstream
                        analyses.
                  \item Quantification methods, exemplified by the V-Well assay, are
                        highly efficient even with low cell numbers.
            \end{itemize}
\end{itemize}



Thus, this adopted two distinct techniques for isolating and quantifying
directly interacting subpopulations, each optimizing for different outcomes, but
also supporting the separation of subpopulations within direct intercellular
interactions.

Still, it is theoretically possible to insert microscopy steps into the
\ac{WPSC} method to scan the well bottom for later cell counting. Also, this
work effectively isolated cell pellets from the V-well plate for subsequent
fixation and cell cycle profiling. The process was
tedious and required multiple technical replicates to achieve sufficient cell
numbers for analysis. It also required removing \ac{hMSC} from the V-well nadir
to prevent contamination during pellet aspiration.

Together, while both methods can combine quantification and isolation, they
are optimized towards either of them. Knowing these strengths and weaknesses
could help to advance these methods in future studies.





\textbf{Rationales of the Well Plate Sandwich Centrifugation}: Inspired by the
principles of both flipping and V-Well adhesion assays, we developed the Well
Plate Sandwich Centrifugation (\acf{WPSC}) method to address the challenges of
isolating cell populations. This method innovatively combines elements from both
techniques to provide a more reliable approach to cell isolation. One of the key
advantages of WPSC is its ability to reduce the variability commonly introduced
by manual pipetting. Instead of relying on aspiration, which introduce
variability in cell collection and requires touching the well bottom for
complete removal of medium, WPSC employs centrifugation to remove non-adhering
cells. Medium is then returned by pipetting to repeat the process and maximize
non-adhering cell collection, as the number of detachable cells plateau after
few rounds of centrifugation. Hence, this approach compromises
between minimizing washing variability and isolating larger quantities of cells.

The 96 well plate format has advantages, reducing spilling when flipping the
sandwich, as surface tension kept fluids in place. The 96 well plate format also
reduces per-well variability by performing the same washing procedure up to 96
times.

The slow centrifugation speeds used during \ac{WPSC} are also decided after
thorough consideration. For this, one has to discuss how exactly
non-adhering cells detach during centrifugation. While centrifugal force is an
obvious factor, the properties of cell adhesion are unclear under dry conditions
during centrifugation. The author assumed that the cells are being pulled along
by the medium as it is centrifuged into the catching plate. Hence, the centrifugation
speed was chosen as fast enough to transfer the medium, without completely
drying the co-culture plate and minimizing overall cell stress.

A significant challenge in \ac{WPSC} is the dissociation of \MAina from the hMSC
monolayer. WPSC employs two distinct techniques to achieve this dissociation.
The first technique involves repeated treatment with the gentle digestive enzyme
Accutase followed by \acp{MACS}. \ac{MACS}, despite being effective, is costly,
time-consuming, reduces overall cell yield, and potentially introduces biases
due to CD45 antibody selection and the requirement for cold-treatment. The
second technique utilizes strong pipetting to physically detach non-adhering
cells (termed \emph{`Wash'}). It is important to note that these techniques did
not affect the protocol on detaching \nMAina from the co-culture, hence
providing for a consistent ratio of isolated \MAina to \nMAina across all
experiments. Ultimately, we preferred \emph{Wash}, as \ac{MACS} had to be performed
on all samples to ensure comparability, reducing overall cell yield which became
limiting for downstream applications, especially for \nMAina cells. Both methods
achieved comparable purity of \MAina cells, with few hMSCs per $10e4$
\MAina cells (purity assessment not shown). \emph{Wash} probably pofited from
the highly durable nature of primary hMSC monolayers, whereas \emph{MACS}
required dissociation of the co-culture.

Together, \ac{WPSC} offers a versatile solution for isolating hMSC-interacting
myeloma cells. It successfully balances the need for precision with the ability
to handle larger cell quantities. \ac{WPSC} could be adapted to other cell types
that combines monolayer forming and suspension cells.



\textbf{Key Points:} Ultimately, this work established two methodologies
that could represent a significant advancement in the field of adhesion assays,
providing cost-effective, precise, reliable, and reproducible techniques for
both isolating and quantifying subpopulations within co-cultures of directly
interacting cell types. They offered valuable insights into the mechanisms of MM
detachment and are potentially applicable to other research questions that focus
on multicellular interactions and complex growth scenarios.




%%%%%%%%%%%%


% \def\footattachdetachdynamics{1}{ \footnote{\emph{Attachment/Detachment
%                   Dynamics} = The Observation and Measurement of time-dependent changes in
%             cell adhesion. The time required for attachment, detachment, and migration
%             of cells gives insight about its disseminative potential. This term
%             differentiates itself from cell adhesion through the time component and an
%             emphasis on explaining detachment events
%       }\label{foot:attachdetach_dynamics} }

% \newcommand{\footattachdetachdynamics}{%
%       \footnote{\emph{Attachment/Detachment Dynamics} = The Observation and
%             Measurement of time-dependent changes in cell adhesion. The
%             time required for attachment, detachment, and migration of
%             cells gives insight about its disseminative potential. This
%             term differentiates itself from cell adhesion through the time
%             component and an emphasis on explaining detachment
%             events}\label{foot:attachdetach_dynamics}
% }%

% % > Define the footnote and its mark manually
% \newcommand{\definefootattachdetachdynamics}{%
%     \footnotetext[1]{\emph{Attachment/Detachment Dynamics} = The Observation and
%     Measurement of time-dependent changes in cell adhesion. The time required
%     for attachment, detachment, and migration of cells gives insight about its
%     disseminative potential. This term differentiates itself from cell adhesion
%     through the time component and an emphasis on explaining detachment
%     events}\label{foot:attachdetach_dynamics}%
% }
% % > Command to reference the footnote by its mark
% \newcommand{\footattachdetachdynamics}{%
%     \footnotemark[1]%
% }

% \definefootattachdetachdynamics


\unnsubsection{Constructing a Hypothetical Framework of Dissemination}%
\label{sec:discussion_framework}%
Until today, a mechanistic understanding of myeloma dissemination is still
lacking. \citet{zeissigTumourDisseminationMultiple2020} has described
dissemination as a multistep process along the lifetime of a myeloma cell.
However, since it provides evidence for only individual steps\dashed{each step
      being very complex itself}, the connection of these steps remains unproven and
therefore a hypothetical construct. Scientific evidence for dissemination
becomes \emph{fragmented}, with each piece contributing but a small part to the
big picture. This includes the evidence gained from this work, especially since
its validity is limited to the \INA cell line. To regain understanding of
dissemination as a process, the following sections attempt to integrate the
findings of this with available literature, and formulates hypotheses that could
help design further experiments to validate commonalities in dissemination.
Given the focus on direct observations of attachment/detachment
% dynamics\footattachdetachdynamics, 
this framework carries the name \emph{Dynamic
      Adhesion Hypothetical Framework for Myeloma Dissemination}.
      
% asdf\footattachdetachdynamics

% - Cell attachment/detachment dynamics = Time-dependent changes in adhesion, measuring time required
% for attachment, detachment and migration of cells, including

% - Cell attachment/detachment behavior = All observed and molecular phenomena leading to
% attachment, detachment or migration


% \textbf{Definition of Cell Adhesion Behavior}
% Cell adhesion behavior encompasses not only the molecular basis for cell
% attachment but also dynamic processes that influence cell migration and tissue
% colonization. This includes:
% \begin{itemize}
%       \item \textbf{Dynamic Attachments, Detachments \& Migration:} How and when
%       cells form and break connections with each other and the extracellular
%       matrix (ECM) in various physiological contexts such as development, wound
%       healing, and immune responses.
%       \item \textbf{Regulatory Mechanisms:} How various signaling pathways and
%       molecular regulators initiate attachment, detachment, and migration. This
%       includes how cells adapt their adhesion characteristics in response to
%       changes in their environment, such as variations in ECM composition or
%       mechanical forces.
%       \item \textbf{Changes Induced by Attachments, Detachments \& Migration:}
%       How the cell adhesion process influences other cellular behaviors, such as
%       cell migration, proliferation, and differentiation.
% \end{itemize}


\textbf{Cell Attachment/Detachment Behavior and Predicting Dissemination}
% Understanding the attachment and detachment dynamics of myeloma cells is crucial
% for predicting how these cells disseminate throughout the body. Our observations
% indicate that INA-6 myeloma cells dynamically regulate adhesion factors in
% response to direct contact with human mesenchymal stem cells (hMSCs),
% highlighting the influence of microenvironmental interactions.
% Predicting when and how myeloma cells
% detach from the primary tumor and colonize new niches is essential for
% understanding the progression of MM and developing therapeutic strategies to
% prevent metastasis.
% Dissemination is a major contributor to MM progression and
% poor prognosis, enabling MM cells to colonize niches that favor survival,
%  are less accessible to therapy, and develop
% heterogenous subclones \cite{forsterMolecularImpactTumor2022,
%       keatsClonalCompetitionAlternating2012}.

% - Explain how this work established attachment/detachment dynamics 
% - Explain how attachment/detachment dynamics is better captured by the term "cell adhesion behaviour", emphasising the direct observation of these processes
% - Explain how attachment/detachment dynamics help predicting dissemination
% - Give a short reminder how dissemination is hampering therapy

The study of attachment and detachment dynamics, termed here as \textit{cell
      adhesion behavior}, provides a comprehensive view of the adaptive interactions
myeloma cells engage in within their microenvironment. This research has
established a detailed observational basis for these dynamics by directly
tracking the behavior of INA-6 myeloma cells in response to their interactions
with human mesenchymal stem cells (hMSCs). The term "cell adhesion behavior"
encapsulates not just the biochemical events of cell attachment but also the
dynamic processes of cellular migration and detachment, crucial for
understanding myeloma dissemination.



Dynamic observations of INA-6 cells have revealed that changes in adhesion
factor expression are closely linked to cell-cell contact and environmental
cues, thereby illustrating the plasticity of myeloma cells in modifying their
adhesive properties \cite{huDevelopmentCellAdhesionbased2024,
      mrozikTherapeuticTargetingNcadherin2015}. These insights into cell adhesion
behavior underscore the ability of myeloma cells to adapt rapidly to varying
microenvironments, facilitating their detachment and subsequent dissemination.

Predicting the detachment and dissemination of myeloma cells is pivotal for
understanding the progression of the disease. Dissemination allows myeloma cells
to colonize new niches favorable for survival, often in quiescent states that
are less accessible to therapeutic interventions. This poses significant
challenges for current treatment strategies, as disseminated myeloma cells can
evade systemic therapies and contribute to a poorer prognosis
\cite{forsterMolecularImpactTumor2022,
      keatsClonalCompetitionAlternating2012}. Understanding these attachment/detachment
dynamics through the lens of cell adhesion behavior not only provides insights
into the biological underpinnings of myeloma progression but also highlights the
importance of developing targeted therapies that can interfere with these
dynamic processes to prevent the spread of the disease.

% asdf2\footattachdetachdynamics

\textbf{Overview of Key Hypotheses}
The Dynamic Adhesion Hypothetical Framework is structured around four key
hypotheses, each addressing a fundamental aspect of myeloma cell dissemination:
\begin{enumerate}
      \item Myeloma cells dynamically change adhesion behaviour during
            dissemination in response to microenvironmental interactions
      \item Dynamic change in cell adhesion behaviour is a hallmark of
            aggressive myeloma
      \item Detachment can be triggered by multiple cues of varying nature,
            including Non-molecular ones like external mechanical forces, cell
            biological ones like cell division or specific cell arrangements,
            molecular ones like an accumulation of changes in cell adhesion
            behaviour, or maybe even pure chance.

            % \item Detachment is triggered by external mechanical forces on cell
            % conglomerates previously sensitized by changes in cell adhesion
            % behaviour

      \item Cell adhesion behaviour is highly diverse, both in patients and cell
            lines

            % \item Dynamic changes in cell adhesion behavior as a response to
            %       microenvironmental interactions and disease progression.
            % \item The role of aggressive disease characteristics in enhancing the
            %       adaptability of adhesion factor expression.
            % \item The impact of mechanical forces and cellular interactions on the
            %       detachment of myeloma cells from the bone marrow.
            % \item Genetic variability among patients and how it influences adhesion
            %       behavior and treatment responses, acknowledging that this framework
            %       currently does not deeply integrate genetic predispositions which could
            %       further refine its predictive power \cite{kumarMultipleMyelomasCurrent2018a,
            %             hoangMutationalProcessesContributing2019}.
      \item
\end{enumerate}

This introduction sets the stage for a detailed exploration of each hypothesis,
linking empirical data with theoretical constructs to provide a comprehensive
framework of myeloma cell dissemination that can inform both future research and
clinical practice.


% ======================================================================
\unnsubsection{Hypothesis 1: Myeloma Cells Dynamically Change Adhesion Behaviour during
      Dissemination}
\label{sec:discussion_order_adhesion}%
lorem ipsum


\textbf{Cell Adhesion Behavior and Predicting Dissemination}



% ======================================================================
\unnsubsection{Hypothesis 2: Dynamic Change in Cell Adhesion Behaviour is a Hallmark of
      Aggressive Myeloma}%
\label{sec:discussion_dynamic_changes_aggressive_hallmark}%
lorem ipsum




% ======================================================================
\unnsubsection{Hypothesis 3: Cell Adhesion Behaviour is Highly Diverse}
\label{sec:discussion_diversity_patients_cell_lines}%


% ======================================================================
\unnsubsection{Hypothesis 4: Detachment Can be Triggered by Multiple Cues of
      Varying Nature}%
\label{sec:discussion_detachment_mechanical_forces}%

Detachment is triggered by external mechanical forces on cell
conglomerates previously sensitized by changes in cell adhesion behaviour



%%%%%%%%%%%%%%%%%%%%%%%%%%%%%%%%%%%%%%%%%%%%%%%%%%%%%%%%%%%%%%%%%%%%%%%%%%%%%%%

%%%%%%%%%%%%%%%%%%%%%%%%%%%%%%%%%%%%%%%%%%%%%%%%%%%%%%%%%%%%%%%%%%%%%%%%%%%%%%%

Hypothetical Framework

% \unnsubsection{A Bottom-Up Approach to Modelling Adhesion Factor Regulation}%
% \unnsubsection{Integrating Adhesion Factor Expression with Attachment/Detachment Events}%
% \unnsubsection{How to Gain a Mechanistic Understanding of Dissemination..?}%
% \unnsubsection{A Bottom Up Approach to Understanding Myeloma Dissemination}

% The Dynamic Adhesion Hypothetical Framework for Myeloma Dissemination.

% The need for integrating Aggression and Location into one Framework of Dissemination

\label{sec:discussion_order_adhesion}%
Overall, cell adhesion play a pivotal role in the attachment/detachment dynamics of
myeloma, hence influencing the dissemination of myeloma cells. This is
exemplified in this work, where \INA cells dynamically upregulate adhesion
factors in direct contact with \acp{hMSC}. Predicting how and when myeloma cells
regulate adhesion activity is a key question in understanding dissemination,
since that

potentially preventing it during therapy.

Research on cell adhesion is progressing:
Promising prognostic factors and therapeutic targets are being identified
\cite{mrozikTherapeuticTargetingNcadherin2015,
      solimandoJAMAPrognosticFactor2018}, as well as MM subpopulations that are both
defined by adhesion gene expression and associated with dissemination
\cite{akhmetzyanovaDynamicCD138Surface2020,
      brandlJunctionalAdhesionMolecule2022}.


A recent study by \citet{huDevelopmentCellAdhesionbased2024} developed a cell
adhesion-based prognostic model for MM, calculating an adhesion-related risk
score (ARRS) based on expression of only twelve adhesion related genes.

However, a mechanistic understanding of dissemination is still lacking. This
work did combine both molecular approaches with studying attachment/detachment
dynamics, and found connections between adhesion factor expression and disease
stage. The following paragraphs will discuss dynamic regulation of adhesion factors
and the role of disease stage in this process, but also discusses the cues that trigger
detachments.


The author argues that under
How adhesion
factor regulation impacts the attachment/detachment dynamics of disseminating
myeloma cells.

The author argues that research
is limited by the lack of integrating cell biological principles of niche
interaction into the analysis of adhesion factor regulation.


of adhesion factor regulation in MM
The following sections attempt to model the dynamics of adhesion factor
regulation based on the results of this work and the current literature.

% Too many factors remain unexplored, such as the location of myeloma cells,
% and the disease stage.


% Such approach is reminiscent of the endeavor of identifying subtypes of MM for
% improved risk stratification and potential personalized therapies. 



% Using adhesion molecules as prognostic factors is far less advanced, yet their relevance
% in preventing disease progression stands firm. 

% Classifying MM subtypes based on
% \acp{CAM} expression or adhesion behavior could be a similar approach, but isn't
% nearly as advanced (??). However, 




% However, a unifying model of adhesion
% factor regulation in MM is still lacking.

% Here, the author df

% However, a unifying 

%       A similar classification by adhesion
% factors is nowhere near as advanced. 




% but could curtain potential breakthroughs in
% preventing dissemination.

% Hence, the following discussion attempts to present
% knowledge and challenges in the field of adhesion factors in MM, and how they
% could be studied.
% Advances in this field 


Myeloma cells are isolated from patients at a certain stage from a certain
location. As summarized by \citet{zeissigTumourDisseminationMultiple2020},
dissemination could be a dynamic process during the lifetime of a myleoma cell
that managed to exit the \ac{BMME} into blood circulation. This implies that
myeloma cells could change their adhesion factors during their course of
dissemination to adapt to their current location for specialized tasks like
exiting the \ac{BMME} or intra-/extravasation. However, this work and evidence
from the literature suggest that different disease stages handle the regulation
of adhesion factors differently. Hence, this work defines not only location but
also disease stage as two dimensions with different implications for adhesive
behaviors.


The following paragraphs construct a narrative and then later checks for every
step if there is evidence for it in this work or the literature.

First let's construct a framework that's at least reasonable, but not necessarily
backed up by evidence:

Three dimensions where changes in adhesion factors are expected. These dimensions
make up a space, where every point describes an adhesive behavior of myeloma cells.
1 Location of Myeloma Cells (BM, vascular)
2 Disease Stage (asymptomatic MM, MM, MM relapse)
3 Cues that might trigger changes, or processes associated with changes or detachment


One important dimension that is missing here is the genetic background of the
myeloma cells. These are based on recurrent patterns of chromosomal aberrations
or mutational signatures, defining structural and single nucleotide variants
\cite{kumarMultipleMyelomasCurrent2018a,
      hoangMutationalProcessesContributing2019}. The prognostic value of genetic
variants in MM is well established \cite{sharmaPrognosticRoleMYC2021}, and their
identification is becoming precise and cost-effective using \emph{optical
      genome mapping}, making progress towards personalized therapies
\cite{zouComprehensiveApproachEvaluate2024,
      budurleanIntegratingOpticalGenome2024}. The prognostic value of adhesion factor
expression is nowhere nearly as advanced, with establishing cell adhesion as a
reliable prognostic factor only recently
\cite{huDevelopmentCellAdhesionbased2024}.


% Research on adhesion factors is nowhere
% near as advanced for classifying MM subtypes, although promising targets are
% being identified \cite{mrozikTherapeuticTargetingNcadherin2015,
% solimandoJAMAPrognosticFactor2018}, as well as subpopulations associated with
% dissemination defined by adhesion factor expression
% \cite{akhmetzyanovaDynamicCD138Surface2020,
% brandlJunctionalAdhesionMolecule2022}. 




Why are these dimensions important and how could they be studied?

1 Location: Knowing how an MM cell can change their adhesive properties during its course of
dissemination is crucial for understanding the process itself. These changes
could be studied by tracking the expression of adhesion factors in MM cells at
different locations in mouse models. For humans, designing studies that gather
biopsies at different locations from the same patient, e.g. bone marrow and cirulating
myeloma cells could be a starting point.

2 Studying the adhesive changes during MM progression is
interesting, as it could unravel a specialized treatment strategy
that could maybe prevent dissemination.

3 The cues that trigger the detachment of MM cells are not well understood. It
could be that MM cells detach due to a combination of factors, such as loss of
adhesion factors, changes in the BM microenvironment, or cell division or
even completely random. Knowing specific dissemination signals helps preventing
dissemination.


How could these dimensions they be studied?

1 Location: These changes could be studied by tracking the expression of adhesion factors in MM cells at
different locations in mouse models. For humans, designing studies that gather
biopsies at different locations from the same patient, e.g. bone marrow and cirulating
myeloma cells could be a starting point.

2 Progression: Databases of expression from Myeloma cells gathered from bone
marrow \ac{MGUS}, \ac{aMM}, \ac{MM}, \ac{MMR} already exist
\citet{akhmetzyanovaDynamicCD138Surface2020,
      seckingerCD38ImmunotherapeuticTarget2018}. Going through such databases gives a
good overview. One could categorize genelists using curated databases, get lists
associated with extravasation, intravasation, Bone marrow adhesion. For every
gene of these genelists, they could be filtered for significant differences
between the stages. Further categorizations of pairwise comparisons of stages
are required. but overall, these genelists could be a starting point for This
approach is similar to the genelists published in chapter 1, with the difference
that the genelist was furthere filtered by the RNAseq results of \textit{in
      vitro} experiments.

3 Cues: Identifying such signals might be challenging without
having understood the other two dimensions first.


How does limited understanding of one dimension prevent the understanding of the
other dimensions?

Location \& Progression: If we don't know the expression profile of an MM cell depending on their
source, results become incomparable.

Location \& Cues: If we don't know the cues that trigger detachment, we can't
predict where the MM cells will detach.




What biological implications do these dimensions have?

1 Location of Myeloma Cells:
- Different locations could require different adhesion factors:
- Circulating MM cells do not need adhesion, probably losing adhesion factors
- BM cells express adhesion factors to adhere to the Bone marrow microenvironment (MSCs, adipocytes, and osteoblasts)
- Extravasating/intravasating cells need adhesion factors for endothelium
- Extramedullary cells need adhesion factors for respective tissues

2 Disease Stage:
- Higher disease stages imply changes in adhesion factors that favor aggressiveness.
- Aggressiveness includes:
- Better Colonization of new niches, including extramedullary ones
- This implies a more diverse set of available adhesion factors
- Faster regulation to adapt to new niches
- Better survival in circulation

3 Cues or associated processes:
- Different cues could trigger different adhesional changes
- Soluble signals?
- Loss of CD138 \cite{akhmetzyanovaDynamicCD138Surface2020}
- Detachment through intercellular effects: cell division, Saturation of hMSC adhesion surface
- Detachment with mechanical influence: External forces and instability after aggregate size
-


What new implications do these dimensions have on targeting adhesion factors for
therapy?

1 Location of Myeloma cells
- Inhibiting adhesion factors could inhibit dissemination at one location
or niche, but also benefit dissemination at another location. Different subsets
of  adhesion factors must be thoroughly evaluated


2 Disease Stages:
- Aggressive MM cells have potantial improved control over adhesion factor expression,
regulating a more diverse set of adhesion factors faster. This poses further challenges to targeting.
It could be smarter to not target effector-molecules, but rather upstream regulators of adhesion.
THis work shows that NF-kappaB signaling, which by itself is not treatable, but regulators
downstream of NF-kappaB were shown to be effective \cite{adamikEZH2HDAC1Inhibition2017,adamikXRK3F2InhibitionP62ZZ2018}

3 Cues or associated processes:
- It could represent a valid strategy to
stimulate myeloma adhesion, provided that targeted adhesion molecule is proven
to not be involved in other steps of dissemination, such as extravasation.
Stimulating adhesion factor expression or activity is harder than inhibition,
yet not impossible. For instance, the short polypeptide SP16 can activate the
receptor LRP1 \dashed{its high expression being associated with improved
      survival of MM patients in this work}, showing promising results during phase I
clinical trial \cite{wohlfordPhaseClinicalTrial2021}, but could potentially
increase survival of MM through PI3K/Akt signaling
\cite{potereDevelopingLRP1Agonists2019, heinemannInhibitingPI3KAKT2022} -







What evidence is there that supports this framework?





\textbf{1 Location of Myeloma Cells}
\begin{itemize}
      \item \textbf{Other Findings}
            \begin{itemize}
                  \item The review by
                        \citet{zeissigTumourDisseminationMultiple2020} could be
                        a starting point. She does not discuss adhesion factors,
                        but seeing dissemination as a multistep process does
                        imply different adhesion factors for different steps.
                  \item Malignant Plasma Cells express different adhesion factors
                        than normal plasma cells \cite{cookRoleAdhesionMolecules1997, bouzerdanAdhesionMoleculesMultiple2022}.

                  \item Adhesion molecules have been a popular target for therapy for a decade \cite{nairChapterSixEmerging2012}
            \end{itemize}

      \item \textbf{Extramedullary Involvement}
            \begin{itemize}
                  \item Extramedullary involvement: HCAM dramatic upregulation of HCAM
                  \item CXCR4, the homing receptor, mediates production of
                        adhesion factors in extramedullary MM cells \cite{roccaroCXCR4RegulatesExtraMedullary2015}
            \end{itemize}

      \item \textbf{Intra-/Extravasation of Myeloma Cells}
            \begin{itemize}
                  \item Blocking Endothelial Adhesion through JAM-A decreases progression: \cite{solimandoHaltingViciousCycle2020}
                  \item N-Cadherin is upregulated in MM compared to healthy plasma cells, and has been shown to be a potential target for therapy \cite{mrozikTherapeuticTargetingNcadherin2015}

            \end{itemize}

      \item \textbf{Circulating Myeloma Cells}
            \begin{itemize}
                  \item This work shows that \nMAina have increased survival
                        during IL-6 deprivation, which could be a mechanism for
                        surviving in circulation.
                  \item Circulating plasma cells are rare, but detectable in
                        peripheral blood
                        \cite{witzigDetectionMyelomaCells1996}
                  \item studies demonstrate that circulating \ac{MM} cells
                        exhibit reduced levels of integrin $\alpha4\beta1$, in
                        contrast to those located in the \ac{BM}
                        \cite{paivaDetailedCharacterizationMultiple2013,
                              paivaCompetitionClonalPlasma2011}
                  \item circulating MM cells were CD138/Syndecan-1 negative
                        \cite{akhmetzyanovaDynamicCD138Surface2020}

            \end{itemize}

      \item \textbf{BM-Resident Myeloma Cells}
            \begin{itemize}
                  \item The role of CXCL12 \dashed{which is highly expressed by
                              MSCs} in inducing adhesion factors in MM is well established
                  \item
                  \item
                  \item THIS WORK: INA-6 cells are highly adhesive to hMSCs, dynamically
                        upregulating adhesion factors when in direct contact with
                        hMSCs, and subsequently losing adhesion factor expression after
                        cell division

                  \item BM-resident MM cells maintain high levels of adhesion
                        molecules to interact with MSCs, adipocytes, and osteoblasts
                        within the BM niche \cite{bouzerdanAdhesionMoleculesMultiple2022, burgerGp130RasMediated2001, chatterjeePresenceBoneMarrow2002}.
            \end{itemize}

\end{itemize}

\begin{enumerate}
      \item \textbf{Disease Stage}
            \begin{itemize}
                  \item THIS WORK: Expression decreases during progression from
                        \ac{MGUS} to \ac{MMR} of adhesion factors involved in hMSC
                        adhesion.
                  \item The idea that MM pathogenesis involves transformative
                        processes has been around for decades
                        \cite{hallekMultipleMyelomaIncreasing1998}, but a
                        detailed understanding of changing adhesive properties
                        is still lacking, especially during the progression of
                        MM.
                  \item It is discussed that myeloma cell lines derived from
                        advanced stages show different expression than newly
                        diagnosed patients, discussing that they come from
                        multiply relapsed patients
                        \cite{sarinEvaluatingEfficacyMultiple2020}. This work
                        also shows that Myeloma cell lines have the lowest
                        expression of adhesion factors compared to all stages of
                        \ac{MM} and \ac{MGUS}.
                  \item For B-Cell Chronic Lymphocytic Leukemia, adhesion
                        molecule expression patterns define distinct phenotypes in
                        disease subsets \cite{derossiAdhesionMoleculeExpression1993}.
                  \item \citet{terposIncreasedCirculatingVCAM12016} reported an
                        increase in adhesion molecule expression of ICAM-1 and
                        VCAM-1 in patients with \ac{MM} compared to those with
                        \ac{MGUS} and \ac{aMM}.
                  \item However, \citet{perez-andresClonalPlasmaCells2005}
                        reported that CD40 is downregulated in \ac{PCL}
                        patients. Hence, different \acp{CAM} could serve
                        ambiguous roles in \ac{MM} progression.
            \end{itemize}

      \item \textbf{Cues or Processes}
            \begin{itemize}
                  \item This work showed that detachment happened mostly
                        mechanically and cell biologically through cell
                        division. - Detachment through intercellular effects:
                        cell division, Saturation of hMSC adhesion surface -
                        Detachment with mechanical influence: External forces
                        and instability after aggregate size.
                  \item Soluble signals within the BM microenvironment, such as
                        cytokines and chemokines, play significant roles in modulating
                        adhesion factor expression in MM cells
                        \cite{aggarwalChemokinesMultipleMyeloma2006, alsayedMechanismsRegulationCXCR42007}.
                  \item CD138 was proposed as a switch between adhesion and
                        migration in MM cells, its blockage triggering migration
                        and intravasation
                        \cite{akhmetzyanovaDynamicCD138Surface2020}.
            \end{itemize}
\end{enumerate}






%%%%%%%%%%%%%%%%%%%%%%%%


% ======================================================================
% \unnsubsection{Dynamic and Niche-Dependent Regulation of Adhesion Factors}%
\label{sec:discussion_dynamic_regulation}%


% 1. Summarize Result
% 2. Explain biological implication
% 3. Summarize key literature that contributes to this implicat
% 4. Judge if the implication hold up against current knowledge
% 5. Name the explicit holes in the current knowledge
% 6. Formulate a hypothesis that could fill these holes

Given the complexity of cell adhesion, and integrating direct observations from
live-cell imaging, one requires to extend the definition of Cell adhesion
to cell adhesion behavior:

Cell Adhesion Behavior =

• Dynamic Attachments, Detachments \& Migration: How and when cells form and
break connections with each other and the ECM in various physiological contexts
like development, wound healing, and immune responses.

• Regulatory Mechanisms: How various signaling pathways and molecular regulators
initiate attachment, detachment \& migration. This includies how cells adapt
their adhesion characteristics in response to changes in their environment, such
as variations in ECM composition or mechanical forces.

• Changes Induced by attachments, detachments \& migration: How the cell
adhesion process influences other cellular behaviors, such as cell migration,
proliferation, and differentiation.


% \textbf{Colonizing New Niches:}
This work showed that \INA cells dynamically upregulate adhesion factors when in
direct contact with \acp{hMSC}. Such adhesion factors are not expressed by \INA
cells without contact to \acp{hMSC}, or by \INA cells emerging as daughter cells
from \MAina cells. This implies that myeloma cells are capable of rapid changes in
adhesion factor expression that are substantially dynamic.
Predicting when a myeloma cell starts regulating adhesion factors is a key
question in understanding dissemination.

The following paragraphs
discuss how the idea of dynamic adhesion factor expression holds up
against current knowledge.



This is in line
substantial dynamics of
myeloma cells to regulate adhesion factors according to their environment.



This implies that myeloma cells
dynamically regulate adhesion factors during colonization of new niches.




INA-6 was initially isolated from plasma cell leukemia as an extramedullary
plasmacytoma located in the pleura from a donor of age.




% \textbf{Extrapolating Dynamic Adhesion Factor Expression onto other Disseminative Steps?:}

For example,
circulating MM cells show lower levels of integrin $\alpha4\beta1$
compared to those residing in the BM. Furthermore, treatment with a syndecan-1 blocking antibody
has been shown to rapidly induce the mobilization of MM cells from the BM to
peripheral blood in mouse models, suggesting that alterations in adhesion
molecule expression facilitate MM cell release
\cite{zeissigTumourDisseminationMultiple2020}.



% \textbf{Losing Adhesion Factors During Progression:}
There is not much more information available on the background of that patient \cite{TwoNewInterleukin6,burgerGp130RasMediated2001}.
But assuming that
This is a highly advanced
stage of myeloma.
However,  Chapter 1 shows that adhesion factors are
lost during MM progression. INA-6 are highly adhesive to hMSCs.
This is a contradiction that needs to be resolved.





% However, INA-6 do not express adhesion factors. They do that only in hMSC presence
% Hence MAINA-6 could be a smaller fraction of MM cells, specialized on preparing a new niche
% for the rest of the MM cells. This could be a reason why they are so adhesive.

This assumption dictates that aggressive myeloma cells gain the ability
to dynamically express adhesion factors.
It could be that INA-6 has gained the capability to express adhesion factors
fast in order to colonize new niches, such as pleura from which they were
isolated.

This shows that not just the stage of the disease, but also the location of the
myeloma cells plays a role when considering adhesion factors.

According to this, this thesis
predicts a low expression of adhesion factors in circulating myeloma cells,
but a high expression in adhesive cells, e.g. non-circulating, or rather those

indeed CD138 paper isolated cells from circulating MM cells \cite{akhmetzyanovaDynamicCD138Surface2020}

indeed, 3 temporal subtypes have been identified, associating higher risk with
faster changes over time \cite{keatsClonalCompetitionAlternating2012}.

% ======================================================================
% \unnsubsection{Subsets of Adhesion Factors Contribute To Different Steps of Dissemination}%
% \label{sec:discussion_subsets_adhesion_factors}% 

Here: Myeloma adhesion to BMME

Literature: Intra-/Extravasation has molecules

This implies that different adhesion factors are required for different steps of
dissemination.




- adhesion molecules during vascular involvement have these adhesion molecules: JAM-C
and CD138.
- NONE of Them were shown in Chapter 2 of this study, (except for JAM-B)


- One has to consider that intravasation and/or intra-/extravasation would require a different
set of adhesion factors than adhesion to BM or extramedullary environments.

This has great implications for targeting adhesion factors for therapy, as it
suggests that different adhesion factors should either be antagonized or
agonized depending on the function of the adhesion factor. According to this
assumption, adhesion factors involved in intra- and extravasation adhesion should be
antagonized, while adhesion factors involved in BM adhesion \dashed{as
      identified in Chapter 2} should be agonized. Indeed, Adhesion factors for endothelium
were shown to decrease tumour burden in mouse models \cite{asosinghUniquePathwayHoming2001a,mrozikTherapeuticTargetingNcadherin2015}

\citet{bouzerdanAdhesionMoleculesMultiple2022}: "Classically, the BMM has been
divided into endosteal and vascular niches"

Together, a detailed mapping of the niches available in the bone marrow is required
to understand the adhesion factors required for each niche. This is a highly
complex task, as the bone marrow is a highly complex organ.



% ======================================================================
% \unnsubsection{What Triggers Release: One Master Switch, Many Small Switches, or is it just Random?}%
% \label{sec:discussion_many_small_switches}%

Papers like \citet{akhmetzyanovaDynamicCD138Surface2020} make it seem as if
there is one molecule that decides if a myeloma cell is circulating or not.

It's less about one clear (molecular) mechanism that decides that a myeloma cell
decides to become a disseminating cell, but rather a indirect consequence of a combination of many
processes.
These processes are:
- Loss of adhesion factors or dynamic expression of adhesion factors
- Loss of dependency from bone marrow microenvironment
- asdf

Our thesis postulates that there is no big switch that decides if a myeloma cell
detaches from the bone marrow, but rather a prolonged process of continuisly
downregulating adhesion factors, a dynamic upregulation of adhesion factors when
they're needed, but the ultimate event that triggers release is better
explained by external mechanical forces intercellular effects (cell division,
saturation of adhesive surface and rising instability of aggregates after
reaching a minimum size).

% Hence, \emph{detachment simply happens} once these processes are present. 



%%%%%%%%%%%%%%%%%%%%%%%%%%%%%%%%%%%%%%%%%%%%%%%%%%%%%%%%%%%%%%%%%%%%%%%%%%%%%%%

%%%%%%%%%%%%%%%%%%%%%%%%%%%%%%%%%%%%%%%%%%%%%%%%%%%%%%%%

% ======================================================================
\unnsubsection{Outlook: High-Value Research Topics for Myeloma Research Arising from this Work}
\label{sec:discussion_potential_breakthroughs}
As an Outlook, the Author lists research topics arising from this work that have
great potential for breakthroughs in myeloma research.

\textbf{Anti tumor effects of MSCs:}
This thesis has discussed the pro-tumor effects of MSCs. However, MSCs have also
been shown to have anti-tumor effects \cite{galderisiMyelomaCellsCan2015}. This
work has also shown that primary \acp{hMSC} can induce apoptosis in \INA6 cells
initially \dashed{probably through the action of death domain receptors},
but inhibit apoptosis during prolonged culturing.

This shows that hMSCs could be leveraged
as a therapeutic target that could prevent myloma progression.




\textbf{Cell Division as a Mechanism for Dissemination Initiation:}
The author describes how the detachment of daughter cells from the mother cell
after a cycle of hMSC-(re)attachment and proliferation could be a key mechanism
in myeloma dissemination. This mechanism was shown in other studies of
intra-/extravasation. The author sees great potential in this mechanism as a target for
future research. It is probably under-researched due to requirement of
sophisticated time-lapse equipment, yet the simplicity of detachment through
cell division is intriguing through its simplicity. It implies asymmetric cell
division. Cancer cells are known to divide asymmetrically, e.g. moving miRNAs to
one daughter cell.

% \textbf{Time as a Key Parameter:}
% The area Thermodynamics of started with scientists measuring how long it takes
% for gases to cool down. The author claims, by measuring the time it takes for
% cancer cells to detach could lead to breakthroughs in research of myeloma
% dissemination.

% - Cell adhesion is highly time-dependent.
% - Cell detachment is required for metastasis and dissemination
% -

% key mechanistic insights

% measuring the minimum time
% for detachments to begin, or the time required for daughter cells to re-attach
% to the hMSC monolayer. Such mechanistic insights



% Time-resolution was mostly
% limited by available disk space. Investing into more hard drives is worth it,
% since

\textbf{Lists of Adhesion Gene Associated With Prolonged Patient Survival:}
The author lists adhesion genes that are associated with prolonged patient
survival. These genes are highly expressed in myeloma samples from patients with
longer overall

At this time we could be on the verge of a new era of myeloma therapy,
including bi-specific antibodies and cell based approaches
\cite{moreNovelImmunotherapiesCombinations2023,
      engelhardtFunctionalCureLongterm2024}. Currently, available CAR-T Cell therapies
(ide-cel, cilta-cel) are extremely expensive, but show complete remission rates
of up to \SI{80}{\percent} and a 18-month progression free survival rate of
\SI{66}{\percent} \cite{bobinRecentAdvancesTreatment2022}. An affordable
``off-the-shelf'' CAR-T Cell product could become reality since the problem of
deadly graft-versus-host disease during allogeneic transplantation seems to be
solvable \cite{qasimMolecularRemissionInfant2017}, hence, research groups and
biotech companies are racing towards developing a safe allogeneic CAR-T Cell
technology \cite{depilOfftheshelfAllogeneicCAR2020}.


the list of genes could be good targets because the BM niche is highly hypoxic,
car t cells are not well, but directing them to the BM niche could increase
efficacy.


\textbf{Find MSC and Myeloma crosstalk:}
Do another GSEA analysis using the list from factors upregulated in
\citet{dotterweichContactMyelomaCells2016}, since there, \INA and primary
\ac{hMSC} were used as well. Redoing an analysis with the background of the
associated procceses gained here could reveal insights on the communication
between \ac{hMSC} and \INA cells.



% ======================================================================
\unnsubsection{\textit{\textbf{Conclusion\,3:} The Dynamic Adhesion Hypothetical Framework for Myeloma Dissemination}}%
\label{sec:discussion_conclusion_cancer}%

lorem ipsum yes yes very bad








