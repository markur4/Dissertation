



% ======================================================================
\unnsubsection{Isolating \& Quantifying Subpopulations within Cells in Direct
      Contact with MSCs}%
\label{sec:discussion_novel_methods}%
This project aimed to develop methodologies for isolating cells after direct
contact with \acp{hMSC}. The primary challenge was the scarcity of \textit{in
      vitro} methods that could effectively separate and isolate adhering cell
subpopulations for subsequent molecular analysis. Most available techniques
predominantly focus on the quantification of cell adhesion
\cite{khaliliReviewCellAdhesion2015, kashefQuantitativeMethodsAnalyzing2015},
and often employ indirect contact setups, complex micromanipulation, or are
unsuitable for using live \acp{hMSC} as the immobilizing surface. To address the
limitations of current adhesion assays, we developed and enhanced innovative
methodologies, specifically the \acf{WPSC} and V-Well adhesion assays.

\textbf{Variability of Washing Steps}: Given the complexity of the requirements,
this project first attempts relied on simple and traditional adhesion assays
that rely on manual washing steps \cite{humphriesCellAdhesionAssays2009}.
Washing involves aspirating the medium, dispensing washing buffer, and
potentially repeating these steps multiple times. This introduces variability
due to differences in pipetting techniques, which affect the accuracy of volume
transfer \cite{guanAssessingVariationsManual2023,
      pushparajRevisitingMicropipettingTechniques2020}. However, adhesion assays don't
rely on precise volume transfer, but accurate detachment of cells adhering at
the well bottom. This introduces a new set of considerations for the pipetting
technique, especially since cells are highly sensitive to shear forces applied
by fluid flow. From the author's experience with washing experiments and
subsequent microscopic evaluations (data not shown), several factors could
contribute to the variability of washing steps:
\begin{enumerate}
      \item The distance of the pipette tip from the well bottom, which
            decreases during aspiration.
      \item The position of the pipette tip relative to the well bottom (center or edge).
      \item The angle of the pipette tip.
      \item The speed of aspiration.
      \item Accidental or intended contact between the pipette tip and the cell layer.
      \item The residual volume left after aspiration.
      \item \textit{The same considerations apply when dispensing the washing buffer.}
\end{enumerate}


In addition to user-dependent factors, other variables such as the cells'
position on the well bottom can significantly impact the outcome. To the
author's experience, cells located at the edge of the well don't detach as
easily as those in the center, while cells touching the edge are almost
impossible to remove. This phenomenon is likely related to the \textit{boundary
      layer effect}, where fluids slow down near the edges of the well
\cite{weyburneNewThicknessShape2014}.

Together, since both user-dependent and independent factors can affect the
outcome of washing steps, adhesive assays that replace washing are highly
desirable. Still, since washing is straightforward and some variability is
alleviated by the disciplined execution of washing protocols, it remains a
common method for adhesion assays.




\textbf{Directly Interacting Cells Contain Unexplored Interaction Scenarios}: It
is evident that direct and indirect contact to \acp{MSC} have varying effects on
myeloma cells. That difference is crucial for understanding changes in the
\ac{BMME} during MM progression \cite{fairfieldMultipleMyelomaCells2020,
      dziadowiczBoneMarrowStromaInduced2022}. These studies utilize well-inserts to
co-culture myeloma cells in close \dashed{indirect} contact with MSCs. However,
such comparison of indirect \textit{vs.} direct co-culturing methods might not
fully represent the complexity of intercellular interactions scenarios found in
the \ac{BMME}. This is exemplified by this project, as it relied on the complex
growth behavior: \INA cells aggregated homotypically in direct
proximity to those adhering heterotypically to \acp{hMSC}, and detached
through cell division. Furthermore, such methods fail to capture the subtle
variations in paracrine signaling concentrations, where even a few micrometers
of distance could significantly alter cellular responses.

Such knowledge shifted this project's point of view as well: Initially, our
hypothesis focused on direct heterotypic interactions, not expecting a \nMAina
population, but rather subpopulations within \MAina cells that are separable by
varying adhesion strengths. Hence, our assay employed strict conditions favoring
one growth scenario \dashed{heterotypic interactions}, with co-cultures
providing unlimited hMSC-surface availability causing predominantly heterotypic
adhesion, while the short incubation time prevented the formation of aggregates.
Despite these measures, our assay still captured cells emerging from recent cell
divisions rather than from weak heterotypic adherence as initially hypothesized.
This demonstrates the robustness of our method in separating subpopulations that
arising from unexpected intercellular interaction scenarios. This can be a major
an advantage over methods that summarize direct interactions as one population.
Analysing the non-adhering subpopulation within directly interacting cells could
provide valuable insights not just in multiple myeloma, but also metastasis of
other cancer types.


\textbf{Minimizing Variability}:
There are innovative adhesion assays that both
support the isolation of nonadherent subpopulations from directly interacting
cells, and avoid variability introduced by washing steps.

One simple method involves flipping over a 96-well plate, with surface tension
preventing medium spills as non-adhering cells fall to the surface for
collection \cite{zepeda-morenoInnovativeMethodQuantification2011}. However, we
found that the medium in fact did spill occasionally (not shown). Other
approaches involve sealing the plate, such as with PCR plate seals, and using
centrifugation to separate cells \cite{reyesCentrifugationCellAdhesion2003,
      chenHighThroughputScreeningTest2021}. Despite our efforts, we could not
consistently avoid air bubbles, which, after flipping, would contact the cell
layer and create dry regions during centrifugation.

The V-Well adhesion assay does not flip, but collects non-adhering cells into
the nadir of V-shaped wells during centrifugation
\cite{weetallHomogeneousFluorometricAssay2001}. This work profited greatly from
this method, while \dashed{to our knowledge} being the first to use cell
monolayers as the immobilizing surface. We value this method for its precision,
as centrifugation applies a uniform and configurable force, while the readout
remains straightforward, relying on the total fluorescent brightness rather than
individual cell counting.




\textbf{Specializing in Quantifying Adhesion or Isolating Subpopulations}: Most
adhesion assays primarily focus on quantification rather than isolation. The
author attempted to combine both quantification and isolation, but found that
the two goals can be mutually exclusive. The author summarizes the key
differences between quantification and isolation approaches as such:

\begin{itemize}
      \item Cell Manipulation for Harvest \textit{vs.} Readout:
            \begin{itemize}
                  \item Isolation methods are designed to manipulate cells for easy
                        harvest. For instance, the \ac{WPSC} method uses a catching
                        plate to collect non-adherent cells for subsequent analysis.
                  \item Quantification methods, on the other hand, manipulate cells
                        to simplify the readout process. For example, the V-Well
                        assay, which pellets cells into a single location, allowing
                        for a pooled fluorescence measurement without the need for
                        extensive cell handling.
            \end{itemize}

      \item Optimization for Subsequent Analysis \textit{vs.} Sample Throughput:
            \begin{itemize}
                  \item Isolation methods are optimized for detailed subsequent
                        analyses, such as RNA or protein analysis. For example,
                        \ac{WPSC} minimizes the introduction of biases such as those
                        from fluorescent staining, making it suitable for downstream
                        molecular assays.
                  \item Quantification methods are optimized for high sample
                        throughput. The V-Well assay, as an end-point assay, is
                        designed to efficiently handle multiple treatments
                        simultaneously, providing quick and comparative results with
                        lower cell numbers.
            \end{itemize}
      \item Handling of Cell Numbers:
            \begin{itemize}
                  \item Isolation methods, such as \ac{WPSC}, require multiple wells
                        (e.g., 96 wells) to gather a sufficient amount of cells per
                        subpopulation, which is crucial for robust downstream
                        analyses.
                  \item Quantification methods, exemplified by the V-Well assay, are
                        highly efficient even with low cell numbers.
            \end{itemize}
\end{itemize}



Thus, this adopted two distinct techniques for isolating and quantifying
directly interacting subpopulations, each optimizing for different outcomes, but
also supporting the separation of subpopulations within direct intercellular
interactions.

Still, it is theoretically possible to insert microscopy steps into the
\ac{WPSC} method to scan the well bottom for later cell counting. Also, this
work effectively isolated cell pellets from the V-well plate for subsequent
fixation and cell cycle profiling. The process was
tedious and required multiple technical replicates to achieve sufficient cell
numbers for analysis. It also required removing \ac{hMSC} from the V-well nadir
to prevent contamination during pellet aspiration.

Together, while both methods can combine quantification and isolation, they
are optimized towards either of them. Knowing these strengths and weaknesses
could help to advance these methods in future studies.





\textbf{Rationales of the Well Plate Sandwich Centrifugation}: Inspired by the
principles of both flipping and V-Well adhesion assays, we developed the Well
Plate Sandwich Centrifugation (\acf{WPSC}) method to address the challenges of
isolating cell populations. This method innovatively combines elements from both
techniques to provide a more reliable approach to cell isolation. One of the key
advantages of WPSC is its ability to reduce the variability commonly introduced
by manual pipetting. Instead of relying on aspiration, which introduce
variability in cell collection and requires touching the well bottom for
complete removal of medium, WPSC employs centrifugation to remove non-adhering
cells. Medium is then returned by pipetting to repeat the process and maximize
non-adhering cell collection, as the number of detachable cells plateau after
few rounds of centrifugation. Hence, this approach compromises
between minimizing washing variability and isolating larger quantities of cells.

The 96 well plate format has advantages, reducing spilling when flipping the
sandwich, as surface tension kept fluids in place. The 96 well plate format also
reduces per-well variability by performing the same washing procedure up to 96
times.

The slow centrifugation speeds used during \ac{WPSC} are also decided after
thorough consideration. For this, one has to discuss how exactly
non-adhering cells detach during centrifugation. While centrifugal force is an
obvious factor, the properties of cell adhesion are unclear under dry conditions
during centrifugation. The author assumed that the cells are being pulled along
by the medium as it is centrifuged into the catching plate. Hence, the centrifugation
speed was chosen as fast enough to transfer the medium, without completely
drying the co-culture plate and minimizing overall cell stress.

A significant challenge in \ac{WPSC} is the dissociation of \MAina from the hMSC
monolayer. WPSC employs two distinct techniques to achieve this dissociation.
The first technique involves repeated treatment with the gentle digestive enzyme
Accutase followed by \acp{MACS}. \ac{MACS}, despite being effective, is costly,
time-consuming, reduces overall cell yield, and potentially introduces biases
due to CD45 antibody selection and the requirement for cold-treatment. The
second technique utilizes strong pipetting to physically detach non-adhering
cells (termed \emph{`Wash'}). It is important to note that these techniques did
not affect the protocol on detaching \nMAina from the co-culture, hence
providing for a consistent ratio of isolated \MAina to \nMAina across all
experiments. Ultimately, we preferred \emph{Wash}, as \ac{MACS} had to be performed
on all samples to ensure comparability, reducing overall cell yield which became
limiting for downstream applications, especially for \nMAina cells. Both methods
achieved comparable purity of \MAina cells, with few hMSCs per $10e4$
\MAina cells (purity assessment not shown). \emph{Wash} probably pofited from
the highly durable nature of primary hMSC monolayers, whereas \emph{MACS}
required dissociation of the co-culture.

Together, \ac{WPSC} offers a versatile solution for isolating hMSC-interacting
myeloma cells. It successfully balances the need for precision with the ability
to handle larger cell quantities. \ac{WPSC} could be adapted to other cell types
that combines monolayer forming and suspension cells.



\textbf{Key Points:} Ultimately, this work established two methodologies
that could represent a significant advancement in the field of adhesion assays,
providing cost-effective, precise, reliable, and reproducible techniques for
both isolating and quantifying subpopulations within co-cultures of directly
interacting cell types. They offered valuable insights into the mechanisms of MM
detachment and are potentially applicable to other research questions that focus
on multicellular interactions and complex growth scenarios.




%%%%%%%%%%%%
% ======================================================================

\newcommand{\footcadd}{%
      \emph{\acf{CADD}} (defined in this work): The observation and measurement
      of time-dependent changes in cell adhesion and detachment events.
      \ac{CADD} characterizes the time cells spend attached, migrating or
      detached and associates these phases with molecular signatures, such as
      \ac{CAM} expression or cell signaling mediated by \acp{CAM} or the
      microenvironment. \ac{CADD} expands traditional \emph{cell adhesion} by a
      time component and implies an intention to predict attachment and
      detachment events. A focus on dynamics is especially relevant for
      suspension cells that exhibit intricate attachment and detachment
      behavior.
      %
}


% \unnsubsection{Constructing a Hypothetical Framework of Dissemination}%
\unnsubsection{Integrating Evidence and Hypotheses for a Mechanistic
      Understanding of Dissemination}%
\label{sec:discussion_framework}%
The results outlined in Chapter 1 encompass various aspects of multiple myeloma
research, including colonization of the \ac{BMME}, myeloma-\ac{MSC}
interactions, and the association of adhesion factor expression with patient
survival and disease stages. Such a broad scope invites the formulation of
generalized conclusions, potentially compromising scientific rigor. The
following sections aim to clearly separate hypotheses from evidence to guide
further research on dissemination.

\textbf{Integrating Observations of \INA in the Multistep Dissemination Model:}
The results gained in this work fit well into the multistep model proposed by
\citet{zeissigTumourDisseminationMultiple2020}. For most steps, observations
were made that could inspire further hypotheses and research:


\begin{enumerate}
      \item \textbf{Retention:}
            \begin{itemize}
                  \item \textit{Observation:} \INA cells attach quickly and
                        strongly to \acp{hMSC}, forming stable aggregates.
                  \item \textit{Hypothesis:} Myeloma cells are retained in the
                        bone marrow microenvironment (BMME) through strong adhesion to
                        \acp{hMSC} and stable homotypic aggregation.
                  \item \textit{Experiment:} Inject \INA cells into mice and
                        examine bone lesions. Compare the growth patterns in mice
                        co-injected with an ICAM-1 or LFA-1$\alpha$ antibody, which
                        dissolve homotypic aggregates \textit{in vitro} and prevent
                        \INA growth \textit{in vivo}
                        \cite{kawanoHomotypicCellAggregations1991a,
                              klauszNovelFcengineeredHuman2017}. If disrupting aggregation
                        leads to diffuse bone colonization rather than focal lesions,
                        it supports the hypothesis that strong adhesion and
                        aggregation are crucial for retention in the \ac{BMME}.
            \end{itemize}
      \item \textbf{Release:}
            \begin{itemize}
                  \item \textit{Observation:} \INA cells detach from \acp{hMSC}
                        through cell division, and external forces can detach single
                        cells from \INA aggregates.
                  \item \textit{Hypothesis:} Myeloma cells detach from the BMME
                        through cell division and external forces after reaching a
                        minimal aggregate size.
                  \item \textit{Experiment:} Inject \INA cells into mice and
                        compare the cell cycle profiles of circulating cells versus
                        those in the bone marrow. Enrichment of G1/G0 cells among
                        circulating cells would support the hypothesis that detachment
                        is more likely shortly after cell division.
            \end{itemize}
      \item \textbf{Intra-/Extravasation:}
            \begin{itemize}
                  \item This study did not make experiments to study for
                        intra-/extravasation, but these phenomena could be
                        explored with similar methods, if MSCs were replaced by
                        endothelial cells.
            \end{itemize}
      \item \textbf{Colonization:}
            \begin{itemize}
                  \item \textit{Observation:} \INA cells exhibit quick
                        attachment to \acp{hMSC} within one hour and rapidly
                        upregulate numerous adhesion factors, including \ac{ECM}
                        factors.
                  \item \textit{Hypothesis:} Quick attachment and fast
                        expression of adhesion factors enhance the potential to
                        colonize new niches. This is particularly relevant as \INA
                        cells were isolated from the pleura, indicating an ability to
                        colonize extramedullary sites
                        \cite{burgerGp130RasMediated2001c}.
                  \item \textit{Experiment:} Inject \INA cells into mice and
                        observe if they colonize extramedullary sites. Compare
                        this to \INA cells with reduced adaptability to test the
                        hypothesis. Research is required to find techniques to
                        reduce such putative adaptability, one potential option
                        is using XRK3F2 to inhibit p62, an upstream activator of
                        NF-$\kappa$B \cite{adamikXRK3F2InhibitionP62ZZ2018}. In
                        fact, NF-$\kappa$B signaling seems a robust target,
                        given that it plays a role both in MM patients
                        \cite{sarinEvaluatingEfficacyMultiple2020}, and inducing
                        adhesion factor expression in \INA (this work).

            \end{itemize}
\end{enumerate}


These hypotheses  \dashed{based on observations from \INA cells} provide a starting
point for understanding myeloma dissemination. While these insights are
specialized for the \INA cell line, they inspire the development of a more
generalized framework applicable to a broader range of myeloma cases.

\textbf{Constructing a Generalizable Hypothetical Framework of Dissemination:}
A mechanistic understanding of myeloma dissemination remains elusive. Although
\citet{zeissigTumourDisseminationMultiple2020} described dissemination as a
multistep process, evidence is largely collected for individual steps, leaving
the connections between these steps unproven. As a result, the process of
dissemination is a patchwork of evidence fragments. The following sections aim
to integrate such fragments, especially those derived from the \INA cell line in
this work, to construct a more coherent understanding of myeloma dissemination.

In this context, the author introduces the \emph{Dynamic Adhesion Hypothetical Framework for
      Myeloma Dissemination}, which leverages direct observations of
\acf{CADD}\footnote{\footcadd\label{foot:cadd}}. \ac{CADD} characterizes
the time-dependent changes in cell adhesion and detachment, associating these
phases with molecular signatures like \ac{CAM} expression or cell signaling
mediated by \acp{CAM} and the microenvironment. By adding a temporal component,
\ac{CADD} aims to predict attachment and detachment events.



\newcommand{\caddadaptation}{ %
      \ac{CADD} is adapted in response to different microenvironments faced
      during dissemination %
}
\newcommand{\caddadaptationtitle}{ %
      \textit{Hypothesis 1}: \acf{CADD} is Adapted during Dissemination%
}%


\newcommand{\caddadaptibility}{ %
      High adaptability of \ac{CADD} is a hallmark of aggressive myeloma %
}%
\newcommand{\caddadaptabilitytitle}{ %
      \textit{Hypothesis 2}: High Adaptability of \ac{CADD} is a Hallmark of
      Aggressive Myeloma %
}%


\newcommand{\cadddiversity}{%
      \ac{CADD} is highly diverse within both patients and cell lines %
}%
\newcommand{\cadddiversitytitle}{ %
      \textit{Hypothesis 3}: \ac{CADD} is Highly Diverse Within both Patients
      and Cell Lines%
}%


\newcommand{\caddtrigger}{%
      Detachment is caused by multiple cues of varying nature, including
      external mechanical forces, cell division, loss of \ac{CAM} expression, or
      even pure chance. }%    
\newcommand{\caddtriggertitle}{ %
      \textit{Hypothesis 4}: Detachment is Caused by Multiple Cues of Varying
      Nature %
}%


\textbf{Key Hypotheses:}
The Dynamic Adhesion Hypothetical Framework is structured around four key
hypotheses, each addressing fundamental aspects of myeloma cell dissemination
based on both literature and the results of this work. These hypotheses are as
follows:

\begin{enumerate}[parsep=4pt]
      \item \caddadaptation
      \item \caddadaptibility
      \item \cadddiversity
      \item \caddtrigger
\end{enumerate}


This framework sets the stage for a detailed exploration of each hypothesis,
linking empirical data with hypothetical constructs to provide a comprehensive
framework that can help to identify commonalities in myeloma dissemination, but
also inform the development of targeted therapies.




% ======================================================================
\unnsubsection{\caddadaptationtitle}%
\label{sec:discussion_caddadaptation}%
As presented in Chapter 1, \MAina cells exhibit rapid upregulation of both
adhesion factors and chemoattractants, adapting their \textit{in vitro}
\ac{CADD} from homotypic aggregation to colonizing \acp{MSC}. This dynamic
behavior includes the loss of adhesion factor expression after cell division,
suggesting that myeloma cells can rapidly change their adhesion factor
expression in a highly dynamic manner. Given that \INA cells were isolated from
an extramedullary site \dashed{the pleura}, such changes likely facilitate
colonization of new microenvironments. This section explores the hypothesis that
MM cells adapt their \ac{CADD} during each step of dissemination.

% Predicting how and when
% myeloma cells regulate \ac{CADD} could be key to understanding and preventing
% dissemination.


\textbf{\ac{CADD} Adaptation Assumes Distinguishable Niches:} The multistep
model proposed by \citet{zeissigTumourDisseminationMultiple2020} posits that
myeloma cells acquire regulatory mechanisms specialized for each step of
dissemination. The author hypothesizes that the different niches involved in
these steps are unique enough to trigger distinct \ac{CADD} adaptations. This
requires thorough knowledge of separate niches.
\citet{granataBoneMarrowNiches2022} categorizes the \ac{BM} into sinusoidal,
arteriolar, and endosteal niches, each spatially and molecularly
distinguishable. The endosteal niche is home to \ac{MSC} and a majority of
plasma cells\footnotequote{We suggest that it is reasonable to approach the
      notion of physical plasma cell survival niches with some skepticism. It is clear
      that most BM plasma cells rely heavily on access to APRIL or BLyS (66, 70), and
      it appears that mature plasma cells are relatively stationary (59). However to
      us, that plasma cells must remain indefinitely in physical survival niches to
      survive is less obvious.}{wilmoreHereThereAnywhere2017}, and the vascular niches
\dashed{sinusoidal and arteriolar} include endothelial cells
\cite{zehentmeierStaticDynamicComponents2014, wilmoreHereThereAnywhere2017}.
Other niches encountered during dissemination include peripheral blood, lymph
nodes, and extramedullary sites. Comprehensive mapping and characterization of
these niches, including their adhesion molecules and soluble factors, is
necessary to understand the adhesion requirements for each niche. This is a
highly complex task, yet summarizing available information per niche could
provide a powerful basis.


\textbf{Distinct Adhesion Phenotypes Transitioning between Niches:}
Adhesion processes are well-documented in MM progression, particularly within
the \ac{BMME} \cite{bouzerdanAdhesionMoleculesMultiple2022}. However, the
dynamism of these processes remains unclear. In other cancers, different
adhesive phenotypes and transitions, such as those seen in
epithelial-mesenchymal transition (EMT), are common
\cite{gengDynamicSwitchTwo2014}. For myeloma, EMT-like phenotypes have been
described, but a clear association with distinct adhesion behaviors is hindered
by the cells maintaining their suspension state
\cite{roccaroCXCR4RegulatesExtraMedullary2015,
      qianSETDB1InducesLenalidomide2023}. This work might be the first to identify
adhesive subtypes through functional separation of detachable myeloma cells. As
presented earlier, expanding these findings could reveal transitions in adhesive
phenotypes during MM dissemination, such as overcoming retention, initiating
release, and establishing colonization.


\textbf{Extramedullary Niche:}
\ac{CADD} adaptation predicts a specialized set of adhesion factors for
extramedullary niches. A distinct phenotype has been proposed for extramedullary
myeloma\footnotequote{Our analysis concluded that the gain of CD44, loss of
CD56, loss of very late antigen-4 (VLA-4), imbalance of the chemokine
receptor-4-chemokine ligand-12 (CXCR4-CXCL12) axis, [...] show an increased
propensity [...] to leave the bone marrow and hone in extramedullary sites
giving rise to more aggressive extramedullary diseases.
}{guptaExtramedullaryMultipleMyeloma2022}, characterized by changes in
expression of CD44, CD56, VLA-4, and CXCR4
\cite{guptaExtramedullaryMultipleMyeloma2022}. The role of CXCR4 in mediating
adhesion factor expression is well established, particularly in extramedullary
MM cells \cite{roccaroCXCR4RegulatesExtraMedullary2015,
guptaExtramedullaryMultipleMyeloma2022}: Extramedullary myeloma cells
overexpress CXCR4, making them more responsive to cues that induce adhesion
factor expression, such as CD44/H-CAM.



\textbf{Vascular Niche:} 
\ac{CADD} adaptation predicts a specialized set of adhesion factors for
endothelial interaction, supporting intravasation and extravasation. Although
not assessed in this thesis, the vascular niche is a popular therapeutic
target for preventing dissemination \cite{neriTargetingAdhesionMolecules2012}.
Key adhesion factors like JAM-A and N-Cadherin have been highlighted as
potential targets \cite{solimandoHaltingViciousCycle2020,
mrozikTherapeuticTargetingNcadherin2015}. These factors were not differentially
expressed between subpopulations isolated in Chapter\,1, suggesting distinct
regulatory mechanisms for vascular versus \ac{MSC} interactions.

\textbf{Circulating MM:} 
An adaptive \ac{CADD} would predict that circulating MM cells lose adhesion
factors. Studies confirm that \dashed{compared to BM-resident cells} circulating
\ac{MM} cells exhibit reduced expression of multiple adhesion factors, including
$\alpha4\beta1$ and CD138/Syndecan-1
\cite{paivaDetailedCharacterizationMultiple2013,
paivaCompetitionClonalPlasma2011, akhmetzyanovaDynamicCD138Surface2020}.
Evidence suggests that a dynamic loss of CD138/Syndecan-1 and gain of JAM-C
causes intravasation, circulation, and dissemination of MM cells
\cite{akhmetzyanovaDynamicCD138Surface2020,
brandlJunctionalAdhesionMolecule2022}. This thesis also shows that \nMAina
cells, after emerging as daughter cells from \MAina, not only lose adhesion
factor expression but also exhibit increased survival during IL-6 deprivation,
potentially aiding survival in circulation.


\textbf{\textit{Intermediary Conclusion:} Evidence for Adhesion Phenotypes Lacks Functional Characterization and Proof of Phenotypic Transitions:}
The concept of \ac{CADD} adaptation is supported by the existence of distinct BM
niches and the identification of separable adhesion phenotypes. However, most
transitions between these phenotypes during dissemination are unexplored.
Functional characterization of adhesive phenotypes and their transitions could
provide a robust framework for understanding dissemination as a multistep
process, reinforcing the dynamic adhesion hypothetical framework. Mapping
adhesive properties for each involved niche could aid this endeavor.



\textbf{Implications for Therapy:}
Adhesion molecules have been targeted for therapy for over a decade
\cite{nairChapterSixEmerging2012, neriTargetingAdhesionMolecules2012}.
Especially inhibiting adhesion molecules involved in interaction with the
endothelium effectively reduces tumor burden in mouse models
\cite{asosinghUniquePathwayHoming2001a,
mrozikTherapeuticTargetingNcadherin2015}. A deeper understanding of how myeloma
cells regulate \ac{CADD} could be key to predicting and preventing
dissemination. \ac{CADD} adaptation suggests that different adhesion factors
should be either antagonized or agonized depending on their role. For instance,
adhesion factors involved in intravasation and extravasation should be
antagonized, while those facilitating BM retention should be
agonized\,\textemdash\autoref{tab:1} provides a list of potential retention
targets. However, care should be taken to not agonize adhesion factors that also
provide survival signals.



\textbf{Considerations for Research on Myeloma Cell Adhesion:}
Studying adhesion factors in MM \textit{in vitro} requires considering the
specific microenvironmental context. Some adhesion factors are not present in MM
cells but can be rapidly expressed with appropriate signals. Also, further
studies should differentiate between initial adhesion and upregulated adhesion
factors. For example, performing a \ac{WPSC} assay after 30 minutes of adhesion
could separate \INA-6 cells based on initial adhesion capability, with RNAseq of
\nMAina \textit{vs} \MAina identifying initial adhesion factors. This
differentiation could be crucial for predicting colonization potential across
niches, as initial adhesion is likely to be essential for subsequent growth in
\ac{BM} or extramedullary environments.


\textbf{\textit{Concluding Remarks:}} The exploration of \ac{CADD} adaptation
across various niches reveals a complex interplay between myeloma cells and
their environments, characterized by a dynamic regulation of adhesion factors.
The evidence presented supports the hypothesis that myeloma cells modify their
adhesion phenotype in response to the unique demands of each microenvironment
they encounter during dissemination. This adaptive capability suggests that
targeting these specific adhesion mechanisms could offer a promising strategy
for therapeutic intervention, particularly in preventing the colonization of new
niches. The distinctions between the adhesion phenotypes among the
niches—vascular, bone marrow, and extramedullary—underscore the necessity for a
targeted approach in therapy, which could involve modulation of specific
adhesion factors to either promote retention or prevent dissemination.

Despite these insights, the current understanding of the functional roles and
transitions of these adhesion phenotypes during myeloma progression remains
incomplete. Future research should focus on delineating these roles more clearly
by functional assays and real-time imaging to capture the dynamic changes in
adhesion factor expression during cell transition between niches. Such studies
will be crucial for validating the \ac{CADD} adaptation hypothesis and for
identifying potential therapeutic targets that could disrupt the dissemination
process at various stages.

\textbf{Future Directions:} It is imperative to further characterize these
adhesion factors in a controlled \textit{in vitro} environment, where specific
microenvironmental contexts are simulated. This approach will allow for a more
nuanced understanding of how adhesion factors are upregulated and their role in
niche-specific colonization. By integrating detailed molecular and cellular
analyses, such as single-cell RNA sequencing and proteomics, researchers can
identify critical adhesion factors that facilitate the initial colonization
processes. This knowledge could then inform the development of interventions
aimed at either enhancing or inhibiting these factors, thereby potentially
providing a more strategic approach to the management and treatment of multiple
myeloma.






% ======================================================================
\unnsubsection{\caddadaptabilitytitle}%
\label{sec:discussion_caddadaptability}%

Chapter 1 demonstrates one peculiar paradox of multiple myeloma
(MM) progression: The expression of adhesion factors is decreased as the
disease advances, but is swiftly increased in direct contact with \acp{MSC}
in \INA cells \dashed{a cell line isolated from highly advanced \ac{PCL}}
This assumption dictates that aggressive myeloma cells gain the ability to
dynamically express adhesion factors particularly fast. This section explores
the hypothesis that the fast adaptability of \ac{CADD} is a hallmark of
aggressive myeloma.

\textbf{Defining aggressiveness:} Aggressiveness in MM is often associated with
higher disease stages.

Is Disease stage a proxy for tumor aggressiveness?

- Higher disease stages imply changes in adhesion factors that favor
aggressiveness. In fact, \ac{ECM} from myeloma patients shows tumor-promoting
properties, much contrasting the tumor-abrogating to \ac{ECM} from healthy donors
\cite{ibraheemBMMSCsderivedECMModifies2019}

- Aggressiveness includes:

- Better Colonization of new niches, including extramedullary ones such as the pleura

- This implies a more diverse set of available adhesion factors

- Faster regulation to adapt to new niches

- Better survival in circulation

\textbf{Evidence for Adaptability:} 

Overall, it is known that plasma cells express different adhesion factors
compared to healthy plasma cells, implying  \cite{cookRoleAdhesionMolecules1997,
      bouzerdanAdhesionMoleculesMultiple2022}.

indeed, 3 temporal subtypes have been identified, associating higher risk with
faster changes over time \cite{keatsClonalCompetitionAlternating2012}.




yes, adhesion has prognostic value: A recent study by
\citet{huDevelopmentCellAdhesionbased2024} developed a cell adhesion-based
prognostic model for MM, calculating an adhesion-related risk score (ARRS) based
on expression of only twelve adhesion related genes.


Supporting Literature:

\begin{enumerate}
      \item \textbf{Disease Stage}
            \begin{itemize}
                  \item THIS WORK: Expression decreases during progression from
                        \ac{MGUS} to \ac{MMR} of adhesion factors involved in hMSC
                        adhesion.
                  \item The idea that MM pathogenesis involves transformative
                        processes has been around for decades
                        \cite{hallekMultipleMyelomaIncreasing1998}, but a
                        detailed understanding of changing adhesive properties
                        is still lacking, especially during the progression of
                        MM.
                  \item It is discussed that myeloma cell lines derived from
                        advanced stages show different expression than newly
                        diagnosed patients, discussing that they come from
                        multiply relapsed patients
                        \cite{sarinEvaluatingEfficacyMultiple2020}. This work
                        also shows that Myeloma cell lines have the lowest
                        expression of adhesion factors compared to all stages of
                        \ac{MM} and \ac{MGUS}.
                  \item For B-Cell Chronic Lymphocytic Leukemia, adhesion
                        molecule expression patterns define distinct phenotypes in
                        disease subsets \cite{derossiAdhesionMoleculeExpression1993}.
                  \item \citet{terposIncreasedCirculatingVCAM12016} reported an
                        increase in adhesion molecule expression of ICAM-1 and
                        VCAM-1 in patients with \ac{MM} compared to those with
                        \ac{MGUS} and \ac{aMM}.
                  \item However, \citet{perez-andresClonalPlasmaCells2005}
                        reported that CD40 is downregulated in \ac{PCL}
                        patients. Hence, different \acp{CAM} could serve
                        ambiguous roles in \ac{MM} progression.
            \end{itemize}

\end{enumerate}



How could this be studied?

Databases of expression from Myeloma cells gathered from bone
marrow \ac{MGUS}, \ac{aMM}, \ac{MM}, \ac{MMR} already exist
\citet{akhmetzyanovaDynamicCD138Surface2020,
      seckingerCD38ImmunotherapeuticTarget2018}. Going through such databases gives a
good overview. One could categorize genelists using curated databases, get lists
associated with extravasation, intravasation, Bone marrow adhesion. For every
gene of these genelists, they could be filtered for significant differences
between the stages. Further categorizations of pairwise comparisons of stages
are required. but overall, these genelists could be a starting point for This
approach is similar to the genelists published in chapter 1, with the difference
that the genelist was furthere filtered by the RNAseq results of \textit{in
      vitro} experiments.



What new implications do these dimensions have on targeting adhesion factors for
therapy?

- Specialized treatment for each stage?

- Aggressive MM cells have potantial improved control over adhesion factor
expression, regulating a more diverse set of adhesion factors faster. This poses
further challenges to targeting. It could be smarter to not target
effector-molecules, but rather upstream regulators of adhesion. THis work shows
that NF-$\kappa$B signaling, which by itself is not treatable, but regulators
downstream of NF-$\kappa$B were shown to be effective
\cite{adamikEZH2HDAC1Inhibition2017,adamikXRK3F2InhibitionP62ZZ2018}




% ======================================================================
\unnsubsection{\cadddiversitytitle}%
\label{sec:discussion_cadddiversity}%

- Describe different cell lines: MM1.S being plastic adhering moderately
MSC-adhering non-aggregating, INA-6 being non adhering aggregate forming and
MSC-adhering, U266 being plastic adhering, non MSC-adhering and
non-aggregating.

- Results from this work: CXCL12 expresion varies from QM between QM

One important dimension that is missing here is the genetic background of the
myeloma cells. These are based on recurrent patterns of chromosomal aberrations
or mutational signatures, defining structural and single nucleotide variants
\cite{kumarMultipleMyelomasCurrent2018a,
      hoangMutationalProcessesContributing2019}. The prognostic value of genetic
variants in MM is well established \cite{sharmaPrognosticRoleMYC2021}, and their
identification is becoming precise and cost-effective using \emph{optical
      genome mapping}, making progress towards personalized therapies
\cite{zouComprehensiveApproachEvaluate2024,
      budurleanIntegratingOpticalGenome2024}. The prognostic value of adhesion factor
expression is nowhere nearly as advanced, with establishing cell adhesion as a
reliable prognostic factor only recently
\cite{huDevelopmentCellAdhesionbased2024}.

What markers can be used to categorize these differences?
- Maybe IL-6 dependency/independency \cite{sprynskiRoleIGF1Major2009}?
- \textit{in vitro} growth characteristics: Plastic adherence, MSC ahderence, aggregation

% ======================================================================
\unnsubsection{\caddtriggertitle}%
\label{sec:discussion_caddtrigger}%

biological implications:
- Different cues could trigger different adhesional changes
- Soluble signals?
- Loss of CD138 \cite{akhmetzyanovaDynamicCD138Surface2020}
- Detachment through intercellular effects: cell division, Saturation of hMSC adhesion surface
- Detachment with mechanical influence: External forces and instability after aggregate size
-


why is this important?
The cues that trigger the detachment of MM cells are not well understood. It
could be that MM cells detach due to a combination of factors, such as loss of
adhesion factors, changes in the BM microenvironment, or cell division or
even completely random. Knowing specific dissemination signals helps preventing
dissemination.


Papers like \citet{akhmetzyanovaDynamicCD138Surface2020} make it seem as if
there is one molecule that decides if a myeloma cell is circulating or not.

It's less about one clear (molecular) mechanism that decides that a myeloma cell
decides to become a disseminating cell, but rather a indirect consequence of a
combination of many processes.
These processes are:
- Loss of adhesion factors or dynamic expression of adhesion factors
- Loss of dependency from bone marrow microenvironment
- asdf

Our thesis postulates that there is no big switch that decides if a myeloma cell
detaches from the bone marrow, but rather a prolonged process of continuisly
downregulating adhesion factors, a dynamic upregulation of adhesion factors when
they're needed, but the ultimate event that triggers release is better
explained by external mechanical forces intercellular effects (cell division,
saturation of adhesive surface and rising instability of aggregates after
reaching a minimum size).

Detachment is triggered by external mechanical forces on cell
conglomerates previously sensitized by changes in cell adhesion behaviour

Supporting Literature:

\begin{enumerate}
      \item \textbf{Cues or Processes}
            \begin{itemize}
                  \item This work showed that detachment happened mostly
                        mechanically and cell biologically through cell
                        division. - Detachment through intercellular effects:
                        cell division, Saturation of hMSC adhesion surface -
                        Detachment with mechanical influence: External forces
                        and instability after aggregate size.
                  \item Soluble signals within the BM microenvironment, such as
                        cytokines and chemokines, play significant roles in
                        modulating adhesion factor expression in MM cells
                        \cite{aggarwalChemokinesMultipleMyeloma2006,
                              alsayedMechanismsRegulationCXCR42007}.
                  \item CD138 was proposed as a switch between adhesion and
                        migration in MM cells, its blockage triggering migration
                        and intravasation
                        \cite{akhmetzyanovaDynamicCD138Surface2020}.
            \end{itemize}
\end{enumerate}



How can this be studied?

Identifying such signals might be challenging without
having understood the other two hypotheses about adaptability first.




What new implications do these dimensions have on targeting adhesion factors for
therapy?


- It could represent a valid strategy to
strengthen myeloma adhesion, provided that targeted adhesion molecule is proven
to not be involved in other steps of dissemination, such as extravasation.
Stimulating adhesion factor expression or activity is harder than inhibition,
yet not impossible. For instance, the short polypeptide SP16 can activate the
receptor LRP1 \dashed{its high expression being associated with improved
      survival of MM patients in this work}, showing promising results during phase I
clinical trial \cite{wohlfordPhaseClinicalTrial2021}, but could potentially
increase survival of MM through PI3K/Akt signaling
\cite{potereDevelopingLRP1Agonists2019, heinemannInhibitingPI3KAKT2022} -

- One could also accept that many cues are simply not controllable, and hope for
systemic therapies like CAR- T Cells






%%%%%%%%%%%%%%%%%%%%%%%%%%%%%



% ======================================================================
\unnsubsection{Outlook: High-Value Research Topics for Myeloma Research Arising
      from this Work}
\label{sec:discussion_potential_breakthroughs}
As an Outlook, the Author lists research topics arising from this work that have
great potential for breakthroughs in myeloma research.

\textbf{Anti tumor effects of MSCs:}
This thesis has discussed the pro-tumor effects of MSCs. However, MSCs have also
been shown to have anti-tumor effects \cite{galderisiMyelomaCellsCan2015}. This
work has also shown that primary \acp{hMSC} can induce apoptosis in \INA6 cells
initially \dashed{probably through the action of death domain receptors},
but inhibit apoptosis during prolonged culturing.

This shows that hMSCs could be leveraged
as a therapeutic target that could prevent myloma progression.




\textbf{Cell Division as a Mechanism for Dissemination Initiation:}
The author describes how the detachment of daughter cells from the mother cell
after a cycle of hMSC-(re)attachment and proliferation could be a key mechanism
in myeloma dissemination. This mechanism was shown in other studies of
intra-/extravasation. The author sees great potential in this mechanism as a
target for future research. It is probably under-researched due to requirement
of sophisticated time-lapse equipment, yet the simplicity of detachment through
cell division is intriguing through its simplicity. It implies asymmetric cell
division. Cancer cells are known to divide asymmetrically, e.g. moving miRNAs to
one daughter cell.

% \textbf{Time as a Key Parameter:}
% The area Thermodynamics of started with scientists measuring how long it takes
% for gases to cool down. The author claims, by measuring the time it takes for
% cancer cells to detach could lead to breakthroughs in research of myeloma
% dissemination.

% - Cell adhesion is highly time-dependent.
% - Cell detachment is required for metastasis and dissemination
% -

% key mechanistic insights

% measuring the minimum time
% for detachments to begin, or the time required for daughter cells to re-attach
% to the hMSC monolayer. Such mechanistic insights



% Time-resolution was mostly
% limited by available disk space. Investing into more hard drives is worth it,
% since

\textbf{Lists of Adhesion Gene Associated With Prolonged Patient Survival:}
The author lists adhesion genes that are associated with prolonged patient
survival. These genes are highly expressed in myeloma samples from patients with
longer overall

At this time we could be on the verge of a new era of myeloma therapy,
including bi-specific antibodies and cell based approaches
\cite{moreNovelImmunotherapiesCombinations2023,
      engelhardtFunctionalCureLongterm2024}. Currently, available CAR-T Cell therapies
(ide-cel, cilta-cel) are extremely expensive, but show complete remission rates
of up to \SI{80}{\percent} and a 18-month progression free survival rate of
\SI{66}{\percent} \cite{bobinRecentAdvancesTreatment2022}. An affordable
``off-the-shelf'' CAR-T Cell product could become reality since the problem of
deadly graft-versus-host disease during allogeneic transplantation seems to be
solvable \cite{qasimMolecularRemissionInfant2017}, hence, research groups and
biotech companies are racing towards developing a safe allogeneic CAR-T Cell
technology \cite{depilOfftheshelfAllogeneicCAR2020}.


the list of genes could be good targets because the BM niche is highly hypoxic,
car t cells are not well, but directing them to the BM niche could increase
efficacy.


\textbf{Find MSC and Myeloma crosstalk:}
Do another GSEA analysis using the list from factors upregulated in
\citet{dotterweichContactMyelomaCells2016}, since there, \INA and primary
\ac{hMSC} were used as well. Redoing an analysis with the background of the
associated procceses gained here could reveal insights on the communication
between \ac{hMSC} and \INA cells.



% ======================================================================
\unnsubsection{\textit{\textbf{Conclusion\,3:} The Dynamic Adhesion Hypothetical
            Framework for Myeloma Dissemination}}%
\label{sec:discussion_conclusion_cancer}%



How does limited understanding of one dimension prevent the understanding of the
other dimensions?

Location \& Progression: If we don't know the expression profile of an MM cell
depending on their

source, results become incomparable.

Location \& Cues: If we don't know the cues that trigger detachment, we can't
predict where the MM cells will detach.



