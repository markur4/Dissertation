



% ======================================================================
\unnsubsection{Isolating \& Quantifying Subpopulations within Cells in Direct Contact with MSCs}%
\label{sec:discussion_novel_methods}%
This project aimed to develop methodologies for isolating cells after direct
contact with \acp{hMSC}. The primary challenge was the scarcity of \textit{in
    vitro} methods that could effectively separate and isolate adhering cell
subpopulations for subsequent molecular analysis. Most available techniques
predominantly focus on the quantification of cell adhesion
\cite{khaliliReviewCellAdhesion2015, kashefQuantitativeMethodsAnalyzing2015},
and often employ indirect contact setups, complex micromanipulation, or are
unsuitable for using live \acp{hMSC} as the immobilizing surface. To address the
limitations of current adhesion assays, we developed and enhanced innovative
methodologies, specifically the \acf{WPSC} and V-Well adhesion assays.

\textbf{Variability of Washing Steps}: Given the complexity of the requirements,
this project first attempts relied on simple and traditional adhesion assays
that rely on manual washing steps \cite{humphriesCellAdhesionAssays2009}.
Washing involves aspirating the medium, dispensing washing buffer, and
potentially repeating these steps multiple times. This introduces variability
due to differences in pipetting techniques, which affect the accuracy of volume
transfer \cite{guanAssessingVariationsManual2023,
    pushparajRevisitingMicropipettingTechniques2020}. However, adhesion assays don't
rely on precise volume transfer, but accurate detachment of cells adhering at
the well bottom. This introduces a new set of considerations for the pipetting
technique, especially since cells are highly sensitive to shear forces applied
by fluid flow. From the author's experience with washing experiments and
subsequent microscopic evaluations (data not shown), several factors could
contribute to the variability of washing steps:
\begin{enumerate}
    \item The distance of the pipette tip from the well bottom, which decreases during aspiration.
    \item The position of the pipette tip relative to the well bottom (center or edge).
    \item The angle of the pipette tip.
    \item The speed of aspiration.
    \item Accidental or intended contact between the pipette tip and the cell layer.
    \item The residual volume left after aspiration.
    \item \textit{The same considerations apply when dispensing the washing buffer.}
\end{enumerate}


In addition to user-dependent factors, other variables such as the cells'
position on the well bottom can significantly impact the outcome. To the
author's experience, cells located at the edge of the well don't detach as
easily as those in the center, while cells touching the edge are almost
impossible to remove. This phenomenon is likely related to the \textit{boundary
    layer effect}, where fluids slow down near the edges of the well
\cite{weyburneNewThicknessShape2014}.

Together, since both user-dependent and independent factors can affect the
outcome of washing steps, adhesive assays that replace washing are highly
desirable. Still, since washing is straightforward and some variability is
alleviated by the disciplined execution of washing protocols, it remains a
common method for adhesion assays.




\textbf{Directly Interacting Cells Contain Unexplored Interaction Scenarios}: It
is evident that direct and indirect contact to \acp{MSC} have varying effects on
myeloma cells. That difference is crucial for understanding changes in the
\ac{BMME} during MM progression \cite{fairfieldMultipleMyelomaCells2020,
    dziadowiczBoneMarrowStromaInduced2022}. These studies utilize well-inserts to
co-culture myeloma cells in close \dashed{indirect} contact with MSCs.
However, such comparison of indirect \textit{vs.} direct co-culturing methods
might not fully represent the complexity of intercellular interactions scenarios
found in the \ac{BMME}. This is exemplified by this project, as it relied on the
complex growth behavior of \INA cells. These aggregated homotypically in direct
proximity to those adhering heterotypically to \acp{hMSC}. Furthermore, such
methods fail to capture the subtle variations in paracrine signaling
concentrations, where even a few micrometers of distance could significantly
alter cellular responses.

Such knowledge shifted this project's point of view as well: Initially, our
hypothesis focused on direct heterotypic interactions, not expecting a \nMAina
population, but rather subpopulations within \MAina cells that are separable by
varying adhesion strengths. Hence, our assay employed strict conditions favoring
one growth scenario \dashed{heterotypic interactions}, with co-cultures providing unlimited hMSC-surface
availability causing predominantly heterotypic adhesion, while the short
incubation time prevented the formation of aggregates. Despite these measures,
our assay still captured cells emerging from recent cell divisions rather than
from weak heterotypic adherence as initially hypothesized. This demonstrates the
robustness of our method in separating subpopulations that arising from unexpected
intercellular interaction scenarios. This can be a major an advantage over
methods that summarize direct interactions as one population. Analysing the
non-adhering subpopulation within directly interacting cells could provide
valuable insights not just in multiple myeloma, but also metastasis of other
cancer types.


\textbf{Minimizing Variability}:
There are innovative adhesion assays that both
support the isolation of nonadherent subpopulations from directly interacting
cells, and avoid variability introduced by washing steps.

One simple method involves flipping over a 96-well plate, with surface tension
preventing medium spills as non-adhering cells fall to the surface for
collection \cite{zepeda-morenoInnovativeMethodQuantification2011}. However, we
found that the medium in fact did spill occasionally (not shown). Other
approaches involve sealing the plate, such as with PCR plate seals, and using
centrifugation to separate cells \cite{reyesCentrifugationCellAdhesion2003,
    chenHighThroughputScreeningTest2021}. Despite our efforts, we could not
consistently avoid air bubbles, which, after flipping, would contact the cell
layer and create dry regions during centrifugation.

The V-Well adhesion assay does not flip, but collects non-adhering cells into
the nadir of V-shaped wells during centrifugation
\cite{weetallHomogeneousFluorometricAssay2001}. This work profited greatly from
this method, while \dashed{to our knowledge} being the first to use cell
monolayers as the immobilizing surface. We value this method for its precision,
as centrifugation applies a uniform and configurable force, while the readout
remains straightforward, relying on the total fluorescent brightness rather than
individual cell counting.




\textbf{Specializing in Quantifying Adhesion or Isolating Subpopulations}: Most
adhesion methods primarily focus on quantification rather than isolation. While
we were able to isolate cell pellets from the V-well plate, the process was
tedious and required multiple technical replicates to achieve sufficient cell
numbers for analysis. The author summarizes the key differences between
quantification and isolation approaches as such:
\begin{itemize}
    \item Isolation methods manipulate cells for collect, whereas quantification
          methods could manipulate cells to simplify readout. For instance, the
          V-well assay pellets cells for one pooled fluorescence readout.
    \item Isolation methods optimize for collecting large cell numbers, whereas
          quantification methods optimize for precise separation.
    \item Isolation methods return at least two different readouts generated
          from adherent and non-adherent populations. Quantification methods
          return one output (population size), allowing higher throughput and
          different treatments.
\end{itemize}

Thus, we adopted two distinct techniques for isolating and quantifying directly
interacting subpopulations, each optimizing for different outcomes while
allowing the development of direct intercellular interactions.






% \textbf{Challenges of Wellplate Sandwich Centrifugation}:
% - Inspired by both
% flipping and V-Well adhesion assays, we developed the \acf{WPSC} method for
% isolation.
% - For Collecting cells, we can risk the involvement of pipette, by
% returning medium through pipetting.
% - Variance of washing is reduced by replacing
% the aspiration step with centrifugation.
% - This represents a compromise between
% precision and isolating larger quantities cells.
% - We use a 96 well plate format
% as it allows for both flipping with reduced risk of medium spilling, but also
% reducing variability, as within-experiment variability is reduced by performing
% the same washing procedure 96 times.
% - In fact, the author assumes that the force
% causing detachment is not directly caused by centrifugation, but rather surface
% tension of the medium pulling cells along as the medium is centirifuged into the
% catching plate.


% - Major challenge solved different ways: How to dissociate \MAina from \ac{MSC} monolayr?

% The Well Plate
% Sandwich Centrifugation (WPSC) used two different techniques to dissociate
% \MAina cells from the hMSC monolayer. This had no impact on the ratio of
% isolated \MAina to \nMAina, since \nMAina isolation was performed prior to
% dissociation using the same protocol consistently. We tried this to test if MACS
% was really necessary, after all it is costly, time-consuming, introduces an antibody bias
% and requires cell cold-treatment during antibody: 
% - First: repeated Accutase treatment followed by magnetic activated
% cell sorting (called \emph{`MACS'}).
% - Second: Strong pipetting (called \emph{`Wash'}) 



\textbf{Challenges of Well Plate Sandwich Centrifugation}: Inspired by the
principles of both flipping and V-Well adhesion assays, we developed the Well
Plate Sandwich Centrifugation (\acf{WPSC}) method to address the challenges of
isolating cell populations. This method innovatively combines elements from both
techniques to provide a more reliable approach to cell isolation. One of the key
advantages of WPSC is its ability to reduce the variability commonly introduced
by manual pipetting. Instead of relying on aspiration, which introduce
variability in cell collection and requires touching the well bottom for
complete removal of medium, WPSC employs centrifugation to collect cells. Medium
is then returned by pipetting to repeat the process and maximize \nMAina
collection. The 96 Well plate format also has advantages, reducing spilling when
flipping the sandwich, but also reducing technical variability by performing the
same washing procedure up to 96 times.

At this point, the author is unsure what exactly causes cell detachment during
centrifugation. While centrifugal force is an obvious factor, the medium could
also be the main driver of detachment, with its surface tension pulling
cells along as it is centrifuged into the catching plate.

% is not directly caused by centrifugation, but rather surface
% tension of the medium pulling cells along as the medium is centrifuged into the
% catching plate.

The WPSC method represents a compromise between precision and the ability to
isolate larger quantities of cells.

% By using a 96-well plate format, the method
% benefits from the reduced risk of medium spilling associated with flipping,
% while also allowing for repeated washing procedures to be performed 96 times
% within a single experiment. This approach not only reduces variability within an
% experiment but also ensures that the same conditions are applied across all
% wells, enhancing the reproducibility of the results. The author hypothesizes
% that the force causing cell detachment in WPSC is not directly due to
% centrifugation, but rather the surface tension of the medium, which pulls the
% cells along as it is centrifuged into the catching plate.

One of the significant challenges addressed by WPSC is the dissociation of
mesenchymal stromal cell (hMSC)-interacting myeloma cells (MA-INA) from the hMSC
monolayer. WPSC employs two distinct techniques to achieve this dissociation
without affecting the ratio of isolated MA-INA to non-interacting myeloma cells
(nMA-INA). This consistency is crucial as the isolation of nMA-INA is performed
prior to dissociation using a standard protocol. The first technique involves
repeated treatment with Accutase followed by magnetic-activated cell sorting
(\emph{`MACS'}), which, despite being effective, is costly, time-consuming, and
introduces potential biases due to the use of antibodies and the requirement for
cell cold-treatment. The s






Together, these novel methodologies represent a significant advancement in the
field, providing cost-effective, precise, reliable, and reproducible techniques
for studying the interactions between \acp{hMSC} and myeloma cells. They offer
valuable insights into the mechanisms of MM detachment and potentially
contribute to a deeper understanding of the dynamic interplay within the BMME.




% ======================================================================
\unnsubsection{Dynamic Regulation of Adhesion Factors During Dissemination}%
\label{sec:discussion_dynamic_regulation}%

One main question arises:

INA-6 was initially isolated from plasma cell leukemia as an extramedullary
plasmacytoma located in the pleura from a donor of age.
There is not much more information available on the background of that patient \cite{TwoNewInterleukin6,burgerGp130RasMediated2001}.
But assuming that
This is a highly advanced
stage of myeloma. However,  Chapter 2 shows that adhesion factors are
lost during MM progression. INA-6 are highly adhesive to hMSCs.



This is a contradiction that needs to be resolved.

For example,
circulating MM cells show lower levels of integrin $\alpha4\beta1$
compared to those residing in the BM. Furthermore, treatment with a syndecan-1 blocking antibody
has been shown to rapidly induce the mobilization of MM cells from the BM to
peripheral blood in mouse models, suggesting that alterations in adhesion
molecule expression facilitate MM cell release
\cite{zeissigTumourDisseminationMultiple2020}.

However, INA-6 do not express adhesion factors. They do that only in hMSC presence
Hence MAINA-6 could be a smaller fraction of MM cells, specialized on preparing a new niche
for the rest of the MM cells. This could be a reason why they are so adhesive.

This assumption dictates that aggressive myeloma cells gain the ability
to dynamically express adhesion factors.
It could be that INA-6 has gained the capability to express adhesion factors
fast in order to colonize new niches, such as pleura from which they were
isolated.

This shows that not just the stage of the disease, but also the location of the
myeloma cells plays a role when considering adhesion factors. According to this, this thesis
predicts a low expression of adhesion factors in circulating myeloma cells,
but a high expression in adhesive cells, e.g. non-circulating, or rather those

indeed CD138 paper isolated cells from circulating MM cells \cite{akhmetzyanovaDynamicCD138Surface2020}

indeed, temporal subclones have been identified \cite{keatsClonalCompetitionAlternating2012}.

% ======================================================================
\unnsubsection{Subsets of Adhesion Factors Contribute To Different Steps of Adhesion}%
\label{sec:discussion_subsets_adhesion_factors}% 

- adhesion molecules during vascular involvement have these adhesion molecules: JAM-C
and CD138.
- NONE of Them were shown in Chapter 2 of this study, (except for JAM-B)


- One has to consider that intravasation and/or extravasation would require a different
set of adhesion factors than adhesion to BM or extramedullary environments.

This has great implications for targeting adhesion factors for therapy, as it
suggests that different adhesion factors should either be antagonized or
agonized depending on the function of the adhesion factor. According to this
assumption, adhesion factors involved in intra- and extravasation adhesion should be
antagonized, while adhesion factors involved in BM adhesion \dashed{as
    identified in Chapter 2} should be agonized. Indeed, Adhesion factors for endothelium
were shown to decrease tumour burden in mouse models \cite{asosinghUniquePathwayHoming2001a,mrozikTherapeuticTargetingNcadherin2015}

\citet{bouzerdanAdhesionMoleculesMultiple2022}: "Classically, the BMM has been
divided into endosteal and vascular niches"

Together, a detailed mapping of the niches available in the bone marrow is required
to understand the adhesion factors required for each niche. This is a highly
complex task, as the bone marrow is a highly complex organ.

% ======================================================================
\unnsubsection{What Triggers Release: One Master Switch, Many Small Switches, or is it just Random?}%
\label{sec:discussion_many_small_switches}%

Papers like \citet{akhmetzyanovaDynamicCD138Surface2020} make it seem as if
there is one molecule that decides if a myeloma cell is circulating or not.

It's less about one clear (molecular) mechanism that decides that a myeloma cell
decides to become a disseminating cell, but rather a indirect consequence of a combination of many
processes.
These processes are:
- Loss of adhesion factors or dynamic expression of adhesion factors
- Loss of dependency from bone marrow microenvironment
- asdf

Our thesis postulates that there is no big switch that decides if a myeloma cell
detaches from the bone marrow, \emph{it simply happens} once these processes are
present.


% ======================================================================
\unnsubsection{Outlook: High-Value Research Topics for Myeloma Research Arising from this Work}
\label{sec:discussion_potential_breakthroughs}
As an Outlook, the Author lists research topics arising from this work that have
great potential for breakthroughs in myeloma research.

\textbf{Anti tumor effects of MSCs:}
This thesis has discussed the pro-tumor effects of MSCs. However, MSCs have also
been shown to have anti-tumor effects \cite{galderisiMyelomaCellsCan2015}. This
work has also shown that primary \acp{hMSC} can induce apoptosis in \INA6 cells
initially \dashed{probably through the action of death domain receptors},
but inhibit apoptosis during prolonged culturing.

This shows that hMSCs could be leveraged
as a therapeutic target that could prevent myloma progression.




\textbf{Cell Division as a Mechanism for Dissemination Initiation:}
The author describes how the detachment of daughter cells from the mother cell
after a cycle of hMSC-(re)attachment and proliferation could be a key mechanism
in myeloma dissemination. This mechanism was shown in other studies of
extravasation. The author sees great potential in this mechanism as a target for
future research. It is probably under-researched due to requirement of
sophisticated time-lapse equipment, yet the simplicity of detachment through
cell division is intriguing through its simplicity. It implies asymmetric cell
division. Cancer cells are known to divide asymmetrically, e.g. moving miRNAs to
one daughter cell.

% \textbf{Time as a Key Parameter:}
% The area Thermodynamics of started with scientists measuring how long it takes
% for gases to cool down. The author claims, by measuring the time it takes for
% cancer cells to detach could lead to breakthroughs in research of myeloma
% dissemination.

% - Cell adhesion is highly time-dependent.
% - Cell detachment is required for metastasis and dissemination
% -

% key mechanistic insights

% measuring the minimum time
% for detachments to begin, or the time required for daughter cells to re-attach
% to the hMSC monolayer. Such mechanistic insights



% Time-resolution was mostly
% limited by available disk space. Investing into more hard drives is worth it,
% since

\textbf{Lists of Adhesion Gene Associated With Prolonged Patient Survival:}
The author lists adhesion genes that are associated with prolonged patient
survival. These genes are highly expressed in myeloma samples from patients with
longer overall

At this time we could be on the verge of a new era of myeloma therapy,
including bi-specific antibodies and cell based approaches
\cite{moreNovelImmunotherapiesCombinations2023,
    engelhardtFunctionalCureLongterm2024}. Currently, available CAR-T Cell therapies
(ide-cel, cilta-cel) are extremely expensive, but show complete remission rates
of up to \SI{80}{\percent} and a 18-month progression free survival rate of
\SI{66}{\percent} \cite{bobinRecentAdvancesTreatment2022}. An affordable
``off-the-shelf'' CAR-T Cell product could become reality since the problem of
deadly graft-versus-host disease during allogeneic transplantation seems to be
solvable \cite{qasimMolecularRemissionInfant2017}, hence, research groups and
biotech companies are racing towards developing a safe allogeneic CAR-T Cell
technology \cite{depilOfftheshelfAllogeneicCAR2020}.


the list of genes could be good targets because the BM niche is highly hypoxic, car t cells
are not well, but directing them to the BM niche could increase efficacy.



% ======================================================================
\unnsubsection{\textit{\textbf{Conclusion\,2:} Cancer \& Myeloma \& Dissemination is bad}}%
\label{sec:discussion_conclusion_cancer}%

lorem ipsum yes yes very bad








