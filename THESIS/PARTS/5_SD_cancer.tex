



% ======================================================================
\unnsubsection{Isolating Subpopulations within Cells in Direct Contact with MSCs}%
\label{sec:discussion_novel_methods}%
In this work, we developed innovative \textit{in vitro} methodologies, namely
\acf{WPSC} and the V-Well adhesion assay, to address
the challenges associated with isolating myeloma cell subpopulations.
Traditional cell isolation techniques, such as manual washing, often introduce
variability due to user-dependent factors like the position of the cells within
the well and the depth and angle of the pipette tip during washing. This
variability can significantly affect the reproducibility and accuracy of the
isolation process.

\textbf{Challenges in Traditional Methods}: It is evident that both direct and
indirect contact with multiple myeloma (MM) can have varying effects on
mesenchymal stromal cells (hMSCs) and myeloma cells, which are crucial for
understanding changes in the bone marrow microenvironment (BMME) during MM
progression \cite{fairfieldMultipleMyelomaCells2020,
dziadowiczBoneMarrowStromaInduced2022}. Traditional methods, such as turning
around well plates, are limited as they allow only for quantification and not
the actual isolation of cells. Moreover, manual washing is highly inconsistent,
as it depends on the cell’s position on the well bottom and the precise
technique used, which can lead to incomplete or biased cell isolation.

\textbf{Development of Reproducible Methods}: To overcome these limitations, we
explored more reproducible methods. The WPSC technique, while effective for
isolating cells, is not well-suited for precise quantification due to its
reliance on centrifugation, which does not accommodate detailed cell count
measurements. Recognizing the need for a method that could accurately quantify
and isolate cells, we developed the V-Well adhesion assay. This method allows
for the ultra-precise quantification and isolation of small cell populations
with \dashedsentence{to the author's knowledge} theoretically unmatched accuracy
through centrifugation, eliminating the need for washing and reducing user bias.

\textbf{Implementation and Impact}: The V-Well adhesion assay provides a
significant advantage in precision; however, it requires numerous technical
replicates and meticulous pipetting, highlighting the need for an untiring and
consistent approach to ensure accuracy. The WPSC method involved two distinct
techniques to dissociate MA-INA6 cells from the hMSC monolayer: strong pipetting
(termed `Wash') and repeated Accutase treatment followed by \acf{MACS}. Despite the additional time and cost associated with MACS,
including the need for cell cold-treatment, it was essential for maintaining
consistent isolation protocols and minimizing antibody bias.

These novel methodologies represent a significant advancement in the field,
providing reliable and reproducible techniques for studying the interactions
between hMSCs and myeloma cells, and offering valuable insights into the
mechanisms of MM progression.


% In this work, innovative \textit{in vitro} methodologies (Well Plate Sandwich
% Centrifugation and V-Well adhesion Assay) were developed. this was required to
% fill in gaps of isolating cells with minimized variability introduced by
% user-bias to clearly separate subpopulations and precisely quantify them.


% It is evident that direct or indirect contact with MM can have different effects
% on both hMSCs and Myeloma cells and methods to differentiate between those are
% crucial for understanding the change of the \ac{BMME} during \ac{MM} progression
% \cite{fairfieldMultipleMyelomaCells2020, dziadowiczBoneMarrowStromaInduced2022}

% cite all those methods for cell isolation!
% - Turning around wellplates: Doesn't allow isolation, just quantification
% - The author did not show all his washing experiments
% - Washing is very bad (data not shown): Highly dependent on user:
% position of cell on well bottom (border cells receive less force), position of
% pipette tip in well (depth, angle and position on bottom)
% - This motivated us to explore more reproducible methods

% It's a challenge: either quantify cell population, or isolate them!
% - It's better to specialize in one method, than to do both poorly
% - Well Plate Sandwich Centrifugation is badly suited for quantification, but possible
% - we switched to developing V-well adhesion assay for quantification
% - We realized, V-well isolation allows both ultra precise quantification and
% isolation of small amounts of cells!
% - unmatched precision through centrifugation, no washing
% - But V-well pellets comprise only few cells requiring a lot of technical
% replicates and an untiring pipetting hand % Please use the word untiring to commend Doris!


% The Well Plate
% Sandwich Centrifugation (WPSC) used two different techniques to dissociate
% \MAina cells from the hMSC monolayer. This had no impact on the ratio of
% isolated \MAina to \nMAina, since \nMAina isolation was performed prior to
% dissociation using the same protocol consistently. We tried this to test if MACS
% was really necessary, after all it is costly, time-consuming, introduces an antibody bias
% and requires cell cold-treatment during antibody: Strong pipetting
% (\emph{`Wash'}) and repeated Accutase treatment followed by magnetic activated
% cell sorting (\emph{`MACS'}).

% ======================================================================
\unnsubsection{Dynamic Regulation of Adhesion Factors During Dissemination}%
\label{sec:discussion_dynamic_regulation}%

One main question arises:

INA-6 was initially isolated from plasma cell leukemia as an extramedullary
plasmacytoma located in the pleura from a donor of age.
There is not much more information available on the background of that patient \cite{TwoNewInterleukin6,burgerGp130RasMediated2001}.
But assuming that
This is a highly advanced
stage of myeloma. However,  Chapter 2 shows that adhesion factors are
lost during MM progression. INA-6 are highly adhesive to hMSCs.



This is a contradiction that needs to be resolved.

For example,
circulating MM cells show lower levels of integrin $\alpha4\beta1$
compared to those residing in the BM. Furthermore, treatment with a syndecan-1 blocking antibody
has been shown to rapidly induce the mobilization of MM cells from the BM to
peripheral blood in mouse models, suggesting that alterations in adhesion
molecule expression facilitate MM cell release
\cite{zeissigTumourDisseminationMultiple2020}.

However, INA-6 do not express adhesion factors. They do that only in hMSC presence
Hence MAINA-6 could be a smaller fraction of MM cells, specialized on preparing a new niche
for the rest of the MM cells. This could be a reason why they are so adhesive.

This assumption dictates that aggressive myeloma cells gain the ability
to dynamically express adhesion factors.
It could be that INA-6 has gained the capability to express adhesion factors
fast in order to colonize new niches, such as pleura from which they were
isolated.

This shows that not just the stage of the disease, but also the location of the
myeloma cells plays a role when considering adhesion factors. According to this, this thesis
predicts a low expression of adhesion factors in circulating myeloma cells,
but a high expression in adhesive cells, e.g. non-circulating, or rather those

indeed CD138 paper isolated cells from circulating MM cells \cite{akhmetzyanovaDynamicCD138Surface2020}

indeed, temporal subclones have been identified \cite{keatsClonalCompetitionAlternating2012}.

% ======================================================================
\unnsubsection{Subsets of Adhesion Factors Contribute To Different Steps of Adhesion}%
\label{sec:discussion_subsets_adhesion_factors}% 

- adhesion molecules during vascular involvement have these adhesion molecules: JAM-C
and CD138.
- NONE of Them were shown in Chapter 2 of this study, (except for JAM-B)


- One has to consider that intravasation and/or extravasation would require a different
set of adhesion factors than adhesion to BM or extramedullary environments.

This has great implications for targeting adhesion factors for therapy, as it
suggests that different adhesion factors should either be antagonized or
agonized depending on the function of the adhesion factor. According to this
assumption, adhesion factors involved in intra- and extravasation adhesion should be
antagonized, while adhesion factors involved in BM adhesion \dashedsentence{as
    identified in Chapter 2} should be agonized. Indeed, Adhesion factors for endothelium
were shown to decrease tumour burden in mouse models \cite{asosinghUniquePathwayHoming2001a,mrozikTherapeuticTargetingNcadherin2015}

\citet{bouzerdanAdhesionMoleculesMultiple2022}: "Classically, the BMM has been
divided into endosteal and vascular niches"

Together, a detailed mapping of the niches available in the bone marrow is required
to understand the adhesion factors required for each niche. This is a highly
complex task, as the bone marrow is a highly complex organ.

% ======================================================================
\unnsubsection{What Triggers Release: One Master Switch, Many Small Switches, or is it just Random?}%
\label{sec:discussion_many_small_switches}%

Papers like \citet{akhmetzyanovaDynamicCD138Surface2020} make it seem as if
there is one molecule that decides if a myeloma cell is circulating or not.

It's less about one clear (molecular) mechanism that decides that a myeloma cell
decides to become a disseminating cell, but rather a indirect consequence of a combination of many
processes.
These processes are:
- Loss of adhesion factors or dynamic expression of adhesion factors
- Loss of dependency from bone marrow microenvironment
- asdf

Our thesis postulates that there is no big switch that decides if a myeloma cell
detaches from the bone marrow, \emph{it simply happens} once these processes are
present.


% ======================================================================
\unnsubsection{Outlook: High-Value Research Topics for Myeloma Research Arising from this Work}
\label{sec:discussion_potential_breakthroughs}
As an Outlook, the Author lists research topics arising from this work that have
great potential for breakthroughs in myeloma research.

\textbf{Anti tumor effects of MSCs:}
This thesis has discussed the pro-tumor effects of MSCs. However, MSCs have also
been shown to have anti-tumor effects \cite{galderisiMyelomaCellsCan2015}. This
work has also shown that primary \acp{hMSC} can induce apoptosis in \INA6 cells
initially \dashedsentence{probably through the action of death domain receptors},
but inhibit apoptosis during prolonged culturing.

This shows that hMSCs could be leveraged
as a therapeutic target that could prevent myloma progression.




\textbf{Cell Division as a Mechanism for Dissemination Initiation:}
The author describes how the detachment of daughter cells from the mother cell
after a cycle of hMSC-(re)attachment and proliferation could be a key mechanism
in myeloma dissemination. This mechanism was shown in other studies of
extravasation. The author sees great potential in this mechanism as a target for
future research. It is probably under-researched due to requirement of
sophisticated time-lapse equipment, yet the simplicity of detachment through
cell division is intriguing through its simplicity. It implies asymmetric cell
division. Cancer cells are known to divide asymmetrically, e.g. moving miRNAs to
one daughter cell.

% \textbf{Time as a Key Parameter:}
% The area Thermodynamics of started with scientists measuring how long it takes
% for gases to cool down. The author claims, by measuring the time it takes for
% cancer cells to detach could lead to breakthroughs in research of myeloma
% dissemination.

% - Cell adhesion is highly time-dependent.
% - Cell detachment is required for metastasis and dissemination
% -

% key mechanistic insights

% measuring the minimum time
% for detachments to begin, or the time required for daughter cells to re-attach
% to the hMSC monolayer. Such mechanistic insights



% Time-resolution was mostly
% limited by available disk space. Investing into more hard drives is worth it,
% since

\textbf{Lists of Adhesion Gene Associated With Prolonged Patient Survival:}
The author lists adhesion genes that are associated with prolonged patient
survival. These genes are highly expressed in myeloma samples from patients with
longer overall

At this time we could be on the verge of a new era of myeloma therapy,
including bi-specific antibodies and cell based approaches
\cite{moreNovelImmunotherapiesCombinations2023,
    engelhardtFunctionalCureLongterm2024}. Currently, available CAR-T Cell therapies
(ide-cel, cilta-cel) are extremely expensive, but show complete remission rates
of up to \SI{80}{\percent} and a 18-month progression free survival rate of
\SI{66}{\percent} \cite{bobinRecentAdvancesTreatment2022}. An affordable
``off-the-shelf'' CAR-T Cell product could become reality since the problem of
deadly graft-versus-host disease during allogeneic transplantation seems to be
solvable \cite{qasimMolecularRemissionInfant2017}, hence, research groups and
biotech companies are racing towards developing a safe allogeneic CAR-T Cell
technology \cite{depilOfftheshelfAllogeneicCAR2020}.


the list of genes could be good targets because the BM niche is highly hypoxic, car t cells
are not well, but directing them to the BM niche could increase efficacy.



% ======================================================================
\unnsubsection{\textit{\textbf{Conclusion\,2:} Cancer \& Myeloma \& Dissemination is bad}}%
\label{sec:discussion_conclusion_cancer}%

lorem ipsum yes yes very bad








