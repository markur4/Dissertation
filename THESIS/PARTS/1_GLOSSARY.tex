


% ======================================================================
%  ## Define footnotes here


% == Imagery ===========================================================

\def\footimagefeatures{%
    \emph{Image Features:} Discernable elements of an image, such as edges,
    corners, directions, colors. These features are mathematically extractable
    using \emph{filters} \dashed{also referred to as \emph{convolution
            kernels}}, which are functions or algorithms applied to the pixel values of
    an image. For instance, \emph{gabor filters} can extract edges of one
    particular direction, resulting in an image of the same size as the input,
    but showing only edges of one direction. \emph{Feature extraction} is the
    process of applying multiple filters, resulting in a stack of filtered
    images called a feature vector. \cite{Szeliski2011,
        guptaDeepLearningImage2019} }

\def\footcnn{%
    \emph{Convolutional Neural Networks (CNN):} Machine learning algorithms that
    use the output of a feature extractor\footref{foot:image_features} to feed
    into a neural network. The network then learns to associate these feature
    vectors with a label, such as \emph{cell} or \emph{background}. This is
    called \emph{supervised learning}. }


% == Adhesion Factors ==================================================

\def\footadhesionfactor{%
    \emph{Adhesion Factor}: Any factor influencing cell adhesion, including
    \dashed{but not limited to} \aclp{CAM} (\acsp{CAM}), \acl{ECM} (\acs{ECM})
    proteins, chemotactic factors associated with inducing adhesion factor
    expression \dashed{such as CXCR4 or CXCL12}, or factors listed in Gene
    ontology terms \textit{“extracellular matrix organization”}, \textit{“ECM
        proteoglycans”}, \textit{“cell-substrate adhesion”}, and \textit{“negative
        regulation of cell-substrate adhesion”}. }%

\def\footretentiveadhesionfactors{%
    \emph{Retentive Adhesion Factors}: The subset of adhesion factors that
    promotes achorage of myeloma cell in the \acs{BMME}, and promote better
    patient survival at high expression as shown in Chapter\,1.
}


% == Cell Adhesion Dynamics & Dramatypes ===============================


\def\footinteractionscenario{%
    \emph{Cell Interaction Scenario} (defined in this work): A combination of
    cellular interaction types describing direct contact or adhesion between
    cells. Cellular interaction types include those between similar cell types
    (homotypic interaction), different cell types (heterotypic interaction), or
    between cells and substrate. A cell interaction scenario usually implies
    that multiple cell interaction types occur simultaneously or in rapid
    succession, for instance, myeloma cells interacting with both other myeloma
    cells and \acsp{MSC}. When interaction scenarios emerge from cell division,
    the term \emph{growth conformation} is also used (see Chapter\,1)%
}


\def\footcad{%
    \emph{\acl{CAD} (\acs{CAD})}: The observation and measurement
    of time-dependent changes in cell adhesion and detachment events. \acs{CAD}
    expands traditional \emph{cell adhesion} by a time component and implies
    an intention to predict the timepoint of detachment events. Such focus on
    dynamics is especially relevant for suspension cells that exhibit
    intricate adhesion behaviors. Chapter\,1 also refers to CAD as
    attachment/detachment dynamics. %
}



% \def\footdramatype{%
%     Environmental influences from the fertilization of an egg [...] through to
%     sexual maturity are referred to as the primary milieu. The interaction
%     between this milieu and the genotype will give rise to the phenotype. The
%     phenotypical properties will subsequently be influenced by the
%     pre-experiment conditions which are referred to as the secondary milieu.
%     As a result, the dramatype is formed. Furthermore the laboratory animal
%     will be affected by experimental procedures and treatments known as the
%     tertiary milieu. %
%     % \citet{zutphenPrinciplesLaboratoryAnimal2001}
% }



\def\footcaddt{%
    \emph{CAD Dramatype} (defined in this work): Specific adhesion behavior
    caused by proximate environmental factors. The term \emph{dramatype} was
    inspired from laboratory animal science
    \cite{zutphenPrinciplesLaboratoryAnimal2001}. A CAD dramatype is
    characterized by the duration cells spend in distinct adhesive states or
    interaction scenarios\footref{foot:interactionscenario}, and the cause of
    transitions between these states and scenarios. Adhesive states include
    attached, migrating, or detached; interaction scenarios include homotypic,
    heterotypic or substrate interactions. CAD dramatypes are associated with
    molecular signatures, such as \acs{CAM} expression patterns or signal
    transduction mediated by proximate environmental factors. The term dramatype
    distinguishes itself from \emph{phenotype} by focusing on dynamic and transient
    states available within transcriptional plasticity, while \emph{phenotypes}
    focus on relatively persistant states such as genetic and epigenetic
    backgrounds.%
    %
}

\def\footadhesiondt{%
    \emph{Adhesion Dramatype} (defined in this work): Short version for
    \acs{CAD} dramatype\footref{foot:caddt}. Since the term \emph{dramatype}
    implies dynamic changes, \emph{CAD dramatypes} and \emph{adhesion
        dramatypes} are interchangable.%
}

\def\footadhesionplasticity{%
    \emph{Adhesional Plasticity}: The overall
    repertoire of adhesion dramatypes\footref{foot:adhesiondt} that individual
    myeloma cells can deploy. }

\def\footmultidimensionaldata{%
    \emph{Multidimensional Data} (specified in this work): Datasets where
    multiple \emph{independent variables} (factors; plotted on `\emph{x}-axis')
    can influence one \emph{dependent variable} (outcome; plotted on
    `\emph{y}-axis'). The terms \emph{independent variable} and \emph{dimension}
    are used interchangeably in this work. Visualizing multidimensional data
    requires further `axes' or dimensions next to that commonly referred to as
    `\emph{x}-axis'. For instance, displaying a second independent variable is
    often achieved by color coding data points or bars for scatter and bar
    plots, respectively. A third dimension can be added by changing the shape of
    datapoints (squares, triangles, etc.). A fourth dimension can be added by
    splitting the data and arranging individual plots next to each other
    (\textit{columns}), each plot representing one level of a fourth independent
    variable. A fifth dimension can be added similarly by arranging plots below
    each other (rows).%
    %
}

\def\footsemibigdata{%
    \emph{Semi-Big Data} (defined in this work): Datasets with a size that's
    manageable with high volume manual analysis, but are prone to long-term
    reproducibility bottlenecks. The semi-big size invites scientists without
    resources or proficiency in scripting automation to rely on manual analysis
    tools, such as \textit{Excel}, \textit{GraphPad Prism}. This reliance
    impedes reproducibility (transparency) because manual methods typically do
    not offer adequate documentation for complex analysis steps. Furthermore,
    the volume of semi-big data makes independent validation with manual methods
    unfeasible.}

\def\semiautomation{
    \emph{Semi-automation} (specified in this work): The process of combining
    manual and automated analysis. %
}

% lorem ipsum dolor sit amet, consectetur adipiscing elit. Donec auctor, nunc
% eget ultricies ultricies, nunc nunc ultricies ultricies, nunc nunc ultricies
% ultricies, nunc nunc ultricies ultricies, nunc nunc ultricies ultricies, nunc
% nunc ultricies ultricies, nunc nunc ultricies ultricies, nunc nunc ultricies
% ultricies, nunc nunc ultricies ultricies, nunc nunc ultricies ultricies, nunc
% nunc ultricies ultricies, nunc nunc ultricies ultricies, nunc nunc ultricies
% ultricies, nunc nunc ultricies ultricies, nunc nunc ultricies ultricies, nunc
% nunc ultricies ultricies, nunc nunc ultricies ultricies, nunc nunc ultricies


% \acresetall

% ======================================================================

% ## Generate list of footnotes
\addcontentsline{toc}{section}{\notessectionname}%

\begin{spacing}{1.2}%
    \theendnotes\label{sec:footnotes}%
\end{spacing}