

\unnsection{Additional Materials \& Methods}%
\label{sec:Additional_Methods}%

\subsection*{Erklärung zur Nutzung von \textit{ChatGPT}, \textit{GitHub Copilot} und sonstiger Software}

Als Grundlage dieses Schreibens nennt der Autor die Eidesstattliche Erklärung
(Affidavit), welche zur Angabe von Hilfsmitteln auffordert, aber nicht explizit
die Nutzung von Hilfsmitteln als Einschränkung der Eigenständigkeit des Autors
betrachtet. Ebenso wird auf die Stellungnahme des Präsidiums der Deutschen
Forschungsgemeinschaft (DFG) zum Einfluss generativer Modelle für die Text- und
Bilderstellung auf die Wissenschaften und das Förderhandeln der DFG (September
2023) verwiesen: Der Einsatz generativer Modelle wie \textit{ChatGPT} soll nicht
ausgeschlossen werden, sondern erfordert transparente und nachvollziehbare
Dokumentation, um wissenschaftliche Integrität zu gewährleisten. Wissenschaftler
müssen offenlegen, ob und wie sie generative Modelle verwendet haben, und
sicherstellen, dass keine Verletzung geistigen Eigentums oder wissenschaftliches
Fehlverhalten vorliegt.

KI-Chatverläufe sind in der digitalen Version dieser Arbeit einsehbar
(\url{https://github.com/markur4/Dissertation}) und dienen als Beweis zur
Einhaltung der hier genannten Prinzipien. Dem ist hinzuzufügen, dass die in den
Chat-Verläufen generierten Texte einen weiteren manuellen Bearbeitungsverlauf
nicht dokumentieren, weswegen auch ignorierte oder auch vollständig vom Autor
umgeschriebene Abschnitte in den Chat-Verläufen enthalten sind.

\noindent Der Autor bemerkte selbstständig folgende Mängel generativer KI-Modelle:
\begin{itemize}
    \item Fehlerhafte Logik
    \item Tendenz, die Meinung des Nutzers zu bestätigen
    \item Mangel an Präzision und Nutzung abstrakter Formulierungen
    \item Halluzinieren von nicht existierenden wissenschaftlichen Quellen
\end{itemize}

\noindent Der Autor entschied sich, auch zur Wahrung wissenschaftlicher Integrität, dass
KI keinen entscheidenden Einfluss auf die Kernaussagen der bearbeiteten
Abschnitte ausüben sollte. Um dies sicherzustellen, wurden generative KI-Modelle
für folgende Zwecke \underline{nicht genutzt}:
\begin{enumerate}
    \item Übernahme wissenschaftlicher Fakten, die über etabliertes Lehrbuchwissen hinausgeht.
    \item Erweiterung des Ideengehalts der Arbeit, die über eine Verfeinerung
          von selbstständig erdachten oder recherchierten Konzepten hinausgeht.
    \item Übernahme KI-generierter Texte ohne nachfolgender Überarbeitung durch den Autor.
    \item Analyse und Interpretation von Daten.
\end{enumerate}

\noindent \textit{ChatGPT} (Versionen 3.5, 4 und 4.0) wurde für folgende Zwecke \underline{benutzt}:
\begin{enumerate}
    \item Ausformulierung kohärenter Textabschnitte basierend auf manuell
          erstellten rohen Textabschnitten und Stichpunkten.
    \item Iterative Verbesserung der Formulierungen: Erstellen einer Erstfassung
          ausgehend von einer Rohfassung, Ausgabe einer durch \textit{ChatGPT} verbesserten
          Version, Abänderung durch den Autor, etc.
    \item Erstellung von zusammenfassenden Abschnitten (z.B.: Summary, Aims,
          Conclusion) mit nachfolgender Überarbeitung.
    \item Prüfung auf akademische Sorgfalt bei fest etablierter Terminologie,
          z.B. um Neudefinitionen bereits bestehender Konzepte zu verhindern.
    \item Übersetzen der Summary von Englisch auf Deutsch
    \item Programmierhilfe zur Formatierung in LaTeX.
    \item Suche nach weiteren Quellen zu bereits erarbeiteten Inhalten, falls nicht halluziniert.
\end{enumerate}

\noindent \textit{GitHub Copilot} (Version 1.198.0) ist als Programmierhilfe
spezialisiert und wurde genutzt, um begonnene Sätze oder Code zu
vervollständigen und Bugs zu beheben.

Aus diesen Grundsätzen ist ersichtlich, dass generative KI-Modelle
ausschließlich als Programmier-, Phrasierungs- und Zusammenfassungshilfe genutzt
wurden, vergleichbar mit als unbedenklich geltenden Tools wie Grammarly. Dem
entsprechend hätte diese Arbeit problemlos ohne \textit{ChatGPT} oder \textit{GitHub Copilot}
angefertigt werden können, was die Eigenständigkeit des Autors sicherstellt.

\noindent Ebenso wurde folgende sonstige Software verwendet:
\begin{itemize}
    \item \textit{Visual Studio Code} (\textit{VS Code}) als Code und LaTeX
          Editor, einschließlich Plugins (z.B. zur Korrektur von Grammatik)
    \item LaTeX (MacTex Distribution)
    \item \textit{Zotero} zur Zitation von Quellen und automatisierten Erstellung des
          Quellenverzeichnisses, einschließlich Plugins (ZotFile, Better BibTex, etc.)
    \item \textit{Affinity Publisher} zur Montage von Abbildungen
\end{itemize}


% \vfill
% \begin{flushleft}
% Würzburg, \\
% \\
% \\
% \\
% \underline{\hspace{6cm}} \\
% Ort, Datum \hspace{5cm} Unterschrift des Autors
% \end{flushleft}

\newpage



\subsection*{Declaration of \textit{ChatGPT}, \textit{GitHub Copilot}, and Other Software Usage}

The basis of this declaration is the Affidavit, which requires the disclosure of
aids but does not explicitly consider the use of aids as a limitation on the
author’s independence. Additionally, the statement by the Presidium of the
German Research Foundation (DFG) on the influence of generative models for text
and image creation on the sciences and DFG funding (September 2023) is
referenced: The use of generative models like \textit{ChatGPT} should not be
excluded but requires transparent and comprehensible documentation to ensure
scientific integrity. Scientists must disclose whether and how they used
generative models and ensure that no intellectual property is violated or
scientific misconduct occurs.

AI-chat logs are accessible in the digital version of this work
(\url{https://github.com/markur4/Dissertation}) and serve as proof of compliance
with the principles outlined here. It should be noted that the texts generated
in the chat logs do not document subsequent manual revision, which is why
ignored or also completely rewritten sections by the author are included in the
chat logs.

\noindent The author independently noticed the following deficiencies of generative AI models:
\begin{itemize}
    \item Logical flaws
    \item Tendency to confirm the user’s opinion
    \item Lack of precision and use of abstract formulations
    \item Hallucination of non-existent scientific sources
\end{itemize}

\noindent The author decided to maintain scientific integrity by preventing AI
from having a decisive influence on the core statements of the processed
sections. To ensure this, generative AI models were \underline{not used} for
the following purposes:
\begin{enumerate}
    \item Adopting scientific facts that transcends established textbook knowledge
    \item Expansion of the idea content of the work that transcends the
          refinement of independently conceived or researched concepts
    \item Adoption of AI-generated texts without subsequent revision by the author.
    \item Analysis and interpretation of data.
\end{enumerate}

\noindent \textit{ChatGPT} (versions 3.5, 4, and 4.0) was \underline{used} for the following purposes:
\begin{enumerate}
    \item Formulation of coherent text sections based on manually created raw text sections and notes
    \item Iterative improvement of formulations: Creating a first draft based on
          a raw draft, generating a version improved by \textit{ChatGPT}, revising by the
          author, etc.
    \item Creation of summarizing sections (e.g., Summary, Aims, Conclusion) with subsequent revision
    \item Checking for academic rigor with well-established terminology, e.g. to
          prevent redefinitions of already existing concepts.
    \item Translating the Summary from English to German
    \item Programming assistance for formatting in LaTeX
    \item Searching for additional sources for already developed content, if not hallucinated
\end{enumerate}

\noindent \textit{GitHub Copilot} (version 1.198.0) is specialized as a programming aid and was
used to complete started sentences or code and to fix bugs.

From these principles, it is evident that generative AI models were used
exclusively as programming, phrasing, and summarizing aids, comparable to tools
such as Grammarly, which are considered unobjectionable. Accordingly, this work
could have been completed without \textit{ChatGPT} or \textit{GitHub Copilot}, ensuring the
author’s independence.

\noindent The following other software was also used:
\begin{itemize}
    \item \textit{Visual Studio Code} (\textit{VS Code}) as code and LaTeX editor, including plugins (e.g., for grammar correction)
    \item LaTeX (MacTex distribution)
    \item \textit{Zotero} for citation of sources and automated creation of the
          bibliography, including plugins (\textit{ZotFile}, \textit{Better BibTex}, etc.)
    \item \textit{Affinity Publisher} for image montage
\end{itemize}



% \vfill
% \begin{flushleft}
% Würzburg, \\
% \\
% \\
% \\
% \underline{\hspace{6cm}} \\
% Place, Date \hspace{5cm} Signature of Author
% \end{flushleft}