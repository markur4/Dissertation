


% ======================================================================
%  ## Define footnotes here


\def\imagefeatures{%
    \emph{Features} are structural elements of an image, such as edges, corners,
    directions, colors. These features are mathematically extractable using
    \emph{filters} \dashed{also referred to as \emph{convolution
            kernels}}, which are functions or algorithms applied to the pixel values of
    an image. For instance, \emph{gabor filters} can extract edges of one
    particular direction, resulting in an image of the same size as the input,
    but showing only edges of one direction. \emph{Feature extraction} is the
    process of applying multiple filters, resulting in a stack of filtered
    images called a feature vector. \cite{szeliskiFeatureDetectionMatching2011,
        guptaDeepLearningImage2019} }

\def\cnn{%
    \emph{Convolutional neural networks} (CNN) are algorithms that use the output of a
    feature extractor\footref{foot:image_features} to feed
    into a neural network. The network then learns to associate these feature
    vectors with a label, such as \emph{cell} or \emph{background}. This is called
    \emph{supervised learning}.
}




\newcommand{\footinteractionscenario}{%
    \emph{Cell Interaction Scenario} (specified in this work): The type of
    cellular interactions and adhesions between cells of the same type
    (homotypic interaction), different types (heterotypic interaction), or between cells
    and the substrate. Complex interaction scenarios can combine all these
    types at the same time. When interaction scenarios emerge from cell
    division, the term \emph{growth conformation} can be used as well (see
    Chapter\,1)%
}


\newcommand{\footcad}{%
    \emph{\acf{CAD}} (specified in this work): The observation and measurement
    of time-dependent changes in cell adhesion and detachment events. \ac{CAD}
    expands traditional \emph{cell adhesion} by a time component and implies
    an intention to predict the timepoint of detachment events. Such focus on
    dynamics is especially relevant for suspension cells that exhibit
    intricate adhesion behaviors. Chapter\,1 also refers to CAD as
    attachment/detachment dynamics. %
}



\newcommand{\footdramatype}{%
    Environmental influences from the fertilization of an egg [...] through to
    sexual maturity are referred to as the primary milieu. The interaction
    between this milieu and the genotype will give rise to the phenotype. The
    phenotypical properties will subsequently be influenced by the
    pre-experiment conditions which are referred to as the secondary milieu.
    As a result, the dramatype is formed. Furthermore the laboratory animal
    will be affected by experimental procedures and treatments known as the
    tertiary milieu. %
    % \citet{zutphenPrinciplesLaboratoryAnimal2001}
}



\newcommand{\footcaddt}{%
    \emph{CAD dramatype} (specified in this work): Specific \ac{CAD} behavior
    caused by proximate environmental factors. A CAD dramatype is characterized
    by the duration cells spend in distinct adhesive states or interaction
    scenarios\footref{foot:interactionscenario}, and the cause of transitions
    between these states and scenarios. Adhesive states include attached,
    migrating, or detached; interaction scenarios include homotypic, heterotypic
    or substrate interactions. CAD dramatypes are associated with molecular
    signatures, such as \ac{CAM} expression patterns or signal transduction
    mediated by proximate environmental factors. %
}

\newcommand{\footadhesiondt}{%
    \emph{Adhesion dramatype} (specified in this work): Short version for
    \ac{CAD} dramatype. Since the term \emph{dramatype} implies dynamic changes,
    \emph{CAD dramatypes} and \emph{adhesion dramatypes} are interchangable.%
}


% ======================================================================
% ## Generate list of footnotes
\addcontentsline{toc}{section}{Notes}%
\theendnotes
% \printfn