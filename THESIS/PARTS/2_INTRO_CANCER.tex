

% From paper 1 (AACR): Multiple myeloma arises from clonal expansion of
% malignant plasma cells in the bone marrow (BM). At diagnosis, myeloma cells
% have disseminated to multiple sites in the skeleton and, in some cases, to
% “virtually any tissue” (1,2). However, the mechanism through which myeloma
% cells initially disseminate remains unclear.

% Dissemination is a multistep process involving invasion, intravasation,
% intravascular arrest, extravasation, and colonization (3). To initiate
% dissemination, myeloma cells overcome adhesion, retention, and dependency on
% the BM microenvironment, which could involve the loss of adhesion factors such
% as CD138 (4,5). BM retention is mediated by multiple factors: First,
% chemokines (CXCL12 and CXCL8) produced by mesenchymal stromal cells (MSCs),
% which attract plasma cells and prime their cytoskeleton and integrins for
% adhesion (6,7). Second, myeloma cells must overcome the anchorage and physical
% boundaries of the extracellular matrix (ECM), consisting of e.g. fibronectin,
% collagens, and proteoglycans such as decorin (8–11). Simultaneously, ECM
% provides signals inducing myeloma cell cycle arrest or progression the cell
% cycle (8,11). ECM is also prone to degradation, which is common in several
% osteotropic cancers, and is the cause of osteolytic bone disease. This is
% driven by a ‘vicious cycle’ that maximizes bone destruction by extracting
% growth factors (EGF and TGF-β) that are stored in calcified tissues (12).
% Third, direct contact with MSCs physically anchors myeloma cells to the BM
% (3,13). Fourth, to disseminate to distant sites, myeloma cells require, at
% least partially, independence from essential growth and survival signals
% provided by MSCs in the form of soluble factors or cell adhesion signaling
% (5,14,15). For example, the VLA4 (Myeloma)–VCAM1 (MSC)-interface activates
% NF-κB in both myeloma and MSCs, inducing IL-6 expression in MSCs. The
% independence from MSCs is then acquired through autocrine survival signaling
% (16,17). In short, anchorage of myeloma cells to MSCs or ECM is a
% ‘double-edged sword’: adhesion counteracts dissemination, but also presents
% signaling cues for growth, survival, and drug resistance (18).

% To address this ambiguity, we developed an in vitro co-culture system modeling
% diverse adhesion modalities to study dissemination, growth, and survival of
% myeloma cells and hMSCs. Co-cultures of hMSCs and the myeloma cell line INA-6
% replicated tight interactions and aggregate growth, akin to "microtumors" in
% Ghobrial’s metastasis concept (19). We characterized the growth conformations
% of hMSCs and INA­6 as homotypic aggregation vs. heterotypic hMSC adherence and
% their effects on myeloma cell survival. We tracked INA­6 detachments from
% aggregates and hMSCs, thereby identifying a potential "disseminated"
% subpopulation lacking strong adhesion. We developed innovative techniques
% (V-well adhesion assay and well plate sandwich centrifugation) to separate
% weakly and strongly adherent subpopulations for the subsequent analysis of
% differential gene expression and cell survival. Notably, our strategy resolves
% the differences in gene expression and growth behavior between cells of one
% cell population in "direct" contact with MSCs. In contrast, previous methods
% differentiated between "direct" and "indirect" cell-cell contact using
% transwell inserts (20). To evaluate whether genes mediating adhesion and
% growth characteristics of INA-6 were associated with patient survival, we
% analyzed publicly available datasets (21,22).






% ======================================================================
% ======================================================================
\unnsubsection{Multiple Myeloma and Other Monoclonal Gammopathies}%
\label{sec:intro_myeloma}%
\emph{Multiple myeloma} (MM) is a hematological malignancy characterized by the
clonal expansion of malignant plasma cells primarily within multiple sites of
the bone marrow (BM). This cancer arises from plasma cells, the
antibody-producing cells of the immune system, which undergo malignant
transformation resulting in uncontrolled growth and disruption of normal bone
marrow function \cite{yangPathogenesisTreatmentMultiple2022}. The prevalence of
multiple myeloma has tripled across both europe and USA from 1980 to 2014 due to
an ageing population \cite{ociasTrendsHematologicalCancer2016,
    turessonRapidlyChangingMyeloma2018}. For 2024, 35780 new MM cases and 12540
deaths are estimated for the USA alone \cite{siegelCancerStatistics20242024}.

To understand the progression of a healthy plasma cell to MM, one can review
other \emph{monoclonal gammopathies}. These are defined by the presence of
monoclonal immunoglobulin in the blood serum which is indicative of abnormal
plasma cell clones overexpressing the same type of dysfunctional antibody.
\cite{kyleMonoclonalGammopathyUndetermined1997,
    fermandMonoclonalGammopathyClinical2018}. When no further disease manifestations
are present, the condition is termed \emph{monoclonal gammopathy of undetermined
    significance} (MGUS), which is the most commonly diagnosed monoclonal gammopathy
\cite{kyleMonoclonalGammopathyUndetermined1997}. MGUS has a 1-\SI{5}{\percent}
annual risk of progression to MM \cite{rajkumarInternationalMyelomaWorking2014}.

To distinguish MM from other monoclonal gammopathies, diagnosis of MM requires
not only identification of a minimum of clonal plasma cells, but also a
\emph{myeloma defining event} which is evidence of malignancy or end-organ
damage, such as hypercalcemia, renal insufficiency, anemia, or bone lesions
\cite{rajkumarInternationalMyelomaWorking2014}. A localized
smaller\footnotequote{Solitary plasmacytoma with \SI{10}{\percent} or more
    clonal plasma cells is regarded as multiple myeloma. [...] If bone marrow has
    less than \SI{10}{\percent} clonal plasma cells, more than one bone lesion is
    required to distinguish [MM] from solitary plasmacytoma with minimal marrow
    involvement.}{rajkumarInternationalMyelomaWorking2014} mass of clonal plasma
cells together with a singular primary bone lesion is diagnosed as
\emph{solitary plasmacytoma} (SP). SP can progress to MM in \SI{32}{\percent} of
cases  (median follow-up of 9.7 years)
\cite{thumallapallySolitaryPlasmacytomaPopulationbased2017,
    gaoSolitaryBonePlasmacytoma2024}. Studies from
\citet{kyleMonoclonalGammopathyUndetermined1997} show that SP cases are rare,
constituting only \SI{2.5}{\percent} of monoclonal gammopathy diagnoses, whereas
MM represent \SI{18}{\percent}. Another rare precursor of MM is
\textit{smouldering} or \textit{asymptomatic MM} (aMM), representing
\SI{3}{\percent} of monoclonal gammopathies
\cite{kyleMonoclonalGammopathyUndetermined1997}. aMM is diagnosed when no
myeloma defining event is detected, although the quantities of clonal plasma
cells or monoclonal protein align with respective criteria for MM diagnosis.
\cite{rajkumarInternationalMyelomaWorking2014}. Recent reports show that if left
untreated, \SI{72}{\percent} of aMM patients progress to MM, whereas early
treatment can lower the progression rate to \SI{11}{\percent} within up to 7.6
years until last follow-up\footnote{For non-high risk aMM patients, treatment
    lowered MM progression rate to \SI{9}{\percent}, compared to \SI{31}{\percent}
    for untreated patients (within up to 6.7 and 7.6 years of follow-up,
    respectively). For high-risk aMM patiens, treatment lowered aMM progression rate
    to \SI{11}{\percent}, compared to \SI{72}{\percent} for untreated patients
    (within up to 5.2 years of follow-up and median time to progression of 2.2
    years, respectively) \cite{abdallahModeProgressionSmoldering2024}.}
\cite{abdallahModeProgressionSmoldering2024,
    mateosmaria-victoriaLenalidomideDexamethasoneHighRisk2013}. MM itself can
progress to advanced stages, such as \emph{extramedullary involvement/disease}
(EMD) which describes colonization of soft tissues outside the bone marrow
\cite{bladeExtramedullaryDiseaseMultiple2022}, but also \emph{plasma cell
    leukemia} (PCL) which is characterized by high levels of circulating plasma
cells \cite{jungUpdatePrimaryPlasma2022}. However, the most common cause of
death is renal failure during the MM stage, caused by excess immunoglobulins or
hypercalcemia due to bone degradation \cite{kunduMultipleMyelomaRenal2022}.

With a 5-year surival rate of \SI{50}{\percent}
\cite{turessonRapidlyChangingMyeloma2018}, MM can be considered incurable and
deadly. MM relapses within the first year in \SI{16}{\percent} of patients,
others face relapse at a later time or only continued response to treatment
\cite{majithiaEarlyRelapseFollowing2016}. Although treatments have improved, the
age-adjusted mortality rate of MM has decreased from 1999 to 2020 by only
\SI{-1.6}{\percent} \cite{doddiDisparitiesMultipleMyeloma2024}.
\citet{engelhardtFunctionalCureLongterm2024} describes the current standard care
for transplant-eligible newly diagnosed MM patients as follows: Induction with a
CD38 antibody, proteasome inhibitor, immunomodulatory drug, and dexamethasone,
potentially followed by bone marrow transplantation and lenalidomide maintenance
\cite{rajkumarMultipleMyelomaCurrent2020}. A major challenge to these treatments
is the continued cycle of remission and relapse, with each relapse generally
being harder to treat \cite{podarRelapsedRefractoryMultiple2021}. Development of
such resistance is well described in the literature, often arising from the
intraclonal genetic heterogeneity within the myeloma cell population and the
protective niche provided by the bone marrow microenvironment
\cite{solimandoDrugResistanceMultiple2022}.



% ======================================================================
% ======================================================================
\unnsubsection{Dissemination of Myeloma Cells}%
\label{sec:intro_myeloma_dissemination}%
As the name suggests, multiple myeloma (MM) involves spreading of clonal plasma
cells in multiple sites within the body, a process that's described with the
term \emph{dissemination}. Although a single large plasmacytoma is still
classified as MM \cite{rajkumarInternationalMyelomaWorking2014}, the presence of
multiple tumor lesions within the bone marrow (BM) is very common. More than one
or 25 such lesions predict poor prognosis for asymptomatic and symptomatic MM
patients, respectively \cite{kastritisPrognosticImportancePresence2014,
    maiMagneticResonanceImagingbased2015a}. Additionally, MM cells can disseminate
to extramedullary sites of virtually any tissue, highlighting MM as a systemic
disease with potential multi-organ impact
\cite{rajkumarMultipleMyelomaCurrent2020,
    bladeExtramedullaryDiseaseMultiple2022}. Hence, dissemination is a major
contributor to MM progression and poor prognosis, enabling MM cells to
colonize new niches that favor survival, quiescent states or are less accessible
for therapy \cite{forsterMolecularImpactTumor2022}.

Dissemination in multiple myeloma (MM) is reminiscent of \emph{metastasis}, a
term typically associated with solid tumors describing the spread of cancer
cells to distant sites. However, it substantially differs from metastasis due to
the hematological or ``liquid'' nature of MM. Long-lived plasma cells originate
from migratory B-cells, negating the need for extensive transformative processes
such as \emph{epithelial to mesenchymal transition}, which is required for
escaping tightly connected solid tissues to enter the bloodstream
\cite{ribattiEpithelialMesenchymalTransitionCancer2020}. Although referred to as
``liquid tumor'', MM cells still accumulate as distinct foci within the bone
marrow, somewhat mirroring the localized growth of solid tumors. This
characteristic has led to MM being proposed as a model for studying solid
``micrometastases'' \cite{ghobrialMyelomaModelProcess2012}, highlighting its
unique blend of liquid and solid tumor properties and providing insights into
the mechanisms of cancer dissemination and colonization of new niches.

The exact mechanisms of MM dissemination are still not entirely understood.
Nevertheless, attempts to structure this process have been made by
\citet{zeissigTumourDisseminationMultiple2020}, describing MM dissemination in
five-steps: \emph{Retention in the BM environment (BME)}, \emph{release from the
BME}, \emph{intravasation}, \emph{extravasation}, and \emph{colonization}.
Critical to these processes is the ability of MM cells to overcome retention and
adhesion within the BME, despite their dependency on the BM for survival
factors. Following release, MM cells undergo \emph{intravasation} into the
bloodstream, where they can circulate before extravasating into new BM sites.
This migration is directed by chemokines and growth factors produced by BM
cells. For instance, CXCL12 and IGF-1 are critical in guiding MM cells back to
the BM, a process called \emph{homing}. In the BM they can colonize and form new
tumor foci. This complex interplay of cell signaling, adhesion, and the
microenvironmental conditions not only dictates the dissemination paths of MM
cells but also influences their survival and proliferation in new niches
\cite{zeissigTumourDisseminationMultiple2020}.


% ======================================================================
% ======================================================================
\unnsubsection{Retention of Myeloma Cells in the Bone Marrow}%
\label{sec:intro_myeloma_retention}%
According to \citet{zeissigTumourDisseminationMultiple2020}, the retention
of MM cells within the BME is mediated by multiple mechanisms. Here, we
categorise retention into \emph{Direct adhesion}, \emph{soluble survival
factors} and \emph{chemotaxis}. Another notable mechanism is the physical
boundary that is bone tissue and extracellular matrix (ECM).

\emph{Direct adhesion} of MM cells to the BM is mediated through ECM
components and cell adhesion to other BM resident cells like osteoblasts,
osteoclasts and mesenchymal stromal cells (BMSCs)
\cite{bouzerdanAdhesionMoleculesMultiple2022}. ECM components include
fibronectin, collagens, and proteoglycans such as decorin
\cite{huDecorinmediatedSuppressionTumorigenesis2021,
    huangHigherDecorinLevels2015,katzAdhesionMoleculesLifelines2010,
    kiblerAdhesiveInteractionsHuman1998}. BM mesenchymal stromal cells (BMSCs) are
vital in this niche, supporting cell adhesion through cell adhesion molecules
(CAMs) but also by secretion of extracellular matrix (ECM) components
\cite{katzAdhesionMoleculesLifelines2010}. Such adhesion acts both as physical
anchorage but also provides signaling cues for growth, survival, and drug
resistance, a classic example being the binding of MM cell integrins, such as
α4β1 (VLA-4) to VCAM-1 on BMSCs, \cite{bouzerdanAdhesionMoleculesMultiple2022}.


\emph{Soluble survival factors} contribute to BM retention, since plasma cells
can not survive outside the bone marrow without them. These signals are secreted
by the bone marrow MSCs and adipocytes
\cite{kiblerAdhesiveInteractionsHuman1998,
garcia-ortizRoleTumorMicroenvironment2021}.


IGF-1 has proven to be the primary survival factors \cite{sprynskiRoleIGF1Major2009}

like IL-6 and IGF-1 are essential to plasma cell
survival and are solely present in the BME, being secreted by BMSCs and other BM
cells \cite{sprynskiRoleIGF1Major2009}

\emph{Chemotaxis} is also crucial for
BM retention. CXCL12 and CXCL8 produced by BMSCs attract MM cells and prime
their cytoskeleton and integrins for adhesion
\cite{aggarwalChemokinesMultipleMyeloma2006,alsayedMechanismsRegulationCXCR42007}.


% ======================================================================
% ======================================================================
\unnsubsection{Release of Myeloma Cells from the Bone Marrow}%
\label{sec:intro_myeloma_release}%
\citet{zeissigTumourDisseminationMultiple2020} describes the release of myeloma
cells from the BME as all steps required for overcoming bone marrow retention.
To the author's knowledge, this step is 
the least understood within the dissemination process. However, \citet{zeissigTumourDisseminationMultiple2020}
outlays many interesting 

 

MM cells must become partially independent from
essential survival factors

. While the specific
microenvironmental stimuli regulating this release are not fully defined,
changes in the expression of adhesion molecules play a role. For example,
circulating MM cells show lower levels of integrin $\alpha4\beta1$
compared to those residing in the BM. Furthermore, treatment with a syndecan-1 blocking antibody
has been shown to rapidly induce the mobilization of MM cells from the BM to
peripheral blood in mouse models, suggesting that alterations in adhesion
molecule expression facilitate MM cell release
\cite{zeissigTumourDisseminationMultiple2020}.



Loss of adhesion
factors such as CD138 could be significant in this context
\cite{akhmetzyanovaDynamicCD138Surface2020,
    brandlJunctionalAdhesionMolecule2022}.

Notably, active degradation of bone tissue removes physical boundaries for MM
cells, allowing them to disseminate to other sites.


This process is driven by a
``vicious cycle'' that maximizes bone destruction by extracting growth factors
(EGF and TGF-β) stored in calcified tissues.

The hypoxic conditions within the BM during MM growth also influence cell
release. Hypoxia induces changes in MM cell adherence to BMSCs and collagen,
which may facilitate their release from the BM niche. Hypoxia-induced factors
like HIF-2α can decrease MM cell responsiveness to stromal-derived CXCL12,
thereby disrupting the CXCR4/CXCL12 retention signal and facilitating MM cell
release \cite{zeissigTumourDisseminationMultiple2020}.







% ======================================================================
% ======================================================================
\unnsubsection{hMSCs: Human Mesenchymal Stromal/Stem Cells}%
\label{sec:intro_hMSCs}%
The previous sections mentioned hMSCs several times, being a critical component
of the bone marrow microenvironment (BME) in the context of multiple myeloma
 \cite{mangoliniBoneMarrowStromal2020}. Before discussing their role in MM
 specifically, it is important to understand what hMSCs are.

Explaining what a mesenchymal stromal cell (MSC) is, can be challenging. MSCs
are derived from multiple different sources, serve a wide array of functions and
are always isolated as a heterogenous group of cells. This makes it particularly
challenging to find a consensus on their exact definition, nomenclature, exact
function and \textit{in vivo} differentiation potential. Therefore, the
following paragraphs provide a brief overview of the biology of MSCs set within
a historical context.

hMSCs first gained popularity as a stem cell. Stem cells lay the foundation of
multicellular organisms. Embryonic stem cells orchestrate the growth and
patterning during embryonic development, while adult stem cells are responsible
for regeneration during adulthood. The classical definition of a stem cell is
that of a relatively undifferentiated cell that divides asymmetrically,
producing another stem cell and a differentiated
cell~\cite{cooperCellMolecularApproach2000, shenghuiMechanismsStemCell2009}.
Because of their significance in biology and regenerative medicine, stem cells
have become a prominent subject in modern research. Especially human mesenchymal
stromal cells (hMSCs) have proven to be a promising candidate in this
context~\cite{ullahHumanMesenchymalStem2015}.

\emph{Mesenchyme} first appears in embryonic development during gastrulation.
There, cells that are committed to a mesodermal fate, lose their cell junctions
and exit the epithelial layer in order to migrate freely. This process is called
epithelial-mesenchymal transition
\cite{tamFormationMesodermalTissues1987,nowotschinCellularDynamicsEarly2010}.
Hence, the term mesenchyme describes non-epithelial embryonic tissue
differentiating into mesodermal lineages such as bone, muscles and blood.
Interestingly, it was shown nearly twenty years earlier that cells within adult
bone marrow seemed to have mesenchymal properties as they were able to
differentiate into bone tissue
\cite{friedensteinOsteogenesisTransplantsBone1966,
    friedensteinOsteogenicPrecursorCells1971, biancoMesenchymalStemCells2014}. This
was the origin of the \emph{``mesengenic process''}-hypothesis: This concept
states that mesenchymal stem cells serve as progenitors for multiple mesodermal
tissues (bone, cartilage, muscle, marrow stroma, tendon, fat, dermis and
connective tissue) during both adulthood and embryonic
development~\cite{caplanMesenchymalStemCells1991,caplanMesengenicProcess1994}.
The mesenchymal nature of these cells (termed bone marrow stromal cells: BMSCs)
was confirmed later when they were shown to differentiate into adipocytic (fat)
and chondrocytic (cartilage)
lineages~\cite{pittengerMultilineagePotentialAdult1999}. Since then, the term
\emph{``mesenchymal stem cell''} (MSC) has grown popular as an adult multipotent
precursor to a couple of mesodermal tissues. hMSCs derived from bone marrow
(BMSCs) were shown to differentiate into osteocytes, chondrocytes, adipocytes
and cardiomyocytes \cite{gronthosSTRO1FractionAdult1994,
    muruganandanAdipocyteDifferentiationBone2009, xuMesenchymalStemCells2004}. Most
impressively, these cells also exhibited ectodermal and endodermal
differentiation potential, as they produced neuronal cells, pancreatic cells and
hepatocytes \cite{barzilayLentiviralDeliveryLMX1a2009,
    wilkinsHumanBoneMarrowderived2009, gabrInsulinproducingCellsAdult2013,
    stockHumanBoneMarrow2014}.

It was later established that cultures with MSC-like
properties can be isolated from ``virtually every post-natal organs and
tissues'', and not just bone
marrow~\cite{dasilvameirellesMesenchymalStemCells2006}. However, depending on
which tissue they originated from, hMSCs can differ greatly in their
transcription profile and \textit{in vivo} differentiation potential
\cite{jansenFunctionalDifferencesMesenchymal2010,sacchettiNoIdenticalMesenchymal2016}.

Since hMSCs are a heterogenous group of cells, they were defined by their
\textit{in vitro} characteristics. A minimal set of criteria are the following
\cite{dominiciMinimalCriteriaDefining2006}: First, hMSCs must be plastic
adherent. Second, they must express or lack a set of specific surface antigens
(positive for CD73, CD90, CD105; negative for CD45, CD34, CD11b, CD19). Third,
hMSCs must differentiate to osteoblasts, adipocytes and chondroblasts \textit{in
    vitro}. Together, hMSCs exhibit diverse differentiation potentials and can be
isolated from multiple sources of the body. 

% TODO: Put this into a context of multiple myeloma
% This offers great opportunity for
% regenerative medicine, if the particular hMSC-subtype is properly characterized.



% ======================================================================
% ======================================================================
\unnsubsection{hMSC­--Myeloma Interactions}
\label{sec:intro_myeloma_hMSC}
As mentioned in previous sections MSCs are key drivers of MM pathology through
mediating retention and survival of MM cells in the bone marrow
\cite{mangoliniBoneMarrowStromal2020}. Here, we outlay details of such
interactions and present further mechanisms how interactions between MM and hMSC
contribute to MM progression.

bone marrow microenvironment



it plays a



Bone is a two-phase system in which the mineral phase provides the stiffness and
the collagen fibers provide the ductility and ability to absorb
energy~\cite{viguet-carrinRoleCollagenBone2006}. On a molecular level, bone
tissue is composed of extracellular matrix (ECM) proteins that are calcified by
hydroxyapatite crystals. This ECM consists mostly of collagen type I, but also
components with major regulatory activity, such as fibronectin and proteoglycans
that are essential for healthy bone
physiology~\cite{alcorta-sevillanoDecipheringRelevanceBone2020}. Bone tissue is
actively remodeled by bone-forming osteoblasts and bone-degrading osteoclasts.
Osteoblasts are derived from mesenchymal stromal cells (MSCs) that reside in the
bone marrow~\cite{friedensteinOsteogenesisTransplantsBone1966,
    pittengerMultilineagePotentialAdult1999}. MSCs also give rise to adipocytes
(ACs) to form Bone Marrow Adipose Tissue (BMAT), which can account for up to
70\% of bone marrow volume~\cite{fazeliMarrowFatBone2013}.

MM indirectly degrades bone tissue by stimulating osteoclasts and inhibiting
osteoblast differentiation, which leads to MM-related bone disease
(MBD)~\cite{glaveyProteomicCharacterizationHuman2017}. MBD is present in 80\% of
patients at diagnosis and is characterized by osteolytic lesions, osteopenia and
pathological fractures~\cite{terposPathogenesisBoneDisease2018}.


from \cite{forsterMolecularImpactTumor2022}:
% """Even though the overall survival and response rate of MM patients have
% significantly improved due to the introduction of novel therapeutic agents, MM
% is still an incurable disease. Different studies have shown that the
% pathophysiology of MM is dynamically supported by strong and dynamic
% interactions with the surrounding microenvironment. Direct and indirect
% signaling pathways between malignant plasma cells and the TME can regulate
% plasma cell adhesion, cellular motility and the generation of new blood vessels.
% Additional factors such as hypoxia or nutrient-deprivation can further support
% the invasiveness of single plasma cell clones that finally enter the blood
% circulation. Cellular and non-cellular contents of the blood can improve the
% survival of disseminated plasma cells and facilitate the engraftment at distant
% body sites. Aside, acquisition of secondary mutations selects clones that are no
% longer dependent on the extrinsic support of the BM. A better understanding of
% the complex interactions between the TME and MM cells might potentially lead to
% new therapeutic approaches and result in an improved progression-free survival
% with reduced risks for disease relapse or extramedullary dissemination. For this
% reason, the TME is a promising therapeutic target, especially for aggressive,
% recurrent and/or advanced disease stages. In order to specify the relevance of
% those therapeutic approaches, future studies must be carried out in a larger and
% multicenter setting."""

% ======================================================================
% ======================================================================
% \unnsubsection{Myeloma Bone Disease}
% \label{sec:intro_myeloma_bone}





