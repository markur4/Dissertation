

% From paper 1 (AACR): Multiple myeloma arises from clonal expansion of
% malignant plasma cells in the bone marrow (BM). At diagnosis, myeloma cells
% have disseminated to multiple sites in the skeleton and, in some cases, to
% “virtually any tissue” (1,2). However, the mechanism through which myeloma
% cells initially disseminate remains unclear.

% Dissemination is a multistep process involving invasion, intravasation,
% intravascular arrest, extravasation, and colonization (3). To initiate
% dissemination, myeloma cells overcome adhesion, retention, and dependency on
% the BM microenvironment, which could involve the loss of adhesion factors such
% as CD138 (4,5). BM retention is mediated by multiple factors: First,
% chemokines (CXCL12 and CXCL8) produced by mesenchymal stromal cells (MSCs),
% which attract plasma cells and prime their cytoskeleton and integrins for
% adhesion (6,7). Second, myeloma cells must overcome the anchorage and physical
% boundaries of the extracellular matrix (ECM), consisting of e.g. fibronectin,
% collagens, and proteoglycans such as decorin (8–11). Simultaneously, ECM
% provides signals inducing myeloma cell cycle arrest or progression the cell
% cycle (8,11). ECM is also prone to degradation, which is common in several
% osteotropic cancers, and is the cause of osteolytic bone disease. This is
% driven by a ‘vicious cycle’ that maximizes bone destruction by extracting
% growth factors (EGF and TGF-β) that are stored in calcified tissues (12).
% Third, direct contact with MSCs physically anchors myeloma cells to the BM
% (3,13). Fourth, to disseminate to distant sites, myeloma cells require, at
% least partially, independence from essential growth and survival signals
% provided by MSCs in the form of soluble factors or cell adhesion signaling
% (5,14,15). For example, the VLA4 (Myeloma)–VCAM1 (MSC)-interface activates
% NF-κB in both myeloma and MSCs, inducing IL-6 expression in MSCs. The
% independence from MSCs is then acquired through autocrine survival signaling
% (16,17). In short, anchorage of myeloma cells to MSCs or ECM is a
% ‘double-edged sword’: adhesion counteracts dissemination, but also presents
% signaling cues for growth, survival, and drug resistance (18).

% To address this ambiguity, we developed an in vitro co-culture system modeling
% diverse adhesion modalities to study dissemination, growth, and survival of
% myeloma cells and hMSCs. Co-cultures of hMSCs and the myeloma cell line INA-6
% replicated tight interactions and aggregate growth, akin to "microtumors" in
% Ghobrial’s metastasis concept (19). We characterized the growth conformations
% of hMSCs and INA­6 as homotypic aggregation vs. heterotypic hMSC adherence and
% their effects on myeloma cell survival. We tracked INA­6 detachments from
% aggregates and hMSCs, thereby identifying a potential "disseminated"
% subpopulation lacking strong adhesion. We developed innovative techniques
% (V-well adhesion assay and well plate sandwich centrifugation) to separate
% weakly and strongly adherent subpopulations for the subsequent analysis of
% differential gene expression and cell survival. Notably, our strategy resolves
% the differences in gene expression and growth behavior between cells of one
% cell population in "direct" contact with MSCs. In contrast, previous methods
% differentiated between "direct" and "indirect" cell-cell contact using
% transwell inserts (20). To evaluate whether genes mediating adhesion and
% growth characteristics of INA-6 were associated with patient survival, we
% analyzed publicly available datasets (21,22).




% ======================================================================
% ======================================================================
\unnsubsection{Human Mesenchymal Stem/Stromal Cells}%
\label{sec:intro_hMSCs}%
Explaining what a mesenchymal stromal cell (MSC) is, can be challenging. MSCs
are derived from multiple different sources, serve a wide array of functions and
are always isolated as a heterogenous group of cells. This makes it particularly
challenging to find a consensus on their exact definition, nomenclature, exact
function and \textit{in vivo} differentiation potential. Therefore, the
following paragraphs provide a brief overview of the biology of MSCs set within
a historical context.

hMSCs first gained popularity as a stem cell. Stem cells lay the foundation of
multicellular organisms. Embryonic stem cells orchestrate the growth and
patterning during embryonic development, while adult stem cells are responsible
for regeneration during adulthood. The classical definition of a stem cell is
that of a relatively undifferentiated cell that divides asymmetrically,
producing another stem cell and a differentiated
cell~\cite{cooperCellMolecularApproach2000, shenghuiMechanismsStemCell2009}.
Because of their significance in biology and regenerative medicine, stem cells
have become a prominent subject in modern research. Especially human mesenchymal
stromal cells (hMSCs) have proven to be a promising candidate in this
context~\cite{ullahHumanMesenchymalStem2015}.

\emph{Mesenchyme} first appears in embryonic development during gastrulation.
There, cells that are committed to a mesodermal fate, lose their cell junctions
and exit the epithelial layer in order to migrate freely. This process is called
epithelial-mesenchymal transition
\cite{tamFormationMesodermalTissues1987,nowotschinCellularDynamicsEarly2010}.
Hence, the term mesenchyme describes non-epithelial embryonic tissue
differentiating into mesodermal lineages such as bone, muscles and blood.
Interestingly, it was shown nearly twenty years earlier that cells within adult
bone marrow seemed to have mesenchymal properties as they were able to
differentiate into bone tissue
\cite{friedensteinOsteogenesisTransplantsBone1966,
    friedensteinOsteogenicPrecursorCells1971, biancoMesenchymalStemCells2014}. This
was the origin of the \emph{``mesengenic process''}-hypothesis: This concept
states that mesenchymal stem cells serve as progenitors for multiple mesodermal
tissues (bone, cartilage, muscle, marrow stroma, tendon, fat, dermis and
connective tissue) during both adulthood and embryonic
development~\cite{caplanMesenchymalStemCells1991,caplanMesengenicProcess1994}.
The mesenchymal nature of these cells (termed bone marrow stromal cells: BMSCs)
was confirmed later when they were shown to differentiate into adipocytic (fat)
and chondrocytic (cartilage)
lineages~\cite{pittengerMultilineagePotentialAdult1999}. Since then, the term
\emph{``mesenchymal stem cell''} (MSC) has grown popular as an adult multipotent
precursor to a couple of mesodermal tissues. hMSCs derived from bone marrow
(BMSCs) were shown to differentiate into osteocytes, chondrocytes, adipocytes
and cardiomyocytes \cite{gronthosSTRO1FractionAdult1994,
    muruganandanAdipocyteDifferentiationBone2009, xuMesenchymalStemCells2004}. Most
impressively, these cells also exhibited ectodermal and endodermal
differentiation potential, as they produced neuronal cells, pancreatic cells and
hepatocytes \cite{barzilayLentiviralDeliveryLMX1a2009,
    wilkinsHumanBoneMarrowderived2009, gabrInsulinproducingCellsAdult2013,
    stockHumanBoneMarrow2014}.

It was later established that cultures with MSC-like
properties can be isolated from ``virtually every post-natal organs and
tissues'', and not just bone
marrow~\cite{dasilvameirellesMesenchymalStemCells2006}. However, depending on
which tissue they originated from, hMSCs can differ greatly in their
transcription profile and \textit{in vivo} differentiation potential
\cite{jansenFunctionalDifferencesMesenchymal2010,sacchettiNoIdenticalMesenchymal2016}.

Since hMSCs are a heterogenous group of cells, they were defined by their
\textit{in vitro} characteristics. A minimal set of criteria are the following
\cite{dominiciMinimalCriteriaDefining2006}: First, hMSCs must be plastic
adherent. Second, they must express or lack a set of specific surface antigens
(positive for CD73, CD90, CD105; negative for CD45, CD34, CD11b, CD19). Third,
hMSCs must differentiate to osteoblasts, adipocytes and chondroblasts \textit{in
    vitro}. Together, hMSCs exhibit diverse differentiation potentials and can be
isolated from multiple sources of the body. This offers great opportunity for
regenerative medicine, if the particular hMSC-subtype is properly characterized.





% ======================================================================
% ======================================================================
\unnsubsection{Multiple Myeloma and Other Monoclonal Gammopathies}%
\label{sec:intro_myeloma}%
\emph{Multiple myeloma} (MM) is a hematological malignancy characterized by the
clonal expansion of malignant plasma cells primarily within multiple sites of
the bone marrow (BM). This cancer arises from plasma cells, the
antibody-producing cells of the immune system, which undergo malignant
transformation resulting in uncontrolled growth and disruption of normal bone
marrow function \cite{yangPathogenesisTreatmentMultiple2022}. The prevalence of
multiple myeloma has tripled across both europe and USA from 1980 to 2014 due to
an ageing population \cite{ociasTrendsHematologicalCancer2016,
    turessonRapidlyChangingMyeloma2018}. For 2024, 35780 new MM cases and 12540
deaths are estimated for the USA alone \cite{siegelCancerStatistics20242024}.

To understand the progression of a healthy plasma cell to MM, one can review
other \emph{monoclonal gammopathies}. These are defined by the presence of
monoclonal immunoglobulin in the blood serum which is indicative of abnormal
plasma cell clones overexpressing the same type of dysfunctional antibody.
\cite{kyleMonoclonalGammopathyUndetermined1997,
    fermandMonoclonalGammopathyClinical2018}. When no further disease manifestations
are present, the condition is termed \emph{monoclonal gammopathy of undetermined
    significance} (MGUS), which is the most commonly diagnosed monoclonal gammopathy
\cite{kyleMonoclonalGammopathyUndetermined1997}. MGUS has a 1-\SI{5}{\percent}
annual risk of progression to MM \cite{rajkumarInternationalMyelomaWorking2014}.

To distinguish MM from other monoclonal gammopathies, diagnosis of MM requires
not only identification of a minimum of clonal plasma cells, but also a
\emph{myeloma defining event} which is evidence of malignancy or end-organ
damage, such as hypercalcemia, renal insufficiency, anemia, or bone lesions
\cite{rajkumarInternationalMyelomaWorking2014}. A localized
smaller\footnotequote{Solitary plasmacytoma with \SI{10}{\percent} or more
    clonal plasma cells is regarded as multiple myeloma. [...] If bone marrow has
    less than \SI{10}{\percent} clonal plasma cells, more than one bone lesion is
    required to distinguish [MM] from solitary plasmacytoma with minimal marrow
    involvement.}{rajkumarInternationalMyelomaWorking2014} mass of clonal plasma
cells together with a singular primary bone lesion is diagnosed as
\emph{solitary plasmacytoma} (SP). SP can progress to MM in \SI{32}{\percent} of
cases  (median follow-up of 9.7 years)
\cite{thumallapallySolitaryPlasmacytomaPopulationbased2017,
    gaoSolitaryBonePlasmacytoma2024}. Studies from
\citet{kyleMonoclonalGammopathyUndetermined1997} show that SP cases are rare,
constituting only \SI{2.5}{\percent} of monoclonal gammopathy diagnoses, whereas
MM represent \SI{18}{\percent}. Another rare precursor of MM is
\textit{smouldering} or \textit{asymptomatic MM} (aMM), representing
\SI{3}{\percent} of monoclonal gammopathies
\cite{kyleMonoclonalGammopathyUndetermined1997}. aMM is diagnosed when no
myeloma defining event is detected, although the quantities of clonal plasma
cells or monoclonal protein align with respective criteria for MM diagnosis.
\cite{rajkumarInternationalMyelomaWorking2014}. Recent reports show that if left
untreated, \SI{72}{\percent} of aMM patients progress to MM, whereas early
treatment can lower the progression rate to \SI{11}{\percent} within up to 7.6
years until last follow-up\footnote{For non-high risk aMM patients, treatment
    lowered MM progression rate to \SI{9}{\percent}, compared to \SI{31}{\percent}
    for untreated patients (within up to 6.7 and 7.6 years of follow-up,
    respectively). For high-risk aMM patiens, treatment lowered aMM progression rate
    to \SI{11}{\percent}, compared to \SI{72}{\percent} for untreated patients
    (within up to 5.2 years of follow-up and median time to progression of 2.2
    years, respectively) \cite{abdallahModeProgressionSmoldering2024}.}
\cite{abdallahModeProgressionSmoldering2024,
    mateosmaria-victoriaLenalidomideDexamethasoneHighRisk2013}. MM itself can
progress to advanced stages, such as \emph{extramedullary involvement/disease}
(EMD) which describes colonization of soft tissues outside the bone marrow
\cite{bladeExtramedullaryDiseaseMultiple2022}, but also \emph{plasma cell
    leukemia} (PCL) which is characterized by high levels of circulating plasma
cells \cite{jungUpdatePrimaryPlasma2022}. However, the most common cause of
death is renal failure during the MM stage, caused by excess immunoglobulins or
hypercalcemia due to bone degradation \cite{kunduMultipleMyelomaRenal2022}.

With a 5-year surival rate of \SI{50}{\percent}
\cite{turessonRapidlyChangingMyeloma2018}, MM can be considered incurable and
deadly. MM relapses within the first year in \SI{16}{\percent} of patients,
others face relapse at a later time or only continued response to treatment
\cite{majithiaEarlyRelapseFollowing2016}. Although treatments have improved, the
age-adjusted mortality rate of MM has decreased from 1999 to 2020 by only
\SI{-1.6}{\percent} \cite{doddiDisparitiesMultipleMyeloma2024}.
\citet{engelhardtFunctionalCureLongterm2024} describes the current standard care
for transplant-eligible newly diagnosed MM patients as follows: Induction with a
CD38 antibody, proteasome inhibitor, immunomodulatory drug, and dexamethasone,
potentially followed by bone marrow transplantation and lenalidomide maintenance
\cite{rajkumarMultipleMyelomaCurrent2020}. A major challenge to these treatments
is the continued cycle of remission and relapse, with each relapse generally
being harder to treat \cite{podarRelapsedRefractoryMultiple2021}. Development of
such resistance is well described in the literature, often arising from the
intraclonal genetic heterogeneity within the myeloma cell population and the
protective niche provided by the bone marrow microenvironment
\cite{solimandoDrugResistanceMultiple2022}.



% ======================================================================
% ======================================================================
\unnsubsection{Dissemination of Myeloma Cells}%
\label{sec:intro_myeloma_dissemination}%
As the name suggests, multiple myeloma (MM) involves spreading of clonal plasma
cells in multiple sites within the body, a process that's described with the
term \emph{dissemination}. Although a single large plasmacytoma is still
classified as MM \cite{rajkumarInternationalMyelomaWorking2014}, the presence of
multiple tumor lesions within the bone marrow (BM) is very common. More than one
or 25 such lesions predict poor prognosis for asymptomatic and symptomatic MM
patients, respectively \cite{kastritisPrognosticImportancePresence2014,
    maiMagneticResonanceImagingbased2015a}. Additionally, MM cells can disseminate
to extramedullary sites of virtually any tissue, highlighting MM as a systemic
disease with potential multi-organ impact
\cite{rajkumarMultipleMyelomaCurrent2020,
    bladeExtramedullaryDiseaseMultiple2022}. Hence, dissemination is a major
contributor to MM progression and poor prognosis.

Dissemination in multiple myeloma (MM) is reminiscent of \emph{metastasis}, a
term typically associated with solid tumors describing the spread of cancer
cells to distant sites. However, it substantially differs from metastasis due to
the hematological or ``liquid'' nature of MM. Long-lived plasma cells originate
from migratory B-cells, negating the need for extensive transformative processes
such as \emph{epithelial to mesenchymal transition}, which is required for
escaping tightly connected solid tissues to enter the bloodstream
\cite{ribattiEpithelialMesenchymalTransitionCancer2020}. Although referred to as
``liquid tumor'', MM cells still accumulate as distinct foci within the bone
marrow, somewhat mirroring the localized growth of solid tumors. This
characteristic has led to MM being proposed as a model for studying solid
"micrometastases" \cite{ghobrialMyelomaModelProcess2012}, highlighting its
unique blend of liquid and solid tumor properties and providing insights into
the mechanisms of cancer dissemination and colonization of new niches.





The exact mechanisms of MM dissemination are still not entirely understood.
Nevertheless, attempts to structure the process have been made by
\citet{zeissigTumourDisseminationMultiple2020}, describing MM dissemination in
five-steps: retention in the BM environment (BME), release from the BME,
intravasation, extravasation, and colonization
\cite{zeissigTumourDisseminationMultiple2020}. Critical to these processes is
the ability of MM cells to overcome retention and adhesion within the BME,
despite their dependency on the BM for survival factors. 

BME plays a role in dissemination \cite{forsterMolecularImpactTumor2022}:
"Preventing the infiltration and spread of myeloma cells to sites where they are
capable to turn into quiescent states or colonize niches that are less
accessible for standard therapies might play a significant role in overcoming
EMM."

Loss of adhesion
factors such as CD138 could be significant in this context
\cite{akhmetzyanovaDynamicCD138Surface2020,
brandlJunctionalAdhesionMolecule2022}. 

The BME plays a crucial role in the
regulation of dissemination, where strategies to prevent the infiltration and
spread of MM cells into less accessible niches for standard therapies could be
pivotal in overcoming extramedullary myeloma (EMM)
\cite{forsterMolecularImpactTumor2022}.

% dissemination is still widely unclear
% - but \citet{zeissigTumourDisseminationMultiple2020} has reviewed it into 5 steps:
% - retention in the BME, release from the BME
% - intravasation, intravascular arrest, extravasation,
% colonization
% - A critical step is to in overcome retention and adhesion to the bone marrow microenvironment (BME), but also the dependency on the BM
% microenvironment to survival factors.
% - loss of adhesion factors such as CD138
% - BME plays a role in dissemination \cite{forsterMolecularImpactTumor2022}:
% "Preventing the infiltration and spread of myeloma cells to sites where they are
% capable to turn into quiescent states or colonize niches that are less
% accessible for standard therapies might play a significant role in overcoming
% EMM."

According to Zeissig et al., the retention of MM cells within the BM stromal
niche is mediated by various mechanisms. BM mesenchymal stromal cells (BMSCs)
are vital in this niche, supporting MM cell growth through direct adhesion and
secreted extracellular matrix (ECM) components. The binding of MM cell
integrins, such as α4β1 to vascular cell-adhesion molecule 1 (VCAM-1) and
fibronectin on BMSCs, is particularly crucial. This binding decreases MM cell
response to chemotactic factors \textit{in vitro}
\cite{shibayamaLamininFibronectinPromote1995a}, thus playing a significant role
in their retention within the BM \cite{zeissigTumourDisseminationMultiple2020}.

Release from the BM niche involves MM cells overcoming strong adhesive
interactions, which act as retention signals. While the specific
microenvironmental stimuli regulating this release are not fully defined,
changes in the expression of adhesion molecules play a role. For example,
circulating MM cells show lower levels of integrin α4β1 
compared to those residing in the BM. Furthermore, treatment with a syndecan-1 blocking antibody
has been shown to rapidly induce the mobilization of MM cells from the BM to
peripheral blood in mouse models, suggesting that alterations in adhesion
molecule expression facilitate MM cell release
\cite{zeissigTumourDisseminationMultiple2020}.

The hypoxic conditions within the BM during MM growth also influence cell
release. Hypoxia induces changes in MM cell adherence to BMSCs and collagen,
which may facilitate their release from the BM niche. Hypoxia-induced factors
like HIF-2α can decrease MM cell responsiveness to stromal-derived CXCL12,
thereby disrupting the CXCR4/CXCL12 retention signal and facilitating MM cell
release \cite{zeissigTumourDisseminationMultiple2020}.

Following release, MM cells undergo intravasation into the bloodstream, where
they can circulate before extravasating into new BM sites. This migration is
directed by chemokines and growth factors produced by BM cells. For instance,
CXCL12 and IGF-1 are critical in guiding MM cells back to the BM, where they can
colonize and form new tumor foci. This complex interplay of cell signaling,
adhesion, and the microenvironmental conditions not only dictates the
dissemination paths of MM cells but also influences their survival and
proliferation in new niches \cite{zeissigTumourDisseminationMultiple2020}


paragraphs from \cite{zeissigTumourDisseminationMultiple2020}:

% 2. The Process of Dissemination in MM 
% """ Similar to the process of solid tumour
% metastasis, the dissemination of MM PCs is associated with a loss of their
% adherence to cells of the BM microenvironment that favours MM PC retention,
% allowing the cells to exit the niche. The tumour cells must then undergo
% trans-endothelial migration, mediated by chemoattractants and adhesive
% interactions, and intravasate (move from the BM into the bloodstream) where they
% are carried to a secondary site. In order to home to new BM sites, the tumour
% cells must arrest on the BM endothelium and extravasate (move out of the blood
% and into tissues), following chemotactic factors produced by BM cells. The final
% stage of MM PC dissemination is associated with the colonisation of new BM sites
% which supports tumour cell growth. The following sections will describe the
% molecular mechanisms that are involved in MM PC dissemination, focusing on both
% the intrinsic properties of the MM PCs that support their dissemination, as well
% as the extrinsic stimuli that drive this process.

% 2.1. Retention
% within the BM Stromal Niche 
% BM mesenchymal stromal cells (BMSCs) are a critical
% component in the BM niche that supports the growth of MM PCs. Adhesion of MM PCs
% to BMSCs, and extracellular matrix (ECM) components that are secreted by these
% cells, is a crucial mechanism by which MM PCs are maintained within the niche
% (Figure 1). Binding of the integrin α4β1 (also known as very late antigen 4,
% VLA-4), expressed by MM PCs, to vascular cell–adhesion molecule 1 (VCAM-1) and
% fibronectin, expressed by BMSCs, is one of the key factors mediating the strong
% adhesion of MM PCs to BMSCs [22]. Notably, integrin α4β1-mediate binding to
% fibronectin decreases the response of MM cell lines to chemotactic factors in
% vitro, supporting its role in MM PC retention [23]. Additionally, the integrin
% αLβ2 complex (also known as lymphocyte function-associated antigen 1, LFA-1) on
% MM PCs mediates binding to intracellular adhesion molecule 1 (ICAM-1) on BMSCs
% [24]. Other adhesive factors expressed by MM PCs include CD44 variants CD44v6
% and CD44v9, which mediate adhesion to BMSCs [25], and syndecan-1 (also known as
% CD138), which is involved in adhesion to type 1 collagen [26]. Cancers 2020, 12,
% x FOR PEER REVIEW 3 of 20 known as lymphocyte function-associated antigen 1,
% LFA-1) on MM PCs mediates binding to intracellular adhesion molecule 1 (ICAM-1)
% on BMSCs [24]. Other adhesive factors expressed by MM PCs include CD44 variants
% CD44v6 and CD44v9, which mediate adhesion to BMSCs [25], and syndecan-1 (also
% known as CD138), which is involved i

% 2.2. Release
% from the BM Niche 
% In order to be released from the BM and enter into the
% circulation, MM PCs must overcome the aforementioned adhesive interactions that
% act as a strong BM retention signal (Figure 1). While the microenvironmental
% stimuli that regulate the release from the BM niche are unclear, decreased
% expression of key factors involved in adhesion in the stromal niche may play a
% role. Studies analysing the expression of cell surface adhesion factors in PB
% PCs compared with BM PCs from MM patients have shown that circulating MM PCs
% express lower levels of integrin α4β1 compared with BM-resident MM PCs [41,42].
% In addition, studies show that there are decreased levels of the activated form
% of integrin β1 in MM PCs in the PB compared with the BM in MM patients [43],
% suggesting that downregulation of its active form may in part facilitate release
% from the BM. Syndecan-1 expression has also been shown to be decreased in PB MM
% PCs compared with their BM counterparts [41] and MM PC syndecan-1 expression is
% associated with decreased dissemination in an in vivo model of MM [44]. Notably,
% treatment with a syndecan-1 blocking antibody rapidly induced the mobilisation
% of MM PCs from the BM to the PB in a syngeneic mouse model of MM [44].
% Additionally, MM cell line expression of the enzyme heparanase-1, which is
% responsible for cleaving proteoglycans including syndecan-1 from the cell
% surface, significantly increased the spontaneous dissemination of MM cells in
% vivo, suggesting that shedding of syndecan-1 may promote dissemination [45].
% Finally, PCs from PCL patients have been reported to have decreased CD40
% expression compared with MGUS PCs, supporting a potential role for loss of CD40
% expression in release of MM PCs from the BM niche [46].

% 2.3. Microenvironmental Control of Release from the BM Niche 
% Release from the BM niche may also be
% facilitated by signals from the micro-environment (Figure 1). The BM becomes
% increasingly hypoxic during MM tumour growth, with highly hypoxic regions
% arising within the tumour mass due to rapid tumour cell growth and abnormal
% blood vessel formation (reviewed by Martin et al. [47]). Moreover, the role of
% hypoxia in tumour progression, dissemination and angiogenesis has been
% demonstrated in mouse models of MM [48–50]. In particular, Azab and colleagues
% demonstrated in a mouse model of MM that MM PC hypoxia, as assessed by
% pimonidazole staining, strongly correlated with both BM tumour burden and
% numbers of circulating MM PCs [48]. Furthermore, transcriptomic analyses of
% paired BM and PB tumour samples from MM patients have revealed a hypoxia-related
% gene expression signature in MM PCs in the circulation, suggesting that hypoxia
% may selectively induce the mobilisation of MM PCs from the BM [51]. In addition,
% culturing MM cell lines in hypoxic conditions significantly reduced their
% adherence to BMSCs or to collagen in vitro [48,52], suggesting a mechanism
% whereby hypoxia may induce the release of MM PCs from the BM niche. Hypoxia has
% been shown to increase MM PC expression of the transcription factors Snail and
% Twist1 that are master regulators of the epithelial to mesenchymal transition
% (EMT), suggesting that, like in epithelial cancers, an EMT-like process may also
% be occurring in MM to allow release from the niche [48,53]. Studies by our group
% have demonstrated that the induction of hypoxia inducible factor HIF-2α in MM
% cell lines can lead to decreased response to stromal cell-derived CXCL12, which
% may also facilitate release from the niche. HIF-2α increases CXCL12 expression
% by MM cell lines [54] which, in turn, reduces CXCR4 cell-surface levels on MM
% cells [35], forming a feedback loop that leads to a decrease in CXCR4 signalling
% and a desensitisation to exogenous CXCL12 [35]. In addition, our group has
% demonstrated that hypoxic activation of HIF-2α leads to upregulation of the C-C
% chemokine receptor 1 (CCR1) in MM PCs, which may also contribute to their
% preferential mobilisation [35]. Notably, treatment of CCR1-positive MM cell
% lines with the CCR1 ligand C-C chemokine ligand 3 (CCL3; also known as
% macrophage inflammatory protein 1 alpha [MIP-1 α]) abrogates migration towards
% CXCL12 in vitro, suggesting that CCL3/CCR1 signalling can desensitise cells to
% exogenous CXCL12 [35]. Furthermore, either CCR1 knockout or treatment with a
% small molecule CCR1 inhibitor strongly inhibits the spontaneous dissemination of
% human MM cell lines in an intratibial model of myeloma [55]. Taken together,
% these studies suggest that hypoxia may mediate a disruption to the CXCR4/CXCL12
% retention signal to allow MM PC release from the BM."""





Further processes are intravasation, circulation, extravasation and colonization

% 2.6. Establishment and Colonisation of MM PCs in a New BM Niche Following
% extravasation, MM PCs further respond to locally produced chemokines and growth
% factors to direct their movement to their ultimate location in the BM (Figure
% 2). The migration of MM PCs towards specific niches in the BM may be driven by a
% range of chemoattractant molecules that are produced by BM cells, including
% BMSCs, osteoclasts and macrophages, which play an important role in the MM BM
% niche (Table 1). For both normal PCs and MM PCs, CXCL12 represents the
% predominant signal that is thought to drive homing from the PB into the BM
% [28,30,76,77] and subsequently also leads to MM PC retention in the BM
% (described in the retention within the niche section, above). CXCL12 expression
% is higher in the BM than in the PB [28,35], consistent with its abundant
% expression by BMSCs [29–32], establishing a gradient that enables homing to the
% BM. MM PCs have a strong migratory response towards CXCL12 [27,76], with CXCL12
% inducing cytoskeletal remodelling that enables MM PC migration [28,35,70].
% Blockade of CXCL12-CXCR4 binding slows homing of human MM cell lines from the PB
% and accumulation in the BM in vivo [28,78]. BMSCs also produce hepatocyte growth
% factor (HGF) [79–81], which further enhances the chemotactic effect of CXCL12 on
% MM PCs [82]. Furthermore, BMSC production of exosomes containing chemokines
% including CCL2, CCL3, and CXCL12 may also play a role in the migration of MM PCs
% towards BMSCs [83]. BMSC-derived exosomes increase the migration of the mouse
% myeloma cell line 5T33MM in vitro, which can be reversed with inhibitors of
% CXCR4 or CCR2 [83]. Insulin-like growth factor 1 (IGF-1) has also been shown to
% be a key promigratory (chemokinetic) and chemoattractant factor for murine [84]
% and human [82] MM cell lines in vitro which enhances the response to CXCL12 in
% vitro [82]. In the mouse 5T MM models, IGF-1 is a critical promoter of homing of
% MM PCs to the BM [84,85]. Notably, ablation of macrophages, an abundant source
% of IGF-1 in the BM, is sufficient to inhibit the homing of 5TGM1 cells to the BM
% [85]. Monocytes and macrophages also highly express the CCR1 ligand CCL3
% [86,87], which is a promigratory factor for primary MM PCs and MM cell lines in
% vitro [76,87,88]. Osteoclasts are also a predominant source of a number of
% factors that induce MM PC migration in in vitro assays, including CCL3 [87,89],
% the CCR2 ligands CCL2, CCL7 (also known as monocyte chemotactic protein-3
% [MCP-3]) and CCL8 (also known as monocyte chemotactic protein-2 [MCP-2]) [87,90]
% and the CXCR3 ligand CXCL10 (also known as interferon gamma-induced protein 10
% [IP-10]) [76,91,92]. It is yet to be established whether these
% osteoclast-derived factors play a role in MM PC homing in an in vivo setting.
% Other cells of the BM microenvironment, including mesenchymal lineage cells such
% as adipocytes, may also play a role in the BM homing process. For example,
% adipocyte conditioned media induces the migration of MM cell lines in vitro
% [93–95] at least in part through expression of the chemoattractants CXCL12 and
% CCL2 [94]. Table 1. Receptors and secreted factors involved in the dissemination
% process in multiple myeloma. Receptor on MM PCs Pro-Disseminatory Factor
% Predominant Source in The Myeloma BM Suggested Role in Dissemination CXCR4
% [33–35] CXCL12 Hypoxic MM PCs [35,54] Mobilisation of MM PCs from the BM [35,96]
% Endothelial cells [29,33,67,68] Arrest of circulating MM PCs in the BM
% vasculature [33,67–69] BMSCs [29–32] Migration and homing to BM niches
% [27,28,76,77] CCR1 [34,35,55] CCL3 MM PCs [35,89,97,98] Mobilisation of MM PCs
% from the BM [35,55] Osteoclasts, macrophages [86,87,89] Migration and homing to
% BM niches [87–89] CCR2 [33,34,87] CCL2 Endothelial cells [33,58] Migration
% towards BMECs [33,58,59] BMSCs, osteoclasts, adipocytes [32,81,87,90,94]
% Migration and homing to BM niches [33,58] CCL7, CCL8 Osteoclasts [87] Migration
% and homing to BM niches [90] IGF-1R [99,100] IGF-1 Macrophages [85,87,101]
% Migration and homing to BM niches [84,85] Met [100,102] HGF BMSCs [79–81]
% Synergises with CXCL12 to increase MM PC migration [82] Recent animal studies
% from our group [103] and those of others [104,105] have shed light on the fate
% of disseminating clones, and shown that the dissemination process is extremely
% inefficient: of the hundreds of MM cells that may reach the BM following
% intravenous injection, very few ultimately proliferate to form macroscopic
% tumours, with the remaining cells being maintained in a non-proliferative state
% [103–105]. This suggests that the BM microenvironment in which the MM PCs
% ultimately reside may determine whether an individual tumour cell is destined
% for dormancy or proliferation [104,106]. At this time, it remains unclear
% whether the homing of MM PCs to these niches is largely a stochastic process, or
% whether MM PCs are driven by specific microenvironment-derived factors to home
% to defined niches that ultimately determine their fate for dormancy or growth.


% ======================================================================
% ======================================================================
\unnsubsection{Myeloma-hMSC Interactions}
\label{sec:intro_myeloma_hMSC}

bone marrow microenvironment

Since plasma cells can not survive outside the bone marrow, MM cells also
require survival signals for growth and disease progression. These signals are
produced by the bone marrow microenvironment, including ECM, MSCs and
ACs~\cite{kiblerAdhesiveInteractionsHuman1998,
    garcia-ortizRoleTumorMicroenvironment2021}.

it plays a

from \cite{forsterMolecularImpactTumor2022}:
% """Even though the overall survival and response rate of MM patients have
% significantly improved due to the introduction of novel therapeutic agents, MM
% is still an incurable disease. Different studies have shown that the
% pathophysiology of MM is dynamically supported by strong and dynamic
% interactions with the surrounding microenvironment. Direct and indirect
% signaling pathways between malignant plasma cells and the TME can regulate
% plasma cell adhesion, cellular motility and the generation of new blood vessels.
% Additional factors such as hypoxia or nutrient-deprivation can further support
% the invasiveness of single plasma cell clones that finally enter the blood
% circulation. Cellular and non-cellular contents of the blood can improve the
% survival of disseminated plasma cells and facilitate the engraftment at distant
% body sites. Aside, acquisition of secondary mutations selects clones that are no
% longer dependent on the extrinsic support of the BM. A better understanding of
% the complex interactions between the TME and MM cells might potentially lead to
% new therapeutic approaches and result in an improved progression-free survival
% with reduced risks for disease relapse or extramedullary dissemination. For this
% reason, the TME is a promising therapeutic target, especially for aggressive,
% recurrent and/or advanced disease stages. In order to specify the relevance of
% those therapeutic approaches, future studies must be carried out in a larger and
% multicenter setting."""

% ======================================================================
% ======================================================================
\unnsubsection{Myeloma Bone Disease}
\label{sec:intro_myeloma_bone}
Bone is a two-phase system in which the mineral phase provides the stiffness and
the collagen fibers provide the ductility and ability to absorb
energy~\cite{viguet-carrinRoleCollagenBone2006}. On a molecular level, bone
tissue is composed of extracellular matrix (ECM) proteins that are calcified by
hydroxyapatite crystals. This ECM consists mostly of collagen type I, but also
components with major regulatory activity, such as fibronectin and proteoglycans
that are essential for healthy bone
physiology~\cite{alcorta-sevillanoDecipheringRelevanceBone2020}. Bone tissue is
actively remodeled by bone-forming osteoblasts and bone-degrading osteoclasts.
Osteoblasts are derived from mesenchymal stromal cells (MSCs) that reside in the
bone marrow~\cite{friedensteinOsteogenesisTransplantsBone1966,
    pittengerMultilineagePotentialAdult1999}. MSCs also give rise to adipocytes
(ACs) to form Bone Marrow Adipose Tissue (BMAT), which can account for up to
70\% of bone marrow volume~\cite{fazeliMarrowFatBone2013}.

MM indirectly degrades bone tissue by stimulating osteoclasts and inhibiting
osteoblast differentiation, which leads to MM-related bone disease
(MBD)~\cite{glaveyProteomicCharacterizationHuman2017}. MBD is present in 80\% of
patients at diagnosis and is characterized by osteolytic lesions, osteopenia and
pathological fractures~\cite{terposPathogenesisBoneDisease2018}.





