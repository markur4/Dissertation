

% From paper 1 (AACR): Multiple myeloma arises from clonal expansion of
% malignant plasma cells in the bone marrow (BM). At diagnosis, myeloma cells
% have disseminated to multiple sites in the skeleton and, in some cases, to
% “virtually any tissue” (1,2). However, the mechanism through which myeloma
% cells initially disseminate remains unclear.

% Dissemination is a multistep process involving invasion, intravasation,
% intravascular arrest, extravasation, and colonization (3). To initiate
% dissemination, myeloma cells overcome adhesion, retention, and dependency on
% the BM microenvironment, which could involve the loss of adhesion factors such
% as CD138 (4,5). BM retention is mediated by multiple factors: First,
% chemokines (CXCL12 and CXCL8) produced by mesenchymal stromal cells (MSCs),
% which attract plasma cells and prime their cytoskeleton and integrins for
% adhesion (6,7). Second, myeloma cells must overcome the anchorage and physical
% boundaries of the extracellular matrix (ECM), consisting of e.g. fibronectin,
% collagens, and proteoglycans such as decorin (8–11). Simultaneously, ECM
% provides signals inducing myeloma cell cycle arrest or progression the cell
% cycle (8,11). ECM is also prone to degradation, which is common in several
% osteotropic cancers, and is the cause of osteolytic bone disease. This is
% driven by a ‘vicious cycle’ that maximizes bone destruction by extracting
% growth factors (EGF and TGF-β) that are stored in calcified tissues (12).
% Third, direct contact with MSCs physically anchors myeloma cells to the BM
% (3,13). Fourth, to disseminate to distant sites, myeloma cells require, at
% least partially, independence from essential growth and survival signals
% provided by MSCs in the form of soluble factors or cell adhesion signaling
% (5,14,15). For example, the VLA4 (Myeloma)–VCAM1 (MSC)-interface activates
% NF-κB in both myeloma and MSCs, inducing IL-6 expression in MSCs. The
% independence from MSCs is then acquired through autocrine survival signaling
% (16,17). In short, anchorage of myeloma cells to MSCs or ECM is a
% ‘double-edged sword’: adhesion counteracts dissemination, but also presents
% signaling cues for growth, survival, and drug resistance (18).

% To address this ambiguity, we developed an in vitro co-culture system modeling
% diverse adhesion modalities to study dissemination, growth, and survival of
% myeloma cells and hMSCs. Co-cultures of hMSCs and the myeloma cell line INA-6
% replicated tight interactions and aggregate growth, akin to "microtumors" in
% Ghobrial’s metastasis concept (19). We characterized the growth conformations
% of hMSCs and INA­6 as homotypic aggregation vs. heterotypic hMSC adherence and
% their effects on myeloma cell survival. We tracked INA­6 detachments from
% aggregates and hMSCs, thereby identifying a potential "disseminated"
% subpopulation lacking strong adhesion. We developed innovative techniques
% (V-well adhesion assay and well plate sandwich centrifugation) to separate
% weakly and strongly adherent subpopulations for the subsequent analysis of
% differential gene expression and cell survival. Notably, our strategy resolves
% the differences in gene expression and growth behavior between cells of one
% cell population in "direct" contact with MSCs. In contrast, previous methods
% differentiated between "direct" and "indirect" cell-cell contact using
% transwell inserts (20). To evaluate whether genes mediating adhesion and
% growth characteristics of INA-6 were associated with patient survival, we
% analyzed publicly available datasets (21,22).


% ======================================================================
% ======================================================================
\unnsubsection{Human Mesenchymal Stem/Stromal Cells}%
\label{sec:intro_hMSCs}%
Explaining what a mesenchymal stromal cell (MSC) is, is not such an easy task as
one might expect. MSCs are derived from multiple\ MSCs different sources, serve
a wide array of functions and are always isolated as a heterogenous group of
cells. This makes it particularly challenging to find a consensus on their exact
definition, nomenclature, exact function and \textit{in vivo} differentiation
potential. Therefore, the most effective approach to describe hMSCs is to
present their historical context.

hMSCs first gained popularity as a stem cell. Stem cells lay the foundation of
multicellular organisms. Embryonic stem cells orchestrate the growth and
patterning during embryonic development, while adult stem cells are responsible
for regeneration during adulthood. The classical definition of a stem cell is
that of a relatively undifferentiated cell that divides asymmetrically,
producing another stem cell and a differentiated
cell~\cite{cooperCellMolecularApproach2000, shenghuiMechanismsStemCell2009}.
Because of their significance in biology and regenerative medicine, stem cells
have become a prominent subject in modern research. Especially human mesenchymal
stromal cells (hMSCs) have proven to be a promising candidate in this
context~\cite{ullahHumanMesenchymalStem2015}.

Mesenchyme first appears in embryonic development during gastrulation. There,
cells that are committed to a mesodermal fate, lose their cell junctions and
exit the epithelial layer in order to migrate freely. This process is called
epithelial-mesenchymal transition
\cite{tamFormationMesodermalTissues1987,nowotschinCellularDynamicsEarly2010}.
Hence, the term mesenchyme describes non-epithelial embryonic tissue
differentiating into mesodermal lineages such as bone, muscles and blood.
Interestingly, it was shown nearly twenty years earlier that cells within adult
bone marrow seemed to have mesenchymal properties as they were able to
differentiate into bone tissue
\cite{friedensteinOsteogenesisTransplantsBone1966,friedensteinOsteogenicPrecursorCells1971,biancoMesenchymalStemCells2014}.
This was the origin of the ``mesengenic process''-hypothesis: This concept
states that mesenchymal stem cells serve as progenitors for multiple mesodermal
tissues (bone, cartilage, muscle, marrow stroma, tendon, fat, dermis and
connective tissue) during both adulthood and embryonic
development~\cite{caplanMesenchymalStemCells1991,caplanMesengenicProcess1994}.
The mesenchymal nature of these cells (termed bone marrow stromal cells: BMSCs)
was confirmed later when they were shown to differentiate into adipocytic (fat)
and chondrocytic (cartilage)
lineages~\cite{pittengerMultilineagePotentialAdult1999}. Since then, the term
``mesenchymal stem cell'' (MSC) has grown popular as an adult multipotent
precursor to a couple of mesodermal tissues. hMSCs derived from bone marrow
(hMSCs) were shown to differentiate into osteocytes, chondrocytes, adipocytes
and cardiomyocytes
\cite{gronthosSTRO1FractionAdult1994,muruganandanAdipocyteDifferentiationBone2009,xuMesenchymalStemCells2004}
Most impressively, these cells also exhibited
ectodermal and endodermal differentiation potential, as they produced
neuronal cells, pancreatic cells and hepatocytes
\cite{barzilayLentiviralDeliveryLMX1a2009,wilkinsHumanBoneMarrowderived2009,gabrInsulinproducingCellsAdult2013,stockHumanBoneMarrow2014}.

Furthermore, cultures with MSC properties can be established from ``virtually
every post-natal organs and tissues'', and not just bone
marrow~\cite{dasilvameirellesMesenchymalStemCells2006}. However, it has to be
noted that hMSCs can differ greatly in their transcription profile and
\textit{in vivo} differentiation potential depending on which tissue they
originated from
\cite{jansenFunctionalDifferencesMesenchymal2010,sacchettiNoIdenticalMesenchymal2016}.

Since ``hMSCs' are a heterogenous group of cells, they were defined by their
\textit{in vitro} characteristics. A minimal set of criteria are the following
\cite{dominiciMinimalCriteriaDefining2006}: First, hMSCs must be plastic
adherent. Second, they must express or lack a set of specific surface antigens
(positive for CD73, CD90, CD105; negative for CD45, CD34, CD11b, CD19). Third,
hMSCs must differentiate to osteoblasts, adipocytes and chondroblasts \textit{in
    vitro}. Together, hMSCs exhibit diverse differentiation potentials and can be
isolated from multiple sources of the body. This offers great opportunity for
regenerative medicine, if the particular hMSC-subtype is properly characterized.


% ======================================================================
% ======================================================================
\unnsubsection{Multiple Myeloma}
\label{sec:intro_myeloma}
Multiple myeloma arises from clonal expansion of malignant plasma cells in the
bone marrow (BM). At diagnosis, myeloma cells have disseminated to multiple
sites in the skeleton and, in some cases, to virtually any
tissue~\cite{rajkumarMultipleMyelomaCurrent2020,
    bladeExtramedullaryDiseaseMultiple2022}.


% ======================================================================
% ======================================================================
\unnsubsection{Myeloma-hMSC Interactions}
\label{sec:intro_myeloma_hMSC}
Since plasma cells can not survive outside the bone marrow, MM cells also
require survival signals for growth and disease progression. These signals are
produced by the bone marrow microenvironment, including ECM, MSCs and
ACs~\cite{kiblerAdhesiveInteractionsHuman1998,
    garcia-ortizRoleTumorMicroenvironment2021}.


% ======================================================================
% ======================================================================
\unnsubsection{Myeloma Bone Disease}
\label{sec:intro_myeloma_bone}
Bone is a two-phase system in which the mineral phase provides the stiffness and
the collagen fibers provide the ductility and ability to absorb
energy~\cite{viguet-carrinRoleCollagenBone2006}. On a molecular level, bone
tissue is composed of extracellular matrix (ECM) proteins that are calcified by
hydroxyapatite crystals. This ECM consists mostly of collagen type I, but also
components with major regulatory activity, such as fibronectin and proteoglycans
that are essential for healthy bone
physiology~\cite{alcorta-sevillanoDecipheringRelevanceBone2020}. Bone tissue is
actively remodeled by bone-forming osteoblasts and bone-degrading osteoclasts.
Osteoblasts are derived from mesenchymal stromal cells (MSCs) that reside in the
bone marrow~\cite{friedensteinOsteogenesisTransplantsBone1966,
    pittengerMultilineagePotentialAdult1999}. MSCs also give rise to adipocytes
(ACs) to form Bone Marrow Adipose Tissue (BMAT), which can account for up to
70\% of bone marrow volume~\cite{fazeliMarrowFatBone2013}.

MM indirectly degrades bone tissue by stimulating osteoclasts and inhibiting
osteoblast differentiation, which leads to MM-related bone disease
(MBD)~\cite{glaveyProteomicCharacterizationHuman2017}. MBD is present in 80\% of
patients at diagnosis and is characterized by osteolytic lesions, osteopenia and
pathological fractures~\cite{terposPathogenesisBoneDisease2018}.


% ======================================================================
% ======================================================================
\unnsubsection{Dissemination of Myeloma Cells}
\label{sec:intro_myeloma_dissemination}
dissemination is still widely unclear
- multistep process
- invasion, intravasation, intravascular arrest, extravasation,
colonization
- overcome adhesion, retention, and dependency on the BM
microenvironment
- loss of adhesion factors such as CD138

% > New Page to separate biological part from the technical part 