

% ======================================================================
% == Chapter 2 ========================================================
% ======================================================================
% Zusätzliche Einleitung und Diskussion: Herr Kuric soll bitte eine ergänzende
% Einleitung und Diskussion in dem Kapitel zu seiner JOSS-Publikation
% hinzufügen, die folgende Informationen enthalten:

% Darstellung von Umfeld, Aufgabenstellung und Signifikanz für die
% biomedizinische Anwendung
% (warum braucht es die Biomedizin?)

% Darstellung der Anforderungen an die Programmierung: Integration von
% Informationen, die die spezifischen Anforderungen verdeutlichen,
% welche die Programmierung der Software erforderlich gemacht haben.
% (warum habe ich es gebraucht?) 

% Darstellung der Nutzbarkeit für Naturwissenschaftler: Klare und auch
% für Nicht-Informatiker verständliche Darstellung, welche konkreten
% Anwendungsmöglichkeiten die Software für Naturwissenschaftler bietet.
% (warum ist es nützlich?)

% Mit einer angemessenen Einleitung und Diskussion würde zum einen den
% Guidelines der GSLS entsprochen, die für alle zugrunde gelegt werden.
% Zum anderen würde es auch dem großen Anteil von Nicht-Informatikern in
% der GSLS erlauben, den Hintergrund und die Signifikanz und
% Verwendungsmöglichkeiten des entwickelten Codes besser zu verstehen.
% Die GSLS hat sich mit der Aufnahme von informatischen Projekten
% interdisziplinär geöffnet. Gleichzeitig erwarten wir damit aber auch
% von den Doktorierenden den Willen, die Arbeit in einer
% interdisziplinären Form in der Thesis zu präsentieren.

\unnsection{Chapter 2: Semi-Automating Data Analysis with \texttt{plotastic}}
\label{sec:C2}
\vspace{-\baselineskip} % > Remove space made by empty lines

% ## Reset reference counters of figs, tabs, so each chapter starts at 1
\setcounter{figure}{0}
\setcounter{table}{0}


% ======================================================================
% == Abstract
% ======================================================================
% \addcontentsline{toc}{subsection}{Abstract} % > Add to Table of Contents

% % > In the paper, it's called summary, but I rename it into abstract
% \renewcommand{\abstractname}{Abstract} % > Rename what is displayed
% \begin{abstract}
%       \texttt{plotastic} addresses the challenges of transitioning from
%       exploratory data analysis to hypothesis testing in Python's data science
%       ecosystem. Bridging the gap between \texttt{seaborn} and
%       \texttt{pingouin}, this library offers a unified environment for plotting
%       and statistical analysis. It simplifies the workflow with user-friendly
%       syntax and seamless integration with familiar \texttt{seaborn} parameters
%       (y, x, hue, row, col). Inspired by \texttt{seaborn}'s consistency,
%       \texttt{plotastic} utilizes a \texttt{DataAnalysis} object to
%       intelligently pass parameters to \texttt{pingouin} statistical functions.
%       Hence, statistics and plotting are performed on the same set of
%       parameters, so that the strength of \texttt{seaborn} in visualizing
%       multidimensional data is extended onto statistical analysis. In essence,
%       \texttt{plotastic} translates \texttt{seaborn} parameters into statistical
%       terms, configures statistical protocols based on intuitive plotting syntax
%       and returns a \texttt{matplotlib} figure with known customization options
%       and more. This approach streamlines data analysis, allowing researchers to
%       focus on correct statistical testing and less about specific syntax and
%       implementations.
% \end{abstract}

\customabstract{c2:abstract}{Abstract}{
      \texttt{plotastic} addresses the challenges of transitioning from
      exploratory data analysis to hypothesis testing in Python's data science
      ecosystem. Bridging the gap between \texttt{seaborn} and
      \texttt{pingouin}, this library offers a unified environment for plotting
      and statistical analysis. It simplifies the workflow with user-friendly
      syntax and seamless integration with familiar \texttt{seaborn} parameters
      (y, x, hue, row, col). Inspired by \texttt{seaborn}'s consistency,
      \texttt{plotastic} utilizes a \texttt{DataAnalysis} object to
      intelligently pass parameters to \texttt{pingouin} statistical functions.
      Hence, statistics and plotting are performed on the same set of
      parameters, so that the strength of \texttt{seaborn} in visualizing
      multidimensional data is extended onto statistical analysis. In essence,
      \texttt{plotastic} translates \texttt{seaborn} parameters into statistical
      terms, configures statistical protocols based on intuitive plotting syntax
      and returns a \texttt{matplotlib} figure with known customization options
      and more. This approach streamlines data analysis, allowing researchers to
      focus on correct statistical testing and less about specific syntax and
      implementations.
}

\newpage



% ======================================================================
% == Introduction
% ======================================================================
\unnsubsection{Introduction}\label{C2:introduction}
%%%
The reproducibility crisis in research highlights a significant challenge in
contemporary biosciences, where a substantial portion of studies faces
reproducibility
issues~\cite{baker500ScientistsLift2016,begleyReproducibilityScienceImproving2015}.
One critical yet often overlooked aspect contributing to this crisis is data
management. The literature most often refers to \textit{big data} as the main
challenge~\cite{gomez-cabreroDataIntegrationEra2014}. However, these challenges
are also present in smaller datasets, which the author refers to as
\textit{semi-big data}. This term describes datasets that -- while not extensive
enough to necessitate advanced computational tools typically reserved for
\textit{big data} -- are sufficiently large to render manual analysis very
time-intensive.~Semi-big data is often generated by methods like automated
microscopy or multiplex qPCR, which produce volumes of data that are manageable
on a surface level, but pose substantial barriers for in-depth, manual
reproducibility~\cite{bustinReproducibilityBiomedicalResearch2014}. This is
further complicated by the complexity inherent in multidimensional datasets
\cite{krzywinskiMultidimensionalData2013}. Modern biosciences describe processes (e.g. cell-adhesion) that are highly dependent on multiple experimental parameters, like time and concentration \cite{reblTimedependentMetabolicActivity2010,
      mckayCellcellAdhesionMolecules1997}. Without a clearly
documented data analysis protocol and standardized data formats,
such analyses become nontransparent and not reproducible~\cite{bustinReproducibilityBiomedicalResearch2014}.

Although the awareness for multidimensional data is increasing
\cite{dunnExploringVisualizingMultidimensional2017}, intuitive tools for the
analysis of such data are still lacking.


% For example, the qPCR experiment from Chapter 1, Fig. 4 involves the analysis of
% 19 genes across in three subpopulations, including eleven biological and three
% technical replicates, resulting in a total of 1881 data points that are all
% assigned to a complex set of experimental variables. 



The evolving standards in data analysis advocate for the standardization of
analytical pipelines, rationalization of sample sizes, and enhanced
infrastructure for data storage, addressing some of these
challenges~\cite{goodmanWhatDoesResearch2016,wilkinsonFAIRGuidingPrinciples2016}.
However, these advancements can place undue pressure on researchers,
particularly those with limited training in statistics, underscoring the need
for intuitive, user-friendly analytical
tools~\cite{gosselinInsufficientTransparencyStatistical2021,
      armstrongWhenUseBonferroni2014, gomez-lopezPrecisionMedicineNeeds2019,
      leekStatisticsValuesAre2015, incertiYouStillUsing2019a}

In this context, \texttt{plotastic} emerges as a tool designed to democratize access to
sophisticated statistical analysis, offering a user-centric interface that
caters to researchers across varying levels of statistical proficiency. By
integrating robust statistical methodologies within an accessible framework,
\texttt{plotastic} aims to contribute to enhancing the reproducibility and
integrity of research in the biosciences~\cite{gomez-cabreroDataIntegrationEra2014}.



initially, the need to develop \texttt{plotastic} arose during this project. The
first is to address the author's need for a tool that could handle the complex,
multidimensional data generated by e.g. qPCR experiments. These experiments
typically involve the analysis of multiple genes across several time points and
biological replicates, resulting in datasets that are challenging to analyze
manually. The author's experience with traditional statistical software, such
as Prism, revealed that these tools required extensive manual input, making
them unsuitable for the efficient analysis of complex, multidimensional data. -
The second was to increase speed. THis is required for developing methods

Also, Adhesion experiments are highly time-dependent, adding a required factor to every experiment

Since \texttt{plotastic} optimizes the analysis of \textit{semi-big data}, we
introduce the term \textit{semi-automation} to distinguish itself from the fully
automated pipelines used for \textit{big data}. Semi-automation is
defined as the following aspects:
\begin{enumerate}
      \item \textbf{Semi-big input:} The input size is oriented towards
            \textit{semi-big data}, which is characterized as being manageable by
            manual analysis, yet highly time inefficient, and probably impossible
            to re-analyse by someone else than the researcher.
      \item \textbf{Standardized input} The input follows a standardized format (e.g. long-format)
      \item \textbf{Minimize user configuration:} User configuration is strictly
            minimized. The user is never asked to pass the same parameters twice. This
            reduces the risk of human error and time spent on configuration.
      \item \textbf{Default configuration provides acceptable results:} If the
            user does not provide any manual configuration, the pipeline should
            provide acceptable results. Options should be provided to allow a
            level of flexibility to adapt the pipeline to the user's needs.
      \item \textbf{Small Reviewable Processing Steps:} The analysis steps are
            structured into small processes that can be combined to form a
            complete analysis pipeline. That way, each step can act as a stage for
            quality control to improve error detection and troubleshooting. For a
            statistical analysis, that means the processing steps are separated
            into 3 steps, those being assumption testing, factor analysis and
            post-hoc testing.
      \item \textbf{Isolated Steps:} Processing steps should work independently
            from another, in the best case only depending on the raw data input.
            If a processing step depends on the output from other steps, the
            software should tell the user what exact steps it expects.
      \item \textbf{Human readable outputs:} Every processing step may provide
            an output that is not necessarily standardized, but is required to be
            human readable to ensure reviewability.

\end{enumerate}

% NOT found, don't care
% Marino, S., et al. (2018). Big Data and Models of Culture: Ongoing Challenges and Possible Solutions. Big Data & Society, 5(2).

% NOT FOUND, required
% Lynch, C., et al. (2019). Big data: How do we measure up? Current Opinion in Systems Biology, 15, 77-84.

% Wrong citation
% Ioannidis, J. P., Fanelli, D., Dunne, D. D., & Goodman, S. N. (2014). Meta-research: Evaluation and Improvement of Research Methods and Practices. PLoS Biol, 12(10), e1002016.
% Peng, R. D. (2011). Reproducible research in computational science. Science, 334(6060), 1226-1227.





%%%%%%%%%%%%%%%%%%%%%%%%%%%%%%%%%%%%


Challenges:
- Reproducibility crisis?
- Data is exploding
- Demands for rigorous statistical analysis are increasing
- Biologists are not trained in statistics


The demands are rising: \cite{moreno-indiasStatisticalMachineLearning2021}
% Nevertheless, although many statistics and machine learning approaches and tools
% have been developed, new techniques are needed to deal with emerging
% applications and the vast heterogeneity of microbiome data. We review and
% discuss emerging applications of statistical and machine learning techniques in
% human microbiome studies and introduce the COST Action CA18131 “ML4Microbiome”
% that brings together microbiome researchers and machine learning experts to
% address current challenges such as standardization of analysis pipelines for
% reproducibility of data analysis results, benchmarking, improvement, or
% development of existing and new tools and ontologies.
% sample size, open source will be critical



As laid out in the introduction, one can doubt if a PhD student without coding
skillsis at its max efficiency.

Why does Biomedicine need plotastic?:
- Thorough analysis has become a standard, with assumption testing, omnibus
tests and post-hoc analyses for every experiment.
- But data is increasing
- Example of my data?
- The number of dedicated statisticians is limited
- The know-how of statistics in biology is limited, for example, Some authors
ignored the problem of multiple testing while others used the method
uncritically with no rationale or discussion \cite{pernegerWhatWrongBonferroni1998,armstrongWhenUseBonferroni2014}


Why did I need plotastic?

Why do biologists need plotastic?
- Assays output more data in shorter time, e.g. multiplex qPCR
- example: 20 genes, 3 timepoints, 11 biological replicates, (all 3
technical replicates already averaged)
- 20 * 3 * 11 = 660 data points

this is multidimensional data:  660 data points spread across two dimensions: time
and gene

-in manual analysis e.g. in Excel, the user has to manually select the
data, copy it, paste it into a new sheet, and then perform the
statistical test. In Prism, the user has to select the data, click on
the statistical test, and then select the data again. This is not only
time-consuming, but also prone to

- Re-Analysis: The user has to repeat the process for every gene and
timepoint. This is not only time-consuming, but also prone to errors.

shortly Describe Main Packages in more detail:
- seaborn: It multidimensional data
- pingouin: It's a statistical package




% == Paper as pdf ======================================================
% > You could import .pdf here, but chapter based theses should apply the 
% > manuscripts into the formatting of the thesis
% \addpdf[.93]
% {Software Article: Journal of Open Source Software}
% {PUBLICATIONS/§-kuricPlotasticBridgingPlotting2024.pdf}




% ======================================================================
% == Statement of Need
% ======================================================================
\unnsubsection{Statement of Need}
\ Python's data science ecosystem provides powerful tools for both visualization
and statistical testing. However, the transition from exploratory data analysis
to hypothesis testing can be cumbersome, requiring users to switch between
libraries and adapt to different syntaxes.~\texttt{seaborn} has become a popular
choice for plotting in Python, offering an intuitive interface. Its statistical
functionality focuses on descriptive plots and bootstrapped confidence
intervals~\cite{waskomSeabornStatisticalData2021}. The library \texttt{pingouin}
offers an extensive set of statistical tests, but it lacks integration with
common plotting capabilities~\cite{vallatPingouinStatisticsPython2018}.
\texttt{statannotations} integrates statistical testing with plot annotations,
but uses a complex interface and is limited to pairwise
comparisons~\cite{charlierTrevismdStatannotationsV02022}.

\texttt{plotastic} addresses this gap by offering a unified environment for
plotting and statistical analysis. With an emphasis on user-friendly
syntax and integration of familiar \texttt{seaborn} parameters, it simplifies
the process for users already comfortable with \texttt{seaborn}. The library
ensures a smooth workflow, from data import to hypothesis testing and
visualization.



\newpage



% ======================================================================
% == Example
% ======================================================================
\unnsubsection{Example}
\
The following code demonstrates how \texttt{plotastic} analyzes the
example dataset ``fmri'', similar to \citet{waskomSeabornStatisticalData2021}
(\autoref{fig:examplefmri}).

\begin{lstlisting}[
    language=Python, 
    style=pythonstyle,
    xrightmargin=0.03\textwidth,
    ]
### IMPORT PLOTASTIC
import plotastic as plst

# IMPORT EXAMPLE DATA
DF, _dims = plst.load_dataset("fmri", verbose = False)

# EXPLICITLY DEFINE DIMENSIONS TO FACET BY
dims = dict(
      y = "signal",    # y-axis, dependent variable
      x = "timepoint", # x-axis, independent variable (within-subject factor)
      hue = "event",   # color,  independent variable (within-subject factor)
      col = "region"   # axes,   grouping variable
)
# INITIALIZE DATAANALYSIS OBJECT
DA = plst.DataAnalysis(
      data=DF,           # Dataframe, long format
      dims=dims,         # Dictionary with y, x, hue, col, row 
      subject="subject", # Datapoints are paired by subject (optional)
      verbose=False,     # Print out info about the Data (optional)
)
# STATISTICAL TESTS
DA.check_normality()   # Check Normality
DA.check_sphericity()  # Check Sphericity
DA.omnibus_rm_anova()  # Perform RM-ANOVA
DA.test_pairwise()     # Perform Posthoc Analysis

# PLOTTING
(DA
.plot_box_strip()     # Pre-built plotting function initializes plot
.annotate_pairwise(   # Annotate results from DA.test_pairwise()
      include="__HUE" # Use only significant pairs across each hue
      ) 
)
\end{lstlisting}



\includeimage[0.9]{
      FIGS/C2_fmri2.png
}\captionof{figure}{
      Example figure of \texttt{plotastic} (version 0.1). Image style was set
      by \texttt{plt.style.use(`ggplot')}
}\label{fig:examplefmri}



\begin{table}[H]
      \small
      \centering
      \caption{Results from \texttt{DA.check\_sphericity()}. \texttt{plotastic}
            assesses sphericity after grouping the data by all grouping dimensions
            (hue, row, col). For example, \texttt{DA.check\_sphericity()} grouped the
            `fmri' dataset by ``region'' (col) and ``event'' (hue), performing four
            subsequent sphericity tests for four datasets.}
      \begin{tabular}{llcccccc}
            \hline
            `region', `event'  & spher & W        & chi2   & dof & pval & group count & n per group \\
            \hline
            `frontal', `cue'   & True  & 3.26e+20 & -462.7 & 44  & 1    & 10          & [14]        \\
            `frontal', `stim'  & True  & 2.45e+17 & -392.2 & 44  & 1    & 10          & [14]        \\
            `parietal', `cue'  & True  & 1.20e+20 & -452.9 & 44  & 1    & 10          & [14]        \\
            `parietal', `stim' & True  & 2.44e+13 & -301.9 & 44  & 1    & 10          & [14]        \\
            \hline
      \end{tabular}
      \label{tab:sphericity}
\end{table}



\begin{table}[H]
      \small
      \centering
      \caption{Results of \texttt{DA.omnibus\_rm\_anova()}. \texttt{plotastic}
            performs one two-factor RM-ANOVA per axis (grouping the data by row and
            col dimensions) using x and hue as the within-factors. For this example,
            \texttt{DA.omnibus\_rm\_anova()} grouped the `fmri' dataset by ``region''
            (col), performing two subsequent two-factor RM-ANOVAs. Within-factors are
            ``timepoint'' (x) and ``event'' (hue). For conciseness, GG-Correction and
            effect sizes are not shown.}
      \begin{tabular}{llccccccc}
            \hline
            `region'   & Source            & SS    & ddof1 & ddof2 & MS    & F      & p-unc    & stars \\
            \hline
            `parietal' & timepoint         & 1.583 & 9     & 117   & 0.175 & 26.20  & 3.40e-24 & ****  \\
            `parietal' & event             & 0.770 & 1     & 13    & 0.770 & 85.31  & 4.48e-07 & ****  \\
            `parietal' & timepoint * event & 0.623 & 9     & 117   & 0.069 & 29.54  & 3.26e-26 & ****  \\
            `frontal'  & timepoint         & 0.686 & 9     & 117   & 0.076 & 15.98  & 8.28e-17 & ****  \\
            `frontal'  & event             & 0.240 & 1     & 13    & 0.240 & 23.44  & 3.21e-4  & ***   \\
            `frontal'  & timepoint * event & 0.242 & 9     & 117   & 0.026 & 13.031 & 3.23e-14 & ****  \\
            \hline
      \end{tabular}
      \label{tab:RMANOVA}
      \vspace{\vfull} % > Add vertical space since a caption is next
\end{table}


% ======================================================================
% == Overview
% ======================================================================
\unnsubsection{Overview}
\
The functionality of \texttt{plotastic} revolves around a seamless integration
of statistical analysis and plotting, leveraging the capabilities of
\texttt{pingouin}, \texttt{seaborn}, \texttt{matplotlib} and
\texttt{statannotations} \cite{vallatPingouinStatisticsPython2018,
      waskomSeabornStatisticalData2021, hunterMatplotlib2DGraphics2007,
      charlierTrevismdStatannotationsV02022}. It utilizes long-format \texttt{pandas}
\texttt{DataFrames} as its primary input, aligning with the conventions of
\texttt{seaborn} and ensuring compatibility with existing data
structures~\cite{wickhamTidyData2014a, reback2020pandas,
      mckinneyDataStructuresStatistical2010}.

\texttt{plotastic} was inspired by \texttt{seaborn} using the same set of intuitive
and consistent parameters (y, x, hue, row, col) found in each of its
plotting functions \cite{waskomSeabornStatisticalData2021}. These parameters
intuitively delineate the data dimensions plotted, yielding `facetted'
subplots, each presenting y against x. This allows for rapid and
insightful exploration of multidimensional relationships.~\texttt{plotastic}
extends this principle to statistical analysis by storing these
\texttt{seaborn} parameters (referred to as dimensions) in a \texttt{DataAnalysis}
object and intelligently passing them to statistical functions of the
\texttt{pingouin} library. This approach is based on the impression that most
decisions during statistical analysis can be derived from how the user
decides to arrange the data in a plot. This approach also prevents code
repetition and streamlines statistical analysis. For example, the
subject keyword is specified only once during \texttt{DataAnalysis}
initialisation, and \texttt{plotastic} selects the appropriate paired or
unpaired version of the test. Using \texttt{pingouin} alone requires the user
to manually pick the correct test and to repeatedly specify the subject
keyword in each testing function.

In essence, \texttt{plotastic} translates plotting parameters into their
statistical counterparts. This translation minimizes user input and also
ensures a coherent and logical connection between plotting and
statistical analysis. The goal is to allow the user to focus on choosing
the correct statistical test (e.g. parametric vs. non-parametric) and
worry less about specific implementations.

At its core, \texttt{plotastic} employs iterators to systematically group data
based on various dimensions, aligning the analysis with the distinct
requirements of tests and plots. Normality testing is performed on each
individual sample, which is achieved by splitting the data by all
grouping dimensions and also the x-axis (hue, row, col, x). Sphericity
and homoscedasticity testing is performed on a complete sampleset listed
on the x-axis, which is achieved by splitting the data by all grouping
dimensions (hue, row, col)  (\autoref{tab:sphericity}). For omnibus and
posthoc analyses, data is grouped by the row and col dimensions in
parallel to the \texttt{matplotlib} axes, before performing one two-factor
analysis per axis using x and hue as the within/between-factors.
(\autoref{tab:RMANOVA}).

\texttt{DataAnalysis} visualizes data through predefined plotting functions
designed for drawing multi-layered plots. A notable emphasis within
\texttt{plotastic} is placed on showcasing individual datapoints alongside
aggregated means or medians. In detail, each plotting function
initializes a \texttt{matplotlib} figure and axes using \texttt{plt.subplots()} while
returning a \texttt{DataAnalysis} object for method chaining. Axes are
populated by \texttt{seaborn} plotting functions (e.g., \texttt{sns.boxplot()}),
leveraging automated aggregation and error bar displays. Keyword
arguments are passed to these \texttt{seaborn} functions, ensuring the same
degree of customization. Users can further customize plots
by chaining \texttt{DataAnalysis} methods or by applying common \texttt{matplotlib} code
to override \texttt{plotastic} settings. Figures are exported using
\texttt{plt.savefig()}.

\texttt{plotastic} also focuses on annotating statistical information within
plots, seamlessly incorporating p-values from pairwise comparisons using
\texttt{statannotations} \cite{charlierTrevismdStatannotationsV02022}. This
integration simplifies the interface and enables options for pair
selection in multidimensional plots, enhancing both user experience and
interpretability.

For statistics, \texttt{plotastic} integrates with the \texttt{pingouin} library
to support classical assumption and hypothesis testing, covering
parametric/non-parametric and paired/non-paired variants. Assumptions such as
normality, homoscedasticity, and sphericity are tested. Omnibus tests include
two-factor RM-ANOVA, ANOVA, Friedman, and Kruskal-Wallis. Posthoc tests are
implemented through \texttt{pingouin.pairwise\_tests()}, offering (paired)
t-tests, Wilcoxon, and Mann-Whitney-U.

To sum up, \texttt{plotastic} stands as a unified and user-friendly solution
catering to the needs of researchers and data scientists, seamlessly
integrating statistical analysis with the power of plotting in Python.
It streamlines the workflow, translates \texttt{seaborn} parameters into
statistical terms, and supports extensive customization options for both
analysis and visualization.

\newpage




% ======================================================================
% == Sub-Discussion 
% ======================================================================
\unnsubsection{Discussion}\label{C2:discussion}
\
Is plotastic tested? Coverage? Does it cover every feature? What is not covered

% The Architecture of plotastic is shown in \refappendixfig{fig:classdiagr}.

The full code of an example analysis is shown in \autoref{apdx:example_analysis}.





Is plotastic USABLE for biologists?
- Yes but use is limited by minimal knowledge of Python
- However, that is subject to change as Python is becoming more popular
in biology and AI assisted coding decreased the barrier to entry
significantly. Tools like github copilot are able to generate code, fix
bugs and suggest improvements. This is a game changer for biologists
that are not familiar with programming.
- Furthermore, installing and using plotastic for biologists is overestimated. These
steps re needed:
- Install anaconda from the internet
- Open the terminal
- Type \texttt{pip install plotastic}
- Check Rea

- lack of GUI? Jupyter Notebooks include buttons to save a figure, and can be
extended with features like sorting displayed tables (?) are compatible with
packages with GUIs, like. However, GUIs do not promise an easier use, in fact,
given the text-based approach of codes, this makes code very compatible with
Large Language Models, offering a powerful debugging platform. In fact, tools like vscode offer GitHub copilot for free for academics,
integrating with the interface and displaying suggestions for code completion.

A GUI on the other hand is
hard to design, maintain, is prone to errors and when users face an error, there
is no options for workarounds


The evaluation of plotastic within this thesis reflects its potential to address
key challenges in the field of data analysis. The software integrates a
comprehensive suite of statistical tests, such as ANOVA and t-tests, designed
for adaptability and ease of use, leveraging the functionalities of pingouin.

In the context of the reproducibility crisis in scientific research, plotastic
offers noteworthy contributions, though it is not positioned as a universal
remedy. The tool's unique approach to integrating statistical analysis with
visual representation establishes a new paradigm, promoting methodological
transparency. By mandating that statistical analyses accompany relevant
graphical outputs, plotastic ensures that analyses are not only conducted with
proper scientific rigor but also documented in a manner that facilitates
replication, provided the user possesses proficiency in Python.

\citet{pengReproducibleResearchComputational2011} claims that exploration of the
data and the analysis code may be sufficient to verify the quality of the
scientific claims.

\citet{mesirovAccessibleReproducibleResearch2010} called out for "simple,
intuitive ways to both capture and embed our computational work directly into
our papers". In this thesis, combining Jupyter Notebooks with plotastic could
represent an intriguing solution to solve that problem.

Statistical or Data illiteracy is a critical issue that's well researched for both clinicians and
biomedical researchers \cite{lakhlifiIllusionKnowledgeStatistics2023,federerDataLiteracyTraining2016}.
plotastic removes a lot of need for statistical knowledge, yet still performs complex statistical analyses,
if the measurements were correctly performed and the results are correctly interpreted.
Using unpaired non-parametric tests, plotastic provides statistically sound for a wide array of study types without
the need for the user to understand the underlying statistical principles. This is a significant advantage for
biologists, who often lack the necessary statistical training to perform these analyses manually.

Usability is a critical attribute of analytical software, particularly as
researchers confront increasingly complex datasets. While the developer's
intimate familiarity with plotastic may bias perceptions of its ease of use, it
is recognized that novices may initially encounter challenges. Nevertheless,
plotastic is distinguished by its user-friendly interface, enabling users with
minimal statistical training to perform sophisticated analyses by intuitively
mapping plotting concepts to statistical operations.

The transition to a new analytical framework, especially one that incorporates
coding, presents a learning curve. However, the advantages of plotastic in terms
of analytical clarity, speed, and depth are anticipated to outweigh these
initial challenges. Support mechanisms, such as assistance from advanced AI like
ChatGPT, are available to mitigate these hurdles, supporting users across
varying levels of expertise.

In conclusion, plotastic is posited as a valuable tool in the landscape of
scientific research, offering a means to enhance the reproducibility and
efficiency of data analysis. Its development ethos emphasizes simplifying
complex analytical tasks, thereby contributing to the broader goal of fostering
transparent and reproducible research practices.


DO we apply the principles of Semi-Automation to the software?

what features are missing?
- Bivariate analysis
- Filer to help save the output?
- StatResults: System to suggest the correct test, based on the data
