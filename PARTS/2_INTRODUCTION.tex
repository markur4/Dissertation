

\unnsection{Introduction}
To provide a comprehensive background for the following chapters that
focus on the interaction of human mesenchymal
stromal cells (hMSCs) with multiple myeloma (MM) cells, this

% From paper 1 (AACR): Multiple myeloma arises from clonal expansion of
% malignant plasma cells in the bone marrow (BM). At diagnosis, myeloma cells
% have disseminated to multiple sites in the skeleton and, in some cases, to
% “virtually any tissue” (1,2). However, the mechanism through which myeloma
% cells initially disseminate remains unclear.

% Dissemination is a multistep process involving invasion, intravasation,
% intravascular arrest, extravasation, and colonization (3). To initiate
% dissemination, myeloma cells overcome adhesion, retention, and dependency on
% the BM microenvironment, which could involve the loss of adhesion factors such
% as CD138 (4,5). BM retention is mediated by multiple factors: First,
% chemokines (CXCL12 and CXCL8) produced by mesenchymal stromal cells (MSCs),
% which attract plasma cells and prime their cytoskeleton and integrins for
% adhesion (6,7). Second, myeloma cells must overcome the anchorage and physical
% boundaries of the extracellular matrix (ECM), consisting of e.g. fibronectin,
% collagens, and proteoglycans such as decorin (8–11). Simultaneously, ECM
% provides signals inducing myeloma cell cycle arrest or progression the cell
% cycle (8,11). ECM is also prone to degradation, which is common in several
% osteotropic cancers, and is the cause of osteolytic bone disease. This is
% driven by a ‘vicious cycle’ that maximizes bone destruction by extracting
% growth factors (EGF and TGF-β) that are stored in calcified tissues (12).
% Third, direct contact with MSCs physically anchors myeloma cells to the BM
% (3,13). Fourth, to disseminate to distant sites, myeloma cells require, at
% least partially, independence from essential growth and survival signals
% provided by MSCs in the form of soluble factors or cell adhesion signaling
% (5,14,15). For example, the VLA4 (Myeloma)–VCAM1 (MSC)-interface activates
% NF-κB in both myeloma and MSCs, inducing IL-6 expression in MSCs. The
% independence from MSCs is then acquired through autocrine survival signaling
% (16,17). In short, anchorage of myeloma cells to MSCs or ECM is a
% ‘double-edged sword’: adhesion counteracts dissemination, but also presents
% signaling cues for growth, survival, and drug resistance (18).

% To address this ambiguity, we developed an in vitro co-culture system modeling
% diverse adhesion modalities to study dissemination, growth, and survival of
% myeloma cells and hMSCs. Co-cultures of hMSCs and the myeloma cell line INA-6
% replicated tight interactions and aggregate growth, akin to "microtumors" in
% Ghobrial’s metastasis concept (19). We characterized the growth conformations
% of hMSCs and INA­6 as homotypic aggregation vs. heterotypic hMSC adherence and
% their effects on myeloma cell survival. We tracked INA­6 detachments from
% aggregates and hMSCs, thereby identifying a potential "disseminated"
% subpopulation lacking strong adhesion. We developed innovative techniques
% (V-well adhesion assay and well plate sandwich centrifugation) to separate
% weakly and strongly adherent subpopulations for the subsequent analysis of
% differential gene expression and cell survival. Notably, our strategy resolves
% the differences in gene expression and growth behavior between cells of one
% cell population in "direct" contact with MSCs. In contrast, previous methods
% differentiated between "direct" and "indirect" cell-cell contact using
% transwell inserts (20). To evaluate whether genes mediating adhesion and
% growth characteristics of INA-6 were associated with patient survival, we
% analyzed publicly available datasets (21,22).



% ======================================================================
\unnsubsection{Human Mesenchymal Stem/Stromal Cells}

Explaining what a mesenchymal stromal cell (MSC) is, is not such an easy task as
one might expect. MSCs are derived from multiple\ MSCs different sources, serve
a wide array of functions and are always isolated as a heterogenous group of
cells. This makes it particularly challenging to find a consensus on their exact
definition, nomenclature, exact function and \textit{in vivo} differentiation
potential. Therefore, the most effective approach to describe hMSCs is to
present their historical context.

hMSCs first gained popularity as a stem cell. Stem cells lay the foundation of
multicellular organisms. Embryonic stem cells orchestrate the growth and
patterning during embryonic development, while adult stem cells are responsible
for regeneration during adulthood. The classical definition of a stem cell is
that of a relatively undifferentiated cell that divides asymmetrically,
producing another stem cell and a differentiated
cell~\cite{cooperCellMolecularApproach2000, shenghuiMechanismsStemCell2009}.
Because of their significance in biology and regenerative medicine, stem cells
have become a prominent subject in modern research. Especially human mesenchymal
stromal cells (hMSCs) have proven to be a promising candidate in this
context~\cite{ullahHumanMesenchymalStem2015}.

Mesenchyme first appears in embryonic development during gastrulation. There,
cells that are committed to a mesodermal fate, lose their cell junctions and
exit the epithelial layer in order to migrate freely. This process is called
epithelial-mesenchymal transition
\cite{tamFormationMesodermalTissues1987,nowotschinCellularDynamicsEarly2010}.
Hence, the term mesenchyme describes non-epithelial embryonic tissue
differentiating into mesodermal lineages such as bone, muscles and blood.
Interestingly, it was shown nearly twenty years earlier that cells within adult
bone marrow seemed to have mesenchymal properties as they were able to
differentiate into bone tissue
\cite{friedensteinOsteogenesisTransplantsBone1966,friedensteinOsteogenicPrecursorCells1971,biancoMesenchymalStemCells2014}.
This was the origin of the ``mesengenic process''-hypothesis: This concept
states that mesenchymal stem cells serve as progenitors for multiple mesodermal
tissues (bone, cartilage, muscle, marrow stroma, tendon, fat, dermis and
connective tissue) during both adulthood and embryonic
development~\cite{caplanMesenchymalStemCells1991,caplanMesengenicProcess1994}.
The mesenchymal nature of these cells (termed bone marrow stromal cells: BMSCs)
was confirmed later when they were shown to differentiate into adipocytic (fat)
and chondrocytic (cartilage)
lineages~\cite{pittengerMultilineagePotentialAdult1999}. Since then, the term
``mesenchymal stem cell'' (MSC) has grown popular as an adult multipotent
precursor to a couple of mesodermal tissues. hMSCs derived from bone marrow
(hMSCs) were shown to differentiate into osteocytes, chondrocytes, adipocytes
and cardiomyocytes
\cite{gronthosSTRO1FractionAdult1994,muruganandanAdipocyteDifferentiationBone2009,xuMesenchymalStemCells2004}
Most impressively, these cells also exhibited
ectodermal and endodermal differentiation potential, as they produced
neuronal cells, pancreatic cells and hepatocytes
\cite{barzilayLentiviralDeliveryLMX1a2009,wilkinsHumanBoneMarrowderived2009,gabrInsulinproducingCellsAdult2013,stockHumanBoneMarrow2014}.

Furthermore, cultures with MSC properties can be established from ``virtually
every post-natal organs and tissues'', and not just bone
marrow~\cite{dasilvameirellesMesenchymalStemCells2006}. However, it has to be
noted that hMSCs can differ greatly in their transcription profile and
\textit{in vivo} differentiation potential depending on which tissue they
originated from
\cite{jansenFunctionalDifferencesMesenchymal2010,sacchettiNoIdenticalMesenchymal2016}.

Since ``hMSCs" are a heterogenous group of cells, they were defined by their
\textit{in vitro} characteristics. A minimal set of criteria are the following
\cite{dominiciMinimalCriteriaDefining2006}: First, hMSCs must be plastic
adherent. Second, they must express or lack a set of specific surface antigens
(positive for CD73, CD90, CD105; negative for CD45, CD34, CD11b, CD19). Third,
hMSCs must differentiate to osteoblasts, adipocytes and chondroblasts \textit{in
vitro}. Together, hMSCs exhibit diverse differentiation potentials and can be
isolated from multiple sources of the body. This offers great opportunity for
regenerative medicine, if the particular hMSC-subtype is properly characterized.


% ======================================================================
\unnsubsection{Multiple Myeloma}
Multiple myeloma arises from clonal expansion of malignant plasma cells in the
bone marrow (BM). At diagnosis, myeloma cells have disseminated to multiple
sites in the skeleton and, in some cases, to virtually any
tissue~\cite{rajkumarMultipleMyelomaCurrent2020,
bladeExtramedullaryDiseaseMultiple2022}.


% ======================================================================
\unnsubsection{Myeloma-hMSC Interactions}
Since plasma cells can not survive outside the bone marrow, MM cells also
require survival signals for growth and disease progression. These signals are
produced by the bone marrow microenvironment, including ECM, MSCs and
ACs~\cite{kiblerAdhesiveInteractionsHuman1998,
garcia-ortizRoleTumorMicroenvironment2021}.


% ======================================================================
\unnsubsection{Myeloma Bone Disease}
Bone is a two-phase system in which the mineral phase provides the stiffness and
the collagen fibers provide the ductility and ability to absorb
energy~\cite{viguet-carrinRoleCollagenBone2006}. On a molecular level, bone
tissue is composed of extracellular matrix (ECM) proteins that are calcified by
hydroxyapatite crystals. This ECM consists mostly of collagen type I, but also
components with major regulatory activity, such as fibronectin and proteoglycans
that are essential for healthy bone
physiology~\cite{alcorta-sevillanoDecipheringRelevanceBone2020}. Bone tissue is
actively remodeled by bone-forming osteoblasts and bone-degrading osteoclasts.
Osteoblasts are derived from mesenchymal stromal cells (MSCs) that reside in the
bone marrow~\cite{friedensteinOsteogenesisTransplantsBone1966,
pittengerMultilineagePotentialAdult1999}. MSCs also give rise to adipocytes
(ACs) to form Bone Marrow Adipose Tissue (BMAT), which can account for up to
70\% of bone marrow volume~\cite{fazeliMarrowFatBone2013}.

MM indirectly degrades bone tissue by stimulating osteoclasts and inhibiting
osteoblast differentiation, which leads to MM-related bone disease
(MBD)~\cite{glaveyProteomicCharacterizationHuman2017}. MBD is present in 80\% of
patients at diagnosis and is characterized by osteolytic lesions, osteopenia and
pathological fractures~\cite{terposPathogenesisBoneDisease2018}.


% ======================================================================
\unnsubsection{Dissemination of Myeloma Cells}
dissemination is still widely unclear
- multistep process
- invasion, intravasation, intravascular arrest, extravasation,
colonization
- overcome adhesion, retention, and dependency on the BM
microenvironment
- loss of adhesion factors such as CD138

% > New Page to separate biological part from the technical part 
\newpage


% ======================================================================
% == Coding part
% ======================================================================

% ======================================================================
\unnsubsection{The Increasing Role of Software in Biomedicine}
The increasing importance of software in biomedicine is a direct consequence of
the increasing complexity of biological data. The biological community has been
producing more data in shorter time, posing new
challenges~\cite{yangScalabilityValidationBig2017}. - RNAseq - single cell
rnaseq - sequence

Artificial intelligence (AI) has been a game changer in the field of
biomedicine. The early development of AI itself was driven by radiology,
where it was designed to detect pathologies in medical images.


% ======================================================================
\unnsubsection{Code Quality Ensures Scientific Reproducibility}
A main reason to write software is to define re-usable instructions for task
automation~\cite{narztReusabilityConceptProcess1998}. However, the complexity of
the code makes it prone to errors and can prevent usage by persons other than
the author himself. This is a problem for the general scientific community, as
the software is often the only way to reproduce the results of a
study~\cite{sandveTenSimpleRules2013}. Hence, modern journals aim to enforce
standards to software development, including software written and used by
biological researchers~\cite{smithJournalOpenSource2018}. Here, we provide a
brief overview of the standards utilized by \texttt{plotastic} that to ensure
its reliability and reproducibility by the scientific community.

Modern software development is a long-term commitment of maintaining and
improving code after initial release~\cite{boswellArtReadableCode2011}. Hence,
it is good practice to write the software such that it is scalable, maintainable
and usable. Scalability or, to be precise, structural scalability means that the
software can easily be expanded with new features without major modifications to
its architecture \cite{bondiCharacteristicsScalabilityTheir2000}. This is
achieved by writing the software in a modular fashion, where each module is
responsible for a single function. Maintainability means that the software can
easily be fixed from bugs and adapted to new requirements
\cite{kazmanMaintainability2020}. This is achieved by writing the code in a
clear and readable manner, and by writing tests that ensure that the code works
as expected~\cite{boswellArtReadableCode2011}. Usability is hard to
define~\cite{brookeSUSQuickDirty1996}, yet one can consider a software as usable
if the commands have intuitive names and if the software's manual, termed
``documentation'', is up-to-date and easy to understand for new users with
minimal coding experience. A software package that has not received an update
for a long time (approx. one year) could be considered abandoned. Abandoned
software is unlikely to be fully functional, since it relies on other software
(dependencies) that has changed in functionality or introduce bugs that were not
expected by the developers of all dependencies. Together, software that's
scalable, maintainable and usable requires continuous changes to its codebase.
There are best practices that standardize the continuous change of the codebase,
including version control, continuous integration (often referred to as CI), and
software testing.

Version control is a system that records changes to the codebase line by line,
allowing the documentation of the history of the codebase, including who made
which changes and when. This is required to isolate new and experimental
features into newer versions and away from the stable version that's known to
work. The most popular version control system is Git, which is considered the
industry standard for software development~\cite{chaconGitBook2024}. Git can use
GitHub.com as a platform to store and host codebases in the form of software
repositories. GitHub's most famous feature is called ``pull request''. A pull
request is a request from anyone registered on GitHub to include their changes
to the codebase (as in ``please pull this into your main code''). One could see
pull requests as the identifying feature of the open source community, since it
exposes the codebase to potentially thousands of independent developers,
reaching a workforce that is impossible to achieve with closed source models
used by paid software companies.

Continuous integration (CI) is a software development practice in which
developers integrate code changes into a shared repository several times a
day~\cite{duvall2007continuous}. Each integration triggers the test suite,
aiming to detect errors as soon as possible. The test suite includes building
the software, setting up an environment for the software to run and then
executing the programmed tests, ensuring that the software runs as a whole.
Continuous integration is often used together with software branches. Branches
are independent copies of the codebase that are meant to be merged back into the
original code once the changes are finished. Since branches accumulate multiple
changes over time, this can lead to minor incompatibilities between the branches
of all developers (integration conflicts), which is something that CI helps to
prevent.

Continuous integration especially relies on a thorough software testing suite.
Software testing is the practice of writing code that checks if the codebase
works as expected~\cite{10.5555/2161638}. The main type of software testing is
unit testing, which tests the smallest units of the codebase (functions and
classes) in isolation (\autoref{lst:unit_test}). The quality of the software
testing suite is measured by the code coverage, the precision of the tests, and
the number of test-cases that are checked. The code coverage is the percentage
of the codebase that is called by the testing functions, which should be as
close to 100\% as possible, although it does not measure how well the code is
tested. The precision of the test is not a measurable quantity, but it
represents if the tests truly checks if the code works as expected. The number
of test-cases is the number of different scenarios that are checked by the
testing functions, for example testing every possible option or combinations of
options for functions that have multiple options. The most popular software
testing framework for python is \texttt{pytest}, which is utilized by
\texttt{plotastic}~\cite{pytestx.y}.

\def\mycaption{ Example of an arbitrary python function and its respective unit
    test function. The first function simply returns the number 5. The second
    function tests if the first function indeed returns the number 5. The test
    function is named with the prefix ``\texttt{test\_}'' and is placed in a
    file that ends with the suffix ``\texttt{\_test.py}''. The test function is
    executed by the testing framework \texttt{pytest}. Note that code after
    ``\texttt{\#}'' is considered a comment and won't be executed.}
\begin{lstlisting}[
    language=Python, 
    style=pythonstyle,
    label=lst:unit_test, 
    caption=\mycaption,
    ]
    # Define a function called "give_me_five" that returns the number 5
    def give_me_five():
        return 5
    # Define a test function asserting that "give_me_five" returns 5
    def test_give_me_five():
        assert give_me_five() == 5 
\end{lstlisting}


% ======================================================================
\unnsubsection{Python as a Programming Language}
Here, we provide a general overview of the python programming language,
explaining terms like ``type'', ``method'', etc., in order to prepare readers
without prior programming experience for the following chapters. We also
describe the design principles of python to lay out the key concepts that
differentiate python compared to other programming languages.

Languages such as python are considered ``high-level'', which means that it is
designed to be easy to read and write, but also independent of
hardware~\cite{PythonLanguageReference}. A key principle of python is the
emphasis on implementing a syntax that is concise and close to human language
(\autoref{lst:readable}, \autoref{lst:not_readable}).

\def\mycaption{ Example of readable python code. This one-line code
    returns the words (string) \texttt{"Hello, World!"} when executed. The command
    is straightforward and easy to understand.}
\begin{lstlisting}[
    language=Python, 
    style=pythonstyle,
    label=lst:readable,
    caption=\mycaption
    ]
    print("Hello, World!")
    # Output: Hello, World!
\end{lstlisting}

\def\mycaption{ Example of less readable code written in the low-level
    programming language C. This code is doing exactly the same as the python
    code in \autoref{lst:readable}. The command is harder to understand because
    more steps are needed to access the same functionality, including the
    definition of a function}
\begin{lstlisting}[
    language=C, 
    style=defaultstyle,
    label=lst:not_readable, 
    caption=\mycaption
    ]
    #include <stdio.h>
    int main() {
        printf("Hello, World!");
        return 0;
    }
    // Output: Hello, World!
\end{lstlisting}

Furthermore, python is an interpreted language, which means that the code is
executed line by line. This makes coding easier because the programmer can see
the results of the code immediately after writing it, and error messages point
to the exact line where the error occurred. This is in contrast to compiled
languages, where the code has to be compiled into machine code before it can be
executed. The advantage of compiled languages is that the code runs faster,
because the machine code is optimized for the hardware.

Python automates tasks that would otherwise require an advanced understanding of
computer hardware, like the need for manual allocation of memory space. This is
achieved by using a technique called ``garbage collection'', which automatically
frees memory space that is no longer needed by the program. This is a feature
that is not present in low-level programming languages like C or C++, that were
designed to maximize control over hardware.

Another hallmark of python is its dynamic typing system. In python the type  is
inferred automatically during code execution (\autoref{lst:dynamic_typing}).
This is in contrast to statically typed languages like C, where the type of a
variable has to be declared explicitly and cannot be changed during code
execution (\autoref{lst:static_typing})~\cite{PythonLanguageReference}. This
makes python a very beginner-friendly language, since one does not have to keep
track of the type of each variable. However, this also makes python a slower
language, because the interpreter has to check the type of each variable during
code execution. Also, developing code with dynamic typing systems is prone to
introducing bugs (``type errors''), because implicit type conversions can lead
to unexpected behavior. Hence, larger python projects require disciplined
adherence to programming conventions. One such convention is type hinting, which
is a way to explicitly state the type of a variable, and is used to make the
code more readable and understandable for other developers, and allows for
development software to detect type errors before execution.
(\autoref{lst:type_hint})~\cite{vanrossumPEP484Type2014}.


\def\mycaption{ Example of dynamic typing in python. The variable ``\texttt{a}''
    is assigned the value 5, which is an integer. The variable ``\texttt{a}'' is
    then assigned the value ``\texttt{Hello, World!}'', which is a string.
    Python allows  Note that code after ``\texttt{\#}'' is considered a comment
    and won't be executed.}
\begin{lstlisting}[
    language=Python,
    style=pythonstyle,
    label=lst:dynamic_typing,
    caption=\mycaption,
    ]
    a = 5  # Type integer
    a = 5.0  # Type float
    a = "Hello, World!"  # Type string
    a = True  # Type boolean
    a = False  # Type boolean
    a = [1, 2, 3]  # Type list of integers
    a = {"name": "Regina"}  # Type dictionary
\end{lstlisting}
\def\mycaption{ Example of static typing in C. The variable ``a'' is declared as
    an integer, and can only store integers. The variable ``a'' is then assigned
    the value 5, which is an integer. The variable ``a'' is then assigned the
    value ``Hello, World!'', which is a string. This results in a compilation
    error, because the variable ``a'' can only store integers.}
\begin{lstlisting}[
    language=C,
    style=defaultstyle,
    label=lst:static_typing,
    caption=\mycaption,
    ]
    int a;  // Declare type as integer
    a = 5;
    a = "Hello, World!";  // Compilation error!
\end{lstlisting}
\def\mycaption{ Example of type hints used in python. Explicitly stating
    the type of the variable is optional and does not change the
    behavior of the code as shown in \autoref{lst:dynamic_typing}.}
\begin{lstlisting}[
    language=Python,
    style=pythonstyle,
    label=lst:type_hint,
    caption=\mycaption,
    ]
    a: int = 5
    a: str = "Hello, World!"
\end{lstlisting}


Python supports both functional and object-oriented programming paradigms. In
functional programming, the code is written in a way that the program is a
sequence of function calls, where each function call returns a value that is
used in the next function call (\autoref{lst:functional}). This approach is
useful when multiple actions have to be performed on the same data and the
structure of the data is relatively simple, for example a string of a gene.

\def\mycaption{ Example of functional programming in Python. The code
    defines a function called ``\texttt{find\_restriction\_site}'' that
    finds the position of a restriction site in a gene. The function
    ``\texttt{cut}'' uses the function ``\texttt{find\_restriction\_site}''
    to cut the gene at the restriction site.}
\begin{lstlisting}[
    language=Python,
    style=pythonstyle,
    label=lst:functional,
    caption=\mycaption,
    ]
    gene1 = "TGAGCTGAGCTGATGCGCTATATTTAGGCG"
    
    def find_restriction_site(gene: str):
        return gene.find("GCGC")
    
    def cut(gene: str):
        position = find_restriction_site(gene)
        return gene[position:]   
        
    gene1_cut = cut(gene1)
    print(gene1_cut)
    # Output: GCGCTATATTTAGGCG
    
    
\end{lstlisting}

% # Define a function to introduce a person
% def introduce(person):
%     say_hello()
%     print("I am " + person['name'] + "!")

% # Create a person
% marie = create_person("Marie")

% # Introduce the person
% introduce(marie)

% # Output: Hello, I am Marie!

When the data itself gains in complexity, an object-oriented approach is more
suitable (\autoref{lst:oop}). Object-oriented programming is a programming
paradigm that uses objects and classes. An object is a collection of both data
and functions, and a class is a blueprint for creating objects. Functions that
are associated with a class are called methods.


The main benefit of using an object oriented versus a
functional approach is that every method has access to the data of the
object

\def\mycaption{ Example of a class in python. The class is called
    \texttt{Person} and has two functions, ``\texttt{\_\_init\_\_}'' and
    ``\texttt{say\_hello}''. The function ``\texttt{\_\_init\_\_}'' is called
    when creating (``initializing'') an object and fills the object with data.
    The parameter ``\texttt{self}'' is used to reference the object itself
    internally. The function ``\texttt{say\_hello}'' is a method that prints the
    string ``Hello!'' when called. The method ``\texttt{introduce}'' is a method
    that calls the method ``\texttt{say\_hello}'' and comines the string ``I
    am'' with the name of the object. }
\begin{lstlisting}[
    language=Python,
    style=pythonstyle,
    label=lst:oop,
    caption=\mycaption,
    ]
    class Gene:
        def __init__(self, sequence: str, organism: str, exons: list):
            self.sequence: str = sequence
            self.organism: str = organism
            self.exons: list = exons
            self.promotor: str = self.find_promotor()
        def find_promotor(self):
            return self.sequence.find("TATA")
        def find_restriction_site(self):
            return self.sequence.find("GCGC")
        def cut(self):
            position = self.find_restriction_site()
            return self.sequence[position:]
            
    gene1 = Gene(
        sequence="TGAGCTGAGCTGATGCGCTATATTTAGGCG", 
        organism="Human"
        )
    gene1_cut = gene1.cut()
    print(gene1_cut)
    # Output: GCGCTATATTTAGGCG
\end{lstlisting}
% # Define a class called "Person"
% class Person:
%     def __init__(self, name: str):
%         self.name: str = name
%     def say_hello(self):
%         print(f"Hello!")
%     def introduce(self):
%         self.say_hello()
%         print("I am " + self.name + "!")

% regina = Person("Regina")  # Create an object of the class Person
% regina.introduce()  # Call the method introduce

% # Output: Hello, I am Regina!

Initially, object-oriented programming was designed to
be used when the functions rely on the state of the data and
modifications of such state.

This state could be for example a Table containing data, and the functions could
be the operations that modify the table. When designing software, both paradigms
can be used together, where object oriented programming is often used to design
the overall architecture of the software, and functional programming is used to
write the code that implements the functionalities.




% ======================================================================
\unnsubsection{Data Science with Python}
the ease of use has made python a very popular language~\cite{rayhanRisePythonSurvey2023}

Like any other programming language, python alone does not provide specialized
tools like those used for data analysis \cite{PythonLanguageReference}. However,
python was designed to be extended by packages developed by its users. A python
package consists of multiple python modules, where each module is a text-file
with a \texttt{.py} ending containing python code. Famous examples of such
packages are \texttt{pytorch} and \texttt{tensorflows}, that are used to build
models of artificial intelligence, including ChatGPT
\cite{paszkePyTorchImperativeStyle2019, abadiTensorFlowLargeScaleMachine2016,
radfordLanguageModelsAre2019}. Here, we outlay the most important packages used
for \texttt{plotastic}.

Interactive Python
- Jupyter

Python overcame the issues of interpreted language by utilizing Code
written in C
numpy:
- Acceleration,
- SIMD instructions

Tabular operations
- pandas

Data visualization
- matplotlib
- seaborn


Inferential Statistics
- pingouin

AI:
- pytorch and tensorflow
- example: VGG19 is just a few lines of code (\autoref{lst:vgg19}) asdfdf


% ======================================================================
\unnsubsection{Convolutional Neural Networks}
This work greatly benefited from the use of convolutional neural
networks provided by the Zeiss ZEN software. Here, we provide a brief
