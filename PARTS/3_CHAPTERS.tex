
% Zusätzliche Einleitung und Diskussion: Herr Kuric soll bitte eine ergänzende Einleitung und Diskussion in dem Kapitel zu seiner JOSS-Publikation hinzufügen, die folgende Informationen enthalten:

% Darstellung von Umfeld, Aufgabenstellung und Signifikanz für die
% biomedizinische Anwendung
% (warum braucht es die Biomedizin?)

% Darstellung der Anforderungen an die Programmierung: Integration von
% Informationen, die die spezifischen Anforderungen verdeutlichen,
% welche die Programmierung der Software erforderlich gemacht haben.
% (warum habe ich es gebraucht?) 

% Darstellung der Nutzbarkeit für Naturwissenschaftler: Klare und auch
% für Nicht-Informatiker verständliche Darstellung, welche konkreten
% Anwendungsmöglichkeiten die Software für Naturwissenschaftler bietet.
% (warum ist es nützlich?)

% Mit einer angemessenen Einleitung und Diskussion würde zum einen den
% Guidelines der GSLS entsprochen, die für alle zugrunde gelegt werden.
% Zum anderen würde es auch dem großen Anteil von Nicht-Informatikern in
% der GSLS erlauben, den Hintergrund und die Signifikanz und
% Verwendungsmöglichkeiten des entwickelten Codes besser zu verstehen.
% Die GSLS hat sich mit der Aufnahme von informatischen Projekten
% interdisziplinär geöffnet. Gleichzeitig erwarten wir damit aber auch
% von den Doktorierenden den Willen, die Arbeit in einer
% interdisziplinären Form in der Thesis zu präsentieren.



% ======================================================================
% == Chapter 1
% ======================================================================
\unnsection{Chapter 1: Modelling Myeloma Dissemination \textit{in vitro}}
% Title: Keep it Together: Modelling Myeloma Dissemination in vitro with
% hMSC- Interacting Subpopulations of INA-6 Cells and their
% Aggregation/Detachment Dynamics 


% == Paper 1 ===========================================================
% ## Import paper here
\addpdf{Research Article: Cancer Research Communications}{PUBLICATIONS/AACR.pdf}


% == Paper 1 Supplemental ==============================================
% ## Import paper here
\addpdf{Methods: Supplementary Figures and Methods}{PUBLICATIONS/? AACR Supplemental.pdf}



% ======================================================================
% == Chapter 2
% ======================================================================
\unnsection{Chapter 2: Semi-Automation of Data Analysis}
% Article title: "plotastic: Bridging Plotting and Statistics in Python"

% == Sub-Introduction ==================================================
% GSLS asked me to nest the second chapter between another introduction
% and discussion


\unnsubsection{Introduction}

As laid out in the introduction, one can doubt if a PhD student without coding
skillsis at its max efficiency.

Why does Biomedicine need plotastic?:
- Thorough analysis has become a standard, with assumption testing, omnibus
tests and post-hoc analyses for every experiment.
- But data is increasing
    - Example of my data?
- The number of dedicated statisticians is limited
- The know-how of statistics in biology is limited, for example, Some authors
ignored the problem of multiple testing while others used the method
uncritically with no rationale or discussion \cite{pernegerWhatWrongBonferroni1998,armstrongWhenUseBonferroni2014}


Why did I need plotastic?

Why do biologists need plotastic?
- Assays output more data in shorter time, e.g. multiplex qPCR
- example: 20 genes, 3 timepoints, 11 biological replicates, (all 3
technical replicates already averaged)
- 20 * 3 * 11 = 660 data points

this is multidimensional data:  660 data points spread across two dimensions: time
and gene

-in manual analysis e.g. in Excel, the user has to manually select the
data, copy it, paste it into a new sheet, and then perform the
statistical test. In Prism, the user has to select the data, click on
the statistical test, and then select the data again. This is not only
time-consuming, but also prone to

- Re-Analysis: The user has to repeat the process for every gene and
timepoint. This is not only time-consuming, but also prone to errors.

shortly Describe Main Packages in more detail:
- seaborn: It multidimensional data
- pingouin: It's a statistical package


% == Paper 2 ===========================================================
% ## Import paper here
\addpdf[.93]
{Software Article: Journal of Open Source Software}
{PUBLICATIONS/§-kuricPlotasticBridgingPlotting2024.pdf}


% == Sub-Discussion ====================================================
\unnsubsection{Discussion}



Is plotastic USABLE for biologists?
- Yes but use is limited by minimal knowledge of Python
- However, that is subject to change as Python is becoming more popular
in biology and AI assisted coding decreased the barrier to entry
significantly. Tools like github copilot are able to generate code, fix
bugs and suggest improvements. This is a game changer for biologists
that are not familiar with programming.
- Furthermore, installing and using plotastic for biologists is overestimated. These
steps re needed:
- Install anaconda from the internet
- Open the terminal
- Type \texttt{pip install plotastic}
- Check Rea
