

% ======================================================================
% == Abstract
% ======================================================================


% == from paper ========================================================
% Abstract 1st paper (AACR):
% Multiple myeloma involves early dissemination of malignant plasma
% cells across the bone marrow; however, the initial steps of
% dissemination remain unclear. Human bone marrow-derived mesenchymal
% stromal cells (hMSCs) stimulate myeloma cell expansion (e.g., IL-6)
% and simultaneously retain myeloma cells via chemokines (e.g., CXCL12)
% and adhesion factors. Hence, we hypothesized that the imbalance
% between cell division and retention drives dissemination. 

% We present an in vitro model using primary hMSCs co-cultured with
% INA-6 myeloma cells. Time-lapse microscopy revealed proliferation and
% attachment/detachment dynamics. Separation techniques (V-well adhesion
% assay and well plate sandwich centrifugation) were established to
% isolate MSC-interacting myeloma subpopulations that were characterized
% by RNAseq, cell viability and apoptosis. Results were correlated with
% gene expression data (n=837) and survival of myeloma patients (n=536). 

% On dispersed hMSCs, INA-6 saturate hMSC-surface before proliferating
% into large homotypic aggregates, from which single cells detached
% completely. On confluent hMSCs, aggregates were replaced by strong
% heterotypic hMSC-INA-6 interactions, which modulated apoptosis
% time-dependently. Only INA-6 daughter cells (nMA-INA6) detached from
% hMSCs by cell division but sustained adherence to hMSC-adhering mother
% cells (MA-INA6).

% Isolated nMA-INA6 indicated hMSC-autonomy through superior viability
% after IL­6 withdrawal and upregulation of proliferation-related genes.
% MA-INA6 upregulated adhesion and retention factors (CXCL12), that,
% intriguingly, were highly expressed in myeloma samples from patients
% with longer overall and progression-free survival, but their
% expression decreased in relapsed myeloma samples. 

% Altogether, in vitro dissemination of INA-6 is driven by detaching
% daughter cells after a cycle of hMSC-(re)attachment and proliferation,
% involving adhesion factors that represent a bone marrow-retentive
% phenotype with potential clinical relevance.


% Summary 2nd paper (JOSS):
% plotastic addresses the challenges of transitioning from exploratory
% data analysis to hypothesis testing in Python’s data science ecosystem.
% Bridging the gap between seaborn and pingouin, this library offers a
% unified environment for plotting and statistical analysis. It simplifies
% the workflow with a user-friendly syntax and seamless integration with
% familiar seaborn parameters (y, x, hue, row, col). Inspired by seaborn’s
% consistency, plotastic utilizes a DataAnalysis object to intelligently
% pass parameters to pingouin statistical functions. The library
% systematically groups the data according to the needs of statistical
% tests and plots, conducts visualisation, analyses and supports extensive
% customization options. In essence, plotastic establishes a protocol for
% configuring statical analyses through plotting parameters. This approach
% streamlines the process, translating seaborn parameters into statistical
% terms, allowing researchers to focus on correct statistical testing and
% less about specific syntax and implementations.

% Statement of need 2nd paper (JOSS):
% Python’s data science ecosystem provides powerful tools for both
% visualization and statistical testing. However, the transition from
% exploratory data analysis to hypothesis testing can be cumbersome,
% requiring users to switch between libraries and adapt to different
% syntaxes.

% seaborn has become a popular choice for plotting in Python, offering an
% intuitive interface. Its statistical functionality focuses on
% descriptive plots and bootstrapped confidence intervals (Waskom, 2021).
% The library pingouin offers an extensive set of statistical tests, but
% it lacks integration with common plotting capabilities (Vallat, 2018).
% statannotations integrates statistical testing with plot annotations,
% but uses a complex interface and is limited to pairwise comparisons
% (Charlier et al., 2022).

% plotastic addresses this gap by offering a unified environment for
% plotting and statistical analysis. With an emphasis on user-friendly
% syntax and integration with familiar seaborn parameters, it simplifies
% the process for users already comfortable seaborn. The library ensures a
% smooth workflow, from data import to hypothesis testing and
% visualization


% == English ===========================================================
\renewcommand{\abstractname}{Summary}
\begin{abstract}
    Lorem Ipsum dolor sit amet, consetetur sadipscing elitr, sed diam nonumy
    eirmod tempor invidunt ut labore et dolore magna aliquyam erat, sed diam
    voluptua. At vero eos et accusam et justo duo dolores et ea rebum. Stet
    clita kasd gubergren, no sea takimata sanctus est Lorem ipsum dolor sit
    amet. Lorem ipsum dolor sit amet, consetetur sadipscing elitr, sed diam
    nonumy eirmod tempor invidunt ut labore et dolore magna aliquyam erat,
    sed diam voluptua. At vero eos et accusam et justo duo dolores et ea
    rebum. Stet clita kasd gubergren, no sea takimata sanctus est Lorem
    ipsum
\end{abstract}

% == German ============================================================
\renewcommand{\abstractname}{Zusammenfassung}
\begin{abstract}
    Lorem Ipsum dolor sit amet, consetetur sadipscing elitr, sed diam nonumy
    eirmod tempor invidunt ut labore et dolore magna aliquyam erat, sed diam
    voluptua. At vero eos et accusam et justo duo dolores et ea rebum. Stet
    clita kasd gubergren, no sea takimata sanctus est Lorem ipsum dolor sit
    amet. Lorem ipsum dolor sit amet, consetetur sadipscing elitr, sed diam
    nonumy eirmod tempor invidunt ut labore et dolore magna aliquyam erat,
    sed diam voluptua. At vero eos et accusam et justo duo dolores et ea
    rebum. Stet clita kasd gubergren, no sea takimata sanctus est Lorem
    ipsum bla
\end{abstract}